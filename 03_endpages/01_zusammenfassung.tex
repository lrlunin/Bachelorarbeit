\chapter{Diskussion und Zusammenfassung}
Das Ziel dieser Arbeit war es, ein Experiment zu entwerfen und durchzuführen, welches nachweist, dass die Detektion resonanter Kleinwinkelstreuung von einer magnetisch lateral heterogenen Probe an einer Laborquelle für weiche Röntgenstrahlen möglich ist. Als Röntgenquelle wurde eine Laser-getriebene Plasmaquelle eingesetzt, in der die Röntgenstrahlung aus dem breiten Emissionspektrum mit einer Reflexionszonenplatte nahe um die Gd-Resonanzenergie fokussiert wird. Das Experiment sollte zeigen, ob die zu erwartende geringe Kohärenz der Quelle für den Streuversuch ausreichend ist und ob magnetischer Streukontrast auch für unpolarisierte Strahlung möglich ist. Größte Herausforderung war aber die Detektion des zu erwartenden extrem kleinen Streusignals in der Größenordung von nur einigen wenigen Photonen pro Puls.

\noindent
Die wesentliche Neuerung des durchgeführten Experiments ist die Einzelpuls-basierte Detektion des Signals. Diese Art der Detektion in Verbindung mit Algorithmen zum Auffinden einzelner Photonen-Ereignisse in jeder Einzelaufnahme des Detektors sollte es ermöglichen, das \glsfirst{snr} so zu verbessern, dass auch extrem kleine Signale wie die Kleinwinkelstreuung gemessen werden können. Für dieses Detektionsschema war es nötig, die Röntgenquelle und einen neuartigen schnellen Flächendetektor elektrisch zu synchronisieren und den Detektor in das Datenerfassungssystem der Laborquelle zu integrieren. Die Algorithmen zum Trennen von Photonen-Ereignissen und Hintergrundrauschen wurden im Zuge dieses Experiments genau analysiert.

\noindent
Der verwendete MÖNCH Detektor wurde hierfür vorab ausführlich getestet und charakterisiert. Dabei wurde eine hohe Quanteneffizienz von über \SI{90}{\percent} bei einer Photonenenergien an den Gd M-Kanten (ca.\ \SI{1200}{\eV}) gemessen. Auch das Rauschverhalten als Funktion der Belichtungszeit wurde als wichtige Kenngröße bestimmt. Dabei hat sich gezeigt, dass das Rauschen generell recht hoch im Vergleich zum Photonensignal ist. Dieses Signal verteilt sich zu dem auf mehrere Pixel. So enthält der hellste Pixel nur ca.\ \SI{64}{\percent} des gesamten Signals. Das Detektorrauschen kann aber durch Absenken der Belichtungszeit deutlich reduziert werden. Um diesen Vorteil nutzen zu können, ist aber eine genaue Synchronisation zwischen Quelle und Detektor erforderlich.

\noindent
Bei der Bestimmung der Rauschcharakteristik ist aber auch offenbar geworden, dass das hier eingesetzte Modul die veröffentlichte Spezifikation \cite{ramilli-measurements-2017} nicht erreicht (siehe Abb.~\ref{fig:noise_moench}). Während bei einer Belichtungszeit unterhalb von \SI{10}{\micro s} die Abweichungen noch klein sind (\SI{<10}{\percent}), erreichen sie bei \SI{1}{ms} Belichtungszeit fast \SI{50}{\percent}. Möglicherweise gibt es hier über eine Optimierung der Detektorparameter mit Unterstützung des Herstellers noch Raum für Verbesserungen.

\noindent
Der MÖNCH-Detektor wurde mit der Laser-getriebenen Röntgenlaborquelle so synchronisiert, dass jeder Puls der Röntgenquelle mit dem Detektor als einzelne Aufnahme aufgezeichnet werden konnte. Die Belichtungszeit ließ sich bis zu \SI{1}{\micro\second} reduzieren. Die kürzeste Belichtungszeit des MÖNCH-Detektors von \SI{100}{\nano\second} konnte aufgrund des Jitters zwischen dem Trigger-Generator und dem Detektor nicht eingestellt werden, was im Endeffekt aber nur zu einer geringen Verbeserung des Detektorrauschens geführt hätte. Der Rauschabstand zum zentralen Pixel eines Photonenereignisses (an der Gd M5-Kante) betrug $\num{5.8}\sigma_R$ (\SI{7.6}{\dB}). Wie in dieser Arbeit gezeigt wurde, reicht dieser Abstand zur Detektion von einzelnen Photonen bei der verwendeten Photonenenergie über einen Schwellenwert-Algorithmus aus.
Wegen des extrem niedrigen detektierten Photonenflusses von nur ca.\ 60 Photonen pro Puls, ist die Beobachtung eines Streumusters beim Aufsummieren einzelner Streubilder ohne diese Auswertung unmöglich.

\noindent
Es wurden zwei Algorithmen zur Auswertung der aufgenommenen Streubilder implementiert. Im Schwellenwert-Algorithmus wird das Signal in einem Pixel als ein Photon bestimmt, wenn dieses einen vorgegebenen Schwellenwert überschreitet. Im Clustering-Algorithmus werden zunächst die Summen von Pixel-Clustern abgebildet, die mit dem vorgegebenen Schwellenwert verglichen werden. Beiden Algorithmen waren in der Lage, einzelne gestreute Photonen durch Diskreditierung vom Hintergrundrauschen zu trennen. Obwohl der Clustering-Algorithmus für den Fall der starken Spreizung des Ein-Photon-Signals entwickelt wurde, wurde eine große Anzahl an fehldetektierten Photonen beobachtet. Der Einsatz des Schwellenwert-Algorithmus war hingegen effizienter als der Clustering-Algorithmus und ermöglichte, das erwartete Streumuster zu beobachten, obwohl weniger gestreuten Photonen detektiert wurden.  

\noindent
Aus den Ergebnissen dieser Arbeit lassen sich mehrere Schlussfolgerungen hinsichtlich des gesamten Mess- und Auswertungsverfahrens ziehen. So wurde mit dem Experiment nachgewiesen, dass die \gls{xmcd}-basierte Streuung sich auch mit unpolarisierter Strahlung beobachten lässt. Durch die erfolgreichen Streuexperimente konnte außerdem gezeigt werden, dass mindestens über die Länge der Domänenperiode (\SI{600}{\nano\meter}) die Laser-getriebene Quelle genügend kohärent ist, um ein Interferenzmuster zu erzeugen. Natürlich setzt sich bei einer gesamten Beleuchtungsgröße von einigen hundert Mikrometern das Streubild inkohärent aus vielen kohärent beleuchteten lokalen Bereichen zusammen.

\noindent
PXS-Laborquellen für weiche Röntgenstrahlung wurden bisher hauptsächlich für Untersuchungen mit Röntgenabsorptionsspektroskopie eingesetzt \cite{witte_electronic_2018, jonas_transient_2020, stiel_towards_2021}. Kürzlich wurde gezeigt, dass sich auch Röntgendiffraktionsexperiemente an magnetischen Proben an diesen Quellen realisieren lassen \cite{schick_laser-driven_2021}. Dabei wurden auch magnetische Multilagenschichten verwendet und räumliche Information über die magnetische Ordnung dieser Schichten in die \emph{Probentiefe} gewonnen. Die Kleinwinkelstreuversuche, die in dieser Arbeit durchgeführt wurden, ermöglichen Zugang zur räumlichen Struktur von magnetischen Proben in der lateralen Richtung, also zu magnetischen Texturen in der Probenebene. Die hier gezeigte Messung ist nach meinem Wissen die erste Labormessung eines resonanten Kleinwinkelstreusignals mit weichen Röntgenstrahlen (also im Bereich von Photonenenergien \SIrange{300}{2000}{eV}). Dieser Energiebereich ist wissenschaftlich von hoher Bedeutung, da er viele spektroskopisch interessante kern-nahe resonante Übergänge fast aller chemischen Elemente enthält, darunter eben auch die Übergänge mit dem stärksten \gls{xmcd}-Kontrast. Bisher wurden diese Experimente ausschließlich an Synchrotronstrahlungsquellen und XFELs durchgeführt.

\noindent
Der verwendete Aufbau ermöglicht bereits zeitlich hoch-aufgelöste Messungen (zeitliche Auflösung ca.\ \SI{10}{ps} \cite{schick_laser-driven_2021}) nach Laseranregung, Messungen im angelegten Magnetfeld und Messungen bei verschiedenen Temperaturen (über einen Kryostaten) der Probe. Diese Möglichkeiten lassen sich jetzt direkt mit der räumlichen Auflösung im Nanometerbereich, die die Kleinwinkelstreuung bietet, kombinieren. Die Aufnahme des dargestellten Streumusters aus der Summe von \SI{50000}{\captures} dauerte ca.\ 10 Minuten in Echtzeit. Dadurch und durch die lange Betriebsdauer der Quelle von mehreren Stunden über Tage hinweg können somit zum ersten Mal ganze Messreihen zur resonanten Kleinwinkelstreuung im Labor aufgenommen werden. 

\noindent
Das Messverfahren bietet damit aussichtsreiche Möglichkeiten zum Beispiel für die Untersuchung der Ausbildung von magnetischen Texturen nach Laseranregung. Kürzlich wurde die Erzeugung einer Phase von magnetischen Skyrmionen in Co-basierten Mehrschichtsystemen beobachtet \cite{buttner_observation_2021}. Die Prozesse spielen sich im Zeitbereich von einigen 100 Pikosekunden ab und können derzeit nur am XFEL beobachtet werden. Wie in dieser Arbeit gezeigt, lassen sich die Streuinformationen sogar mit spektroskopischen Informationen kombinieren. Diese Informationen können z.B. hilfreich für die Normierung der Streuintensität sein.

\noindent
Es können einige Schritte unternommen werden, um den aktuellen experimentellen Aufbau sowie die Auswertungsverfahren zu verbessern. Die Probe kann kleiner gemacht werden, um die Überlagerung des Streurings mit dem Direktstrahl zu vermeiden. Die hohe Zahl der fehldetektierten Photonen beim Clustering-Algorithmus, die den Algorithmus nicht zur Anwendung kommen ließ, kann dadurch gesenkt werden, dass kleinere Clustering-Kerne benutzt werden. Die Dunkelbilder, die gemittelt werden und von jedem Streubild subtrahiert werden, können viel häufiger aufgenommen werden. Dafür kann z.B.\ die Leerlaufzeit des Detektors zwischen den Röntgenpulsen ausgenutzt werden. Mit der Auslesezeit von ca.\ \SI{250}{\micro\second} und der Belichtungszeit von \SI{1}{\micro\second} könnten bis zu 30 Dunkelbilder innerhalb der \SI{10}{\milli\second} Periode zwischen zwei Streubildern aufgenommen werden.

\noindent
Messungen an einer Fe L3 Resonanzenergie wären ebenfalls von großem Interesse. Der Photonenfluss der Quelle ist bei diesen niedrigeren Energien fast dreimal so hoch. Darüber hinaus ist die Reflexionszonenplatte für Fe ca.\ vier Mal effizienter als die Reflexionszonenplatte für Gd. Allerdings ist das Ein-Photon-Signal von Fe L3 im Vergleich mit Gd M5 um \SI{60}{\percent} niedriger, was es nummerisch komplizierter macht, die Photonen effizient zu erkennen.  

\noindent
Zusammengefasst wurde in dieser Arbeit das erste resonante Kleinwinkelstreusignal von einer magnetischen Probe mit einer Laborquelle im weichen Röntgenbereich aufgenommen. Als Probe dienten magnetische Domänen in einem Fe/Gd-Mehrschichtsystem. Technisch möglich gemacht wurde diese Messung durch die Implementierung eines schnellen Detektors zur Erfassung des Streusignals einzelner Röntgenpulse und die nachfolgende numerische Detektion einzelner Photonenereignisse in den Aufnahmen. Die Quelle ermöglicht zeitaufgelöste Messungen mit einer Auflösung von \SI{10}{ps}. Messungen von Magnetisierungsdynamik mit dieser Kombination aus räumlicher und zeitlicher Auflösung waren bisher nur am XFEL möglich und können jetzt in einer flexiblen Probenumgebung im Labor durchgeführt werden.