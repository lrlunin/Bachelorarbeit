\newacronym[user1=\emph{engl. „plasma X-ray source“}]{pxs}{PXS}{Laser-getriebene Plasma-Röntgenquelle}
\newacronym[user1=\emph{engl. „refection zone plate“}]{rzp}{RZP}{Reflexionszonenplatte}
\chapter{Laser-getriebenes Instrument für resonante Streuung mit weichen Röntgenstrahlen}
\label{text:quelle_roentgen}
Typischerweise werden resonante Streuexperimente an Synchrotronstrahlungsquellen oder \gls{fel}s durchgeführt. Wegen der kleinen magnetischen Streuquerschnitte wird ein hoher Photonenfluss benötigt. Die Photonenenergie muss frei durchstimmbar sein und eine Kontrolle über die Polarisation ist vorteilhaft. Die Strahllinie P04 an der Synchrotronstrahlungsquelle PETRA III soll hier als Beispiel solcher Röntgenquellen dienen \cite{viefhaus_variable_2013}.
% \noindent
% Der Strahl an der Strahlinie ist stark kollimiert. So ist dessen Fläche von 10 bis \SI{50}{\micro\meter\squared} groß, wobei der Photonenfluss mehr als $\SI{1e12}{\photons\per\second}$ beträgt. Die Strahlung wird in Form der zeitlich periodischen Pulsen emittiert. Die typische Pulsdauer $t_\text{PETRA}$ ist \SI{44}{\pico\second} und Pulsperiode $T_\text{PETRA}$ kann zwischen den Werten \qtylist{8;16;192}{\nano\second} variiert werden \cite{petra-values-website}.
\begin{figure}[H]
    \centering
    \input{aufbau_skizze.pdf_tex}
    \caption{Schematische Skizze zur Versuchsgeometrie. Röntgenstrahlung (blauer Strahl) wird erzeugt, indem ein scharf fokussierter Laserstrahl (roter Strahl) auf einen Wolframzylinder gerichtet wird. Die erzeugte Röntgenstrahlung wird mit einer Reflexionszonenplatte gebündelt, wobei die Photonenenergien entlang der vertikalen Achse aufgelöst werden. So dient ein horizontaler Spalt zum Durchlass bestimmter Photonenenergien, die auf die Probe abgebildet werden. Der an der Probe gestreute Strahl wird mit einem Detektorsensor aufgenommen. Eine zu klein eingestellte Spalthöhe kann jedoch unerwünschte Beugungseffekte an den Spaltkanten erzeugen, welche sich am Detektorsensor beobachten lassen.}
    \label{fig:pxs_aufbau}
\end{figure}
%textwidht = 6.49733inch
\noindent
Als Quelle der weichen Röntgenstrahlung dient hingegen eine \gls{pxs}, deren Aufbau und Eigenschaften in \cite{schick_laser-driven_2021} detailliert beschrieben sind. Ihrem Funktionsprinzip liegt die Emission von Wolfram im weichen Röntgenbereich zugrunde, die durch die hochenergetischen Laserpulse getrieben wird \cite{mantouvalou_high_2015}. So wird die Röntgenstrahlung an der \gls{pxs} ebenso in Form von Pulsen emittiert. Im Vergleich zu PETRA III, wo die typische Pulsdauer \SI{44}{\pico\second} (RMS), also ca. \SI{100}{\pico\second} (FWHM), und Pulsperiode \SI{192}{\nano\second}  betragen, ist die Pulsdauer der \gls{pxs} mit \SI{10}{\pico\second} (FWHM) wesentlich kürzer, wobei ihre Pulsperiode viel größer ist und im Bereich von \SI{10}{\milli\second} liegt.
\begin{figure}[H]
    \centering
    \input{images/xps/xps_spectrum_700_1400_mrad.pgf}
    \caption{Das Spektrum der benutzten \gls{pxs}. Unter der Abkürzung „BB“ ist die Bandbreite zu verstehen. Die schwarzen Hilfslinien mit den Titeln entsprechen den Photonenenergien $h\nu_{\text{Fe, L3}} = \SI{706.97}{\eV}$ bzw. $h\nu_{\text{Gd, M5}} = \SI{1184,79}{\eV}$. Adaptiert von \cite{schick_laser-driven_2021}, mit Genehmigung von \href{https://orcid.org/0000-0001-7988-6489}{D. Schick}.}
    \label{fig:pxs_spectrum}
\end{figure}
\noindent
Im Gegensatz zur hoch kollimierten Strahlung an der Strahllinie P04 strahlt die \gls{pxs} die Photonen des ganzen Spektrums (s. Abb. \ref{fig:pxs_spectrum}) in alle Richtungen aus.

\noindent
Nichtsdestotrotz ist es möglich, die Strahlung zu bündeln und einen bestimmten Photonenenergiebereich zu selektieren. Dafür wird die emittierte Strahlung mithilfe einer \gls{rzp} fokussiert. Zur Verfügung stehen zwei \gls{rzp}, die für die Zielphotonenenergien von Fe (\SI{705}{\eV}) und Gd (\SI{1189}{\eV}) konstruiert wurden. Der mit der \gls{rzp} fokussierte Strahl wird in Form einer Sanduhr auf der Detektorfläche (s. Abb. \ref{fig:butterfly_moench}) abgebildet, wobei die Zielphotonenenergie in der Sanduhrtaille liegt und die benachbarten Photonenenergien entlang der horizontalen Symmetrieachse des Strahlprofils um den Fokuspunkt energetisch aufgelöst werden.
\begin{figure}[H]
    \centering
    \input{images/xps/butterfly_two_shots.pgf}
    \label{fig:butterfly_moench_sum}
    \caption{Das direkte Strahprofil auf dem Detektor, das mit der \gls{rzp} für Gd fokussierst wurde. Abgebildet sind (a) die Summe von 26026 Pulsen und (b) ein einzelner Puls. Der hellste Bereich liegt um \SI{1189}{\eV}; die Photonenenergie nimmt von unten nach oben ab. Der Bereich, in dem die Photonenenergie $h\nu_{\text{Gd, M5}} = \SI{1184,79}{\eV}$ abgebildet wurde, ist mit dem rotem Viereck markiert.}
    \label{fig:butterfly_moench}
\end{figure}
\noindent
Es ist schwer, den Photonenfluss der \gls{pxs} genau anzugeben, weil er nicht nur von der Wahl der \gls{rzp} abhängt, sondern auch von ihrer Position und dem  Photonenenergiebereich.

\noindent
Es kann beispielsweise der Photonenfluss der \gls{pxs} an der Photonenenergie $h\nu_\text{Gd, M5}$ abgeschätzt werden. So werden ca. \SI{617}{\photons} pro Puls innerhalb des roten Vierecks in Abb. \ref{fig:butterfly_moench} detektiert. In Hinblick auf die Pulsperiode \SI{10}{\milli\second} ergibt sich der Photonenfluss in Höhe von ca. \SI[per-mode = symbol]{6.2e4}{\photons\per\second}, wobei der Photonenfluss an der Strahllinie P04 höher als \SI[per-mode = symbol]{1e12}{\photons\per\second} ist.

\noindent
Für die Detektion des Streusignals an der \gls{pxs} werden also erheblich längere Belichtungszeiten des Detektors benötigt. In diesem Fall gibt es zwei Strategien für das Aufnahmeverfahren:

\noindent
Die Belichtungszeit wird hoch gesetzt, sodass eine Aufnahme Streusignal mehrerer Pulse der \gls{pxs} enthält. In diesem Fall müssen sowohl das Detektordunkelrauschen, das typischerweise mit der Zunahme der Belichtungszeit steigt, als auch das Ausleserauschen klein gegenüber dem Streusignal sein.

\noindent
Die andere Methode basiert auf dem Trennen des Hintergrundrauschens vom Streusignal. Die Belichtungszeit wird möglichst klein gesetzt, sodass jeder Puls der \gls{pxs} einzeln aufgenommen wird und das Detektordunkelrauschen gesenkt wird. In diesem Fall ist das Ausleserauschen nun dominierend und muss klein gegenüber dem Ein-Photon-Signal sein.

\noindent
Die beiden Methoden haben jeweils gewisse Vor- und Nachteile. Die erste Aufnahmemethode ist einfach zu realisieren. Der Dynamikumfang des Detektors muss jedoch idealerweise groß genug sein, damit das gemessene Signal innerhalb des gesamten Sensors linear abgebildet wird. In der Wirklichkeit ist dies kaum realisierbar. So werden typischerweise potenziell überbelichtete Sensorbereiche des Detektors vom Röntgenstrahl abgeschirmt, was weiter das Normierungsverfahren erschwert. Die zweite Aufnahmemethode lässt sich gut mit einem Anregungs-Abfrage-Experiment (\emph{engl. pump-probe experiment}) kombinieren und bietet besseren Dynamikumfang. Die Photonenerkennung kann jedoch Artefakte verursachen.

\noindent
Die Wahl der Lösung und die Bedingungen, die dem Detektor gegebenenfalls auferlegt werden, werden im nächsten Kapitel genauer diskutiert.


% 11 mittig: "So dient ein vertikaler Spalt..." ich glaube, der Spalt ist horizontal, oder?
% 16 \gls{pxs}: X-ray inkonsequent in der Groß-/Kleinschreibung: zuvor großes X, kleines -ray
% 27 Ich glaube, das Dysprosium interessiert an dieser Stelle nicht. Würde ich umformulieren, um das auszulassen.
% 28-31 (Abb. 8a) die Skalierung ist nicht besonders gut gelungen (Auswahl der Zwischenschritte); es ist nicht erkennbar, ob linear oder logarithmisch skaliert ist
% 39 Die Art, wie Photonen s⁻¹ in der PDF dargestellt wird, ist irgendwie unglücklich. Kann man da einen Multiplikations-Operator dazwischen setzen?
% 54 Vllt kann man auch deinen ursprünglichen Satz beibehalten und lediglich das "diskutiert" am Ende gegen "gehen" ersetzen
