\newacronym{moke}{MOKE}{magnetooptischen Kerr-Effekts}
\newacronym[user1=\emph{engl. magnetic force microscope}]{mfm}{MFM}{Magnetkraftmikroskop}
\newacronym[user1=\emph{engl. X-ray magnetic circular dichroism}]{xmcd}{XMCD}{Röntgenzirkulardichroismus}
\chapter{Prinzipien der resonanten Röntgenstrahlung}
\label{text:streuung}
Die in dieser Arbeit durchgeführten Experimente basieren auf den Effekten der resonanten Röntgentreuung. Die Darstellung der Zusammenhänge orientiert sich  hauptsächlich an \cite{kortright_resonant_2013}. 

\noindent
Die resonante Röntgenstreuung wird durch den elementspezifischen atomaren Streufaktor
\begin{align}
f(h\nu) = (\mathbf{\hat{e}_f^*} \cdot \mathbf{\hat{e}_0})f_c - i(\mathbf{\hat{e}_f^*} \times \mathbf{\hat{e}_0})\cdot\mathbf{\hat{m}}f_{m1} + \dots
\label{eq:atomfaktoren}
\end{align}
beschrieben, wobei $\mathbf{\hat{e}_f}$ und $\mathbf{\hat{e}_0}$ - die einfallenden und auslaufenden normierten Polarisationsvektoren, $\mathbf{\hat{m}}$ - der Einheitsvektor entlang der Magnetisierungsachse und $f_c(h\nu), f_{m1}(h\nu), f_{m2}(h\nu) \in \mathbb{C}$ - die energieabhängigen Streukoeffizienten des Atoms. Letztere werden auch häufig durch die Paare von reellen Koeffzienten
\begin{equation}
    \begin{split}
    f_c(h\nu) &= f_{c,1}(h\nu) + if_{c,2}(h\nu)\\
    f_{m1}(h\nu) &= f_{m1,1}(h\nu) + if_{m1,2}(h\nu)
\end{split}
\label{eq:komplex_paar}
\end{equation}
gegeben.

\noindent
Die normierten Polarisationsvektoren sind in der Basis von \emph{Jones-Vektoren} gegeben. D.h., dass der Wellenvektor der einlaufenden Welle $\mathbf{k_0} = \mathbf{e_{z}}$ ein $z$-Einheitsvektor ist und die lineare Polarisation in $x$- bzw. $y$-Richtung durch die Vektoren
\begin{equation}
    \mathbf{\hat{e}_{\text{lin $x$}}} = \begin{pmatrix}
    1\\
    0
    \end{pmatrix}
    \text{bzw. }
    \mathbf{\hat{e}_{\text{lin $y$}}} = \begin{pmatrix}
    0\\
    1
    \end{pmatrix}
    \label{eq:lin_jones}
\end{equation}
beschrieben wird, wobei die rechts bzw. links zirkulare Polarisation sich durch die Vektoren 
\begin{equation}
    \mathbf{\hat{e}_+} = \frac{1}{\sqrt{2}}\begin{pmatrix}
    1\\
    +i
    \end{pmatrix}
    \text{bzw. }
    \mathbf{\hat{e}_-} = \frac{1}{\sqrt{2}}\begin{pmatrix}
    1\\
    -i
    \end{pmatrix}
    \label{eq:circ_jones}
\end{equation}
darstellen lässt. Unter anderem gilt die folgende Beziehung zwischen den Vektorenpaaren:
\begin{equation}
    \mathbf{\hat{e}_{\text{lin $x$}}} = \frac{1}{\sqrt{2}}(\mathbf{\hat{e}_+} + \mathbf{\hat{e}_-}) \text{ bzw. } \mathbf{\hat{e}_{\text{lin $y$}}} = \frac{-i}{\sqrt{2}}(\mathbf{\hat{e}_+} - \mathbf{\hat{e}_-})
    \label{eq:lin_circ_jones}
\end{equation}

\noindent
Die Streukoeffizienten lassen sich nach der physikalischen Natur der Streuung voneinander unterscheiden. Der Streuung, die mit dem Koeffizienten $f_c$ beschrieben wird, liegt die Streuung an den gebundenen Elektronen des Atoms zugrunde. Für die Beschreibung der Streuung, die auf den magnetischen Effekten basiert, dienen die Koeffizienten $f_{m1}$ und $f_{m2}$.


\noindent
Für die Photonenfrequenz $\nu_{\text{nichtresonant}}$, die weit von der Resonanzabsorptionsfrequenz der Elektronen im Atom entfernt liegt, ist der magnetische Streukoeffizient $f_{m1}$ annähernd gleich Null. Aus diesem Grund hängt der Streufaktor $f(h\nu_{\text{nichtresonant}})$ lediglich vom ersten Summand in Gl. (\ref{eq:atomfaktoren}) ab.
% \begin{figure}[ht]
%     \centering
%     \input{wellenvektoren.pdf_tex}
%     \caption{Caption}
%     \label{fig:wellenvektoren}
% \end{figure}

\noindent
Im Photonenfrequenzbereich $\nu_\text{resonant}$, der nah an der Resonanzabsorptionsfrequenz der Elektronen im Atom liegt, nimmt der Streukoeffizient $f_{m1}$ zu und leisten einen wesentlichen Beitrag zum Gesamtwert des Streufaktors $f$. Es ist auch wichtig zu beachten, dass der Streufaktor $f(h\nu_\text{resonant})$ nun sowohl von den Polarisationsvektoren $\mathbf{\hat{e}_f}$ und $\mathbf{\hat{e}_0}$, als auch von der räumlichen Ausrichtung der Atommagnetisierungsachse $\mathbf{\hat{m}}$ in Bezug auf $\mathbf{\hat{e}_f}$ und $\mathbf{\hat{e}_0}$ abhängt. Dieser Effekt heißt \gls{xmcd} und wird grundsätzlich für zirkular polarisiertes Licht beobachtet.

\noindent
Für diese Arbeit wird insbesondere der Fall betrachtet, wenn die Magnetisierung in der Probe lateral periodisch ist, und zwar mikroskopische magnetische Domänen senkrecht zur Probenfläche bildet. % Weiter wird die Wechselwirkung mit einem magnetischkontrasten Material genauer betrachtet. Die theoretischen Grundlagen der physikalischen Prozessen, die zur Entstehung einer räumlich periodischen abwechselnden Magnetisierungsstruktur im Festkörper führen, gehören nicht zum Hauptthema dieser Arbeit und werden daher weggelassen.  

\noindent
Die Transmissionsgeometrie der Streuung an einer Probe mit der senkrechten Ausrichtung der magnetischen Domänen ist in Abb. \ref{fig:transmission_geometrie} dargestellt.
\begin{figure}[H]
    \centering
    \input{magnet_domain.pdf_tex}
    \caption{Der einlaufende Wellenvektor $\mathbf{k_0}$ mit dem Polarisationsvektor $\mathbf{\hat{e}_0}$ und der auslaufende Wellenvektor $\mathbf{k_f}$ mit dem Polarisationsvektor $\mathbf{\hat{e}_f}$, sowie der Streuvektor $\mathbf{q}$.}
    \label{fig:transmission_geometrie}
\end{figure}
\noindent
Die vektorielle Differenz zwischen den auslaufenden und einlaufenden Wellenvektoren $\mathbf{k_f}$ und $\mathbf{k_0}$ wird zunächst als Streuvektor $\mathbf{q}$ bezeichnet.

\noindent
Betrachtet man den Fall von zirkular polarisierter ein- und auslaufender Strahlung, ergibt sich aus den Gleichungen (\ref{eq:atomfaktoren}) und (\ref{eq:lin_jones}) die Beziehung
\begin{equation}
    f_{\pm}(h\nu)=f_{c}\mp \mathbf{k_0}\mathbf{\hat{m}} f_{m1} = f_{c}\mp \hat{m}_\perp f_{m1}
    \label{eq:faktor_zirc}
\end{equation}
%  Kurze Einleitung, wie man die Röntgenstrahlung durch Belichten eines wolframzylinders bekommt (verweis auf TU-Paper). DAS SPEKTRUM von Pape
% \section{Resonante Röntgenstreuung von mesoskopischen magnetischen Texturen}
Nach der \emph{Bornschen Näherung} lässt sich die Streuamplitude an den $N$-Atomen als die Summe phasen der Streuungen an jedem $n$-ten Atom darstellen. Die Streuung am $n$-ten Atom, das am Ort $\mathbf{r_n}$ liegt, wird als Produkt dessen Streufaktors $f_{\pm}^n$ mit dem Phasenfaktor $e^{i\mathbf{qr_n}}$ bezeichnet. Die Streuintensität berechnet sich dann aus dem Betragsquadrat der Amplitude:
\begin{equation}
    I_{\pm} \propto \left| \sum_{n}^N f_{\pm}^n e^{i\mathbf{qr_n}} \right|^2 \stackrel{{(\ref{eq:faktor_zirc})}}{\propto} \left| \sum_{n}^N (f_{c}^n\mp {\hat{m}_\perp}^n f_{m1}^n) e^{i\mathbf{qr_n}} \right|^2
    \label{eq:intencity_circ}
\end{equation}
Man führt die Strukturfaktoren $F_c$ und $F_m$ ein, die als
\begin{equation}
    \begin{split}
     F_c(\mathbf{q}) &= \sum^N_n f_c^n e^{i\mathbf{qr_n}}\\
     F_m(\mathbf{q}) &= \sum^N_n {\hat{m}_\perp}^n f_{m1}^n e^{i\mathbf{qr_n}}
    \end{split}
    \label{eq:fc_fm_substitution}
\end{equation}
definiert werden. Somit lässt sich Gl. (\ref{eq:intencity_circ}) wie folgt umschreiben:
\begin{equation}
     I_{\pm} \propto \left| F_c \mp F_m \right|^2 = \left| F_c \right|^2 \mp 2\text{Re}(F_c^*F_m) + \left| F_m \right|^2,
\end{equation}
wobei $\text{Re}(z)$ der reelle Teil der komplexen Größe ist.

\noindent
Nach Gl. (\ref{eq:lin_circ_jones}) kann die Intensität der Streuung des linear polarisierten Lichtes $I_{\text{lin}}$ als die Summe
\begin{equation}
    I_{\text{lin}} \propto I_+ + I_- \propto  \left| F_c \right|^2 + \left| F_m \right|^2
    % \left| F_c \right|^2 \underbrace{-2\text{Re}(F_c^*F_m) + 2\text{Re}(F_c^*F_m)}_{=0} + \left| F_m \right|^2 =
\end{equation}
bestimmt werden, wobei die Mischterme $\pm2\text{Re}(F_c^*F_m)$ einander ausgleichen. Das Ergebnis, dass der \gls{xmcd}-Effekt sowohl für zirkular als auch für linear polarisierte Strahlung zu beobachten ist, ist insbesondere im Kontext dieser Arbeit wichtig, weil die benutzte Röntgenquelle Strahlung ohne bevorzugte Polarisation emittiert.

\noindent
Das Betragsquadrat vom Strukturfaktor $\left|F_m\right|^2$ kann in Hinblick auf die Gleichungen (\ref{eq:komplex_paar}) und (\ref{eq:fc_fm_substitution}) umgeschrieben werden:
\begin{equation}
    \left|F_m(\mathbf{q})\right|^2 =  \left|\sum^N_n {\hat{m}_\perp}^n \left[f_{m1,1}^n + if_{m1,2}^n\right] e^{i\mathbf{qr_n}}\right|^2
    \label{eq:f_m}
\end{equation}
Wird die Streuung unter der Resonanzbedingung für ein chemisches Element betrachtet, können die Koeffizienten $f_{m1,1/2}$ anderer chemischer Elemente als vernachlässigbar klein weggelassen werden, wenn diese spektral weit genug voneinander entfernt liegen. In dieser Arbeit werden somit die resonanten Photonenenergien der Fe L3-Absorptionslinie $h\nu_{\text{Fe, L3}} = \SI{706.97}{\eV}$ und Gd M5-Absorptionslinie $h\nu_{\text{Gd, M5}} = \SI{1184.79}{\eV}$ unabhängig voneinander betrachtet. Diese Annahme
\begin{equation}
    f_{m1,1/2}^n = f_{m1,1/2} = \text{konst. } \forall n \in N
\end{equation}
lässt Gl. (\ref{eq:f_m}) weiter vereinfachen zu
\begin{equation}
   \left|F_m(\mathbf{q})\right|^2 = \left|f_{m1,1} + if_{m1,2}\right|^2 \cdot \left|\sum^N_n {\hat{m}_\perp}^n e^{i\mathbf{qr_n}}\right|^2 = \left(f_{m1,1}^2 + f_{m1,2}^2\right)S_m(\mathbf{q}),
\end{equation}
wobei $S_m(\mathbf{q})$ das Betragsquadrat der Fourier-Transformierten der magnetischen Struktur der Probe ist.

\noindent
Schließlich ergibt sich der folgende Zusammenhang für die Streuintensität:
\begin{equation}
    I_{\text{lin}} \propto  \left| F_c(\mathbf{q}) \right|^2 + \left(f_{m1,1}^2 + f_{m1,2}^2\right)S_m(\mathbf{q})
    \label{eq:intensitat_fourier}
\end{equation}
\noindent
Analog zur Proportionalität des Betragsquadrats des Strukturfaktors $\left|F_m(\mathbf{q})\right|^2$ der Fourier-Trans\-for\-mier\-ten der magnetischen Struktur $S_m(\mathbf{q})$ ist das Betragsquadrat des Strukturfaktors $\left|F_c(\mathbf{q})\right|^2$ proportional zur Fourier-Transformierten des Kristallgitters $S_c(\mathbf{q})$. Die magnetische Struktur verändert sich jedoch typischerweise auf einer Skala von \SI{10}{\nano\meter} Mikrometer, wobei typische Atomabstände innerhalb eines Kristallgitters einige Ångström betragen. So ist der Beitrag von $S_c(\mathbf{q})$ zur gesamten Streuintensität nur bei deutlich größeren Beträgen von Streuvektoren $q$ (größeren Streuwinkeln) zu beobachten.%liegen im Bereich von  sind viel größerordnung, d.h. dass unter der sehr kleinen winkeln (kleine streuvektoren) wird die magnetische streuung betrachet.

\noindent
So beschreibt der Faktor $S_m(\mathbf{q})$ die räumliche Streuintensitätsverteilung bei den kleinen Streuwinkeln. Im nächsten Abschnitt wird darüber diskutiert, wie sich der Faktor $S_m(\mathbf{q})$ in Bezug auf die Zusammensetzung der Probe verändert.

\section{Magnetische Dünnschichtsysteme}
In diesem Abschnitt werden physikalische Grundlagen betrachtet, die zur Bildung der magnetischen Domänen in Festkörpern, und zwar in Dünnschichtsystemen, führen. Solche Systeme bestehen häufig aus Dutzenden von alternierenden Schichten von 3d-Übergangsmetallen und schweren Metallen Seltenen Erden, die bis einige \text{Ångström} dick sind. In dieser Arbeit wird das Beispiel von einem Dünnschichtsystem aus Eisen und Gadolinium betrachtet.

\noindent
Die Domänenbildung lässt sich auf die Balance von drei dominierenden Wechselwirkungen zurückführen: Austauschwechselwirkung, uniaxiale magnetische Anisotropie und magnetische Streufelder \cite{hubert_magnetic_1998}, \cite{hellwig_domain_2007}. Ihre Beiträge zur Gesamtenergie werden zunächst einzeln erfasst.

\newacronym{fm}{FM}{ferromagnetisch}
\newacronym{am}{AFM}{antiferromagnetisch}
\noindent
Die Austauschwechselwirkung ist sehr kurzreichweitig und ist zur Ausrichtung der Elektornenspins (Kopplung) entscheidend. Sind die Elektronenspins zueinander antiparallel orientiert, nennt man diese Art der Kopplung \gls{am}. Sind sie parallel orientiert, heißt diese Kopplung \gls{fm}. Die Art der Kopplung wirkt sich unmittelbar auf die Gesamtmagnetisierung der Probe aus. Dies ist schematisch in Abb. \ref{fig:am_fm_kopplung} dargestellt. In einem Dünnschichtsystem aus Fe und Gd sind die magnetischen Momente innerhalb einer Schicht \gls{fm} gekoppelt, wobei die benachbarten Schichten \gls{am} gekoppelt sind. Die resultierende Magnetisierung des Dünnschichtsystems lässt sich als die Summe der Fe/Gd-Doppelschichtmagnetisierung darstellen, wobei die Doppelschichtmagnetisierung die Differenz der magnetischen Momente von Fe und Gd ist. Diese ist durch die \gls{am} Kopplung relativ klein, aber nicht null, da der absolute Betrag des magnetischen Momentes von Gd bei Raumtemperatur höher als der von Fe ist \cite{hintermayr_structure_2021_fixed}. Daher soll die Schichtenzahl sehr hoch sein, um eine höhere Gesamtmagnetisierung zu erhalten, welche durch die höheren Streufelder die Ausbildung von magnetischen Domänen begünstigt.
\begin{figure}[H]
    \centering
    \input{ferromagnet_antiferromagnet_wall.pdf_tex}
    \caption{schematisch dargestellte \gls{am}-gekoppelte Domänen zwischen den benachbarten Schichten am Beispiel von Gd/Fe. Die charakteristische Domänenbreite ist mit $\lambda_{\text{AF}}$ bezeichnet. Im vergrößerten Bereich (rechts) ist die räumliche Drehung der Domänen dargestellt. Eine solche Anordnung heißt Bloch-Wand.}
    \label{fig:am_fm_kopplung}
\end{figure}
\noindent
Die Anisotropie führt zum Entstehen der Vorzugsachsen der Magnetisierung innerhalb eines Mediums. Typischerweise liegen die Vorzugsachsen in der Materialebene. Im Falle des Mehrschichtsystems aus Schwermetallen, wo die Schichten nur wenige \text{Ånsgtröm} dick sind, überwiegen die Grenzschichteffekte, wobei die Bahnmomenten, als auch die Spinmomenten aufgrund der Spin-Bahn-Kopplung, an der Grenzschicht senkrecht zur Ebene ausrichten. Dies führt zur Ausbildung einer senkrechten Anisotropie.

\noindent
Die Magnetisierung senkrecht zur Ebene erhöht jedoch die Streufeldenergie. Diese kann minimiert werden, indem sich Domänen mit entgegengesetzter magnetischer Polarisation innerhalb des Systems bilden. Die Periodizität der Domänen und ihre Breite hängen von den Schichtdicken und der Schichtenzahl ab. 

\noindent
Die Bildung eines bestimmten Systemzustandes hängt von mehreren inneren und äußeren Faktoren ab. Zu denen gehören die bereits oben erwähnten Schichtdicken und Schichtenzahl, sowie die Temperatur, die Stärke des externen Magnetfeldes und dessen Historie. Auch Anregungen mit ultrakurzen Laserpulsen können zur Ausbildung von charakteristischen Domänenstrukturen führen \cite{buttner_observation_2021}.
%sind die \gls{am}- und \gls{fm}-Kopplung und die resultierende Domänengrößen $\lambda_{\text{AF}}$ und $\lambda_{\text{FM}}$ schematisch dargestellt.

% * Heterostrukturen aus einem magnetischen Übergangsmetall (Fe, Co, Ni) und einem anderen schweren Metall (Pt, Pd, Au, Gd) können eine senkrechte magnetische Anisotropie ausbilden. Diese Anisotropie entsteht durch Spin-Bahn-Ww an der Grenzschicht. Das funktioniert aber nur, wenn diese Grenzschichteffekte stark gegenüber der Formanisotropie sind, die eigentlich eine in-plane Magnetisierung bevorzugt. Deshalb müssen die Schichten sehr dünn sein.
% * Um das magnetische Moment zu erhöhen, macht man deshalb ML mit vielen dieser Grenzschichten.
% * In Fe/Gd besteht noch der Spezialfall, dass Gd auch eine Magnetisierung hat und die Schichten ferrimagnetisch koppeln bzw. noch komplexere Spinstruktuen ausbilden
% * ML mit einem hohen magnetischen Moment und senkrechter Anisotropie haben im gesättigten Zustand ein hohes Streufeld, das sehr viel Energie kostet. Um diese zu minimieren, bilden die ML Domänen mit entgegengesetzter Magnetisierung aus. Die Bildung von Domänenwänden kostet aber auch wieder Energie (weil die Drehung der Spin in der Domänenwand gegen die AustauschWw geht), was zu einer gewissen Gleichgewichtsbreite der Domänen führt und damit zu sehr regelmäßigen Mustern.


\section{Probenherstellung}
Zur Optimierung des zu erwartende Streusignals wurden zwei Multilagenproben hergestellt. Mithilfe des Programmpakets „\emph{udkm1Dsim}“ \cite{schick_udkm1dsim_2021} wird die Transmission und der magnetische Kontrast in Bezug auf die Wiederholungszahl $N$ der [Fe(\SI{0.41}{\nano\meter})/Gd(\SI{0.45}{\nano\meter})] Doppelschicht in einer Probe bei den Photonenenergien $h\nu_{\text{Fe, L3}} = \SI{706,97}{\eV}$ und $h\nu_{\text{Gd, M5}} = \SI{1184,79}{\eV}$ untersucht. Die Wahl dieser zwei Elemente wird später in Kapitel \ref{text:quelle_roentgen} thematisiert. Die Werte der Koeffizienten $f_{m1,1}$ und $f_{m1,2}$ von Fe und Gd wurden jeweils aus \cite[Abb. 4]{kortright_resonant_2000} und \cite[Abb. 2]{prieto-x-ray-2005} abgeleitet. In der Zusammensetzung der Probe müssen auch ein \SI{200}{\nano\meter} SiN-Substrat und zwei \SI{3}{\micro\meter} bzw. \SI{2}{\nano\meter} Schichten aus Tantal, die als Haft- bzw. Deckschicht dienen, mitberechnet.

\noindent
Mit dem Programmpaket werden die Transmissionen T$_+$ und T$_-$ für das rechts und links polarisierte Licht berechnet. Die Transmissionsrate T und der magnetische Kontrast P werden ermittelt über
\begin{equation}
    \text{T} = \frac{\text{T}_+ + \text{T}_-}{2}; \text{P} = \frac{\text{T}_+ - \text{T}_-}{\text{T}}
\end{equation}
und zur Bestimmung der Gütezahl der Streuungsintensität TP$^2$ verwendet. Die Gütezahl $\text{TP}^2$ ist ein Maß für die Streuungsintensität und muss maximiert werden. Die verfügbare Herstellungstechnologie von mehrschichtigen Systemen limitiert die maximale Schichtenzahl einer Probe. Ohne allzu große technische Herausforderungen könnte eine Probe mit 200 Wiederholungen erstellt werden. Aus diesem Grund wird das Maximum von TP$^2$ im Bereich der Wiederholungszahl $N$ von 1 bis 200 gesucht.
\begin{figure}[H]
    \centering
    %% Creator: Matplotlib, PGF backend
%%
%% To include the figure in your LaTeX document, write
%%   \input{<filename>.pgf}
%%
%% Make sure the required packages are loaded in your preamble
%%   \usepackage{pgf}
%%
%% Also ensure that all the required font packages are loaded; for instance,
%% the lmodern package is sometimes necessary when using math font.
%%   \usepackage{lmodern}
%%
%% Figures using additional raster images can only be included by \input if
%% they are in the same directory as the main LaTeX file. For loading figures
%% from other directories you can use the `import` package
%%   \usepackage{import}
%%
%% and then include the figures with
%%   \import{<path to file>}{<filename>.pgf}
%%
%% Matplotlib used the following preamble
%%   \usepackage{amsmath} \usepackage[utf8]{inputenc} \usepackage[T1]{fontenc} \usepackage[output-decimal-marker={,},print-unity-mantissa=false]{siunitx} \sisetup{per-mode=fraction, separate-uncertainty = true, locale = DE} \usepackage[acronym, toc, section=section, nonumberlist, nopostdot]{glossaries-extra} \DeclareSIUnit\adu{\text{ADU}} \DeclareSIUnit\px{\text{px}} \DeclareSIUnit\photons{\text{Pho\-to\-nen}} \DeclareSIUnit\photon{\text{Pho\-ton}}
%%
\begingroup%
\makeatletter%
\begin{pgfpicture}%
\pgfpathrectangle{\pgfpointorigin}{\pgfqpoint{6.185863in}{3.682112in}}%
\pgfusepath{use as bounding box, clip}%
\begin{pgfscope}%
\pgfsetbuttcap%
\pgfsetmiterjoin%
\pgfsetlinewidth{0.000000pt}%
\definecolor{currentstroke}{rgb}{1.000000,1.000000,1.000000}%
\pgfsetstrokecolor{currentstroke}%
\pgfsetstrokeopacity{0.000000}%
\pgfsetdash{}{0pt}%
\pgfpathmoveto{\pgfqpoint{0.000000in}{0.000000in}}%
\pgfpathlineto{\pgfqpoint{6.185863in}{0.000000in}}%
\pgfpathlineto{\pgfqpoint{6.185863in}{3.682112in}}%
\pgfpathlineto{\pgfqpoint{0.000000in}{3.682112in}}%
\pgfpathlineto{\pgfqpoint{0.000000in}{0.000000in}}%
\pgfpathclose%
\pgfusepath{}%
\end{pgfscope}%
\begin{pgfscope}%
\pgfsetbuttcap%
\pgfsetmiterjoin%
\definecolor{currentfill}{rgb}{1.000000,1.000000,1.000000}%
\pgfsetfillcolor{currentfill}%
\pgfsetlinewidth{0.000000pt}%
\definecolor{currentstroke}{rgb}{0.000000,0.000000,0.000000}%
\pgfsetstrokecolor{currentstroke}%
\pgfsetstrokeopacity{0.000000}%
\pgfsetdash{}{0pt}%
\pgfpathmoveto{\pgfqpoint{0.833025in}{2.646053in}}%
\pgfpathlineto{\pgfqpoint{5.483025in}{2.646053in}}%
\pgfpathlineto{\pgfqpoint{5.483025in}{3.534288in}}%
\pgfpathlineto{\pgfqpoint{0.833025in}{3.534288in}}%
\pgfpathlineto{\pgfqpoint{0.833025in}{2.646053in}}%
\pgfpathclose%
\pgfusepath{fill}%
\end{pgfscope}%
\begin{pgfscope}%
\pgfsetbuttcap%
\pgfsetroundjoin%
\definecolor{currentfill}{rgb}{0.000000,0.000000,0.000000}%
\pgfsetfillcolor{currentfill}%
\pgfsetlinewidth{0.803000pt}%
\definecolor{currentstroke}{rgb}{0.000000,0.000000,0.000000}%
\pgfsetstrokecolor{currentstroke}%
\pgfsetdash{}{0pt}%
\pgfsys@defobject{currentmarker}{\pgfqpoint{0.000000in}{-0.048611in}}{\pgfqpoint{0.000000in}{0.000000in}}{%
\pgfpathmoveto{\pgfqpoint{0.000000in}{0.000000in}}%
\pgfpathlineto{\pgfqpoint{0.000000in}{-0.048611in}}%
\pgfusepath{stroke,fill}%
}%
\begin{pgfscope}%
\pgfsys@transformshift{0.851570in}{2.646053in}%
\pgfsys@useobject{currentmarker}{}%
\end{pgfscope}%
\end{pgfscope}%
\begin{pgfscope}%
\pgfsetbuttcap%
\pgfsetroundjoin%
\definecolor{currentfill}{rgb}{0.000000,0.000000,0.000000}%
\pgfsetfillcolor{currentfill}%
\pgfsetlinewidth{0.803000pt}%
\definecolor{currentstroke}{rgb}{0.000000,0.000000,0.000000}%
\pgfsetstrokecolor{currentstroke}%
\pgfsetdash{}{0pt}%
\pgfsys@defobject{currentmarker}{\pgfqpoint{0.000000in}{-0.048611in}}{\pgfqpoint{0.000000in}{0.000000in}}{%
\pgfpathmoveto{\pgfqpoint{0.000000in}{0.000000in}}%
\pgfpathlineto{\pgfqpoint{0.000000in}{-0.048611in}}%
\pgfusepath{stroke,fill}%
}%
\begin{pgfscope}%
\pgfsys@transformshift{1.987412in}{2.646053in}%
\pgfsys@useobject{currentmarker}{}%
\end{pgfscope}%
\end{pgfscope}%
\begin{pgfscope}%
\pgfsetbuttcap%
\pgfsetroundjoin%
\definecolor{currentfill}{rgb}{0.000000,0.000000,0.000000}%
\pgfsetfillcolor{currentfill}%
\pgfsetlinewidth{0.803000pt}%
\definecolor{currentstroke}{rgb}{0.000000,0.000000,0.000000}%
\pgfsetstrokecolor{currentstroke}%
\pgfsetdash{}{0pt}%
\pgfsys@defobject{currentmarker}{\pgfqpoint{0.000000in}{-0.048611in}}{\pgfqpoint{0.000000in}{0.000000in}}{%
\pgfpathmoveto{\pgfqpoint{0.000000in}{0.000000in}}%
\pgfpathlineto{\pgfqpoint{0.000000in}{-0.048611in}}%
\pgfusepath{stroke,fill}%
}%
\begin{pgfscope}%
\pgfsys@transformshift{3.146435in}{2.646053in}%
\pgfsys@useobject{currentmarker}{}%
\end{pgfscope}%
\end{pgfscope}%
\begin{pgfscope}%
\pgfsetbuttcap%
\pgfsetroundjoin%
\definecolor{currentfill}{rgb}{0.000000,0.000000,0.000000}%
\pgfsetfillcolor{currentfill}%
\pgfsetlinewidth{0.803000pt}%
\definecolor{currentstroke}{rgb}{0.000000,0.000000,0.000000}%
\pgfsetstrokecolor{currentstroke}%
\pgfsetdash{}{0pt}%
\pgfsys@defobject{currentmarker}{\pgfqpoint{0.000000in}{-0.048611in}}{\pgfqpoint{0.000000in}{0.000000in}}{%
\pgfpathmoveto{\pgfqpoint{0.000000in}{0.000000in}}%
\pgfpathlineto{\pgfqpoint{0.000000in}{-0.048611in}}%
\pgfusepath{stroke,fill}%
}%
\begin{pgfscope}%
\pgfsys@transformshift{4.305458in}{2.646053in}%
\pgfsys@useobject{currentmarker}{}%
\end{pgfscope}%
\end{pgfscope}%
\begin{pgfscope}%
\pgfsetbuttcap%
\pgfsetroundjoin%
\definecolor{currentfill}{rgb}{0.000000,0.000000,0.000000}%
\pgfsetfillcolor{currentfill}%
\pgfsetlinewidth{0.803000pt}%
\definecolor{currentstroke}{rgb}{0.000000,0.000000,0.000000}%
\pgfsetstrokecolor{currentstroke}%
\pgfsetdash{}{0pt}%
\pgfsys@defobject{currentmarker}{\pgfqpoint{0.000000in}{-0.048611in}}{\pgfqpoint{0.000000in}{0.000000in}}{%
\pgfpathmoveto{\pgfqpoint{0.000000in}{0.000000in}}%
\pgfpathlineto{\pgfqpoint{0.000000in}{-0.048611in}}%
\pgfusepath{stroke,fill}%
}%
\begin{pgfscope}%
\pgfsys@transformshift{5.464481in}{2.646053in}%
\pgfsys@useobject{currentmarker}{}%
\end{pgfscope}%
\end{pgfscope}%
\begin{pgfscope}%
\pgfsetbuttcap%
\pgfsetroundjoin%
\definecolor{currentfill}{rgb}{0.000000,0.000000,0.000000}%
\pgfsetfillcolor{currentfill}%
\pgfsetlinewidth{0.602250pt}%
\definecolor{currentstroke}{rgb}{0.000000,0.000000,0.000000}%
\pgfsetstrokecolor{currentstroke}%
\pgfsetdash{}{0pt}%
\pgfsys@defobject{currentmarker}{\pgfqpoint{0.000000in}{-0.027778in}}{\pgfqpoint{0.000000in}{0.000000in}}{%
\pgfpathmoveto{\pgfqpoint{0.000000in}{0.000000in}}%
\pgfpathlineto{\pgfqpoint{0.000000in}{-0.027778in}}%
\pgfusepath{stroke,fill}%
}%
\begin{pgfscope}%
\pgfsys@transformshift{1.060194in}{2.646053in}%
\pgfsys@useobject{currentmarker}{}%
\end{pgfscope}%
\end{pgfscope}%
\begin{pgfscope}%
\pgfsetbuttcap%
\pgfsetroundjoin%
\definecolor{currentfill}{rgb}{0.000000,0.000000,0.000000}%
\pgfsetfillcolor{currentfill}%
\pgfsetlinewidth{0.602250pt}%
\definecolor{currentstroke}{rgb}{0.000000,0.000000,0.000000}%
\pgfsetstrokecolor{currentstroke}%
\pgfsetdash{}{0pt}%
\pgfsys@defobject{currentmarker}{\pgfqpoint{0.000000in}{-0.027778in}}{\pgfqpoint{0.000000in}{0.000000in}}{%
\pgfpathmoveto{\pgfqpoint{0.000000in}{0.000000in}}%
\pgfpathlineto{\pgfqpoint{0.000000in}{-0.027778in}}%
\pgfusepath{stroke,fill}%
}%
\begin{pgfscope}%
\pgfsys@transformshift{1.291998in}{2.646053in}%
\pgfsys@useobject{currentmarker}{}%
\end{pgfscope}%
\end{pgfscope}%
\begin{pgfscope}%
\pgfsetbuttcap%
\pgfsetroundjoin%
\definecolor{currentfill}{rgb}{0.000000,0.000000,0.000000}%
\pgfsetfillcolor{currentfill}%
\pgfsetlinewidth{0.602250pt}%
\definecolor{currentstroke}{rgb}{0.000000,0.000000,0.000000}%
\pgfsetstrokecolor{currentstroke}%
\pgfsetdash{}{0pt}%
\pgfsys@defobject{currentmarker}{\pgfqpoint{0.000000in}{-0.027778in}}{\pgfqpoint{0.000000in}{0.000000in}}{%
\pgfpathmoveto{\pgfqpoint{0.000000in}{0.000000in}}%
\pgfpathlineto{\pgfqpoint{0.000000in}{-0.027778in}}%
\pgfusepath{stroke,fill}%
}%
\begin{pgfscope}%
\pgfsys@transformshift{1.523803in}{2.646053in}%
\pgfsys@useobject{currentmarker}{}%
\end{pgfscope}%
\end{pgfscope}%
\begin{pgfscope}%
\pgfsetbuttcap%
\pgfsetroundjoin%
\definecolor{currentfill}{rgb}{0.000000,0.000000,0.000000}%
\pgfsetfillcolor{currentfill}%
\pgfsetlinewidth{0.602250pt}%
\definecolor{currentstroke}{rgb}{0.000000,0.000000,0.000000}%
\pgfsetstrokecolor{currentstroke}%
\pgfsetdash{}{0pt}%
\pgfsys@defobject{currentmarker}{\pgfqpoint{0.000000in}{-0.027778in}}{\pgfqpoint{0.000000in}{0.000000in}}{%
\pgfpathmoveto{\pgfqpoint{0.000000in}{0.000000in}}%
\pgfpathlineto{\pgfqpoint{0.000000in}{-0.027778in}}%
\pgfusepath{stroke,fill}%
}%
\begin{pgfscope}%
\pgfsys@transformshift{1.755608in}{2.646053in}%
\pgfsys@useobject{currentmarker}{}%
\end{pgfscope}%
\end{pgfscope}%
\begin{pgfscope}%
\pgfsetbuttcap%
\pgfsetroundjoin%
\definecolor{currentfill}{rgb}{0.000000,0.000000,0.000000}%
\pgfsetfillcolor{currentfill}%
\pgfsetlinewidth{0.602250pt}%
\definecolor{currentstroke}{rgb}{0.000000,0.000000,0.000000}%
\pgfsetstrokecolor{currentstroke}%
\pgfsetdash{}{0pt}%
\pgfsys@defobject{currentmarker}{\pgfqpoint{0.000000in}{-0.027778in}}{\pgfqpoint{0.000000in}{0.000000in}}{%
\pgfpathmoveto{\pgfqpoint{0.000000in}{0.000000in}}%
\pgfpathlineto{\pgfqpoint{0.000000in}{-0.027778in}}%
\pgfusepath{stroke,fill}%
}%
\begin{pgfscope}%
\pgfsys@transformshift{2.219217in}{2.646053in}%
\pgfsys@useobject{currentmarker}{}%
\end{pgfscope}%
\end{pgfscope}%
\begin{pgfscope}%
\pgfsetbuttcap%
\pgfsetroundjoin%
\definecolor{currentfill}{rgb}{0.000000,0.000000,0.000000}%
\pgfsetfillcolor{currentfill}%
\pgfsetlinewidth{0.602250pt}%
\definecolor{currentstroke}{rgb}{0.000000,0.000000,0.000000}%
\pgfsetstrokecolor{currentstroke}%
\pgfsetdash{}{0pt}%
\pgfsys@defobject{currentmarker}{\pgfqpoint{0.000000in}{-0.027778in}}{\pgfqpoint{0.000000in}{0.000000in}}{%
\pgfpathmoveto{\pgfqpoint{0.000000in}{0.000000in}}%
\pgfpathlineto{\pgfqpoint{0.000000in}{-0.027778in}}%
\pgfusepath{stroke,fill}%
}%
\begin{pgfscope}%
\pgfsys@transformshift{2.451021in}{2.646053in}%
\pgfsys@useobject{currentmarker}{}%
\end{pgfscope}%
\end{pgfscope}%
\begin{pgfscope}%
\pgfsetbuttcap%
\pgfsetroundjoin%
\definecolor{currentfill}{rgb}{0.000000,0.000000,0.000000}%
\pgfsetfillcolor{currentfill}%
\pgfsetlinewidth{0.602250pt}%
\definecolor{currentstroke}{rgb}{0.000000,0.000000,0.000000}%
\pgfsetstrokecolor{currentstroke}%
\pgfsetdash{}{0pt}%
\pgfsys@defobject{currentmarker}{\pgfqpoint{0.000000in}{-0.027778in}}{\pgfqpoint{0.000000in}{0.000000in}}{%
\pgfpathmoveto{\pgfqpoint{0.000000in}{0.000000in}}%
\pgfpathlineto{\pgfqpoint{0.000000in}{-0.027778in}}%
\pgfusepath{stroke,fill}%
}%
\begin{pgfscope}%
\pgfsys@transformshift{2.682826in}{2.646053in}%
\pgfsys@useobject{currentmarker}{}%
\end{pgfscope}%
\end{pgfscope}%
\begin{pgfscope}%
\pgfsetbuttcap%
\pgfsetroundjoin%
\definecolor{currentfill}{rgb}{0.000000,0.000000,0.000000}%
\pgfsetfillcolor{currentfill}%
\pgfsetlinewidth{0.602250pt}%
\definecolor{currentstroke}{rgb}{0.000000,0.000000,0.000000}%
\pgfsetstrokecolor{currentstroke}%
\pgfsetdash{}{0pt}%
\pgfsys@defobject{currentmarker}{\pgfqpoint{0.000000in}{-0.027778in}}{\pgfqpoint{0.000000in}{0.000000in}}{%
\pgfpathmoveto{\pgfqpoint{0.000000in}{0.000000in}}%
\pgfpathlineto{\pgfqpoint{0.000000in}{-0.027778in}}%
\pgfusepath{stroke,fill}%
}%
\begin{pgfscope}%
\pgfsys@transformshift{2.914631in}{2.646053in}%
\pgfsys@useobject{currentmarker}{}%
\end{pgfscope}%
\end{pgfscope}%
\begin{pgfscope}%
\pgfsetbuttcap%
\pgfsetroundjoin%
\definecolor{currentfill}{rgb}{0.000000,0.000000,0.000000}%
\pgfsetfillcolor{currentfill}%
\pgfsetlinewidth{0.602250pt}%
\definecolor{currentstroke}{rgb}{0.000000,0.000000,0.000000}%
\pgfsetstrokecolor{currentstroke}%
\pgfsetdash{}{0pt}%
\pgfsys@defobject{currentmarker}{\pgfqpoint{0.000000in}{-0.027778in}}{\pgfqpoint{0.000000in}{0.000000in}}{%
\pgfpathmoveto{\pgfqpoint{0.000000in}{0.000000in}}%
\pgfpathlineto{\pgfqpoint{0.000000in}{-0.027778in}}%
\pgfusepath{stroke,fill}%
}%
\begin{pgfscope}%
\pgfsys@transformshift{3.378240in}{2.646053in}%
\pgfsys@useobject{currentmarker}{}%
\end{pgfscope}%
\end{pgfscope}%
\begin{pgfscope}%
\pgfsetbuttcap%
\pgfsetroundjoin%
\definecolor{currentfill}{rgb}{0.000000,0.000000,0.000000}%
\pgfsetfillcolor{currentfill}%
\pgfsetlinewidth{0.602250pt}%
\definecolor{currentstroke}{rgb}{0.000000,0.000000,0.000000}%
\pgfsetstrokecolor{currentstroke}%
\pgfsetdash{}{0pt}%
\pgfsys@defobject{currentmarker}{\pgfqpoint{0.000000in}{-0.027778in}}{\pgfqpoint{0.000000in}{0.000000in}}{%
\pgfpathmoveto{\pgfqpoint{0.000000in}{0.000000in}}%
\pgfpathlineto{\pgfqpoint{0.000000in}{-0.027778in}}%
\pgfusepath{stroke,fill}%
}%
\begin{pgfscope}%
\pgfsys@transformshift{3.610044in}{2.646053in}%
\pgfsys@useobject{currentmarker}{}%
\end{pgfscope}%
\end{pgfscope}%
\begin{pgfscope}%
\pgfsetbuttcap%
\pgfsetroundjoin%
\definecolor{currentfill}{rgb}{0.000000,0.000000,0.000000}%
\pgfsetfillcolor{currentfill}%
\pgfsetlinewidth{0.602250pt}%
\definecolor{currentstroke}{rgb}{0.000000,0.000000,0.000000}%
\pgfsetstrokecolor{currentstroke}%
\pgfsetdash{}{0pt}%
\pgfsys@defobject{currentmarker}{\pgfqpoint{0.000000in}{-0.027778in}}{\pgfqpoint{0.000000in}{0.000000in}}{%
\pgfpathmoveto{\pgfqpoint{0.000000in}{0.000000in}}%
\pgfpathlineto{\pgfqpoint{0.000000in}{-0.027778in}}%
\pgfusepath{stroke,fill}%
}%
\begin{pgfscope}%
\pgfsys@transformshift{3.841849in}{2.646053in}%
\pgfsys@useobject{currentmarker}{}%
\end{pgfscope}%
\end{pgfscope}%
\begin{pgfscope}%
\pgfsetbuttcap%
\pgfsetroundjoin%
\definecolor{currentfill}{rgb}{0.000000,0.000000,0.000000}%
\pgfsetfillcolor{currentfill}%
\pgfsetlinewidth{0.602250pt}%
\definecolor{currentstroke}{rgb}{0.000000,0.000000,0.000000}%
\pgfsetstrokecolor{currentstroke}%
\pgfsetdash{}{0pt}%
\pgfsys@defobject{currentmarker}{\pgfqpoint{0.000000in}{-0.027778in}}{\pgfqpoint{0.000000in}{0.000000in}}{%
\pgfpathmoveto{\pgfqpoint{0.000000in}{0.000000in}}%
\pgfpathlineto{\pgfqpoint{0.000000in}{-0.027778in}}%
\pgfusepath{stroke,fill}%
}%
\begin{pgfscope}%
\pgfsys@transformshift{4.073654in}{2.646053in}%
\pgfsys@useobject{currentmarker}{}%
\end{pgfscope}%
\end{pgfscope}%
\begin{pgfscope}%
\pgfsetbuttcap%
\pgfsetroundjoin%
\definecolor{currentfill}{rgb}{0.000000,0.000000,0.000000}%
\pgfsetfillcolor{currentfill}%
\pgfsetlinewidth{0.602250pt}%
\definecolor{currentstroke}{rgb}{0.000000,0.000000,0.000000}%
\pgfsetstrokecolor{currentstroke}%
\pgfsetdash{}{0pt}%
\pgfsys@defobject{currentmarker}{\pgfqpoint{0.000000in}{-0.027778in}}{\pgfqpoint{0.000000in}{0.000000in}}{%
\pgfpathmoveto{\pgfqpoint{0.000000in}{0.000000in}}%
\pgfpathlineto{\pgfqpoint{0.000000in}{-0.027778in}}%
\pgfusepath{stroke,fill}%
}%
\begin{pgfscope}%
\pgfsys@transformshift{4.537263in}{2.646053in}%
\pgfsys@useobject{currentmarker}{}%
\end{pgfscope}%
\end{pgfscope}%
\begin{pgfscope}%
\pgfsetbuttcap%
\pgfsetroundjoin%
\definecolor{currentfill}{rgb}{0.000000,0.000000,0.000000}%
\pgfsetfillcolor{currentfill}%
\pgfsetlinewidth{0.602250pt}%
\definecolor{currentstroke}{rgb}{0.000000,0.000000,0.000000}%
\pgfsetstrokecolor{currentstroke}%
\pgfsetdash{}{0pt}%
\pgfsys@defobject{currentmarker}{\pgfqpoint{0.000000in}{-0.027778in}}{\pgfqpoint{0.000000in}{0.000000in}}{%
\pgfpathmoveto{\pgfqpoint{0.000000in}{0.000000in}}%
\pgfpathlineto{\pgfqpoint{0.000000in}{-0.027778in}}%
\pgfusepath{stroke,fill}%
}%
\begin{pgfscope}%
\pgfsys@transformshift{4.769067in}{2.646053in}%
\pgfsys@useobject{currentmarker}{}%
\end{pgfscope}%
\end{pgfscope}%
\begin{pgfscope}%
\pgfsetbuttcap%
\pgfsetroundjoin%
\definecolor{currentfill}{rgb}{0.000000,0.000000,0.000000}%
\pgfsetfillcolor{currentfill}%
\pgfsetlinewidth{0.602250pt}%
\definecolor{currentstroke}{rgb}{0.000000,0.000000,0.000000}%
\pgfsetstrokecolor{currentstroke}%
\pgfsetdash{}{0pt}%
\pgfsys@defobject{currentmarker}{\pgfqpoint{0.000000in}{-0.027778in}}{\pgfqpoint{0.000000in}{0.000000in}}{%
\pgfpathmoveto{\pgfqpoint{0.000000in}{0.000000in}}%
\pgfpathlineto{\pgfqpoint{0.000000in}{-0.027778in}}%
\pgfusepath{stroke,fill}%
}%
\begin{pgfscope}%
\pgfsys@transformshift{5.000872in}{2.646053in}%
\pgfsys@useobject{currentmarker}{}%
\end{pgfscope}%
\end{pgfscope}%
\begin{pgfscope}%
\pgfsetbuttcap%
\pgfsetroundjoin%
\definecolor{currentfill}{rgb}{0.000000,0.000000,0.000000}%
\pgfsetfillcolor{currentfill}%
\pgfsetlinewidth{0.602250pt}%
\definecolor{currentstroke}{rgb}{0.000000,0.000000,0.000000}%
\pgfsetstrokecolor{currentstroke}%
\pgfsetdash{}{0pt}%
\pgfsys@defobject{currentmarker}{\pgfqpoint{0.000000in}{-0.027778in}}{\pgfqpoint{0.000000in}{0.000000in}}{%
\pgfpathmoveto{\pgfqpoint{0.000000in}{0.000000in}}%
\pgfpathlineto{\pgfqpoint{0.000000in}{-0.027778in}}%
\pgfusepath{stroke,fill}%
}%
\begin{pgfscope}%
\pgfsys@transformshift{5.232676in}{2.646053in}%
\pgfsys@useobject{currentmarker}{}%
\end{pgfscope}%
\end{pgfscope}%
\begin{pgfscope}%
\pgfsetbuttcap%
\pgfsetroundjoin%
\definecolor{currentfill}{rgb}{0.000000,0.000000,0.000000}%
\pgfsetfillcolor{currentfill}%
\pgfsetlinewidth{0.803000pt}%
\definecolor{currentstroke}{rgb}{0.000000,0.000000,0.000000}%
\pgfsetstrokecolor{currentstroke}%
\pgfsetdash{}{0pt}%
\pgfsys@defobject{currentmarker}{\pgfqpoint{-0.048611in}{0.000000in}}{\pgfqpoint{-0.000000in}{0.000000in}}{%
\pgfpathmoveto{\pgfqpoint{-0.000000in}{0.000000in}}%
\pgfpathlineto{\pgfqpoint{-0.048611in}{0.000000in}}%
\pgfusepath{stroke,fill}%
}%
\begin{pgfscope}%
\pgfsys@transformshift{0.833025in}{2.646053in}%
\pgfsys@useobject{currentmarker}{}%
\end{pgfscope}%
\end{pgfscope}%
\begin{pgfscope}%
\definecolor{textcolor}{rgb}{0.000000,0.000000,0.000000}%
\pgfsetstrokecolor{textcolor}%
\pgfsetfillcolor{textcolor}%
\pgftext[x=0.488889in, y=2.598228in, left, base]{\color{textcolor}\rmfamily\fontsize{10.000000}{12.000000}\selectfont \num{0.00}}%
\end{pgfscope}%
\begin{pgfscope}%
\pgfsetbuttcap%
\pgfsetroundjoin%
\definecolor{currentfill}{rgb}{0.000000,0.000000,0.000000}%
\pgfsetfillcolor{currentfill}%
\pgfsetlinewidth{0.803000pt}%
\definecolor{currentstroke}{rgb}{0.000000,0.000000,0.000000}%
\pgfsetstrokecolor{currentstroke}%
\pgfsetdash{}{0pt}%
\pgfsys@defobject{currentmarker}{\pgfqpoint{-0.048611in}{0.000000in}}{\pgfqpoint{-0.000000in}{0.000000in}}{%
\pgfpathmoveto{\pgfqpoint{-0.000000in}{0.000000in}}%
\pgfpathlineto{\pgfqpoint{-0.048611in}{0.000000in}}%
\pgfusepath{stroke,fill}%
}%
\begin{pgfscope}%
\pgfsys@transformshift{0.833025in}{2.868111in}%
\pgfsys@useobject{currentmarker}{}%
\end{pgfscope}%
\end{pgfscope}%
\begin{pgfscope}%
\definecolor{textcolor}{rgb}{0.000000,0.000000,0.000000}%
\pgfsetstrokecolor{textcolor}%
\pgfsetfillcolor{textcolor}%
\pgftext[x=0.488889in, y=2.820287in, left, base]{\color{textcolor}\rmfamily\fontsize{10.000000}{12.000000}\selectfont \num{0.25}}%
\end{pgfscope}%
\begin{pgfscope}%
\pgfsetbuttcap%
\pgfsetroundjoin%
\definecolor{currentfill}{rgb}{0.000000,0.000000,0.000000}%
\pgfsetfillcolor{currentfill}%
\pgfsetlinewidth{0.803000pt}%
\definecolor{currentstroke}{rgb}{0.000000,0.000000,0.000000}%
\pgfsetstrokecolor{currentstroke}%
\pgfsetdash{}{0pt}%
\pgfsys@defobject{currentmarker}{\pgfqpoint{-0.048611in}{0.000000in}}{\pgfqpoint{-0.000000in}{0.000000in}}{%
\pgfpathmoveto{\pgfqpoint{-0.000000in}{0.000000in}}%
\pgfpathlineto{\pgfqpoint{-0.048611in}{0.000000in}}%
\pgfusepath{stroke,fill}%
}%
\begin{pgfscope}%
\pgfsys@transformshift{0.833025in}{3.090170in}%
\pgfsys@useobject{currentmarker}{}%
\end{pgfscope}%
\end{pgfscope}%
\begin{pgfscope}%
\definecolor{textcolor}{rgb}{0.000000,0.000000,0.000000}%
\pgfsetstrokecolor{textcolor}%
\pgfsetfillcolor{textcolor}%
\pgftext[x=0.488889in, y=3.042346in, left, base]{\color{textcolor}\rmfamily\fontsize{10.000000}{12.000000}\selectfont \num{0.50}}%
\end{pgfscope}%
\begin{pgfscope}%
\pgfsetbuttcap%
\pgfsetroundjoin%
\definecolor{currentfill}{rgb}{0.000000,0.000000,0.000000}%
\pgfsetfillcolor{currentfill}%
\pgfsetlinewidth{0.803000pt}%
\definecolor{currentstroke}{rgb}{0.000000,0.000000,0.000000}%
\pgfsetstrokecolor{currentstroke}%
\pgfsetdash{}{0pt}%
\pgfsys@defobject{currentmarker}{\pgfqpoint{-0.048611in}{0.000000in}}{\pgfqpoint{-0.000000in}{0.000000in}}{%
\pgfpathmoveto{\pgfqpoint{-0.000000in}{0.000000in}}%
\pgfpathlineto{\pgfqpoint{-0.048611in}{0.000000in}}%
\pgfusepath{stroke,fill}%
}%
\begin{pgfscope}%
\pgfsys@transformshift{0.833025in}{3.312229in}%
\pgfsys@useobject{currentmarker}{}%
\end{pgfscope}%
\end{pgfscope}%
\begin{pgfscope}%
\definecolor{textcolor}{rgb}{0.000000,0.000000,0.000000}%
\pgfsetstrokecolor{textcolor}%
\pgfsetfillcolor{textcolor}%
\pgftext[x=0.488889in, y=3.264405in, left, base]{\color{textcolor}\rmfamily\fontsize{10.000000}{12.000000}\selectfont \num{0.75}}%
\end{pgfscope}%
\begin{pgfscope}%
\pgfsetbuttcap%
\pgfsetroundjoin%
\definecolor{currentfill}{rgb}{0.000000,0.000000,0.000000}%
\pgfsetfillcolor{currentfill}%
\pgfsetlinewidth{0.803000pt}%
\definecolor{currentstroke}{rgb}{0.000000,0.000000,0.000000}%
\pgfsetstrokecolor{currentstroke}%
\pgfsetdash{}{0pt}%
\pgfsys@defobject{currentmarker}{\pgfqpoint{-0.048611in}{0.000000in}}{\pgfqpoint{-0.000000in}{0.000000in}}{%
\pgfpathmoveto{\pgfqpoint{-0.000000in}{0.000000in}}%
\pgfpathlineto{\pgfqpoint{-0.048611in}{0.000000in}}%
\pgfusepath{stroke,fill}%
}%
\begin{pgfscope}%
\pgfsys@transformshift{0.833025in}{3.534288in}%
\pgfsys@useobject{currentmarker}{}%
\end{pgfscope}%
\end{pgfscope}%
\begin{pgfscope}%
\definecolor{textcolor}{rgb}{0.000000,0.000000,0.000000}%
\pgfsetstrokecolor{textcolor}%
\pgfsetfillcolor{textcolor}%
\pgftext[x=0.488889in, y=3.486463in, left, base]{\color{textcolor}\rmfamily\fontsize{10.000000}{12.000000}\selectfont \num{1.00}}%
\end{pgfscope}%
\begin{pgfscope}%
\definecolor{textcolor}{rgb}{0.000000,0.000000,0.000000}%
\pgfsetstrokecolor{textcolor}%
\pgfsetfillcolor{textcolor}%
\pgftext[x=0.433333in,y=3.090170in,,bottom,rotate=90.000000]{\color{textcolor}\rmfamily\fontsize{10.000000}{12.000000}\selectfont \begin{tabular}{c}Transmis-\\sionsrate \(\displaystyle \text{T}\)\end{tabular}}%
\end{pgfscope}%
\begin{pgfscope}%
\pgfpathrectangle{\pgfqpoint{0.833025in}{2.646053in}}{\pgfqpoint{4.650000in}{0.888235in}}%
\pgfusepath{clip}%
\pgfsetrectcap%
\pgfsetroundjoin%
\pgfsetlinewidth{1.505625pt}%
\definecolor{currentstroke}{rgb}{0.121569,0.466667,0.705882}%
\pgfsetstrokecolor{currentstroke}%
\pgfsetdash{}{0pt}%
\pgfpathmoveto{\pgfqpoint{0.851570in}{3.333274in}}%
\pgfpathlineto{\pgfqpoint{0.921111in}{3.283070in}}%
\pgfpathlineto{\pgfqpoint{0.990653in}{3.236807in}}%
\pgfpathlineto{\pgfqpoint{1.060194in}{3.193906in}}%
\pgfpathlineto{\pgfqpoint{1.129735in}{3.153963in}}%
\pgfpathlineto{\pgfqpoint{1.222457in}{3.104862in}}%
\pgfpathlineto{\pgfqpoint{1.315179in}{3.060183in}}%
\pgfpathlineto{\pgfqpoint{1.407901in}{3.019710in}}%
\pgfpathlineto{\pgfqpoint{1.500623in}{2.983262in}}%
\pgfpathlineto{\pgfqpoint{1.593344in}{2.950621in}}%
\pgfpathlineto{\pgfqpoint{1.686066in}{2.921509in}}%
\pgfpathlineto{\pgfqpoint{1.778788in}{2.895594in}}%
\pgfpathlineto{\pgfqpoint{1.871510in}{2.872516in}}%
\pgfpathlineto{\pgfqpoint{1.987412in}{2.847112in}}%
\pgfpathlineto{\pgfqpoint{2.103315in}{2.824934in}}%
\pgfpathlineto{\pgfqpoint{2.242397in}{2.801821in}}%
\pgfpathlineto{\pgfqpoint{2.381480in}{2.781860in}}%
\pgfpathlineto{\pgfqpoint{2.543743in}{2.761888in}}%
\pgfpathlineto{\pgfqpoint{2.706006in}{2.744966in}}%
\pgfpathlineto{\pgfqpoint{2.891450in}{2.728815in}}%
\pgfpathlineto{\pgfqpoint{3.100074in}{2.714036in}}%
\pgfpathlineto{\pgfqpoint{3.331879in}{2.700953in}}%
\pgfpathlineto{\pgfqpoint{3.586864in}{2.689633in}}%
\pgfpathlineto{\pgfqpoint{3.888210in}{2.679336in}}%
\pgfpathlineto{\pgfqpoint{4.235917in}{2.670542in}}%
\pgfpathlineto{\pgfqpoint{4.653165in}{2.663122in}}%
\pgfpathlineto{\pgfqpoint{5.186316in}{2.656897in}}%
\pgfpathlineto{\pgfqpoint{5.464481in}{2.654631in}}%
\pgfpathlineto{\pgfqpoint{5.464481in}{2.654631in}}%
\pgfusepath{stroke}%
\end{pgfscope}%
\begin{pgfscope}%
\pgfpathrectangle{\pgfqpoint{0.833025in}{2.646053in}}{\pgfqpoint{4.650000in}{0.888235in}}%
\pgfusepath{clip}%
\pgfsetrectcap%
\pgfsetroundjoin%
\pgfsetlinewidth{1.505625pt}%
\definecolor{currentstroke}{rgb}{1.000000,0.498039,0.054902}%
\pgfsetstrokecolor{currentstroke}%
\pgfsetdash{}{0pt}%
\pgfpathmoveto{\pgfqpoint{0.851570in}{3.450909in}}%
\pgfpathlineto{\pgfqpoint{0.897931in}{3.394629in}}%
\pgfpathlineto{\pgfqpoint{0.944292in}{3.342915in}}%
\pgfpathlineto{\pgfqpoint{0.990653in}{3.295360in}}%
\pgfpathlineto{\pgfqpoint{1.037013in}{3.251583in}}%
\pgfpathlineto{\pgfqpoint{1.083374in}{3.211255in}}%
\pgfpathlineto{\pgfqpoint{1.129735in}{3.174079in}}%
\pgfpathlineto{\pgfqpoint{1.176096in}{3.139776in}}%
\pgfpathlineto{\pgfqpoint{1.245638in}{3.093166in}}%
\pgfpathlineto{\pgfqpoint{1.315179in}{3.051720in}}%
\pgfpathlineto{\pgfqpoint{1.384720in}{3.014792in}}%
\pgfpathlineto{\pgfqpoint{1.454262in}{2.981823in}}%
\pgfpathlineto{\pgfqpoint{1.523803in}{2.952330in}}%
\pgfpathlineto{\pgfqpoint{1.593344in}{2.925893in}}%
\pgfpathlineto{\pgfqpoint{1.662886in}{2.902148in}}%
\pgfpathlineto{\pgfqpoint{1.755608in}{2.874132in}}%
\pgfpathlineto{\pgfqpoint{1.848330in}{2.849701in}}%
\pgfpathlineto{\pgfqpoint{1.941051in}{2.828323in}}%
\pgfpathlineto{\pgfqpoint{2.056954in}{2.805224in}}%
\pgfpathlineto{\pgfqpoint{2.172856in}{2.785493in}}%
\pgfpathlineto{\pgfqpoint{2.311939in}{2.765463in}}%
\pgfpathlineto{\pgfqpoint{2.451021in}{2.748682in}}%
\pgfpathlineto{\pgfqpoint{2.613285in}{2.732400in}}%
\pgfpathlineto{\pgfqpoint{2.798728in}{2.717244in}}%
\pgfpathlineto{\pgfqpoint{3.007352in}{2.703614in}}%
\pgfpathlineto{\pgfqpoint{3.239157in}{2.691712in}}%
\pgfpathlineto{\pgfqpoint{3.517323in}{2.680797in}}%
\pgfpathlineto{\pgfqpoint{3.841849in}{2.671434in}}%
\pgfpathlineto{\pgfqpoint{4.235917in}{2.663466in}}%
\pgfpathlineto{\pgfqpoint{4.722706in}{2.657033in}}%
\pgfpathlineto{\pgfqpoint{5.348579in}{2.652143in}}%
\pgfpathlineto{\pgfqpoint{5.464481in}{2.651515in}}%
\pgfpathlineto{\pgfqpoint{5.464481in}{2.651515in}}%
\pgfusepath{stroke}%
\end{pgfscope}%
\begin{pgfscope}%
\pgfpathrectangle{\pgfqpoint{0.833025in}{2.646053in}}{\pgfqpoint{4.650000in}{0.888235in}}%
\pgfusepath{clip}%
\pgfsetrectcap%
\pgfsetroundjoin%
\pgfsetlinewidth{1.003750pt}%
\definecolor{currentstroke}{rgb}{0.000000,0.000000,0.000000}%
\pgfsetstrokecolor{currentstroke}%
\pgfsetdash{}{0pt}%
\pgfpathmoveto{\pgfqpoint{3.517323in}{2.646053in}}%
\pgfpathlineto{\pgfqpoint{3.517323in}{3.534288in}}%
\pgfusepath{stroke}%
\end{pgfscope}%
\begin{pgfscope}%
\pgfpathrectangle{\pgfqpoint{0.833025in}{2.646053in}}{\pgfqpoint{4.650000in}{0.888235in}}%
\pgfusepath{clip}%
\pgfsetrectcap%
\pgfsetroundjoin%
\pgfsetlinewidth{1.003750pt}%
\definecolor{currentstroke}{rgb}{0.000000,0.000000,0.000000}%
\pgfsetstrokecolor{currentstroke}%
\pgfsetdash{}{0pt}%
\pgfpathmoveto{\pgfqpoint{2.242397in}{2.646053in}}%
\pgfpathlineto{\pgfqpoint{2.242397in}{3.534288in}}%
\pgfusepath{stroke}%
\end{pgfscope}%
\begin{pgfscope}%
\pgfsetrectcap%
\pgfsetmiterjoin%
\pgfsetlinewidth{0.803000pt}%
\definecolor{currentstroke}{rgb}{0.000000,0.000000,0.000000}%
\pgfsetstrokecolor{currentstroke}%
\pgfsetdash{}{0pt}%
\pgfpathmoveto{\pgfqpoint{0.833025in}{2.646053in}}%
\pgfpathlineto{\pgfqpoint{0.833025in}{3.534288in}}%
\pgfusepath{stroke}%
\end{pgfscope}%
\begin{pgfscope}%
\pgfsetrectcap%
\pgfsetmiterjoin%
\pgfsetlinewidth{0.803000pt}%
\definecolor{currentstroke}{rgb}{0.000000,0.000000,0.000000}%
\pgfsetstrokecolor{currentstroke}%
\pgfsetdash{}{0pt}%
\pgfpathmoveto{\pgfqpoint{5.483025in}{2.646053in}}%
\pgfpathlineto{\pgfqpoint{5.483025in}{3.534288in}}%
\pgfusepath{stroke}%
\end{pgfscope}%
\begin{pgfscope}%
\pgfsetrectcap%
\pgfsetmiterjoin%
\pgfsetlinewidth{0.803000pt}%
\definecolor{currentstroke}{rgb}{0.000000,0.000000,0.000000}%
\pgfsetstrokecolor{currentstroke}%
\pgfsetdash{}{0pt}%
\pgfpathmoveto{\pgfqpoint{0.833025in}{2.646053in}}%
\pgfpathlineto{\pgfqpoint{5.483025in}{2.646053in}}%
\pgfusepath{stroke}%
\end{pgfscope}%
\begin{pgfscope}%
\pgfsetrectcap%
\pgfsetmiterjoin%
\pgfsetlinewidth{0.803000pt}%
\definecolor{currentstroke}{rgb}{0.000000,0.000000,0.000000}%
\pgfsetstrokecolor{currentstroke}%
\pgfsetdash{}{0pt}%
\pgfpathmoveto{\pgfqpoint{0.833025in}{3.534288in}}%
\pgfpathlineto{\pgfqpoint{5.483025in}{3.534288in}}%
\pgfusepath{stroke}%
\end{pgfscope}%
\begin{pgfscope}%
\definecolor{textcolor}{rgb}{0.000000,0.000000,0.000000}%
\pgfsetstrokecolor{textcolor}%
\pgfsetfillcolor{textcolor}%
\pgftext[x=2.196036in,y=3.338876in,right,base]{\color{textcolor}\rmfamily\fontsize{10.000000}{12.000000}\selectfont \(\displaystyle \bar{N} = 61\)}%
\end{pgfscope}%
\begin{pgfscope}%
\definecolor{textcolor}{rgb}{0.000000,0.000000,0.000000}%
\pgfsetstrokecolor{textcolor}%
\pgfsetfillcolor{textcolor}%
\pgftext[x=3.470962in,y=3.338876in,right,base]{\color{textcolor}\rmfamily\fontsize{10.000000}{12.000000}\selectfont \(\displaystyle N = 116\)}%
\end{pgfscope}%
\begin{pgfscope}%
\pgfsetbuttcap%
\pgfsetmiterjoin%
\definecolor{currentfill}{rgb}{1.000000,1.000000,1.000000}%
\pgfsetfillcolor{currentfill}%
\pgfsetfillopacity{0.800000}%
\pgfsetlinewidth{1.003750pt}%
\definecolor{currentstroke}{rgb}{0.800000,0.800000,0.800000}%
\pgfsetstrokecolor{currentstroke}%
\pgfsetstrokeopacity{0.800000}%
\pgfsetdash{}{0pt}%
\pgfpathmoveto{\pgfqpoint{3.583040in}{3.010378in}}%
\pgfpathlineto{\pgfqpoint{5.385803in}{3.010378in}}%
\pgfpathquadraticcurveto{\pgfqpoint{5.413581in}{3.010378in}}{\pgfqpoint{5.413581in}{3.038156in}}%
\pgfpathlineto{\pgfqpoint{5.413581in}{3.437066in}}%
\pgfpathquadraticcurveto{\pgfqpoint{5.413581in}{3.464843in}}{\pgfqpoint{5.385803in}{3.464843in}}%
\pgfpathlineto{\pgfqpoint{3.583040in}{3.464843in}}%
\pgfpathquadraticcurveto{\pgfqpoint{3.555262in}{3.464843in}}{\pgfqpoint{3.555262in}{3.437066in}}%
\pgfpathlineto{\pgfqpoint{3.555262in}{3.038156in}}%
\pgfpathquadraticcurveto{\pgfqpoint{3.555262in}{3.010378in}}{\pgfqpoint{3.583040in}{3.010378in}}%
\pgfpathlineto{\pgfqpoint{3.583040in}{3.010378in}}%
\pgfpathclose%
\pgfusepath{stroke,fill}%
\end{pgfscope}%
\begin{pgfscope}%
\pgfsetrectcap%
\pgfsetroundjoin%
\pgfsetlinewidth{1.505625pt}%
\definecolor{currentstroke}{rgb}{0.121569,0.466667,0.705882}%
\pgfsetstrokecolor{currentstroke}%
\pgfsetdash{}{0pt}%
\pgfpathmoveto{\pgfqpoint{3.610817in}{3.360677in}}%
\pgfpathlineto{\pgfqpoint{3.749706in}{3.360677in}}%
\pgfpathlineto{\pgfqpoint{3.888595in}{3.360677in}}%
\pgfusepath{stroke}%
\end{pgfscope}%
\begin{pgfscope}%
\definecolor{textcolor}{rgb}{0.000000,0.000000,0.000000}%
\pgfsetstrokecolor{textcolor}%
\pgfsetfillcolor{textcolor}%
\pgftext[x=3.999706in,y=3.312066in,left,base]{\color{textcolor}\rmfamily\fontsize{10.000000}{12.000000}\selectfont \(\displaystyle h\nu_{\text{Fe, L3}}=\SI{706,97}{\eV}\)}%
\end{pgfscope}%
\begin{pgfscope}%
\pgfsetrectcap%
\pgfsetroundjoin%
\pgfsetlinewidth{1.505625pt}%
\definecolor{currentstroke}{rgb}{1.000000,0.498039,0.054902}%
\pgfsetstrokecolor{currentstroke}%
\pgfsetdash{}{0pt}%
\pgfpathmoveto{\pgfqpoint{3.610817in}{3.154277in}}%
\pgfpathlineto{\pgfqpoint{3.749706in}{3.154277in}}%
\pgfpathlineto{\pgfqpoint{3.888595in}{3.154277in}}%
\pgfusepath{stroke}%
\end{pgfscope}%
\begin{pgfscope}%
\definecolor{textcolor}{rgb}{0.000000,0.000000,0.000000}%
\pgfsetstrokecolor{textcolor}%
\pgfsetfillcolor{textcolor}%
\pgftext[x=3.999706in,y=3.105666in,left,base]{\color{textcolor}\rmfamily\fontsize{10.000000}{12.000000}\selectfont \(\displaystyle h\nu_{\text{Gd, M5}}=\SI{1184,79}{\eV}\)}%
\end{pgfscope}%
\begin{pgfscope}%
\pgfsetbuttcap%
\pgfsetmiterjoin%
\definecolor{currentfill}{rgb}{1.000000,1.000000,1.000000}%
\pgfsetfillcolor{currentfill}%
\pgfsetlinewidth{0.000000pt}%
\definecolor{currentstroke}{rgb}{0.000000,0.000000,0.000000}%
\pgfsetstrokecolor{currentstroke}%
\pgfsetstrokeopacity{0.000000}%
\pgfsetdash{}{0pt}%
\pgfpathmoveto{\pgfqpoint{0.833025in}{1.580170in}}%
\pgfpathlineto{\pgfqpoint{5.483025in}{1.580170in}}%
\pgfpathlineto{\pgfqpoint{5.483025in}{2.468406in}}%
\pgfpathlineto{\pgfqpoint{0.833025in}{2.468406in}}%
\pgfpathlineto{\pgfqpoint{0.833025in}{1.580170in}}%
\pgfpathclose%
\pgfusepath{fill}%
\end{pgfscope}%
\begin{pgfscope}%
\pgfsetbuttcap%
\pgfsetroundjoin%
\definecolor{currentfill}{rgb}{0.000000,0.000000,0.000000}%
\pgfsetfillcolor{currentfill}%
\pgfsetlinewidth{0.803000pt}%
\definecolor{currentstroke}{rgb}{0.000000,0.000000,0.000000}%
\pgfsetstrokecolor{currentstroke}%
\pgfsetdash{}{0pt}%
\pgfsys@defobject{currentmarker}{\pgfqpoint{0.000000in}{-0.048611in}}{\pgfqpoint{0.000000in}{0.000000in}}{%
\pgfpathmoveto{\pgfqpoint{0.000000in}{0.000000in}}%
\pgfpathlineto{\pgfqpoint{0.000000in}{-0.048611in}}%
\pgfusepath{stroke,fill}%
}%
\begin{pgfscope}%
\pgfsys@transformshift{0.851570in}{1.580170in}%
\pgfsys@useobject{currentmarker}{}%
\end{pgfscope}%
\end{pgfscope}%
\begin{pgfscope}%
\pgfsetbuttcap%
\pgfsetroundjoin%
\definecolor{currentfill}{rgb}{0.000000,0.000000,0.000000}%
\pgfsetfillcolor{currentfill}%
\pgfsetlinewidth{0.803000pt}%
\definecolor{currentstroke}{rgb}{0.000000,0.000000,0.000000}%
\pgfsetstrokecolor{currentstroke}%
\pgfsetdash{}{0pt}%
\pgfsys@defobject{currentmarker}{\pgfqpoint{0.000000in}{-0.048611in}}{\pgfqpoint{0.000000in}{0.000000in}}{%
\pgfpathmoveto{\pgfqpoint{0.000000in}{0.000000in}}%
\pgfpathlineto{\pgfqpoint{0.000000in}{-0.048611in}}%
\pgfusepath{stroke,fill}%
}%
\begin{pgfscope}%
\pgfsys@transformshift{1.987412in}{1.580170in}%
\pgfsys@useobject{currentmarker}{}%
\end{pgfscope}%
\end{pgfscope}%
\begin{pgfscope}%
\pgfsetbuttcap%
\pgfsetroundjoin%
\definecolor{currentfill}{rgb}{0.000000,0.000000,0.000000}%
\pgfsetfillcolor{currentfill}%
\pgfsetlinewidth{0.803000pt}%
\definecolor{currentstroke}{rgb}{0.000000,0.000000,0.000000}%
\pgfsetstrokecolor{currentstroke}%
\pgfsetdash{}{0pt}%
\pgfsys@defobject{currentmarker}{\pgfqpoint{0.000000in}{-0.048611in}}{\pgfqpoint{0.000000in}{0.000000in}}{%
\pgfpathmoveto{\pgfqpoint{0.000000in}{0.000000in}}%
\pgfpathlineto{\pgfqpoint{0.000000in}{-0.048611in}}%
\pgfusepath{stroke,fill}%
}%
\begin{pgfscope}%
\pgfsys@transformshift{3.146435in}{1.580170in}%
\pgfsys@useobject{currentmarker}{}%
\end{pgfscope}%
\end{pgfscope}%
\begin{pgfscope}%
\pgfsetbuttcap%
\pgfsetroundjoin%
\definecolor{currentfill}{rgb}{0.000000,0.000000,0.000000}%
\pgfsetfillcolor{currentfill}%
\pgfsetlinewidth{0.803000pt}%
\definecolor{currentstroke}{rgb}{0.000000,0.000000,0.000000}%
\pgfsetstrokecolor{currentstroke}%
\pgfsetdash{}{0pt}%
\pgfsys@defobject{currentmarker}{\pgfqpoint{0.000000in}{-0.048611in}}{\pgfqpoint{0.000000in}{0.000000in}}{%
\pgfpathmoveto{\pgfqpoint{0.000000in}{0.000000in}}%
\pgfpathlineto{\pgfqpoint{0.000000in}{-0.048611in}}%
\pgfusepath{stroke,fill}%
}%
\begin{pgfscope}%
\pgfsys@transformshift{4.305458in}{1.580170in}%
\pgfsys@useobject{currentmarker}{}%
\end{pgfscope}%
\end{pgfscope}%
\begin{pgfscope}%
\pgfsetbuttcap%
\pgfsetroundjoin%
\definecolor{currentfill}{rgb}{0.000000,0.000000,0.000000}%
\pgfsetfillcolor{currentfill}%
\pgfsetlinewidth{0.803000pt}%
\definecolor{currentstroke}{rgb}{0.000000,0.000000,0.000000}%
\pgfsetstrokecolor{currentstroke}%
\pgfsetdash{}{0pt}%
\pgfsys@defobject{currentmarker}{\pgfqpoint{0.000000in}{-0.048611in}}{\pgfqpoint{0.000000in}{0.000000in}}{%
\pgfpathmoveto{\pgfqpoint{0.000000in}{0.000000in}}%
\pgfpathlineto{\pgfqpoint{0.000000in}{-0.048611in}}%
\pgfusepath{stroke,fill}%
}%
\begin{pgfscope}%
\pgfsys@transformshift{5.464481in}{1.580170in}%
\pgfsys@useobject{currentmarker}{}%
\end{pgfscope}%
\end{pgfscope}%
\begin{pgfscope}%
\pgfsetbuttcap%
\pgfsetroundjoin%
\definecolor{currentfill}{rgb}{0.000000,0.000000,0.000000}%
\pgfsetfillcolor{currentfill}%
\pgfsetlinewidth{0.602250pt}%
\definecolor{currentstroke}{rgb}{0.000000,0.000000,0.000000}%
\pgfsetstrokecolor{currentstroke}%
\pgfsetdash{}{0pt}%
\pgfsys@defobject{currentmarker}{\pgfqpoint{0.000000in}{-0.027778in}}{\pgfqpoint{0.000000in}{0.000000in}}{%
\pgfpathmoveto{\pgfqpoint{0.000000in}{0.000000in}}%
\pgfpathlineto{\pgfqpoint{0.000000in}{-0.027778in}}%
\pgfusepath{stroke,fill}%
}%
\begin{pgfscope}%
\pgfsys@transformshift{1.060194in}{1.580170in}%
\pgfsys@useobject{currentmarker}{}%
\end{pgfscope}%
\end{pgfscope}%
\begin{pgfscope}%
\pgfsetbuttcap%
\pgfsetroundjoin%
\definecolor{currentfill}{rgb}{0.000000,0.000000,0.000000}%
\pgfsetfillcolor{currentfill}%
\pgfsetlinewidth{0.602250pt}%
\definecolor{currentstroke}{rgb}{0.000000,0.000000,0.000000}%
\pgfsetstrokecolor{currentstroke}%
\pgfsetdash{}{0pt}%
\pgfsys@defobject{currentmarker}{\pgfqpoint{0.000000in}{-0.027778in}}{\pgfqpoint{0.000000in}{0.000000in}}{%
\pgfpathmoveto{\pgfqpoint{0.000000in}{0.000000in}}%
\pgfpathlineto{\pgfqpoint{0.000000in}{-0.027778in}}%
\pgfusepath{stroke,fill}%
}%
\begin{pgfscope}%
\pgfsys@transformshift{1.291998in}{1.580170in}%
\pgfsys@useobject{currentmarker}{}%
\end{pgfscope}%
\end{pgfscope}%
\begin{pgfscope}%
\pgfsetbuttcap%
\pgfsetroundjoin%
\definecolor{currentfill}{rgb}{0.000000,0.000000,0.000000}%
\pgfsetfillcolor{currentfill}%
\pgfsetlinewidth{0.602250pt}%
\definecolor{currentstroke}{rgb}{0.000000,0.000000,0.000000}%
\pgfsetstrokecolor{currentstroke}%
\pgfsetdash{}{0pt}%
\pgfsys@defobject{currentmarker}{\pgfqpoint{0.000000in}{-0.027778in}}{\pgfqpoint{0.000000in}{0.000000in}}{%
\pgfpathmoveto{\pgfqpoint{0.000000in}{0.000000in}}%
\pgfpathlineto{\pgfqpoint{0.000000in}{-0.027778in}}%
\pgfusepath{stroke,fill}%
}%
\begin{pgfscope}%
\pgfsys@transformshift{1.523803in}{1.580170in}%
\pgfsys@useobject{currentmarker}{}%
\end{pgfscope}%
\end{pgfscope}%
\begin{pgfscope}%
\pgfsetbuttcap%
\pgfsetroundjoin%
\definecolor{currentfill}{rgb}{0.000000,0.000000,0.000000}%
\pgfsetfillcolor{currentfill}%
\pgfsetlinewidth{0.602250pt}%
\definecolor{currentstroke}{rgb}{0.000000,0.000000,0.000000}%
\pgfsetstrokecolor{currentstroke}%
\pgfsetdash{}{0pt}%
\pgfsys@defobject{currentmarker}{\pgfqpoint{0.000000in}{-0.027778in}}{\pgfqpoint{0.000000in}{0.000000in}}{%
\pgfpathmoveto{\pgfqpoint{0.000000in}{0.000000in}}%
\pgfpathlineto{\pgfqpoint{0.000000in}{-0.027778in}}%
\pgfusepath{stroke,fill}%
}%
\begin{pgfscope}%
\pgfsys@transformshift{1.755608in}{1.580170in}%
\pgfsys@useobject{currentmarker}{}%
\end{pgfscope}%
\end{pgfscope}%
\begin{pgfscope}%
\pgfsetbuttcap%
\pgfsetroundjoin%
\definecolor{currentfill}{rgb}{0.000000,0.000000,0.000000}%
\pgfsetfillcolor{currentfill}%
\pgfsetlinewidth{0.602250pt}%
\definecolor{currentstroke}{rgb}{0.000000,0.000000,0.000000}%
\pgfsetstrokecolor{currentstroke}%
\pgfsetdash{}{0pt}%
\pgfsys@defobject{currentmarker}{\pgfqpoint{0.000000in}{-0.027778in}}{\pgfqpoint{0.000000in}{0.000000in}}{%
\pgfpathmoveto{\pgfqpoint{0.000000in}{0.000000in}}%
\pgfpathlineto{\pgfqpoint{0.000000in}{-0.027778in}}%
\pgfusepath{stroke,fill}%
}%
\begin{pgfscope}%
\pgfsys@transformshift{2.219217in}{1.580170in}%
\pgfsys@useobject{currentmarker}{}%
\end{pgfscope}%
\end{pgfscope}%
\begin{pgfscope}%
\pgfsetbuttcap%
\pgfsetroundjoin%
\definecolor{currentfill}{rgb}{0.000000,0.000000,0.000000}%
\pgfsetfillcolor{currentfill}%
\pgfsetlinewidth{0.602250pt}%
\definecolor{currentstroke}{rgb}{0.000000,0.000000,0.000000}%
\pgfsetstrokecolor{currentstroke}%
\pgfsetdash{}{0pt}%
\pgfsys@defobject{currentmarker}{\pgfqpoint{0.000000in}{-0.027778in}}{\pgfqpoint{0.000000in}{0.000000in}}{%
\pgfpathmoveto{\pgfqpoint{0.000000in}{0.000000in}}%
\pgfpathlineto{\pgfqpoint{0.000000in}{-0.027778in}}%
\pgfusepath{stroke,fill}%
}%
\begin{pgfscope}%
\pgfsys@transformshift{2.451021in}{1.580170in}%
\pgfsys@useobject{currentmarker}{}%
\end{pgfscope}%
\end{pgfscope}%
\begin{pgfscope}%
\pgfsetbuttcap%
\pgfsetroundjoin%
\definecolor{currentfill}{rgb}{0.000000,0.000000,0.000000}%
\pgfsetfillcolor{currentfill}%
\pgfsetlinewidth{0.602250pt}%
\definecolor{currentstroke}{rgb}{0.000000,0.000000,0.000000}%
\pgfsetstrokecolor{currentstroke}%
\pgfsetdash{}{0pt}%
\pgfsys@defobject{currentmarker}{\pgfqpoint{0.000000in}{-0.027778in}}{\pgfqpoint{0.000000in}{0.000000in}}{%
\pgfpathmoveto{\pgfqpoint{0.000000in}{0.000000in}}%
\pgfpathlineto{\pgfqpoint{0.000000in}{-0.027778in}}%
\pgfusepath{stroke,fill}%
}%
\begin{pgfscope}%
\pgfsys@transformshift{2.682826in}{1.580170in}%
\pgfsys@useobject{currentmarker}{}%
\end{pgfscope}%
\end{pgfscope}%
\begin{pgfscope}%
\pgfsetbuttcap%
\pgfsetroundjoin%
\definecolor{currentfill}{rgb}{0.000000,0.000000,0.000000}%
\pgfsetfillcolor{currentfill}%
\pgfsetlinewidth{0.602250pt}%
\definecolor{currentstroke}{rgb}{0.000000,0.000000,0.000000}%
\pgfsetstrokecolor{currentstroke}%
\pgfsetdash{}{0pt}%
\pgfsys@defobject{currentmarker}{\pgfqpoint{0.000000in}{-0.027778in}}{\pgfqpoint{0.000000in}{0.000000in}}{%
\pgfpathmoveto{\pgfqpoint{0.000000in}{0.000000in}}%
\pgfpathlineto{\pgfqpoint{0.000000in}{-0.027778in}}%
\pgfusepath{stroke,fill}%
}%
\begin{pgfscope}%
\pgfsys@transformshift{2.914631in}{1.580170in}%
\pgfsys@useobject{currentmarker}{}%
\end{pgfscope}%
\end{pgfscope}%
\begin{pgfscope}%
\pgfsetbuttcap%
\pgfsetroundjoin%
\definecolor{currentfill}{rgb}{0.000000,0.000000,0.000000}%
\pgfsetfillcolor{currentfill}%
\pgfsetlinewidth{0.602250pt}%
\definecolor{currentstroke}{rgb}{0.000000,0.000000,0.000000}%
\pgfsetstrokecolor{currentstroke}%
\pgfsetdash{}{0pt}%
\pgfsys@defobject{currentmarker}{\pgfqpoint{0.000000in}{-0.027778in}}{\pgfqpoint{0.000000in}{0.000000in}}{%
\pgfpathmoveto{\pgfqpoint{0.000000in}{0.000000in}}%
\pgfpathlineto{\pgfqpoint{0.000000in}{-0.027778in}}%
\pgfusepath{stroke,fill}%
}%
\begin{pgfscope}%
\pgfsys@transformshift{3.378240in}{1.580170in}%
\pgfsys@useobject{currentmarker}{}%
\end{pgfscope}%
\end{pgfscope}%
\begin{pgfscope}%
\pgfsetbuttcap%
\pgfsetroundjoin%
\definecolor{currentfill}{rgb}{0.000000,0.000000,0.000000}%
\pgfsetfillcolor{currentfill}%
\pgfsetlinewidth{0.602250pt}%
\definecolor{currentstroke}{rgb}{0.000000,0.000000,0.000000}%
\pgfsetstrokecolor{currentstroke}%
\pgfsetdash{}{0pt}%
\pgfsys@defobject{currentmarker}{\pgfqpoint{0.000000in}{-0.027778in}}{\pgfqpoint{0.000000in}{0.000000in}}{%
\pgfpathmoveto{\pgfqpoint{0.000000in}{0.000000in}}%
\pgfpathlineto{\pgfqpoint{0.000000in}{-0.027778in}}%
\pgfusepath{stroke,fill}%
}%
\begin{pgfscope}%
\pgfsys@transformshift{3.610044in}{1.580170in}%
\pgfsys@useobject{currentmarker}{}%
\end{pgfscope}%
\end{pgfscope}%
\begin{pgfscope}%
\pgfsetbuttcap%
\pgfsetroundjoin%
\definecolor{currentfill}{rgb}{0.000000,0.000000,0.000000}%
\pgfsetfillcolor{currentfill}%
\pgfsetlinewidth{0.602250pt}%
\definecolor{currentstroke}{rgb}{0.000000,0.000000,0.000000}%
\pgfsetstrokecolor{currentstroke}%
\pgfsetdash{}{0pt}%
\pgfsys@defobject{currentmarker}{\pgfqpoint{0.000000in}{-0.027778in}}{\pgfqpoint{0.000000in}{0.000000in}}{%
\pgfpathmoveto{\pgfqpoint{0.000000in}{0.000000in}}%
\pgfpathlineto{\pgfqpoint{0.000000in}{-0.027778in}}%
\pgfusepath{stroke,fill}%
}%
\begin{pgfscope}%
\pgfsys@transformshift{3.841849in}{1.580170in}%
\pgfsys@useobject{currentmarker}{}%
\end{pgfscope}%
\end{pgfscope}%
\begin{pgfscope}%
\pgfsetbuttcap%
\pgfsetroundjoin%
\definecolor{currentfill}{rgb}{0.000000,0.000000,0.000000}%
\pgfsetfillcolor{currentfill}%
\pgfsetlinewidth{0.602250pt}%
\definecolor{currentstroke}{rgb}{0.000000,0.000000,0.000000}%
\pgfsetstrokecolor{currentstroke}%
\pgfsetdash{}{0pt}%
\pgfsys@defobject{currentmarker}{\pgfqpoint{0.000000in}{-0.027778in}}{\pgfqpoint{0.000000in}{0.000000in}}{%
\pgfpathmoveto{\pgfqpoint{0.000000in}{0.000000in}}%
\pgfpathlineto{\pgfqpoint{0.000000in}{-0.027778in}}%
\pgfusepath{stroke,fill}%
}%
\begin{pgfscope}%
\pgfsys@transformshift{4.073654in}{1.580170in}%
\pgfsys@useobject{currentmarker}{}%
\end{pgfscope}%
\end{pgfscope}%
\begin{pgfscope}%
\pgfsetbuttcap%
\pgfsetroundjoin%
\definecolor{currentfill}{rgb}{0.000000,0.000000,0.000000}%
\pgfsetfillcolor{currentfill}%
\pgfsetlinewidth{0.602250pt}%
\definecolor{currentstroke}{rgb}{0.000000,0.000000,0.000000}%
\pgfsetstrokecolor{currentstroke}%
\pgfsetdash{}{0pt}%
\pgfsys@defobject{currentmarker}{\pgfqpoint{0.000000in}{-0.027778in}}{\pgfqpoint{0.000000in}{0.000000in}}{%
\pgfpathmoveto{\pgfqpoint{0.000000in}{0.000000in}}%
\pgfpathlineto{\pgfqpoint{0.000000in}{-0.027778in}}%
\pgfusepath{stroke,fill}%
}%
\begin{pgfscope}%
\pgfsys@transformshift{4.537263in}{1.580170in}%
\pgfsys@useobject{currentmarker}{}%
\end{pgfscope}%
\end{pgfscope}%
\begin{pgfscope}%
\pgfsetbuttcap%
\pgfsetroundjoin%
\definecolor{currentfill}{rgb}{0.000000,0.000000,0.000000}%
\pgfsetfillcolor{currentfill}%
\pgfsetlinewidth{0.602250pt}%
\definecolor{currentstroke}{rgb}{0.000000,0.000000,0.000000}%
\pgfsetstrokecolor{currentstroke}%
\pgfsetdash{}{0pt}%
\pgfsys@defobject{currentmarker}{\pgfqpoint{0.000000in}{-0.027778in}}{\pgfqpoint{0.000000in}{0.000000in}}{%
\pgfpathmoveto{\pgfqpoint{0.000000in}{0.000000in}}%
\pgfpathlineto{\pgfqpoint{0.000000in}{-0.027778in}}%
\pgfusepath{stroke,fill}%
}%
\begin{pgfscope}%
\pgfsys@transformshift{4.769067in}{1.580170in}%
\pgfsys@useobject{currentmarker}{}%
\end{pgfscope}%
\end{pgfscope}%
\begin{pgfscope}%
\pgfsetbuttcap%
\pgfsetroundjoin%
\definecolor{currentfill}{rgb}{0.000000,0.000000,0.000000}%
\pgfsetfillcolor{currentfill}%
\pgfsetlinewidth{0.602250pt}%
\definecolor{currentstroke}{rgb}{0.000000,0.000000,0.000000}%
\pgfsetstrokecolor{currentstroke}%
\pgfsetdash{}{0pt}%
\pgfsys@defobject{currentmarker}{\pgfqpoint{0.000000in}{-0.027778in}}{\pgfqpoint{0.000000in}{0.000000in}}{%
\pgfpathmoveto{\pgfqpoint{0.000000in}{0.000000in}}%
\pgfpathlineto{\pgfqpoint{0.000000in}{-0.027778in}}%
\pgfusepath{stroke,fill}%
}%
\begin{pgfscope}%
\pgfsys@transformshift{5.000872in}{1.580170in}%
\pgfsys@useobject{currentmarker}{}%
\end{pgfscope}%
\end{pgfscope}%
\begin{pgfscope}%
\pgfsetbuttcap%
\pgfsetroundjoin%
\definecolor{currentfill}{rgb}{0.000000,0.000000,0.000000}%
\pgfsetfillcolor{currentfill}%
\pgfsetlinewidth{0.602250pt}%
\definecolor{currentstroke}{rgb}{0.000000,0.000000,0.000000}%
\pgfsetstrokecolor{currentstroke}%
\pgfsetdash{}{0pt}%
\pgfsys@defobject{currentmarker}{\pgfqpoint{0.000000in}{-0.027778in}}{\pgfqpoint{0.000000in}{0.000000in}}{%
\pgfpathmoveto{\pgfqpoint{0.000000in}{0.000000in}}%
\pgfpathlineto{\pgfqpoint{0.000000in}{-0.027778in}}%
\pgfusepath{stroke,fill}%
}%
\begin{pgfscope}%
\pgfsys@transformshift{5.232676in}{1.580170in}%
\pgfsys@useobject{currentmarker}{}%
\end{pgfscope}%
\end{pgfscope}%
\begin{pgfscope}%
\pgfsetbuttcap%
\pgfsetroundjoin%
\definecolor{currentfill}{rgb}{0.000000,0.000000,0.000000}%
\pgfsetfillcolor{currentfill}%
\pgfsetlinewidth{0.803000pt}%
\definecolor{currentstroke}{rgb}{0.000000,0.000000,0.000000}%
\pgfsetstrokecolor{currentstroke}%
\pgfsetdash{}{0pt}%
\pgfsys@defobject{currentmarker}{\pgfqpoint{-0.048611in}{0.000000in}}{\pgfqpoint{-0.000000in}{0.000000in}}{%
\pgfpathmoveto{\pgfqpoint{-0.000000in}{0.000000in}}%
\pgfpathlineto{\pgfqpoint{-0.048611in}{0.000000in}}%
\pgfusepath{stroke,fill}%
}%
\begin{pgfscope}%
\pgfsys@transformshift{0.833025in}{1.614270in}%
\pgfsys@useobject{currentmarker}{}%
\end{pgfscope}%
\end{pgfscope}%
\begin{pgfscope}%
\definecolor{textcolor}{rgb}{0.000000,0.000000,0.000000}%
\pgfsetstrokecolor{textcolor}%
\pgfsetfillcolor{textcolor}%
\pgftext[x=0.488889in, y=1.566445in, left, base]{\color{textcolor}\rmfamily\fontsize{10.000000}{12.000000}\selectfont \num{0.00}}%
\end{pgfscope}%
\begin{pgfscope}%
\pgfsetbuttcap%
\pgfsetroundjoin%
\definecolor{currentfill}{rgb}{0.000000,0.000000,0.000000}%
\pgfsetfillcolor{currentfill}%
\pgfsetlinewidth{0.803000pt}%
\definecolor{currentstroke}{rgb}{0.000000,0.000000,0.000000}%
\pgfsetstrokecolor{currentstroke}%
\pgfsetdash{}{0pt}%
\pgfsys@defobject{currentmarker}{\pgfqpoint{-0.048611in}{0.000000in}}{\pgfqpoint{-0.000000in}{0.000000in}}{%
\pgfpathmoveto{\pgfqpoint{-0.000000in}{0.000000in}}%
\pgfpathlineto{\pgfqpoint{-0.048611in}{0.000000in}}%
\pgfusepath{stroke,fill}%
}%
\begin{pgfscope}%
\pgfsys@transformshift{0.833025in}{1.818723in}%
\pgfsys@useobject{currentmarker}{}%
\end{pgfscope}%
\end{pgfscope}%
\begin{pgfscope}%
\definecolor{textcolor}{rgb}{0.000000,0.000000,0.000000}%
\pgfsetstrokecolor{textcolor}%
\pgfsetfillcolor{textcolor}%
\pgftext[x=0.488889in, y=1.770898in, left, base]{\color{textcolor}\rmfamily\fontsize{10.000000}{12.000000}\selectfont \num{0.25}}%
\end{pgfscope}%
\begin{pgfscope}%
\pgfsetbuttcap%
\pgfsetroundjoin%
\definecolor{currentfill}{rgb}{0.000000,0.000000,0.000000}%
\pgfsetfillcolor{currentfill}%
\pgfsetlinewidth{0.803000pt}%
\definecolor{currentstroke}{rgb}{0.000000,0.000000,0.000000}%
\pgfsetstrokecolor{currentstroke}%
\pgfsetdash{}{0pt}%
\pgfsys@defobject{currentmarker}{\pgfqpoint{-0.048611in}{0.000000in}}{\pgfqpoint{-0.000000in}{0.000000in}}{%
\pgfpathmoveto{\pgfqpoint{-0.000000in}{0.000000in}}%
\pgfpathlineto{\pgfqpoint{-0.048611in}{0.000000in}}%
\pgfusepath{stroke,fill}%
}%
\begin{pgfscope}%
\pgfsys@transformshift{0.833025in}{2.023175in}%
\pgfsys@useobject{currentmarker}{}%
\end{pgfscope}%
\end{pgfscope}%
\begin{pgfscope}%
\definecolor{textcolor}{rgb}{0.000000,0.000000,0.000000}%
\pgfsetstrokecolor{textcolor}%
\pgfsetfillcolor{textcolor}%
\pgftext[x=0.488889in, y=1.975351in, left, base]{\color{textcolor}\rmfamily\fontsize{10.000000}{12.000000}\selectfont \num{0.50}}%
\end{pgfscope}%
\begin{pgfscope}%
\pgfsetbuttcap%
\pgfsetroundjoin%
\definecolor{currentfill}{rgb}{0.000000,0.000000,0.000000}%
\pgfsetfillcolor{currentfill}%
\pgfsetlinewidth{0.803000pt}%
\definecolor{currentstroke}{rgb}{0.000000,0.000000,0.000000}%
\pgfsetstrokecolor{currentstroke}%
\pgfsetdash{}{0pt}%
\pgfsys@defobject{currentmarker}{\pgfqpoint{-0.048611in}{0.000000in}}{\pgfqpoint{-0.000000in}{0.000000in}}{%
\pgfpathmoveto{\pgfqpoint{-0.000000in}{0.000000in}}%
\pgfpathlineto{\pgfqpoint{-0.048611in}{0.000000in}}%
\pgfusepath{stroke,fill}%
}%
\begin{pgfscope}%
\pgfsys@transformshift{0.833025in}{2.227628in}%
\pgfsys@useobject{currentmarker}{}%
\end{pgfscope}%
\end{pgfscope}%
\begin{pgfscope}%
\definecolor{textcolor}{rgb}{0.000000,0.000000,0.000000}%
\pgfsetstrokecolor{textcolor}%
\pgfsetfillcolor{textcolor}%
\pgftext[x=0.488889in, y=2.179803in, left, base]{\color{textcolor}\rmfamily\fontsize{10.000000}{12.000000}\selectfont \num{0.75}}%
\end{pgfscope}%
\begin{pgfscope}%
\pgfsetbuttcap%
\pgfsetroundjoin%
\definecolor{currentfill}{rgb}{0.000000,0.000000,0.000000}%
\pgfsetfillcolor{currentfill}%
\pgfsetlinewidth{0.803000pt}%
\definecolor{currentstroke}{rgb}{0.000000,0.000000,0.000000}%
\pgfsetstrokecolor{currentstroke}%
\pgfsetdash{}{0pt}%
\pgfsys@defobject{currentmarker}{\pgfqpoint{-0.048611in}{0.000000in}}{\pgfqpoint{-0.000000in}{0.000000in}}{%
\pgfpathmoveto{\pgfqpoint{-0.000000in}{0.000000in}}%
\pgfpathlineto{\pgfqpoint{-0.048611in}{0.000000in}}%
\pgfusepath{stroke,fill}%
}%
\begin{pgfscope}%
\pgfsys@transformshift{0.833025in}{2.432080in}%
\pgfsys@useobject{currentmarker}{}%
\end{pgfscope}%
\end{pgfscope}%
\begin{pgfscope}%
\definecolor{textcolor}{rgb}{0.000000,0.000000,0.000000}%
\pgfsetstrokecolor{textcolor}%
\pgfsetfillcolor{textcolor}%
\pgftext[x=0.488889in, y=2.384256in, left, base]{\color{textcolor}\rmfamily\fontsize{10.000000}{12.000000}\selectfont \num{1.00}}%
\end{pgfscope}%
\begin{pgfscope}%
\definecolor{textcolor}{rgb}{0.000000,0.000000,0.000000}%
\pgfsetstrokecolor{textcolor}%
\pgfsetfillcolor{textcolor}%
\pgftext[x=0.433333in,y=2.024288in,,bottom,rotate=90.000000]{\color{textcolor}\rmfamily\fontsize{10.000000}{12.000000}\selectfont \begin{tabular}{c}magnetischer\\ Kontrast P\end{tabular}}%
\end{pgfscope}%
\begin{pgfscope}%
\pgfpathrectangle{\pgfqpoint{0.833025in}{1.580170in}}{\pgfqpoint{4.650000in}{0.888235in}}%
\pgfusepath{clip}%
\pgfsetrectcap%
\pgfsetroundjoin%
\pgfsetlinewidth{1.505625pt}%
\definecolor{currentstroke}{rgb}{0.121569,0.466667,0.705882}%
\pgfsetstrokecolor{currentstroke}%
\pgfsetdash{}{0pt}%
\pgfpathmoveto{\pgfqpoint{0.851570in}{1.620545in}}%
\pgfpathlineto{\pgfqpoint{1.060194in}{1.675048in}}%
\pgfpathlineto{\pgfqpoint{1.431081in}{1.771808in}}%
\pgfpathlineto{\pgfqpoint{1.848330in}{1.881463in}}%
\pgfpathlineto{\pgfqpoint{2.056954in}{1.932868in}}%
\pgfpathlineto{\pgfqpoint{2.265578in}{1.980928in}}%
\pgfpathlineto{\pgfqpoint{2.474202in}{2.025608in}}%
\pgfpathlineto{\pgfqpoint{2.682826in}{2.067088in}}%
\pgfpathlineto{\pgfqpoint{2.891450in}{2.105554in}}%
\pgfpathlineto{\pgfqpoint{3.100074in}{2.141090in}}%
\pgfpathlineto{\pgfqpoint{3.308698in}{2.173666in}}%
\pgfpathlineto{\pgfqpoint{3.517323in}{2.203228in}}%
\pgfpathlineto{\pgfqpoint{3.725947in}{2.229823in}}%
\pgfpathlineto{\pgfqpoint{3.957751in}{2.256118in}}%
\pgfpathlineto{\pgfqpoint{4.189556in}{2.279339in}}%
\pgfpathlineto{\pgfqpoint{4.444541in}{2.301727in}}%
\pgfpathlineto{\pgfqpoint{4.699526in}{2.321126in}}%
\pgfpathlineto{\pgfqpoint{4.977691in}{2.339206in}}%
\pgfpathlineto{\pgfqpoint{5.279037in}{2.355617in}}%
\pgfpathlineto{\pgfqpoint{5.464481in}{2.364299in}}%
\pgfpathlineto{\pgfqpoint{5.464481in}{2.364299in}}%
\pgfusepath{stroke}%
\end{pgfscope}%
\begin{pgfscope}%
\pgfpathrectangle{\pgfqpoint{0.833025in}{1.580170in}}{\pgfqpoint{4.650000in}{0.888235in}}%
\pgfusepath{clip}%
\pgfsetrectcap%
\pgfsetroundjoin%
\pgfsetlinewidth{1.505625pt}%
\definecolor{currentstroke}{rgb}{1.000000,0.498039,0.054902}%
\pgfsetstrokecolor{currentstroke}%
\pgfsetdash{}{0pt}%
\pgfpathmoveto{\pgfqpoint{0.851570in}{1.626527in}}%
\pgfpathlineto{\pgfqpoint{1.106555in}{1.759874in}}%
\pgfpathlineto{\pgfqpoint{1.268818in}{1.841189in}}%
\pgfpathlineto{\pgfqpoint{1.407901in}{1.907284in}}%
\pgfpathlineto{\pgfqpoint{1.523803in}{1.959241in}}%
\pgfpathlineto{\pgfqpoint{1.639705in}{2.008019in}}%
\pgfpathlineto{\pgfqpoint{1.755608in}{2.053404in}}%
\pgfpathlineto{\pgfqpoint{1.871510in}{2.095282in}}%
\pgfpathlineto{\pgfqpoint{1.987412in}{2.133628in}}%
\pgfpathlineto{\pgfqpoint{2.103315in}{2.168494in}}%
\pgfpathlineto{\pgfqpoint{2.219217in}{2.199994in}}%
\pgfpathlineto{\pgfqpoint{2.335119in}{2.228289in}}%
\pgfpathlineto{\pgfqpoint{2.451021in}{2.253573in}}%
\pgfpathlineto{\pgfqpoint{2.566924in}{2.276061in}}%
\pgfpathlineto{\pgfqpoint{2.706006in}{2.299678in}}%
\pgfpathlineto{\pgfqpoint{2.845089in}{2.319990in}}%
\pgfpathlineto{\pgfqpoint{2.984172in}{2.337382in}}%
\pgfpathlineto{\pgfqpoint{3.146435in}{2.354466in}}%
\pgfpathlineto{\pgfqpoint{3.308698in}{2.368594in}}%
\pgfpathlineto{\pgfqpoint{3.494142in}{2.381721in}}%
\pgfpathlineto{\pgfqpoint{3.702766in}{2.393353in}}%
\pgfpathlineto{\pgfqpoint{3.934571in}{2.403211in}}%
\pgfpathlineto{\pgfqpoint{4.212736in}{2.411830in}}%
\pgfpathlineto{\pgfqpoint{4.537263in}{2.418717in}}%
\pgfpathlineto{\pgfqpoint{4.954511in}{2.424265in}}%
\pgfpathlineto{\pgfqpoint{5.464481in}{2.428031in}}%
\pgfpathlineto{\pgfqpoint{5.464481in}{2.428031in}}%
\pgfusepath{stroke}%
\end{pgfscope}%
\begin{pgfscope}%
\pgfpathrectangle{\pgfqpoint{0.833025in}{1.580170in}}{\pgfqpoint{4.650000in}{0.888235in}}%
\pgfusepath{clip}%
\pgfsetrectcap%
\pgfsetroundjoin%
\pgfsetlinewidth{1.003750pt}%
\definecolor{currentstroke}{rgb}{0.000000,0.000000,0.000000}%
\pgfsetstrokecolor{currentstroke}%
\pgfsetdash{}{0pt}%
\pgfpathmoveto{\pgfqpoint{2.242397in}{1.580170in}}%
\pgfpathlineto{\pgfqpoint{2.242397in}{2.468406in}}%
\pgfusepath{stroke}%
\end{pgfscope}%
\begin{pgfscope}%
\pgfpathrectangle{\pgfqpoint{0.833025in}{1.580170in}}{\pgfqpoint{4.650000in}{0.888235in}}%
\pgfusepath{clip}%
\pgfsetrectcap%
\pgfsetroundjoin%
\pgfsetlinewidth{1.003750pt}%
\definecolor{currentstroke}{rgb}{0.000000,0.000000,0.000000}%
\pgfsetstrokecolor{currentstroke}%
\pgfsetdash{}{0pt}%
\pgfpathmoveto{\pgfqpoint{3.517323in}{1.580170in}}%
\pgfpathlineto{\pgfqpoint{3.517323in}{2.468406in}}%
\pgfusepath{stroke}%
\end{pgfscope}%
\begin{pgfscope}%
\pgfsetrectcap%
\pgfsetmiterjoin%
\pgfsetlinewidth{0.803000pt}%
\definecolor{currentstroke}{rgb}{0.000000,0.000000,0.000000}%
\pgfsetstrokecolor{currentstroke}%
\pgfsetdash{}{0pt}%
\pgfpathmoveto{\pgfqpoint{0.833025in}{1.580170in}}%
\pgfpathlineto{\pgfqpoint{0.833025in}{2.468406in}}%
\pgfusepath{stroke}%
\end{pgfscope}%
\begin{pgfscope}%
\pgfsetrectcap%
\pgfsetmiterjoin%
\pgfsetlinewidth{0.803000pt}%
\definecolor{currentstroke}{rgb}{0.000000,0.000000,0.000000}%
\pgfsetstrokecolor{currentstroke}%
\pgfsetdash{}{0pt}%
\pgfpathmoveto{\pgfqpoint{5.483025in}{1.580170in}}%
\pgfpathlineto{\pgfqpoint{5.483025in}{2.468406in}}%
\pgfusepath{stroke}%
\end{pgfscope}%
\begin{pgfscope}%
\pgfsetrectcap%
\pgfsetmiterjoin%
\pgfsetlinewidth{0.803000pt}%
\definecolor{currentstroke}{rgb}{0.000000,0.000000,0.000000}%
\pgfsetstrokecolor{currentstroke}%
\pgfsetdash{}{0pt}%
\pgfpathmoveto{\pgfqpoint{0.833025in}{1.580170in}}%
\pgfpathlineto{\pgfqpoint{5.483025in}{1.580170in}}%
\pgfusepath{stroke}%
\end{pgfscope}%
\begin{pgfscope}%
\pgfsetrectcap%
\pgfsetmiterjoin%
\pgfsetlinewidth{0.803000pt}%
\definecolor{currentstroke}{rgb}{0.000000,0.000000,0.000000}%
\pgfsetstrokecolor{currentstroke}%
\pgfsetdash{}{0pt}%
\pgfpathmoveto{\pgfqpoint{0.833025in}{2.468406in}}%
\pgfpathlineto{\pgfqpoint{5.483025in}{2.468406in}}%
\pgfusepath{stroke}%
\end{pgfscope}%
\begin{pgfscope}%
\pgfsetbuttcap%
\pgfsetmiterjoin%
\definecolor{currentfill}{rgb}{1.000000,1.000000,1.000000}%
\pgfsetfillcolor{currentfill}%
\pgfsetlinewidth{0.000000pt}%
\definecolor{currentstroke}{rgb}{0.000000,0.000000,0.000000}%
\pgfsetstrokecolor{currentstroke}%
\pgfsetstrokeopacity{0.000000}%
\pgfsetdash{}{0pt}%
\pgfpathmoveto{\pgfqpoint{0.833025in}{0.514288in}}%
\pgfpathlineto{\pgfqpoint{5.483025in}{0.514288in}}%
\pgfpathlineto{\pgfqpoint{5.483025in}{1.402523in}}%
\pgfpathlineto{\pgfqpoint{0.833025in}{1.402523in}}%
\pgfpathlineto{\pgfqpoint{0.833025in}{0.514288in}}%
\pgfpathclose%
\pgfusepath{fill}%
\end{pgfscope}%
\begin{pgfscope}%
\pgfsetbuttcap%
\pgfsetroundjoin%
\definecolor{currentfill}{rgb}{0.000000,0.000000,0.000000}%
\pgfsetfillcolor{currentfill}%
\pgfsetlinewidth{0.803000pt}%
\definecolor{currentstroke}{rgb}{0.000000,0.000000,0.000000}%
\pgfsetstrokecolor{currentstroke}%
\pgfsetdash{}{0pt}%
\pgfsys@defobject{currentmarker}{\pgfqpoint{0.000000in}{-0.048611in}}{\pgfqpoint{0.000000in}{0.000000in}}{%
\pgfpathmoveto{\pgfqpoint{0.000000in}{0.000000in}}%
\pgfpathlineto{\pgfqpoint{0.000000in}{-0.048611in}}%
\pgfusepath{stroke,fill}%
}%
\begin{pgfscope}%
\pgfsys@transformshift{1.407901in}{0.514288in}%
\pgfsys@useobject{currentmarker}{}%
\end{pgfscope}%
\end{pgfscope}%
\begin{pgfscope}%
\definecolor{textcolor}{rgb}{0.000000,0.000000,0.000000}%
\pgfsetstrokecolor{textcolor}%
\pgfsetfillcolor{textcolor}%
\pgftext[x=1.407901in,y=0.417066in,,top]{\color{textcolor}\rmfamily\fontsize{10.000000}{12.000000}\selectfont \(\displaystyle {25}\)}%
\end{pgfscope}%
\begin{pgfscope}%
\pgfsetbuttcap%
\pgfsetroundjoin%
\definecolor{currentfill}{rgb}{0.000000,0.000000,0.000000}%
\pgfsetfillcolor{currentfill}%
\pgfsetlinewidth{0.803000pt}%
\definecolor{currentstroke}{rgb}{0.000000,0.000000,0.000000}%
\pgfsetstrokecolor{currentstroke}%
\pgfsetdash{}{0pt}%
\pgfsys@defobject{currentmarker}{\pgfqpoint{0.000000in}{-0.048611in}}{\pgfqpoint{0.000000in}{0.000000in}}{%
\pgfpathmoveto{\pgfqpoint{0.000000in}{0.000000in}}%
\pgfpathlineto{\pgfqpoint{0.000000in}{-0.048611in}}%
\pgfusepath{stroke,fill}%
}%
\begin{pgfscope}%
\pgfsys@transformshift{1.987412in}{0.514288in}%
\pgfsys@useobject{currentmarker}{}%
\end{pgfscope}%
\end{pgfscope}%
\begin{pgfscope}%
\definecolor{textcolor}{rgb}{0.000000,0.000000,0.000000}%
\pgfsetstrokecolor{textcolor}%
\pgfsetfillcolor{textcolor}%
\pgftext[x=1.987412in,y=0.417066in,,top]{\color{textcolor}\rmfamily\fontsize{10.000000}{12.000000}\selectfont \(\displaystyle {50}\)}%
\end{pgfscope}%
\begin{pgfscope}%
\pgfsetbuttcap%
\pgfsetroundjoin%
\definecolor{currentfill}{rgb}{0.000000,0.000000,0.000000}%
\pgfsetfillcolor{currentfill}%
\pgfsetlinewidth{0.803000pt}%
\definecolor{currentstroke}{rgb}{0.000000,0.000000,0.000000}%
\pgfsetstrokecolor{currentstroke}%
\pgfsetdash{}{0pt}%
\pgfsys@defobject{currentmarker}{\pgfqpoint{0.000000in}{-0.048611in}}{\pgfqpoint{0.000000in}{0.000000in}}{%
\pgfpathmoveto{\pgfqpoint{0.000000in}{0.000000in}}%
\pgfpathlineto{\pgfqpoint{0.000000in}{-0.048611in}}%
\pgfusepath{stroke,fill}%
}%
\begin{pgfscope}%
\pgfsys@transformshift{2.566924in}{0.514288in}%
\pgfsys@useobject{currentmarker}{}%
\end{pgfscope}%
\end{pgfscope}%
\begin{pgfscope}%
\definecolor{textcolor}{rgb}{0.000000,0.000000,0.000000}%
\pgfsetstrokecolor{textcolor}%
\pgfsetfillcolor{textcolor}%
\pgftext[x=2.566924in,y=0.417066in,,top]{\color{textcolor}\rmfamily\fontsize{10.000000}{12.000000}\selectfont \(\displaystyle {75}\)}%
\end{pgfscope}%
\begin{pgfscope}%
\pgfsetbuttcap%
\pgfsetroundjoin%
\definecolor{currentfill}{rgb}{0.000000,0.000000,0.000000}%
\pgfsetfillcolor{currentfill}%
\pgfsetlinewidth{0.803000pt}%
\definecolor{currentstroke}{rgb}{0.000000,0.000000,0.000000}%
\pgfsetstrokecolor{currentstroke}%
\pgfsetdash{}{0pt}%
\pgfsys@defobject{currentmarker}{\pgfqpoint{0.000000in}{-0.048611in}}{\pgfqpoint{0.000000in}{0.000000in}}{%
\pgfpathmoveto{\pgfqpoint{0.000000in}{0.000000in}}%
\pgfpathlineto{\pgfqpoint{0.000000in}{-0.048611in}}%
\pgfusepath{stroke,fill}%
}%
\begin{pgfscope}%
\pgfsys@transformshift{3.146435in}{0.514288in}%
\pgfsys@useobject{currentmarker}{}%
\end{pgfscope}%
\end{pgfscope}%
\begin{pgfscope}%
\definecolor{textcolor}{rgb}{0.000000,0.000000,0.000000}%
\pgfsetstrokecolor{textcolor}%
\pgfsetfillcolor{textcolor}%
\pgftext[x=3.146435in,y=0.417066in,,top]{\color{textcolor}\rmfamily\fontsize{10.000000}{12.000000}\selectfont \(\displaystyle {100}\)}%
\end{pgfscope}%
\begin{pgfscope}%
\pgfsetbuttcap%
\pgfsetroundjoin%
\definecolor{currentfill}{rgb}{0.000000,0.000000,0.000000}%
\pgfsetfillcolor{currentfill}%
\pgfsetlinewidth{0.803000pt}%
\definecolor{currentstroke}{rgb}{0.000000,0.000000,0.000000}%
\pgfsetstrokecolor{currentstroke}%
\pgfsetdash{}{0pt}%
\pgfsys@defobject{currentmarker}{\pgfqpoint{0.000000in}{-0.048611in}}{\pgfqpoint{0.000000in}{0.000000in}}{%
\pgfpathmoveto{\pgfqpoint{0.000000in}{0.000000in}}%
\pgfpathlineto{\pgfqpoint{0.000000in}{-0.048611in}}%
\pgfusepath{stroke,fill}%
}%
\begin{pgfscope}%
\pgfsys@transformshift{3.725947in}{0.514288in}%
\pgfsys@useobject{currentmarker}{}%
\end{pgfscope}%
\end{pgfscope}%
\begin{pgfscope}%
\definecolor{textcolor}{rgb}{0.000000,0.000000,0.000000}%
\pgfsetstrokecolor{textcolor}%
\pgfsetfillcolor{textcolor}%
\pgftext[x=3.725947in,y=0.417066in,,top]{\color{textcolor}\rmfamily\fontsize{10.000000}{12.000000}\selectfont \(\displaystyle {125}\)}%
\end{pgfscope}%
\begin{pgfscope}%
\pgfsetbuttcap%
\pgfsetroundjoin%
\definecolor{currentfill}{rgb}{0.000000,0.000000,0.000000}%
\pgfsetfillcolor{currentfill}%
\pgfsetlinewidth{0.803000pt}%
\definecolor{currentstroke}{rgb}{0.000000,0.000000,0.000000}%
\pgfsetstrokecolor{currentstroke}%
\pgfsetdash{}{0pt}%
\pgfsys@defobject{currentmarker}{\pgfqpoint{0.000000in}{-0.048611in}}{\pgfqpoint{0.000000in}{0.000000in}}{%
\pgfpathmoveto{\pgfqpoint{0.000000in}{0.000000in}}%
\pgfpathlineto{\pgfqpoint{0.000000in}{-0.048611in}}%
\pgfusepath{stroke,fill}%
}%
\begin{pgfscope}%
\pgfsys@transformshift{4.305458in}{0.514288in}%
\pgfsys@useobject{currentmarker}{}%
\end{pgfscope}%
\end{pgfscope}%
\begin{pgfscope}%
\definecolor{textcolor}{rgb}{0.000000,0.000000,0.000000}%
\pgfsetstrokecolor{textcolor}%
\pgfsetfillcolor{textcolor}%
\pgftext[x=4.305458in,y=0.417066in,,top]{\color{textcolor}\rmfamily\fontsize{10.000000}{12.000000}\selectfont \(\displaystyle {150}\)}%
\end{pgfscope}%
\begin{pgfscope}%
\pgfsetbuttcap%
\pgfsetroundjoin%
\definecolor{currentfill}{rgb}{0.000000,0.000000,0.000000}%
\pgfsetfillcolor{currentfill}%
\pgfsetlinewidth{0.803000pt}%
\definecolor{currentstroke}{rgb}{0.000000,0.000000,0.000000}%
\pgfsetstrokecolor{currentstroke}%
\pgfsetdash{}{0pt}%
\pgfsys@defobject{currentmarker}{\pgfqpoint{0.000000in}{-0.048611in}}{\pgfqpoint{0.000000in}{0.000000in}}{%
\pgfpathmoveto{\pgfqpoint{0.000000in}{0.000000in}}%
\pgfpathlineto{\pgfqpoint{0.000000in}{-0.048611in}}%
\pgfusepath{stroke,fill}%
}%
\begin{pgfscope}%
\pgfsys@transformshift{4.884970in}{0.514288in}%
\pgfsys@useobject{currentmarker}{}%
\end{pgfscope}%
\end{pgfscope}%
\begin{pgfscope}%
\definecolor{textcolor}{rgb}{0.000000,0.000000,0.000000}%
\pgfsetstrokecolor{textcolor}%
\pgfsetfillcolor{textcolor}%
\pgftext[x=4.884970in,y=0.417066in,,top]{\color{textcolor}\rmfamily\fontsize{10.000000}{12.000000}\selectfont \(\displaystyle {175}\)}%
\end{pgfscope}%
\begin{pgfscope}%
\pgfsetbuttcap%
\pgfsetroundjoin%
\definecolor{currentfill}{rgb}{0.000000,0.000000,0.000000}%
\pgfsetfillcolor{currentfill}%
\pgfsetlinewidth{0.803000pt}%
\definecolor{currentstroke}{rgb}{0.000000,0.000000,0.000000}%
\pgfsetstrokecolor{currentstroke}%
\pgfsetdash{}{0pt}%
\pgfsys@defobject{currentmarker}{\pgfqpoint{0.000000in}{-0.048611in}}{\pgfqpoint{0.000000in}{0.000000in}}{%
\pgfpathmoveto{\pgfqpoint{0.000000in}{0.000000in}}%
\pgfpathlineto{\pgfqpoint{0.000000in}{-0.048611in}}%
\pgfusepath{stroke,fill}%
}%
\begin{pgfscope}%
\pgfsys@transformshift{5.464481in}{0.514288in}%
\pgfsys@useobject{currentmarker}{}%
\end{pgfscope}%
\end{pgfscope}%
\begin{pgfscope}%
\definecolor{textcolor}{rgb}{0.000000,0.000000,0.000000}%
\pgfsetstrokecolor{textcolor}%
\pgfsetfillcolor{textcolor}%
\pgftext[x=5.464481in,y=0.417066in,,top]{\color{textcolor}\rmfamily\fontsize{10.000000}{12.000000}\selectfont \(\displaystyle {200}\)}%
\end{pgfscope}%
\begin{pgfscope}%
\definecolor{textcolor}{rgb}{0.000000,0.000000,0.000000}%
\pgfsetstrokecolor{textcolor}%
\pgfsetfillcolor{textcolor}%
\pgftext[x=3.158025in,y=0.238855in,,top]{\color{textcolor}\rmfamily\fontsize{10.000000}{12.000000}\selectfont Wiederholungszahl \(\displaystyle N\) der Probe SiN(\SI{200}{\micro\meter})/Ta(\SI{3}{\nano\meter})/[Fe(\SI{0.41}{\nano\meter})/Gd(\SI{0.45}{\nano\meter})]\(\displaystyle _{\text{x}N}\)/Ta(\SI{2}{\nano\meter})}%
\end{pgfscope}%
\begin{pgfscope}%
\pgfsetbuttcap%
\pgfsetroundjoin%
\definecolor{currentfill}{rgb}{0.000000,0.000000,0.000000}%
\pgfsetfillcolor{currentfill}%
\pgfsetlinewidth{0.803000pt}%
\definecolor{currentstroke}{rgb}{0.000000,0.000000,0.000000}%
\pgfsetstrokecolor{currentstroke}%
\pgfsetdash{}{0pt}%
\pgfsys@defobject{currentmarker}{\pgfqpoint{-0.048611in}{0.000000in}}{\pgfqpoint{-0.000000in}{0.000000in}}{%
\pgfpathmoveto{\pgfqpoint{-0.000000in}{0.000000in}}%
\pgfpathlineto{\pgfqpoint{-0.048611in}{0.000000in}}%
\pgfusepath{stroke,fill}%
}%
\begin{pgfscope}%
\pgfsys@transformshift{0.833025in}{0.554193in}%
\pgfsys@useobject{currentmarker}{}%
\end{pgfscope}%
\end{pgfscope}%
\begin{pgfscope}%
\definecolor{textcolor}{rgb}{0.000000,0.000000,0.000000}%
\pgfsetstrokecolor{textcolor}%
\pgfsetfillcolor{textcolor}%
\pgftext[x=0.488889in, y=0.506368in, left, base]{\color{textcolor}\rmfamily\fontsize{10.000000}{12.000000}\selectfont \num{0.00}}%
\end{pgfscope}%
\begin{pgfscope}%
\pgfsetbuttcap%
\pgfsetroundjoin%
\definecolor{currentfill}{rgb}{0.000000,0.000000,0.000000}%
\pgfsetfillcolor{currentfill}%
\pgfsetlinewidth{0.803000pt}%
\definecolor{currentstroke}{rgb}{0.000000,0.000000,0.000000}%
\pgfsetstrokecolor{currentstroke}%
\pgfsetdash{}{0pt}%
\pgfsys@defobject{currentmarker}{\pgfqpoint{-0.048611in}{0.000000in}}{\pgfqpoint{-0.000000in}{0.000000in}}{%
\pgfpathmoveto{\pgfqpoint{-0.000000in}{0.000000in}}%
\pgfpathlineto{\pgfqpoint{-0.048611in}{0.000000in}}%
\pgfusepath{stroke,fill}%
}%
\begin{pgfscope}%
\pgfsys@transformshift{0.833025in}{0.760277in}%
\pgfsys@useobject{currentmarker}{}%
\end{pgfscope}%
\end{pgfscope}%
\begin{pgfscope}%
\definecolor{textcolor}{rgb}{0.000000,0.000000,0.000000}%
\pgfsetstrokecolor{textcolor}%
\pgfsetfillcolor{textcolor}%
\pgftext[x=0.488889in, y=0.712453in, left, base]{\color{textcolor}\rmfamily\fontsize{10.000000}{12.000000}\selectfont \num{0.02}}%
\end{pgfscope}%
\begin{pgfscope}%
\pgfsetbuttcap%
\pgfsetroundjoin%
\definecolor{currentfill}{rgb}{0.000000,0.000000,0.000000}%
\pgfsetfillcolor{currentfill}%
\pgfsetlinewidth{0.803000pt}%
\definecolor{currentstroke}{rgb}{0.000000,0.000000,0.000000}%
\pgfsetstrokecolor{currentstroke}%
\pgfsetdash{}{0pt}%
\pgfsys@defobject{currentmarker}{\pgfqpoint{-0.048611in}{0.000000in}}{\pgfqpoint{-0.000000in}{0.000000in}}{%
\pgfpathmoveto{\pgfqpoint{-0.000000in}{0.000000in}}%
\pgfpathlineto{\pgfqpoint{-0.048611in}{0.000000in}}%
\pgfusepath{stroke,fill}%
}%
\begin{pgfscope}%
\pgfsys@transformshift{0.833025in}{0.966362in}%
\pgfsys@useobject{currentmarker}{}%
\end{pgfscope}%
\end{pgfscope}%
\begin{pgfscope}%
\definecolor{textcolor}{rgb}{0.000000,0.000000,0.000000}%
\pgfsetstrokecolor{textcolor}%
\pgfsetfillcolor{textcolor}%
\pgftext[x=0.488889in, y=0.918537in, left, base]{\color{textcolor}\rmfamily\fontsize{10.000000}{12.000000}\selectfont \num{0.04}}%
\end{pgfscope}%
\begin{pgfscope}%
\pgfsetbuttcap%
\pgfsetroundjoin%
\definecolor{currentfill}{rgb}{0.000000,0.000000,0.000000}%
\pgfsetfillcolor{currentfill}%
\pgfsetlinewidth{0.803000pt}%
\definecolor{currentstroke}{rgb}{0.000000,0.000000,0.000000}%
\pgfsetstrokecolor{currentstroke}%
\pgfsetdash{}{0pt}%
\pgfsys@defobject{currentmarker}{\pgfqpoint{-0.048611in}{0.000000in}}{\pgfqpoint{-0.000000in}{0.000000in}}{%
\pgfpathmoveto{\pgfqpoint{-0.000000in}{0.000000in}}%
\pgfpathlineto{\pgfqpoint{-0.048611in}{0.000000in}}%
\pgfusepath{stroke,fill}%
}%
\begin{pgfscope}%
\pgfsys@transformshift{0.833025in}{1.172446in}%
\pgfsys@useobject{currentmarker}{}%
\end{pgfscope}%
\end{pgfscope}%
\begin{pgfscope}%
\definecolor{textcolor}{rgb}{0.000000,0.000000,0.000000}%
\pgfsetstrokecolor{textcolor}%
\pgfsetfillcolor{textcolor}%
\pgftext[x=0.488889in, y=1.124621in, left, base]{\color{textcolor}\rmfamily\fontsize{10.000000}{12.000000}\selectfont \num{0.06}}%
\end{pgfscope}%
\begin{pgfscope}%
\pgfsetbuttcap%
\pgfsetroundjoin%
\definecolor{currentfill}{rgb}{0.000000,0.000000,0.000000}%
\pgfsetfillcolor{currentfill}%
\pgfsetlinewidth{0.803000pt}%
\definecolor{currentstroke}{rgb}{0.000000,0.000000,0.000000}%
\pgfsetstrokecolor{currentstroke}%
\pgfsetdash{}{0pt}%
\pgfsys@defobject{currentmarker}{\pgfqpoint{-0.048611in}{0.000000in}}{\pgfqpoint{-0.000000in}{0.000000in}}{%
\pgfpathmoveto{\pgfqpoint{-0.000000in}{0.000000in}}%
\pgfpathlineto{\pgfqpoint{-0.048611in}{0.000000in}}%
\pgfusepath{stroke,fill}%
}%
\begin{pgfscope}%
\pgfsys@transformshift{0.833025in}{1.378530in}%
\pgfsys@useobject{currentmarker}{}%
\end{pgfscope}%
\end{pgfscope}%
\begin{pgfscope}%
\definecolor{textcolor}{rgb}{0.000000,0.000000,0.000000}%
\pgfsetstrokecolor{textcolor}%
\pgfsetfillcolor{textcolor}%
\pgftext[x=0.488889in, y=1.330706in, left, base]{\color{textcolor}\rmfamily\fontsize{10.000000}{12.000000}\selectfont \num{0.08}}%
\end{pgfscope}%
\begin{pgfscope}%
\definecolor{textcolor}{rgb}{0.000000,0.000000,0.000000}%
\pgfsetstrokecolor{textcolor}%
\pgfsetfillcolor{textcolor}%
\pgftext[x=0.433333in,y=0.958406in,,bottom,rotate=90.000000]{\color{textcolor}\rmfamily\fontsize{10.000000}{12.000000}\selectfont \begin{tabular}{c}Gütezahl TP\(\displaystyle ^2\)\end{tabular}}%
\end{pgfscope}%
\begin{pgfscope}%
\pgfpathrectangle{\pgfqpoint{0.833025in}{0.514288in}}{\pgfqpoint{4.650000in}{0.888235in}}%
\pgfusepath{clip}%
\pgfsetrectcap%
\pgfsetroundjoin%
\pgfsetlinewidth{1.505625pt}%
\definecolor{currentstroke}{rgb}{0.121569,0.466667,0.705882}%
\pgfsetstrokecolor{currentstroke}%
\pgfsetdash{}{0pt}%
\pgfpathmoveto{\pgfqpoint{0.851570in}{0.554662in}}%
\pgfpathlineto{\pgfqpoint{0.897931in}{0.558132in}}%
\pgfpathlineto{\pgfqpoint{0.944292in}{0.564430in}}%
\pgfpathlineto{\pgfqpoint{1.013833in}{0.578037in}}%
\pgfpathlineto{\pgfqpoint{1.083374in}{0.595461in}}%
\pgfpathlineto{\pgfqpoint{1.176096in}{0.622945in}}%
\pgfpathlineto{\pgfqpoint{1.291998in}{0.661695in}}%
\pgfpathlineto{\pgfqpoint{1.639705in}{0.782141in}}%
\pgfpathlineto{\pgfqpoint{1.732427in}{0.810369in}}%
\pgfpathlineto{\pgfqpoint{1.825149in}{0.835650in}}%
\pgfpathlineto{\pgfqpoint{1.917871in}{0.857650in}}%
\pgfpathlineto{\pgfqpoint{2.010593in}{0.876189in}}%
\pgfpathlineto{\pgfqpoint{2.103315in}{0.891210in}}%
\pgfpathlineto{\pgfqpoint{2.196036in}{0.902751in}}%
\pgfpathlineto{\pgfqpoint{2.288758in}{0.910935in}}%
\pgfpathlineto{\pgfqpoint{2.381480in}{0.915960in}}%
\pgfpathlineto{\pgfqpoint{2.474202in}{0.918091in}}%
\pgfpathlineto{\pgfqpoint{2.590104in}{0.917167in}}%
\pgfpathlineto{\pgfqpoint{2.706006in}{0.912885in}}%
\pgfpathlineto{\pgfqpoint{2.845089in}{0.904239in}}%
\pgfpathlineto{\pgfqpoint{3.007352in}{0.890564in}}%
\pgfpathlineto{\pgfqpoint{3.192796in}{0.871642in}}%
\pgfpathlineto{\pgfqpoint{3.424601in}{0.844763in}}%
\pgfpathlineto{\pgfqpoint{3.818668in}{0.795237in}}%
\pgfpathlineto{\pgfqpoint{4.189556in}{0.749859in}}%
\pgfpathlineto{\pgfqpoint{4.467721in}{0.719212in}}%
\pgfpathlineto{\pgfqpoint{4.722706in}{0.694215in}}%
\pgfpathlineto{\pgfqpoint{4.977691in}{0.672170in}}%
\pgfpathlineto{\pgfqpoint{5.255857in}{0.651371in}}%
\pgfpathlineto{\pgfqpoint{5.464481in}{0.637893in}}%
\pgfpathlineto{\pgfqpoint{5.464481in}{0.637893in}}%
\pgfusepath{stroke}%
\end{pgfscope}%
\begin{pgfscope}%
\pgfpathrectangle{\pgfqpoint{0.833025in}{0.514288in}}{\pgfqpoint{4.650000in}{0.888235in}}%
\pgfusepath{clip}%
\pgfsetrectcap%
\pgfsetroundjoin%
\pgfsetlinewidth{1.505625pt}%
\definecolor{currentstroke}{rgb}{1.000000,0.498039,0.054902}%
\pgfsetstrokecolor{currentstroke}%
\pgfsetdash{}{0pt}%
\pgfpathmoveto{\pgfqpoint{0.851570in}{0.556290in}}%
\pgfpathlineto{\pgfqpoint{0.874750in}{0.562286in}}%
\pgfpathlineto{\pgfqpoint{0.897931in}{0.571746in}}%
\pgfpathlineto{\pgfqpoint{0.921111in}{0.584262in}}%
\pgfpathlineto{\pgfqpoint{0.944292in}{0.599476in}}%
\pgfpathlineto{\pgfqpoint{0.967472in}{0.617025in}}%
\pgfpathlineto{\pgfqpoint{1.013833in}{0.657887in}}%
\pgfpathlineto{\pgfqpoint{1.060194in}{0.704545in}}%
\pgfpathlineto{\pgfqpoint{1.129735in}{0.781237in}}%
\pgfpathlineto{\pgfqpoint{1.315179in}{0.991739in}}%
\pgfpathlineto{\pgfqpoint{1.384720in}{1.063777in}}%
\pgfpathlineto{\pgfqpoint{1.431081in}{1.108094in}}%
\pgfpathlineto{\pgfqpoint{1.477442in}{1.149045in}}%
\pgfpathlineto{\pgfqpoint{1.523803in}{1.186405in}}%
\pgfpathlineto{\pgfqpoint{1.570164in}{1.220047in}}%
\pgfpathlineto{\pgfqpoint{1.616525in}{1.249908in}}%
\pgfpathlineto{\pgfqpoint{1.662886in}{1.275988in}}%
\pgfpathlineto{\pgfqpoint{1.709247in}{1.298348in}}%
\pgfpathlineto{\pgfqpoint{1.755608in}{1.317082in}}%
\pgfpathlineto{\pgfqpoint{1.801969in}{1.332316in}}%
\pgfpathlineto{\pgfqpoint{1.848330in}{1.344209in}}%
\pgfpathlineto{\pgfqpoint{1.894690in}{1.352936in}}%
\pgfpathlineto{\pgfqpoint{1.941051in}{1.358682in}}%
\pgfpathlineto{\pgfqpoint{1.987412in}{1.361645in}}%
\pgfpathlineto{\pgfqpoint{2.033773in}{1.362031in}}%
\pgfpathlineto{\pgfqpoint{2.080134in}{1.360041in}}%
\pgfpathlineto{\pgfqpoint{2.149675in}{1.353043in}}%
\pgfpathlineto{\pgfqpoint{2.219217in}{1.341820in}}%
\pgfpathlineto{\pgfqpoint{2.288758in}{1.327009in}}%
\pgfpathlineto{\pgfqpoint{2.358300in}{1.309203in}}%
\pgfpathlineto{\pgfqpoint{2.451021in}{1.281745in}}%
\pgfpathlineto{\pgfqpoint{2.543743in}{1.251100in}}%
\pgfpathlineto{\pgfqpoint{2.682826in}{1.201307in}}%
\pgfpathlineto{\pgfqpoint{3.192796in}{1.013853in}}%
\pgfpathlineto{\pgfqpoint{3.331879in}{0.967374in}}%
\pgfpathlineto{\pgfqpoint{3.470962in}{0.924135in}}%
\pgfpathlineto{\pgfqpoint{3.610044in}{0.884340in}}%
\pgfpathlineto{\pgfqpoint{3.749127in}{0.848026in}}%
\pgfpathlineto{\pgfqpoint{3.888210in}{0.815115in}}%
\pgfpathlineto{\pgfqpoint{4.027293in}{0.785454in}}%
\pgfpathlineto{\pgfqpoint{4.166375in}{0.758843in}}%
\pgfpathlineto{\pgfqpoint{4.305458in}{0.735057in}}%
\pgfpathlineto{\pgfqpoint{4.467721in}{0.710564in}}%
\pgfpathlineto{\pgfqpoint{4.629985in}{0.689225in}}%
\pgfpathlineto{\pgfqpoint{4.815428in}{0.668245in}}%
\pgfpathlineto{\pgfqpoint{5.000872in}{0.650428in}}%
\pgfpathlineto{\pgfqpoint{5.209496in}{0.633614in}}%
\pgfpathlineto{\pgfqpoint{5.441301in}{0.618298in}}%
\pgfpathlineto{\pgfqpoint{5.464481in}{0.616936in}}%
\pgfpathlineto{\pgfqpoint{5.464481in}{0.616936in}}%
\pgfusepath{stroke}%
\end{pgfscope}%
\begin{pgfscope}%
\pgfpathrectangle{\pgfqpoint{0.833025in}{0.514288in}}{\pgfqpoint{4.650000in}{0.888235in}}%
\pgfusepath{clip}%
\pgfsetrectcap%
\pgfsetroundjoin%
\pgfsetlinewidth{1.003750pt}%
\definecolor{currentstroke}{rgb}{0.000000,0.000000,0.000000}%
\pgfsetstrokecolor{currentstroke}%
\pgfsetdash{}{0pt}%
\pgfpathmoveto{\pgfqpoint{2.242397in}{0.514288in}}%
\pgfpathlineto{\pgfqpoint{2.242397in}{1.402523in}}%
\pgfusepath{stroke}%
\end{pgfscope}%
\begin{pgfscope}%
\pgfpathrectangle{\pgfqpoint{0.833025in}{0.514288in}}{\pgfqpoint{4.650000in}{0.888235in}}%
\pgfusepath{clip}%
\pgfsetrectcap%
\pgfsetroundjoin%
\pgfsetlinewidth{1.003750pt}%
\definecolor{currentstroke}{rgb}{0.000000,0.000000,0.000000}%
\pgfsetstrokecolor{currentstroke}%
\pgfsetdash{}{0pt}%
\pgfpathmoveto{\pgfqpoint{3.517323in}{0.514288in}}%
\pgfpathlineto{\pgfqpoint{3.517323in}{1.402523in}}%
\pgfusepath{stroke}%
\end{pgfscope}%
\begin{pgfscope}%
\pgfsetrectcap%
\pgfsetmiterjoin%
\pgfsetlinewidth{0.803000pt}%
\definecolor{currentstroke}{rgb}{0.000000,0.000000,0.000000}%
\pgfsetstrokecolor{currentstroke}%
\pgfsetdash{}{0pt}%
\pgfpathmoveto{\pgfqpoint{0.833025in}{0.514288in}}%
\pgfpathlineto{\pgfqpoint{0.833025in}{1.402523in}}%
\pgfusepath{stroke}%
\end{pgfscope}%
\begin{pgfscope}%
\pgfsetrectcap%
\pgfsetmiterjoin%
\pgfsetlinewidth{0.803000pt}%
\definecolor{currentstroke}{rgb}{0.000000,0.000000,0.000000}%
\pgfsetstrokecolor{currentstroke}%
\pgfsetdash{}{0pt}%
\pgfpathmoveto{\pgfqpoint{5.483025in}{0.514288in}}%
\pgfpathlineto{\pgfqpoint{5.483025in}{1.402523in}}%
\pgfusepath{stroke}%
\end{pgfscope}%
\begin{pgfscope}%
\pgfsetrectcap%
\pgfsetmiterjoin%
\pgfsetlinewidth{0.803000pt}%
\definecolor{currentstroke}{rgb}{0.000000,0.000000,0.000000}%
\pgfsetstrokecolor{currentstroke}%
\pgfsetdash{}{0pt}%
\pgfpathmoveto{\pgfqpoint{0.833025in}{0.514288in}}%
\pgfpathlineto{\pgfqpoint{5.483025in}{0.514288in}}%
\pgfusepath{stroke}%
\end{pgfscope}%
\begin{pgfscope}%
\pgfsetrectcap%
\pgfsetmiterjoin%
\pgfsetlinewidth{0.803000pt}%
\definecolor{currentstroke}{rgb}{0.000000,0.000000,0.000000}%
\pgfsetstrokecolor{currentstroke}%
\pgfsetdash{}{0pt}%
\pgfpathmoveto{\pgfqpoint{0.833025in}{1.402523in}}%
\pgfpathlineto{\pgfqpoint{5.483025in}{1.402523in}}%
\pgfusepath{stroke}%
\end{pgfscope}%
\end{pgfpicture}%
\makeatother%
\endgroup%

    \caption{Die Abhängigkeit der Transmissionsrate $\text{T}$ und der Gütezahl TP$^2$ einer Probe ist in Bezug auf die Wiederholungszahl $N$ abgebildet. Die Maxima der Gütezahl TP$^2$ werden bei dem Wert $N_{\text{Fe, L3}}=72$ für die Photonenenergie $h\nu_{\text{Fe, L3}}=\SI{706,97}{\eV}$ (orange Linie) bzw. bei dem Wert $N_{\text{Gd, M5}}=52$ für die Photonenenergie $h\nu_{\text{Gd, M5}}=\SI{1184,79}{\eV}$ (blaue Linie) erreicht.}
    \label{fig:proben_vergleich_centered}
\end{figure}
\noindent
In Abb. \ref{fig:proben_vergleich_centered} sind die Ergebnisse aufgetragen. Die Extrema der Gütezahl TP$^2$ liegen bei den Werten $N_{\text{Fe, L3}}=72$  bzw. $N_{\text{Gd, M5}}=52$ für die Photonenenergien $h\nu_{\text{Fe, L3}}=\SI{706,97}{\eV}$ bzw. $h\nu_{\text{Gd, M5}}=\SI{1184,79}{\eV}$. Außerdem ist es wichtig, dass der absolute Betrag des Gütezahlmaximums TP$^2$ bei der Photonenenergie $h\nu_{\text{Gd, M5}}$ ca. doppelt so groß ist wie bei der Photonenenrgie $h\nu_{\text{Fe, L3}}$. Aus dieser Perspektive sollen die Experimente unter Verwendung der Photonenenergie $h\nu_{\text{Gd, M5}}$ durchgeführt werden.

\noindent
Als Schlussfolgerung werden insgesamt zwei Proben hergestellt: die erste Probe, die weiter unter dem Namen \textbf{DS211221} bezeichnet wird, wurde auf \SI{200}{\micro\meter} SiN-Substrat gewachsen und besteht aus 61 Doppelschichten [Fe(\SI{0.41}{\nano\meter})/Gd(\SI{0.45}{\nano\meter})] mit einer Haftschicht aus Ta(\SI{3}{\nano\meter}) und einer Deckschicht aus Ta(\SI{2}{\nano\meter}). Die Wiederholungszahl $\bar{N} = 61$ wurde als Mittelwert zwischen den Werten $N_{\text{Fe, L3}}=72$ und $N_{\text{Gd, M5}}=52$ genommen.

\noindent
Die zweite zu produzierende Probe, die folgend unter dem Namen \textbf{DS220126} bezeichnet wird, wurde auf demselben Substrat gewachsen und besteht aus 116 Doppelschichten [Fe(\SI{0.41}{\nano\meter})/Gd(\SI{0.45}{\nano\meter})]. Die Haft- und Deckschicht sind jeweils Ta(\SI{3}{\nano\meter}) und Ta(\SI{2}{\nano\meter}). Für die Wahl der Wiederholungszahl $N=116$ wurde \cite[Abschnitt „Sample Preparation“]{tripathi_dichroic_2011} als Referenz genommen.

\section{Probencharakterisierung}
Eines der Messgeräte, das die magnetische Struktur dünner Schichten mit \si{\nano\meter}-Auflösung abbilden kann, ist ein \gls{mfm}. Die mit dem \gls{mfm} aufgenommenen Domänenmuster und ihre Betragsquadrate von Fourier-Transformierten in logarithmischer Darstellung sind in Abb. \ref{fig:mfm-amplitude-ft} dargestellt.
% \begin{figure}[H]
%     \centering
%     \subfloat[]{\includegraphics[width=0.27\textwidth]{images/mfm/ds211221_R1_membrane_amplitude_crop.png} \label{fig:mfm_ds211221}}
%     \hspace{1cm}
%     \subfloat[]{\includegraphics[width=0.27\textwidth]{images/mfm/ds211221_R1_membrane_amplitude_crop_2um .png} \label{fig:mfm_ds211221_2um}}
%     \hspace{1cm}
%     \subfloat[]{\includegraphics[width=0.27\textwidth]{images/mfm/ds220126_R1_membrane_amplitude_cropped.png} \label{fig:mfm_ds220126}}
%     \caption{Die aufgenommenen Domäne von der Probe (a) \textbf{DS211221} und (b) \textbf{DS220126}. Aufgenommen mit dem Bruker Dimension Icon$^{\text{®}}$ Rastermikroskop im \gls{mfm} Modus. Den kleineren Werten des senkrecht ausgerichteten Anteils der Magnetisierung $\hat{\mathbf{m}}$ entsprechen die dunkleren Farben. Den größeren Werten - die helleren. Die Farbskala kann nicht in die absoluten physikalischen Werte umgerechnet werden. Aus diesem Grund sind die Magnetisierungsstärken verschiedener Proben miteinander nicht zu vergleichen.}
%     \label{fig:mfm_amplitude}
% \end{figure}
\begin{figure}[H]
    \centering
    % \subfloat[]{\input{images/mfm/ft/61_real.pgf}\hspace{1mm}\raisebox{0.55in plus -.5\height}{\scalebox{2}{$\stackrel{\mathcal{F}}{\rightarrow}$}}\label{fig:mfm_ds211221}}
    % \subfloat[]{\hspace{1mm}\input{images/mfm/ft/61_ft.pgf}\label{fig:mfm_ds211221_ft}}
    % \hspace{1cm}
    % \subfloat[]{\input{images/mfm/ft/116_real.pgf}\hspace{1mm}\raisebox{0.55in plus -.5\height}{\scalebox{2}{$\stackrel{\mathcal{F}}{\rightarrow}$}}\label{fig:mfm_ds220126}}
    % \subfloat[]{\hspace{1mm}\input{images/mfm/ft/116_ft.pgf}\label{fig:mfm_ds220126_ft}}
    \input{images/mfm/mfm_fourier.pgf}
    \caption{Die aufgenommenen Domänenmuster der Probe (a) \textbf{DS211221} und (c) \textbf{DS220126}, und (b),(d) - die absoluten Beträge der Fourier-Transformierten von gegebenen Mustern in logarithmischer Skala. Die Domänenmuster wurden mit dem Bruker Dimension Icon$^{\text{®}}$ Rastermikroskop im \gls{mfm}-Modus aufgenommen. Die hellere Farbe entspricht dem größeren Signalwert. Die Farbskala kann nicht in die absoluten physikalischen Werte umgerechnet werden. Die absoluten Magnetisierungsstärken verschiedener Proben sind nicht miteinander vergleichbar.}
    \label{fig:mfm-amplitude-ft}
\end{figure}
\noindent
Es ist leicht zu sehen, dass die Domänenbreite der Probe \textbf{DS211221} in Abb. \ref{fig:mfm-amplitude-ft}a wesentlich größer als die Domänenbreite der Probe \textbf{DS220126} in Abb. \ref{fig:mfm-amplitude-ft}c ist. Darüber hinaus ist die Periodizität von Domänen in der Probe \textbf{DS211221} deutlich niedriger als in der Probe  \textbf{DS220126}. Das spiegelt sich in den Fourier-Transformierten der Domänenmuster beider Proben (Abb. \ref{fig:mfm-amplitude-ft}b und \ref{fig:mfm-amplitude-ft}d) wider, die proportional dem $S_m(\mathbf{q)}$ sind. In der Fourier-Transformierten des Domänenmusters von \textbf{DS220126} ist ein viel schärferes Ringmuster zu sehen, das im Endeffekt bei dem Streuexperiment an der gegebenen Probe zu erwarten ist. Die Domänenbreite der Probe \textbf{DS220126} lässt sich mit \SI{300}{\nano\meter} abschätzen, die ca. \SI{20}{\per\micro\meter} im reziproken Raum entsprechen.

\noindent
Für das Streuexperiment wurde die Probe \textbf{DS220126} wegen der stabilen Domänenperiodizität gewählt, obwohl sie der anderen Probe in der erwarteten Transmissionsrate $\text{T}$ und Gütezahl TP$^2$ unterlegen ist. Für die Probe wurde zusätzlich eine Hysteresekurve mit Hilfe des \gls{moke} aufgenommen. Das Messgerät bietet keinen Zugang zur Domänenstruktur, da weder eine optische Abbildung der Probe gemacht wird, noch die Strahlgröße klein genug ist, um die \si{\micro\meter}-Strukturen aufzulösen. Die Hysteresekurve $\chi(\mu_0 H)$ in Abb. \ref{fig:hysterese_sample} lässt anhand ihres Kurvenverlaufs - ohne nachweisbare Remanenz - erwarten, dass die Domänen in der Probe ohne äußeres magnetisches Feld präsent sind. Die Probe ist beginnend mit dem Betrag des äußeren magnetischen Feldes von $\SI{120}{\milli\tesla}$ gesättigt. So sollten beim Anlegen solcher magnetischer Felder keine Domänen mehr in der Probe existieren.
\begin{figure}[H]
    \centering
    \input{images/moke/ds220107_R1_moke.pgf}
    \caption{Die Hysteresekurve der Probe \textbf{DS220126}, aufgenommen mit \gls{moke}. Das magnetische Feld $\mu_0 H$ wird senkrecht zu der Probenfläche angelegt.}
    \label{fig:hysterese_sample}
\end{figure}
\newpage


% 78-85 der Zeilenumbruch rund um die Abbildung ist in der PDF nicht so gut gelungen. Zur Abbildung: ist es wirklich 2Theta oder nur eins?
% 128-136 Der Satz ist irgendwie nicht schön um die Gleichung herum gelegt
% 154 ich hoffe, die "Anisotropie" war richtig geraten?
% 167 "wobei die Bahnmomenten, sowohl die Spinmomenten aufgrund der Spin-Bahn-Kopplung, an der Grenzschicht senkrecht zur Ebene ausrichten." Gegenvorschlag: "wobei sich die Bahnmomente, sowohl die Spinmomente aufgrund [...], ALS AUCH [...], an der Grenzschicht [...]
% 183 ungefähr mittig: "wird später in Kapitel 3 motiviert" thematisiert statt motiviert?
% 183 Hat das SiN-Substrat wirklich 200nm (nicht eher µm)? Auch in Abb. 3 stehts als nm, ebenso in Zeile 201
% 201 "Mittelwert" ist das falsche Wort; da hätten es 62 Schichten sein müssen. Mir ist schon klar, dass ihr die Anzahl zugunsten TP2 in Richtung Gd verschoben habt... aber... keine Ahnung, Wortfindungsschwierigkeiten
% 226 "sowie (b),(d) die absoluten Beträge der Fourier-Transformierten in logarithmischer Darstellung von gegebenen Mustern" Der Sinn der letzten drei Worte erschließt sich mir nicht ganz
% 233 Bin nicht sicher, ob ich den Satz richtig interpretiert habe
