\newacronym{qe}{QE}{Quanteneffizienz}
\newacronym{fdpa}{FDPA}{fehldetektierten Photonen pro Pixel pro Aufnahme}
\newacronym{photnenfluss}{PF}{Photonenfluss}
\newacronym[user1=\emph{engl. „region of interest“}]{roi}{ROI}{Bereiche von Interesse}
\chapter{Ergebnisse}
\label{text:auswertung}
Der zentrale Aspekt dieser Arbeit war die Durchführung und Auswertung eines Laborversuches zur resonanten Röntgenkleinwinkelstreuung an einer Probe, die magnetischen Domänen mit einer Größe im Nanometerbereich aufweist. Die weiche Röntgenstrahlung wird von der bereits beschriebenen PXS geliefert. Die Vorbereitungen zu diesem Versuch wurden in den vorherigen Kapiteln im Detail beschrieben. Zusammengefasst beinhalteten diese Arbeiten:
\begin{itemize}
    \item Herstellung und Charakterisierung einer geeigneten Probe
    \item Charakterisierung des MÖNCH-Detektor-Moduls
    \item Installation des Detektors am Diffraktionsinstrument an der PXS
    \item Elektronische Synchronisation des Detektors mit der Quelle und Einbinden des Detektors in die Datenerfassung am Instrument
    \item Erstellen von Analysesoftware zur Auswertung der Streumuster.
\end{itemize}

\noindent
Die Streuung wird an der Fe/Gd-Multilagenprobe mit einer Breite der magnetischen Domänen von ca.\ \SI{300}{\nano\meter} beobachtet. Der MÖNCH-Detektor wird in \SI{607(7)}{\milli\meter} Abstand von der Probe in Transmissionsgeometrie befestigt. Mit der \qtyproduct{10 x 10}{\milli\meter} Sensorgröße kann der maximale Streuwinkel $2\theta = \SI{0.47}{\degree}$ detektiert werden. Dem maximalen Streuwinkel entspricht der Betrag des Streuvektors \SI{99}{\per\micro\meter}, wenn Photonen mit der Wellenlänge $\lambda_\text{Gd, M5} = \SI{1.045}{\nano\meter}$ gestreut werden. So kann das erwartete ringförmige Streumuster der Probe mit Radius \SI{20}{\per\micro\meter}, der sich als die Fourier-Transformierte des Domänenmusters ergibt, in dem vorhanden experimentellen Aufbau beobachtet werden.

\noindent
Im Laufe des Experiments werden zuerst \num{10000} Dunkelbilder mit dem MÖNCH-Detektor aufgenommen, die zur Bestimmung des konstanten Offsets jedes Pixels und der mittleren Standardabweichung des Detektorrauschens benutzt werden. Als Nächstes wird das Absorptionsspektrum aufgenommen, um die Korrelation zwischen der Kippung der Reflexionszonenplatte und der auf dem Detektor abgebildeten Photonenenergie festzustellen. Im Anschluss wird der Versuch der Detektion der magnetischen Streuung an der resonanten und nicht-resonanten Photonenenergie durchgeführt.   

\noindent
Die Auswertung der Messreihen wird durch die hohe Zahl an Aufnahmen erschwert. So nehmen \num{50000} Aufnahmen, wie sie für ein Streubild mit hohem SNR benötigt werden, insgesamt ca.\ \qty{180}{\giga\byte} an Speicherplatz ein. Jede Aufnahme wird einzeln, also von anderen Aufnahmen unabhängig, ausgewertet. So ist der Auswertungsvorgang leicht parallelisierbar. Benutzt wird die high-level API Bibliothek \textit{dask-image} \cite{dask-library}, die die parallelisierte und optimierte Ausführung der eingegebenen Funktionen ohne großen technischen Aufwand ermöglicht.

\section{Dunkelbild-Analyse}
Die \num{10000} aufgenommenen Dunkelbilder werden gemittelt und als ein einzelnes Bild gespeichert. Die Mittelung verringert das Rauschen um einen Faktor $\sqrt{10000} = 100$. So wird im gemittelten Bild im Wesentlichen nur der zeitlich konstante Offset in jedem Pixel behalten. Das gemittelte Bild ist in Abb.~\ref{fig:capture_ped_diff}b dargestellt und wird vor der weiteren Analyse von jeder einzelnen Aufnahme subtrahiert.

\noindent
Exemplarisch soll hier die Rauschcharakteristik des Detektors für die im Experiment gewählte Belichtungszeit $\tau = \SI{1}{\micro\second}$ gezeigt werden. Dazu wird das gemittelte Bild von jedem der \num{10000} Dunkelbilder subtrahiert. Die Pixelwerte, die sich als Differenzen ergeben, werden in ein Histogramm eingetragen und diese Verteilung wird mit der Funktion
\begin{equation}
    G(W, \mu, \sigma_R, A) = \frac{A}{\sqrt{2\pi \sigma_R^2}}\exp\left[-\frac{(W - \mu)^2}{2\sigma_R^2}\right]
    \label{eq:gauss_funktion}
\end{equation}
angepasst, welche eine auf die Fläche $A$ normierte Gauß-Funktion mit dem Mittelwert $\mu$ und Standardabweichung $\sigma_R$ ist.
\begin{figure}[H]
    \centering
    \input{images/auswertung/noise_hist_fit.pgf}
    \caption{Histogramm von \num{10000} Dunkelbildern, das mit Fit $G(W,\mu,\sigma, A)$ mit $\mu= \SI{-0.32(2)}{\adu}$, $\sigma_R = \SI{19.9(1)}{\adu}$ und $A = \num{1.59e9}$ angepasst wird.}
    \label{fig:noise_hist_fit}
\end{figure}
\noindent
So wurde die Standardabweichung des Rauschens
\begin{equation}
    \sigma_R = \SI{19.9(1)}{\adu}
    \label{eq:sigma_r}
\end{equation}
ermittelt und wird in der weiteren Analyse benutzt.

\section{Energiekalibration}
Zuerst muss die Photonenenergie kalibriert werden. Die \gls{rzp} dispergiert das \gls{pxs} Spektrum entlang der vertikalen Richtung. Das Einstellen der Photonenenergie am Experiment erfolgt über eine Kippung der \gls{rzp} mithilfe eines Schrittmotors, der mit dem Namen $\varphi_\text{\gls{rzp}}$ bezeichnet wird. Die Motorpositionen sind in Schritten angegeben.

\noindent
Um die Motorpositionen dem auf die Probe abgebildeten Photonenenergiebereich zuzuordnen, wird das Absorptionsspektrum der Probe über die Motorposition $\varphi_\text{\gls{rzp}}$ im Intervall von \num{-115} bis \num{-50} Schritten aufgenommen. Diese Messung wird mit einer CCD-Kamera durchgeführt, die bereits in die Labor-Steuerungssoftware integriert ist und somit die Aufnahmen ermöglicht, die mit der Veränderung der Motorposition synchronisiert sind.

\noindent
Zu jeder Motorposition wird eine Aufnahme gemacht. In jeder Aufnahme werden zwei rechteckige \gls{roi} festgelegt. Die Höhe der ersten \gls{roi} ist möglichst klein gewählt, um eine möglichst hohe Energieauflösung von ca.\ \SI{1}{\eV} zu erhalten. Die zweite \gls{roi} deckt hingegen den ganzen Strahl ab, um die Intensität in dem ersten \gls{roi} zu normieren. 

\noindent
Das gemessene Absorptionsspektrum wird mit dem Referenzabsorptionsspektrum von Gd im weichen Röntgenstrahlungsbereich \cite[Abb. 2]{prieto-x-ray-2005} verglichen. In dem Energieintervall nimmt das Spektrum zwei Maxima an den Gd M4 und Gd M5 Absorptionslinien an. Der Absoprtionskoeffzient für linear polarisierte Strahlung $\bar{\beta}$ ergibt sich als Mittelwert von Absoprtionskoeffzienten für links bzw. rechts zirkular polarisierte Strahlung.

\noindent
Die Energieachse wird linear zur Motorposition $\varphi_\text{\gls{rzp}}$ angepasst, indem die Absorptionslinien Gd M5 und Gd M4 den beiden Peaks im gemessenen Spektrum zugeordnet werden. Das gemessene Spektrum und die Referenz werden in Abb.~\ref{fig:rzp_phi_ev} über der Energieachse aufgetragen.
\begin{figure}[H]
    \centering
    %% Creator: Matplotlib, PGF backend
%%
%% To include the figure in your LaTeX document, write
%%   \input{<filename>.pgf}
%%
%% Make sure the required packages are loaded in your preamble
%%   \usepackage{pgf}
%%
%% Also ensure that all the required font packages are loaded; for instance,
%% the lmodern package is sometimes necessary when using math font.
%%   \usepackage{lmodern}
%%
%% Figures using additional raster images can only be included by \input if
%% they are in the same directory as the main LaTeX file. For loading figures
%% from other directories you can use the `import` package
%%   \usepackage{import}
%%
%% and then include the figures with
%%   \import{<path to file>}{<filename>.pgf}
%%
%% Matplotlib used the following preamble
%%   \usepackage{amsmath} \usepackage[utf8]{inputenc} \usepackage[T1]{fontenc} \usepackage[output-decimal-marker={,},print-unity-mantissa=false]{siunitx} \sisetup{per-mode=fraction, separate-uncertainty = true, locale = DE} \usepackage[acronym, toc, section=section, nonumberlist, nopostdot]{glossaries-extra} \DeclareSIUnit\adu{\text{ADU}} \DeclareSIUnit\px{\text{px}} \DeclareSIUnit\photons{\text{Pho\-to\-nen}} \DeclareSIUnit\photon{\text{Pho\-ton}}
%%
\begingroup%
\makeatletter%
\begin{pgfpicture}%
\pgfpathrectangle{\pgfpointorigin}{\pgfqpoint{6.181121in}{3.201735in}}%
\pgfusepath{use as bounding box, clip}%
\begin{pgfscope}%
\pgfsetbuttcap%
\pgfsetmiterjoin%
\pgfsetlinewidth{0.000000pt}%
\definecolor{currentstroke}{rgb}{1.000000,1.000000,1.000000}%
\pgfsetstrokecolor{currentstroke}%
\pgfsetstrokeopacity{0.000000}%
\pgfsetdash{}{0pt}%
\pgfpathmoveto{\pgfqpoint{0.000000in}{0.000000in}}%
\pgfpathlineto{\pgfqpoint{6.181121in}{0.000000in}}%
\pgfpathlineto{\pgfqpoint{6.181121in}{3.201735in}}%
\pgfpathlineto{\pgfqpoint{0.000000in}{3.201735in}}%
\pgfpathlineto{\pgfqpoint{0.000000in}{0.000000in}}%
\pgfpathclose%
\pgfusepath{}%
\end{pgfscope}%
\begin{pgfscope}%
\pgfsetbuttcap%
\pgfsetmiterjoin%
\definecolor{currentfill}{rgb}{1.000000,1.000000,1.000000}%
\pgfsetfillcolor{currentfill}%
\pgfsetlinewidth{0.000000pt}%
\definecolor{currentstroke}{rgb}{0.000000,0.000000,0.000000}%
\pgfsetstrokecolor{currentstroke}%
\pgfsetstrokeopacity{0.000000}%
\pgfsetdash{}{0pt}%
\pgfpathmoveto{\pgfqpoint{0.452903in}{0.398883in}}%
\pgfpathlineto{\pgfqpoint{6.181121in}{0.398883in}}%
\pgfpathlineto{\pgfqpoint{6.181121in}{2.830646in}}%
\pgfpathlineto{\pgfqpoint{0.452903in}{2.830646in}}%
\pgfpathlineto{\pgfqpoint{0.452903in}{0.398883in}}%
\pgfpathclose%
\pgfusepath{fill}%
\end{pgfscope}%
\begin{pgfscope}%
\pgfsetbuttcap%
\pgfsetmiterjoin%
\definecolor{currentfill}{rgb}{1.000000,1.000000,1.000000}%
\pgfsetfillcolor{currentfill}%
\pgfsetlinewidth{0.000000pt}%
\definecolor{currentstroke}{rgb}{0.000000,0.000000,0.000000}%
\pgfsetstrokecolor{currentstroke}%
\pgfsetstrokeopacity{0.000000}%
\pgfsetdash{}{0pt}%
\pgfpathmoveto{\pgfqpoint{0.452903in}{2.830646in}}%
\pgfpathlineto{\pgfqpoint{6.181121in}{2.830646in}}%
\pgfpathlineto{\pgfqpoint{6.181121in}{2.830646in}}%
\pgfpathlineto{\pgfqpoint{0.452903in}{2.830646in}}%
\pgfpathlineto{\pgfqpoint{0.452903in}{2.830646in}}%
\pgfpathclose%
\pgfusepath{fill}%
\end{pgfscope}%
\begin{pgfscope}%
\pgfsetbuttcap%
\pgfsetroundjoin%
\definecolor{currentfill}{rgb}{0.000000,0.000000,0.000000}%
\pgfsetfillcolor{currentfill}%
\pgfsetlinewidth{0.803000pt}%
\definecolor{currentstroke}{rgb}{0.000000,0.000000,0.000000}%
\pgfsetstrokecolor{currentstroke}%
\pgfsetdash{}{0pt}%
\pgfsys@defobject{currentmarker}{\pgfqpoint{0.000000in}{0.000000in}}{\pgfqpoint{0.000000in}{0.048611in}}{%
\pgfpathmoveto{\pgfqpoint{0.000000in}{0.000000in}}%
\pgfpathlineto{\pgfqpoint{0.000000in}{0.048611in}}%
\pgfusepath{stroke,fill}%
}%
\begin{pgfscope}%
\pgfsys@transformshift{5.433902in}{2.830646in}%
\pgfsys@useobject{currentmarker}{}%
\end{pgfscope}%
\end{pgfscope}%
\begin{pgfscope}%
\definecolor{textcolor}{rgb}{0.000000,0.000000,0.000000}%
\pgfsetstrokecolor{textcolor}%
\pgfsetfillcolor{textcolor}%
\pgftext[x=5.433902in,y=2.927868in,,bottom]{\color{textcolor}\rmfamily\fontsize{10.000000}{12.000000}\selectfont \(\displaystyle {\ensuremath{-}90}\)}%
\end{pgfscope}%
\begin{pgfscope}%
\pgfsetbuttcap%
\pgfsetroundjoin%
\definecolor{currentfill}{rgb}{0.000000,0.000000,0.000000}%
\pgfsetfillcolor{currentfill}%
\pgfsetlinewidth{0.803000pt}%
\definecolor{currentstroke}{rgb}{0.000000,0.000000,0.000000}%
\pgfsetstrokecolor{currentstroke}%
\pgfsetdash{}{0pt}%
\pgfsys@defobject{currentmarker}{\pgfqpoint{0.000000in}{0.000000in}}{\pgfqpoint{0.000000in}{0.048611in}}{%
\pgfpathmoveto{\pgfqpoint{0.000000in}{0.000000in}}%
\pgfpathlineto{\pgfqpoint{0.000000in}{0.048611in}}%
\pgfusepath{stroke,fill}%
}%
\begin{pgfscope}%
\pgfsys@transformshift{4.188728in}{2.830646in}%
\pgfsys@useobject{currentmarker}{}%
\end{pgfscope}%
\end{pgfscope}%
\begin{pgfscope}%
\definecolor{textcolor}{rgb}{0.000000,0.000000,0.000000}%
\pgfsetstrokecolor{textcolor}%
\pgfsetfillcolor{textcolor}%
\pgftext[x=4.188728in,y=2.927868in,,bottom]{\color{textcolor}\rmfamily\fontsize{10.000000}{12.000000}\selectfont \(\displaystyle {\ensuremath{-}80}\)}%
\end{pgfscope}%
\begin{pgfscope}%
\pgfsetbuttcap%
\pgfsetroundjoin%
\definecolor{currentfill}{rgb}{0.000000,0.000000,0.000000}%
\pgfsetfillcolor{currentfill}%
\pgfsetlinewidth{0.803000pt}%
\definecolor{currentstroke}{rgb}{0.000000,0.000000,0.000000}%
\pgfsetstrokecolor{currentstroke}%
\pgfsetdash{}{0pt}%
\pgfsys@defobject{currentmarker}{\pgfqpoint{0.000000in}{0.000000in}}{\pgfqpoint{0.000000in}{0.048611in}}{%
\pgfpathmoveto{\pgfqpoint{0.000000in}{0.000000in}}%
\pgfpathlineto{\pgfqpoint{0.000000in}{0.048611in}}%
\pgfusepath{stroke,fill}%
}%
\begin{pgfscope}%
\pgfsys@transformshift{2.943553in}{2.830646in}%
\pgfsys@useobject{currentmarker}{}%
\end{pgfscope}%
\end{pgfscope}%
\begin{pgfscope}%
\definecolor{textcolor}{rgb}{0.000000,0.000000,0.000000}%
\pgfsetstrokecolor{textcolor}%
\pgfsetfillcolor{textcolor}%
\pgftext[x=2.943553in,y=2.927868in,,bottom]{\color{textcolor}\rmfamily\fontsize{10.000000}{12.000000}\selectfont \(\displaystyle {\ensuremath{-}70}\)}%
\end{pgfscope}%
\begin{pgfscope}%
\pgfsetbuttcap%
\pgfsetroundjoin%
\definecolor{currentfill}{rgb}{0.000000,0.000000,0.000000}%
\pgfsetfillcolor{currentfill}%
\pgfsetlinewidth{0.803000pt}%
\definecolor{currentstroke}{rgb}{0.000000,0.000000,0.000000}%
\pgfsetstrokecolor{currentstroke}%
\pgfsetdash{}{0pt}%
\pgfsys@defobject{currentmarker}{\pgfqpoint{0.000000in}{0.000000in}}{\pgfqpoint{0.000000in}{0.048611in}}{%
\pgfpathmoveto{\pgfqpoint{0.000000in}{0.000000in}}%
\pgfpathlineto{\pgfqpoint{0.000000in}{0.048611in}}%
\pgfusepath{stroke,fill}%
}%
\begin{pgfscope}%
\pgfsys@transformshift{1.698379in}{2.830646in}%
\pgfsys@useobject{currentmarker}{}%
\end{pgfscope}%
\end{pgfscope}%
\begin{pgfscope}%
\definecolor{textcolor}{rgb}{0.000000,0.000000,0.000000}%
\pgfsetstrokecolor{textcolor}%
\pgfsetfillcolor{textcolor}%
\pgftext[x=1.698379in,y=2.927868in,,bottom]{\color{textcolor}\rmfamily\fontsize{10.000000}{12.000000}\selectfont \(\displaystyle {\ensuremath{-}60}\)}%
\end{pgfscope}%
\begin{pgfscope}%
\pgfsetbuttcap%
\pgfsetroundjoin%
\definecolor{currentfill}{rgb}{0.000000,0.000000,0.000000}%
\pgfsetfillcolor{currentfill}%
\pgfsetlinewidth{0.803000pt}%
\definecolor{currentstroke}{rgb}{0.000000,0.000000,0.000000}%
\pgfsetstrokecolor{currentstroke}%
\pgfsetdash{}{0pt}%
\pgfsys@defobject{currentmarker}{\pgfqpoint{0.000000in}{0.000000in}}{\pgfqpoint{0.000000in}{0.048611in}}{%
\pgfpathmoveto{\pgfqpoint{0.000000in}{0.000000in}}%
\pgfpathlineto{\pgfqpoint{0.000000in}{0.048611in}}%
\pgfusepath{stroke,fill}%
}%
\begin{pgfscope}%
\pgfsys@transformshift{0.453204in}{2.830646in}%
\pgfsys@useobject{currentmarker}{}%
\end{pgfscope}%
\end{pgfscope}%
\begin{pgfscope}%
\definecolor{textcolor}{rgb}{0.000000,0.000000,0.000000}%
\pgfsetstrokecolor{textcolor}%
\pgfsetfillcolor{textcolor}%
\pgftext[x=0.453204in,y=2.927868in,,bottom]{\color{textcolor}\rmfamily\fontsize{10.000000}{12.000000}\selectfont \(\displaystyle {\ensuremath{-}50}\)}%
\end{pgfscope}%
\begin{pgfscope}%
\pgfsetbuttcap%
\pgfsetroundjoin%
\definecolor{currentfill}{rgb}{0.000000,0.000000,0.000000}%
\pgfsetfillcolor{currentfill}%
\pgfsetlinewidth{0.602250pt}%
\definecolor{currentstroke}{rgb}{0.000000,0.000000,0.000000}%
\pgfsetstrokecolor{currentstroke}%
\pgfsetdash{}{0pt}%
\pgfsys@defobject{currentmarker}{\pgfqpoint{0.000000in}{0.000000in}}{\pgfqpoint{0.000000in}{0.027778in}}{%
\pgfpathmoveto{\pgfqpoint{0.000000in}{0.000000in}}%
\pgfpathlineto{\pgfqpoint{0.000000in}{0.027778in}}%
\pgfusepath{stroke,fill}%
}%
\begin{pgfscope}%
\pgfsys@transformshift{6.181007in}{2.830646in}%
\pgfsys@useobject{currentmarker}{}%
\end{pgfscope}%
\end{pgfscope}%
\begin{pgfscope}%
\pgfsetbuttcap%
\pgfsetroundjoin%
\definecolor{currentfill}{rgb}{0.000000,0.000000,0.000000}%
\pgfsetfillcolor{currentfill}%
\pgfsetlinewidth{0.602250pt}%
\definecolor{currentstroke}{rgb}{0.000000,0.000000,0.000000}%
\pgfsetstrokecolor{currentstroke}%
\pgfsetdash{}{0pt}%
\pgfsys@defobject{currentmarker}{\pgfqpoint{0.000000in}{0.000000in}}{\pgfqpoint{0.000000in}{0.027778in}}{%
\pgfpathmoveto{\pgfqpoint{0.000000in}{0.000000in}}%
\pgfpathlineto{\pgfqpoint{0.000000in}{0.027778in}}%
\pgfusepath{stroke,fill}%
}%
\begin{pgfscope}%
\pgfsys@transformshift{6.056490in}{2.830646in}%
\pgfsys@useobject{currentmarker}{}%
\end{pgfscope}%
\end{pgfscope}%
\begin{pgfscope}%
\pgfsetbuttcap%
\pgfsetroundjoin%
\definecolor{currentfill}{rgb}{0.000000,0.000000,0.000000}%
\pgfsetfillcolor{currentfill}%
\pgfsetlinewidth{0.602250pt}%
\definecolor{currentstroke}{rgb}{0.000000,0.000000,0.000000}%
\pgfsetstrokecolor{currentstroke}%
\pgfsetdash{}{0pt}%
\pgfsys@defobject{currentmarker}{\pgfqpoint{0.000000in}{0.000000in}}{\pgfqpoint{0.000000in}{0.027778in}}{%
\pgfpathmoveto{\pgfqpoint{0.000000in}{0.000000in}}%
\pgfpathlineto{\pgfqpoint{0.000000in}{0.027778in}}%
\pgfusepath{stroke,fill}%
}%
\begin{pgfscope}%
\pgfsys@transformshift{5.931972in}{2.830646in}%
\pgfsys@useobject{currentmarker}{}%
\end{pgfscope}%
\end{pgfscope}%
\begin{pgfscope}%
\pgfsetbuttcap%
\pgfsetroundjoin%
\definecolor{currentfill}{rgb}{0.000000,0.000000,0.000000}%
\pgfsetfillcolor{currentfill}%
\pgfsetlinewidth{0.602250pt}%
\definecolor{currentstroke}{rgb}{0.000000,0.000000,0.000000}%
\pgfsetstrokecolor{currentstroke}%
\pgfsetdash{}{0pt}%
\pgfsys@defobject{currentmarker}{\pgfqpoint{0.000000in}{0.000000in}}{\pgfqpoint{0.000000in}{0.027778in}}{%
\pgfpathmoveto{\pgfqpoint{0.000000in}{0.000000in}}%
\pgfpathlineto{\pgfqpoint{0.000000in}{0.027778in}}%
\pgfusepath{stroke,fill}%
}%
\begin{pgfscope}%
\pgfsys@transformshift{5.807455in}{2.830646in}%
\pgfsys@useobject{currentmarker}{}%
\end{pgfscope}%
\end{pgfscope}%
\begin{pgfscope}%
\pgfsetbuttcap%
\pgfsetroundjoin%
\definecolor{currentfill}{rgb}{0.000000,0.000000,0.000000}%
\pgfsetfillcolor{currentfill}%
\pgfsetlinewidth{0.602250pt}%
\definecolor{currentstroke}{rgb}{0.000000,0.000000,0.000000}%
\pgfsetstrokecolor{currentstroke}%
\pgfsetdash{}{0pt}%
\pgfsys@defobject{currentmarker}{\pgfqpoint{0.000000in}{0.000000in}}{\pgfqpoint{0.000000in}{0.027778in}}{%
\pgfpathmoveto{\pgfqpoint{0.000000in}{0.000000in}}%
\pgfpathlineto{\pgfqpoint{0.000000in}{0.027778in}}%
\pgfusepath{stroke,fill}%
}%
\begin{pgfscope}%
\pgfsys@transformshift{5.682937in}{2.830646in}%
\pgfsys@useobject{currentmarker}{}%
\end{pgfscope}%
\end{pgfscope}%
\begin{pgfscope}%
\pgfsetbuttcap%
\pgfsetroundjoin%
\definecolor{currentfill}{rgb}{0.000000,0.000000,0.000000}%
\pgfsetfillcolor{currentfill}%
\pgfsetlinewidth{0.602250pt}%
\definecolor{currentstroke}{rgb}{0.000000,0.000000,0.000000}%
\pgfsetstrokecolor{currentstroke}%
\pgfsetdash{}{0pt}%
\pgfsys@defobject{currentmarker}{\pgfqpoint{0.000000in}{0.000000in}}{\pgfqpoint{0.000000in}{0.027778in}}{%
\pgfpathmoveto{\pgfqpoint{0.000000in}{0.000000in}}%
\pgfpathlineto{\pgfqpoint{0.000000in}{0.027778in}}%
\pgfusepath{stroke,fill}%
}%
\begin{pgfscope}%
\pgfsys@transformshift{5.558420in}{2.830646in}%
\pgfsys@useobject{currentmarker}{}%
\end{pgfscope}%
\end{pgfscope}%
\begin{pgfscope}%
\pgfsetbuttcap%
\pgfsetroundjoin%
\definecolor{currentfill}{rgb}{0.000000,0.000000,0.000000}%
\pgfsetfillcolor{currentfill}%
\pgfsetlinewidth{0.602250pt}%
\definecolor{currentstroke}{rgb}{0.000000,0.000000,0.000000}%
\pgfsetstrokecolor{currentstroke}%
\pgfsetdash{}{0pt}%
\pgfsys@defobject{currentmarker}{\pgfqpoint{0.000000in}{0.000000in}}{\pgfqpoint{0.000000in}{0.027778in}}{%
\pgfpathmoveto{\pgfqpoint{0.000000in}{0.000000in}}%
\pgfpathlineto{\pgfqpoint{0.000000in}{0.027778in}}%
\pgfusepath{stroke,fill}%
}%
\begin{pgfscope}%
\pgfsys@transformshift{5.309385in}{2.830646in}%
\pgfsys@useobject{currentmarker}{}%
\end{pgfscope}%
\end{pgfscope}%
\begin{pgfscope}%
\pgfsetbuttcap%
\pgfsetroundjoin%
\definecolor{currentfill}{rgb}{0.000000,0.000000,0.000000}%
\pgfsetfillcolor{currentfill}%
\pgfsetlinewidth{0.602250pt}%
\definecolor{currentstroke}{rgb}{0.000000,0.000000,0.000000}%
\pgfsetstrokecolor{currentstroke}%
\pgfsetdash{}{0pt}%
\pgfsys@defobject{currentmarker}{\pgfqpoint{0.000000in}{0.000000in}}{\pgfqpoint{0.000000in}{0.027778in}}{%
\pgfpathmoveto{\pgfqpoint{0.000000in}{0.000000in}}%
\pgfpathlineto{\pgfqpoint{0.000000in}{0.027778in}}%
\pgfusepath{stroke,fill}%
}%
\begin{pgfscope}%
\pgfsys@transformshift{5.184867in}{2.830646in}%
\pgfsys@useobject{currentmarker}{}%
\end{pgfscope}%
\end{pgfscope}%
\begin{pgfscope}%
\pgfsetbuttcap%
\pgfsetroundjoin%
\definecolor{currentfill}{rgb}{0.000000,0.000000,0.000000}%
\pgfsetfillcolor{currentfill}%
\pgfsetlinewidth{0.602250pt}%
\definecolor{currentstroke}{rgb}{0.000000,0.000000,0.000000}%
\pgfsetstrokecolor{currentstroke}%
\pgfsetdash{}{0pt}%
\pgfsys@defobject{currentmarker}{\pgfqpoint{0.000000in}{0.000000in}}{\pgfqpoint{0.000000in}{0.027778in}}{%
\pgfpathmoveto{\pgfqpoint{0.000000in}{0.000000in}}%
\pgfpathlineto{\pgfqpoint{0.000000in}{0.027778in}}%
\pgfusepath{stroke,fill}%
}%
\begin{pgfscope}%
\pgfsys@transformshift{5.060350in}{2.830646in}%
\pgfsys@useobject{currentmarker}{}%
\end{pgfscope}%
\end{pgfscope}%
\begin{pgfscope}%
\pgfsetbuttcap%
\pgfsetroundjoin%
\definecolor{currentfill}{rgb}{0.000000,0.000000,0.000000}%
\pgfsetfillcolor{currentfill}%
\pgfsetlinewidth{0.602250pt}%
\definecolor{currentstroke}{rgb}{0.000000,0.000000,0.000000}%
\pgfsetstrokecolor{currentstroke}%
\pgfsetdash{}{0pt}%
\pgfsys@defobject{currentmarker}{\pgfqpoint{0.000000in}{0.000000in}}{\pgfqpoint{0.000000in}{0.027778in}}{%
\pgfpathmoveto{\pgfqpoint{0.000000in}{0.000000in}}%
\pgfpathlineto{\pgfqpoint{0.000000in}{0.027778in}}%
\pgfusepath{stroke,fill}%
}%
\begin{pgfscope}%
\pgfsys@transformshift{4.935832in}{2.830646in}%
\pgfsys@useobject{currentmarker}{}%
\end{pgfscope}%
\end{pgfscope}%
\begin{pgfscope}%
\pgfsetbuttcap%
\pgfsetroundjoin%
\definecolor{currentfill}{rgb}{0.000000,0.000000,0.000000}%
\pgfsetfillcolor{currentfill}%
\pgfsetlinewidth{0.602250pt}%
\definecolor{currentstroke}{rgb}{0.000000,0.000000,0.000000}%
\pgfsetstrokecolor{currentstroke}%
\pgfsetdash{}{0pt}%
\pgfsys@defobject{currentmarker}{\pgfqpoint{0.000000in}{0.000000in}}{\pgfqpoint{0.000000in}{0.027778in}}{%
\pgfpathmoveto{\pgfqpoint{0.000000in}{0.000000in}}%
\pgfpathlineto{\pgfqpoint{0.000000in}{0.027778in}}%
\pgfusepath{stroke,fill}%
}%
\begin{pgfscope}%
\pgfsys@transformshift{4.811315in}{2.830646in}%
\pgfsys@useobject{currentmarker}{}%
\end{pgfscope}%
\end{pgfscope}%
\begin{pgfscope}%
\pgfsetbuttcap%
\pgfsetroundjoin%
\definecolor{currentfill}{rgb}{0.000000,0.000000,0.000000}%
\pgfsetfillcolor{currentfill}%
\pgfsetlinewidth{0.602250pt}%
\definecolor{currentstroke}{rgb}{0.000000,0.000000,0.000000}%
\pgfsetstrokecolor{currentstroke}%
\pgfsetdash{}{0pt}%
\pgfsys@defobject{currentmarker}{\pgfqpoint{0.000000in}{0.000000in}}{\pgfqpoint{0.000000in}{0.027778in}}{%
\pgfpathmoveto{\pgfqpoint{0.000000in}{0.000000in}}%
\pgfpathlineto{\pgfqpoint{0.000000in}{0.027778in}}%
\pgfusepath{stroke,fill}%
}%
\begin{pgfscope}%
\pgfsys@transformshift{4.686798in}{2.830646in}%
\pgfsys@useobject{currentmarker}{}%
\end{pgfscope}%
\end{pgfscope}%
\begin{pgfscope}%
\pgfsetbuttcap%
\pgfsetroundjoin%
\definecolor{currentfill}{rgb}{0.000000,0.000000,0.000000}%
\pgfsetfillcolor{currentfill}%
\pgfsetlinewidth{0.602250pt}%
\definecolor{currentstroke}{rgb}{0.000000,0.000000,0.000000}%
\pgfsetstrokecolor{currentstroke}%
\pgfsetdash{}{0pt}%
\pgfsys@defobject{currentmarker}{\pgfqpoint{0.000000in}{0.000000in}}{\pgfqpoint{0.000000in}{0.027778in}}{%
\pgfpathmoveto{\pgfqpoint{0.000000in}{0.000000in}}%
\pgfpathlineto{\pgfqpoint{0.000000in}{0.027778in}}%
\pgfusepath{stroke,fill}%
}%
\begin{pgfscope}%
\pgfsys@transformshift{4.562280in}{2.830646in}%
\pgfsys@useobject{currentmarker}{}%
\end{pgfscope}%
\end{pgfscope}%
\begin{pgfscope}%
\pgfsetbuttcap%
\pgfsetroundjoin%
\definecolor{currentfill}{rgb}{0.000000,0.000000,0.000000}%
\pgfsetfillcolor{currentfill}%
\pgfsetlinewidth{0.602250pt}%
\definecolor{currentstroke}{rgb}{0.000000,0.000000,0.000000}%
\pgfsetstrokecolor{currentstroke}%
\pgfsetdash{}{0pt}%
\pgfsys@defobject{currentmarker}{\pgfqpoint{0.000000in}{0.000000in}}{\pgfqpoint{0.000000in}{0.027778in}}{%
\pgfpathmoveto{\pgfqpoint{0.000000in}{0.000000in}}%
\pgfpathlineto{\pgfqpoint{0.000000in}{0.027778in}}%
\pgfusepath{stroke,fill}%
}%
\begin{pgfscope}%
\pgfsys@transformshift{4.437763in}{2.830646in}%
\pgfsys@useobject{currentmarker}{}%
\end{pgfscope}%
\end{pgfscope}%
\begin{pgfscope}%
\pgfsetbuttcap%
\pgfsetroundjoin%
\definecolor{currentfill}{rgb}{0.000000,0.000000,0.000000}%
\pgfsetfillcolor{currentfill}%
\pgfsetlinewidth{0.602250pt}%
\definecolor{currentstroke}{rgb}{0.000000,0.000000,0.000000}%
\pgfsetstrokecolor{currentstroke}%
\pgfsetdash{}{0pt}%
\pgfsys@defobject{currentmarker}{\pgfqpoint{0.000000in}{0.000000in}}{\pgfqpoint{0.000000in}{0.027778in}}{%
\pgfpathmoveto{\pgfqpoint{0.000000in}{0.000000in}}%
\pgfpathlineto{\pgfqpoint{0.000000in}{0.027778in}}%
\pgfusepath{stroke,fill}%
}%
\begin{pgfscope}%
\pgfsys@transformshift{4.313245in}{2.830646in}%
\pgfsys@useobject{currentmarker}{}%
\end{pgfscope}%
\end{pgfscope}%
\begin{pgfscope}%
\pgfsetbuttcap%
\pgfsetroundjoin%
\definecolor{currentfill}{rgb}{0.000000,0.000000,0.000000}%
\pgfsetfillcolor{currentfill}%
\pgfsetlinewidth{0.602250pt}%
\definecolor{currentstroke}{rgb}{0.000000,0.000000,0.000000}%
\pgfsetstrokecolor{currentstroke}%
\pgfsetdash{}{0pt}%
\pgfsys@defobject{currentmarker}{\pgfqpoint{0.000000in}{0.000000in}}{\pgfqpoint{0.000000in}{0.027778in}}{%
\pgfpathmoveto{\pgfqpoint{0.000000in}{0.000000in}}%
\pgfpathlineto{\pgfqpoint{0.000000in}{0.027778in}}%
\pgfusepath{stroke,fill}%
}%
\begin{pgfscope}%
\pgfsys@transformshift{4.064210in}{2.830646in}%
\pgfsys@useobject{currentmarker}{}%
\end{pgfscope}%
\end{pgfscope}%
\begin{pgfscope}%
\pgfsetbuttcap%
\pgfsetroundjoin%
\definecolor{currentfill}{rgb}{0.000000,0.000000,0.000000}%
\pgfsetfillcolor{currentfill}%
\pgfsetlinewidth{0.602250pt}%
\definecolor{currentstroke}{rgb}{0.000000,0.000000,0.000000}%
\pgfsetstrokecolor{currentstroke}%
\pgfsetdash{}{0pt}%
\pgfsys@defobject{currentmarker}{\pgfqpoint{0.000000in}{0.000000in}}{\pgfqpoint{0.000000in}{0.027778in}}{%
\pgfpathmoveto{\pgfqpoint{0.000000in}{0.000000in}}%
\pgfpathlineto{\pgfqpoint{0.000000in}{0.027778in}}%
\pgfusepath{stroke,fill}%
}%
\begin{pgfscope}%
\pgfsys@transformshift{3.939693in}{2.830646in}%
\pgfsys@useobject{currentmarker}{}%
\end{pgfscope}%
\end{pgfscope}%
\begin{pgfscope}%
\pgfsetbuttcap%
\pgfsetroundjoin%
\definecolor{currentfill}{rgb}{0.000000,0.000000,0.000000}%
\pgfsetfillcolor{currentfill}%
\pgfsetlinewidth{0.602250pt}%
\definecolor{currentstroke}{rgb}{0.000000,0.000000,0.000000}%
\pgfsetstrokecolor{currentstroke}%
\pgfsetdash{}{0pt}%
\pgfsys@defobject{currentmarker}{\pgfqpoint{0.000000in}{0.000000in}}{\pgfqpoint{0.000000in}{0.027778in}}{%
\pgfpathmoveto{\pgfqpoint{0.000000in}{0.000000in}}%
\pgfpathlineto{\pgfqpoint{0.000000in}{0.027778in}}%
\pgfusepath{stroke,fill}%
}%
\begin{pgfscope}%
\pgfsys@transformshift{3.815175in}{2.830646in}%
\pgfsys@useobject{currentmarker}{}%
\end{pgfscope}%
\end{pgfscope}%
\begin{pgfscope}%
\pgfsetbuttcap%
\pgfsetroundjoin%
\definecolor{currentfill}{rgb}{0.000000,0.000000,0.000000}%
\pgfsetfillcolor{currentfill}%
\pgfsetlinewidth{0.602250pt}%
\definecolor{currentstroke}{rgb}{0.000000,0.000000,0.000000}%
\pgfsetstrokecolor{currentstroke}%
\pgfsetdash{}{0pt}%
\pgfsys@defobject{currentmarker}{\pgfqpoint{0.000000in}{0.000000in}}{\pgfqpoint{0.000000in}{0.027778in}}{%
\pgfpathmoveto{\pgfqpoint{0.000000in}{0.000000in}}%
\pgfpathlineto{\pgfqpoint{0.000000in}{0.027778in}}%
\pgfusepath{stroke,fill}%
}%
\begin{pgfscope}%
\pgfsys@transformshift{3.690658in}{2.830646in}%
\pgfsys@useobject{currentmarker}{}%
\end{pgfscope}%
\end{pgfscope}%
\begin{pgfscope}%
\pgfsetbuttcap%
\pgfsetroundjoin%
\definecolor{currentfill}{rgb}{0.000000,0.000000,0.000000}%
\pgfsetfillcolor{currentfill}%
\pgfsetlinewidth{0.602250pt}%
\definecolor{currentstroke}{rgb}{0.000000,0.000000,0.000000}%
\pgfsetstrokecolor{currentstroke}%
\pgfsetdash{}{0pt}%
\pgfsys@defobject{currentmarker}{\pgfqpoint{0.000000in}{0.000000in}}{\pgfqpoint{0.000000in}{0.027778in}}{%
\pgfpathmoveto{\pgfqpoint{0.000000in}{0.000000in}}%
\pgfpathlineto{\pgfqpoint{0.000000in}{0.027778in}}%
\pgfusepath{stroke,fill}%
}%
\begin{pgfscope}%
\pgfsys@transformshift{3.566140in}{2.830646in}%
\pgfsys@useobject{currentmarker}{}%
\end{pgfscope}%
\end{pgfscope}%
\begin{pgfscope}%
\pgfsetbuttcap%
\pgfsetroundjoin%
\definecolor{currentfill}{rgb}{0.000000,0.000000,0.000000}%
\pgfsetfillcolor{currentfill}%
\pgfsetlinewidth{0.602250pt}%
\definecolor{currentstroke}{rgb}{0.000000,0.000000,0.000000}%
\pgfsetstrokecolor{currentstroke}%
\pgfsetdash{}{0pt}%
\pgfsys@defobject{currentmarker}{\pgfqpoint{0.000000in}{0.000000in}}{\pgfqpoint{0.000000in}{0.027778in}}{%
\pgfpathmoveto{\pgfqpoint{0.000000in}{0.000000in}}%
\pgfpathlineto{\pgfqpoint{0.000000in}{0.027778in}}%
\pgfusepath{stroke,fill}%
}%
\begin{pgfscope}%
\pgfsys@transformshift{3.441623in}{2.830646in}%
\pgfsys@useobject{currentmarker}{}%
\end{pgfscope}%
\end{pgfscope}%
\begin{pgfscope}%
\pgfsetbuttcap%
\pgfsetroundjoin%
\definecolor{currentfill}{rgb}{0.000000,0.000000,0.000000}%
\pgfsetfillcolor{currentfill}%
\pgfsetlinewidth{0.602250pt}%
\definecolor{currentstroke}{rgb}{0.000000,0.000000,0.000000}%
\pgfsetstrokecolor{currentstroke}%
\pgfsetdash{}{0pt}%
\pgfsys@defobject{currentmarker}{\pgfqpoint{0.000000in}{0.000000in}}{\pgfqpoint{0.000000in}{0.027778in}}{%
\pgfpathmoveto{\pgfqpoint{0.000000in}{0.000000in}}%
\pgfpathlineto{\pgfqpoint{0.000000in}{0.027778in}}%
\pgfusepath{stroke,fill}%
}%
\begin{pgfscope}%
\pgfsys@transformshift{3.317106in}{2.830646in}%
\pgfsys@useobject{currentmarker}{}%
\end{pgfscope}%
\end{pgfscope}%
\begin{pgfscope}%
\pgfsetbuttcap%
\pgfsetroundjoin%
\definecolor{currentfill}{rgb}{0.000000,0.000000,0.000000}%
\pgfsetfillcolor{currentfill}%
\pgfsetlinewidth{0.602250pt}%
\definecolor{currentstroke}{rgb}{0.000000,0.000000,0.000000}%
\pgfsetstrokecolor{currentstroke}%
\pgfsetdash{}{0pt}%
\pgfsys@defobject{currentmarker}{\pgfqpoint{0.000000in}{0.000000in}}{\pgfqpoint{0.000000in}{0.027778in}}{%
\pgfpathmoveto{\pgfqpoint{0.000000in}{0.000000in}}%
\pgfpathlineto{\pgfqpoint{0.000000in}{0.027778in}}%
\pgfusepath{stroke,fill}%
}%
\begin{pgfscope}%
\pgfsys@transformshift{3.192588in}{2.830646in}%
\pgfsys@useobject{currentmarker}{}%
\end{pgfscope}%
\end{pgfscope}%
\begin{pgfscope}%
\pgfsetbuttcap%
\pgfsetroundjoin%
\definecolor{currentfill}{rgb}{0.000000,0.000000,0.000000}%
\pgfsetfillcolor{currentfill}%
\pgfsetlinewidth{0.602250pt}%
\definecolor{currentstroke}{rgb}{0.000000,0.000000,0.000000}%
\pgfsetstrokecolor{currentstroke}%
\pgfsetdash{}{0pt}%
\pgfsys@defobject{currentmarker}{\pgfqpoint{0.000000in}{0.000000in}}{\pgfqpoint{0.000000in}{0.027778in}}{%
\pgfpathmoveto{\pgfqpoint{0.000000in}{0.000000in}}%
\pgfpathlineto{\pgfqpoint{0.000000in}{0.027778in}}%
\pgfusepath{stroke,fill}%
}%
\begin{pgfscope}%
\pgfsys@transformshift{3.068071in}{2.830646in}%
\pgfsys@useobject{currentmarker}{}%
\end{pgfscope}%
\end{pgfscope}%
\begin{pgfscope}%
\pgfsetbuttcap%
\pgfsetroundjoin%
\definecolor{currentfill}{rgb}{0.000000,0.000000,0.000000}%
\pgfsetfillcolor{currentfill}%
\pgfsetlinewidth{0.602250pt}%
\definecolor{currentstroke}{rgb}{0.000000,0.000000,0.000000}%
\pgfsetstrokecolor{currentstroke}%
\pgfsetdash{}{0pt}%
\pgfsys@defobject{currentmarker}{\pgfqpoint{0.000000in}{0.000000in}}{\pgfqpoint{0.000000in}{0.027778in}}{%
\pgfpathmoveto{\pgfqpoint{0.000000in}{0.000000in}}%
\pgfpathlineto{\pgfqpoint{0.000000in}{0.027778in}}%
\pgfusepath{stroke,fill}%
}%
\begin{pgfscope}%
\pgfsys@transformshift{2.819036in}{2.830646in}%
\pgfsys@useobject{currentmarker}{}%
\end{pgfscope}%
\end{pgfscope}%
\begin{pgfscope}%
\pgfsetbuttcap%
\pgfsetroundjoin%
\definecolor{currentfill}{rgb}{0.000000,0.000000,0.000000}%
\pgfsetfillcolor{currentfill}%
\pgfsetlinewidth{0.602250pt}%
\definecolor{currentstroke}{rgb}{0.000000,0.000000,0.000000}%
\pgfsetstrokecolor{currentstroke}%
\pgfsetdash{}{0pt}%
\pgfsys@defobject{currentmarker}{\pgfqpoint{0.000000in}{0.000000in}}{\pgfqpoint{0.000000in}{0.027778in}}{%
\pgfpathmoveto{\pgfqpoint{0.000000in}{0.000000in}}%
\pgfpathlineto{\pgfqpoint{0.000000in}{0.027778in}}%
\pgfusepath{stroke,fill}%
}%
\begin{pgfscope}%
\pgfsys@transformshift{2.694518in}{2.830646in}%
\pgfsys@useobject{currentmarker}{}%
\end{pgfscope}%
\end{pgfscope}%
\begin{pgfscope}%
\pgfsetbuttcap%
\pgfsetroundjoin%
\definecolor{currentfill}{rgb}{0.000000,0.000000,0.000000}%
\pgfsetfillcolor{currentfill}%
\pgfsetlinewidth{0.602250pt}%
\definecolor{currentstroke}{rgb}{0.000000,0.000000,0.000000}%
\pgfsetstrokecolor{currentstroke}%
\pgfsetdash{}{0pt}%
\pgfsys@defobject{currentmarker}{\pgfqpoint{0.000000in}{0.000000in}}{\pgfqpoint{0.000000in}{0.027778in}}{%
\pgfpathmoveto{\pgfqpoint{0.000000in}{0.000000in}}%
\pgfpathlineto{\pgfqpoint{0.000000in}{0.027778in}}%
\pgfusepath{stroke,fill}%
}%
\begin{pgfscope}%
\pgfsys@transformshift{2.570001in}{2.830646in}%
\pgfsys@useobject{currentmarker}{}%
\end{pgfscope}%
\end{pgfscope}%
\begin{pgfscope}%
\pgfsetbuttcap%
\pgfsetroundjoin%
\definecolor{currentfill}{rgb}{0.000000,0.000000,0.000000}%
\pgfsetfillcolor{currentfill}%
\pgfsetlinewidth{0.602250pt}%
\definecolor{currentstroke}{rgb}{0.000000,0.000000,0.000000}%
\pgfsetstrokecolor{currentstroke}%
\pgfsetdash{}{0pt}%
\pgfsys@defobject{currentmarker}{\pgfqpoint{0.000000in}{0.000000in}}{\pgfqpoint{0.000000in}{0.027778in}}{%
\pgfpathmoveto{\pgfqpoint{0.000000in}{0.000000in}}%
\pgfpathlineto{\pgfqpoint{0.000000in}{0.027778in}}%
\pgfusepath{stroke,fill}%
}%
\begin{pgfscope}%
\pgfsys@transformshift{2.445483in}{2.830646in}%
\pgfsys@useobject{currentmarker}{}%
\end{pgfscope}%
\end{pgfscope}%
\begin{pgfscope}%
\pgfsetbuttcap%
\pgfsetroundjoin%
\definecolor{currentfill}{rgb}{0.000000,0.000000,0.000000}%
\pgfsetfillcolor{currentfill}%
\pgfsetlinewidth{0.602250pt}%
\definecolor{currentstroke}{rgb}{0.000000,0.000000,0.000000}%
\pgfsetstrokecolor{currentstroke}%
\pgfsetdash{}{0pt}%
\pgfsys@defobject{currentmarker}{\pgfqpoint{0.000000in}{0.000000in}}{\pgfqpoint{0.000000in}{0.027778in}}{%
\pgfpathmoveto{\pgfqpoint{0.000000in}{0.000000in}}%
\pgfpathlineto{\pgfqpoint{0.000000in}{0.027778in}}%
\pgfusepath{stroke,fill}%
}%
\begin{pgfscope}%
\pgfsys@transformshift{2.320966in}{2.830646in}%
\pgfsys@useobject{currentmarker}{}%
\end{pgfscope}%
\end{pgfscope}%
\begin{pgfscope}%
\pgfsetbuttcap%
\pgfsetroundjoin%
\definecolor{currentfill}{rgb}{0.000000,0.000000,0.000000}%
\pgfsetfillcolor{currentfill}%
\pgfsetlinewidth{0.602250pt}%
\definecolor{currentstroke}{rgb}{0.000000,0.000000,0.000000}%
\pgfsetstrokecolor{currentstroke}%
\pgfsetdash{}{0pt}%
\pgfsys@defobject{currentmarker}{\pgfqpoint{0.000000in}{0.000000in}}{\pgfqpoint{0.000000in}{0.027778in}}{%
\pgfpathmoveto{\pgfqpoint{0.000000in}{0.000000in}}%
\pgfpathlineto{\pgfqpoint{0.000000in}{0.027778in}}%
\pgfusepath{stroke,fill}%
}%
\begin{pgfscope}%
\pgfsys@transformshift{2.196448in}{2.830646in}%
\pgfsys@useobject{currentmarker}{}%
\end{pgfscope}%
\end{pgfscope}%
\begin{pgfscope}%
\pgfsetbuttcap%
\pgfsetroundjoin%
\definecolor{currentfill}{rgb}{0.000000,0.000000,0.000000}%
\pgfsetfillcolor{currentfill}%
\pgfsetlinewidth{0.602250pt}%
\definecolor{currentstroke}{rgb}{0.000000,0.000000,0.000000}%
\pgfsetstrokecolor{currentstroke}%
\pgfsetdash{}{0pt}%
\pgfsys@defobject{currentmarker}{\pgfqpoint{0.000000in}{0.000000in}}{\pgfqpoint{0.000000in}{0.027778in}}{%
\pgfpathmoveto{\pgfqpoint{0.000000in}{0.000000in}}%
\pgfpathlineto{\pgfqpoint{0.000000in}{0.027778in}}%
\pgfusepath{stroke,fill}%
}%
\begin{pgfscope}%
\pgfsys@transformshift{2.071931in}{2.830646in}%
\pgfsys@useobject{currentmarker}{}%
\end{pgfscope}%
\end{pgfscope}%
\begin{pgfscope}%
\pgfsetbuttcap%
\pgfsetroundjoin%
\definecolor{currentfill}{rgb}{0.000000,0.000000,0.000000}%
\pgfsetfillcolor{currentfill}%
\pgfsetlinewidth{0.602250pt}%
\definecolor{currentstroke}{rgb}{0.000000,0.000000,0.000000}%
\pgfsetstrokecolor{currentstroke}%
\pgfsetdash{}{0pt}%
\pgfsys@defobject{currentmarker}{\pgfqpoint{0.000000in}{0.000000in}}{\pgfqpoint{0.000000in}{0.027778in}}{%
\pgfpathmoveto{\pgfqpoint{0.000000in}{0.000000in}}%
\pgfpathlineto{\pgfqpoint{0.000000in}{0.027778in}}%
\pgfusepath{stroke,fill}%
}%
\begin{pgfscope}%
\pgfsys@transformshift{1.947414in}{2.830646in}%
\pgfsys@useobject{currentmarker}{}%
\end{pgfscope}%
\end{pgfscope}%
\begin{pgfscope}%
\pgfsetbuttcap%
\pgfsetroundjoin%
\definecolor{currentfill}{rgb}{0.000000,0.000000,0.000000}%
\pgfsetfillcolor{currentfill}%
\pgfsetlinewidth{0.602250pt}%
\definecolor{currentstroke}{rgb}{0.000000,0.000000,0.000000}%
\pgfsetstrokecolor{currentstroke}%
\pgfsetdash{}{0pt}%
\pgfsys@defobject{currentmarker}{\pgfqpoint{0.000000in}{0.000000in}}{\pgfqpoint{0.000000in}{0.027778in}}{%
\pgfpathmoveto{\pgfqpoint{0.000000in}{0.000000in}}%
\pgfpathlineto{\pgfqpoint{0.000000in}{0.027778in}}%
\pgfusepath{stroke,fill}%
}%
\begin{pgfscope}%
\pgfsys@transformshift{1.822896in}{2.830646in}%
\pgfsys@useobject{currentmarker}{}%
\end{pgfscope}%
\end{pgfscope}%
\begin{pgfscope}%
\pgfsetbuttcap%
\pgfsetroundjoin%
\definecolor{currentfill}{rgb}{0.000000,0.000000,0.000000}%
\pgfsetfillcolor{currentfill}%
\pgfsetlinewidth{0.602250pt}%
\definecolor{currentstroke}{rgb}{0.000000,0.000000,0.000000}%
\pgfsetstrokecolor{currentstroke}%
\pgfsetdash{}{0pt}%
\pgfsys@defobject{currentmarker}{\pgfqpoint{0.000000in}{0.000000in}}{\pgfqpoint{0.000000in}{0.027778in}}{%
\pgfpathmoveto{\pgfqpoint{0.000000in}{0.000000in}}%
\pgfpathlineto{\pgfqpoint{0.000000in}{0.027778in}}%
\pgfusepath{stroke,fill}%
}%
\begin{pgfscope}%
\pgfsys@transformshift{1.573861in}{2.830646in}%
\pgfsys@useobject{currentmarker}{}%
\end{pgfscope}%
\end{pgfscope}%
\begin{pgfscope}%
\pgfsetbuttcap%
\pgfsetroundjoin%
\definecolor{currentfill}{rgb}{0.000000,0.000000,0.000000}%
\pgfsetfillcolor{currentfill}%
\pgfsetlinewidth{0.602250pt}%
\definecolor{currentstroke}{rgb}{0.000000,0.000000,0.000000}%
\pgfsetstrokecolor{currentstroke}%
\pgfsetdash{}{0pt}%
\pgfsys@defobject{currentmarker}{\pgfqpoint{0.000000in}{0.000000in}}{\pgfqpoint{0.000000in}{0.027778in}}{%
\pgfpathmoveto{\pgfqpoint{0.000000in}{0.000000in}}%
\pgfpathlineto{\pgfqpoint{0.000000in}{0.027778in}}%
\pgfusepath{stroke,fill}%
}%
\begin{pgfscope}%
\pgfsys@transformshift{1.449344in}{2.830646in}%
\pgfsys@useobject{currentmarker}{}%
\end{pgfscope}%
\end{pgfscope}%
\begin{pgfscope}%
\pgfsetbuttcap%
\pgfsetroundjoin%
\definecolor{currentfill}{rgb}{0.000000,0.000000,0.000000}%
\pgfsetfillcolor{currentfill}%
\pgfsetlinewidth{0.602250pt}%
\definecolor{currentstroke}{rgb}{0.000000,0.000000,0.000000}%
\pgfsetstrokecolor{currentstroke}%
\pgfsetdash{}{0pt}%
\pgfsys@defobject{currentmarker}{\pgfqpoint{0.000000in}{0.000000in}}{\pgfqpoint{0.000000in}{0.027778in}}{%
\pgfpathmoveto{\pgfqpoint{0.000000in}{0.000000in}}%
\pgfpathlineto{\pgfqpoint{0.000000in}{0.027778in}}%
\pgfusepath{stroke,fill}%
}%
\begin{pgfscope}%
\pgfsys@transformshift{1.324826in}{2.830646in}%
\pgfsys@useobject{currentmarker}{}%
\end{pgfscope}%
\end{pgfscope}%
\begin{pgfscope}%
\pgfsetbuttcap%
\pgfsetroundjoin%
\definecolor{currentfill}{rgb}{0.000000,0.000000,0.000000}%
\pgfsetfillcolor{currentfill}%
\pgfsetlinewidth{0.602250pt}%
\definecolor{currentstroke}{rgb}{0.000000,0.000000,0.000000}%
\pgfsetstrokecolor{currentstroke}%
\pgfsetdash{}{0pt}%
\pgfsys@defobject{currentmarker}{\pgfqpoint{0.000000in}{0.000000in}}{\pgfqpoint{0.000000in}{0.027778in}}{%
\pgfpathmoveto{\pgfqpoint{0.000000in}{0.000000in}}%
\pgfpathlineto{\pgfqpoint{0.000000in}{0.027778in}}%
\pgfusepath{stroke,fill}%
}%
\begin{pgfscope}%
\pgfsys@transformshift{1.200309in}{2.830646in}%
\pgfsys@useobject{currentmarker}{}%
\end{pgfscope}%
\end{pgfscope}%
\begin{pgfscope}%
\pgfsetbuttcap%
\pgfsetroundjoin%
\definecolor{currentfill}{rgb}{0.000000,0.000000,0.000000}%
\pgfsetfillcolor{currentfill}%
\pgfsetlinewidth{0.602250pt}%
\definecolor{currentstroke}{rgb}{0.000000,0.000000,0.000000}%
\pgfsetstrokecolor{currentstroke}%
\pgfsetdash{}{0pt}%
\pgfsys@defobject{currentmarker}{\pgfqpoint{0.000000in}{0.000000in}}{\pgfqpoint{0.000000in}{0.027778in}}{%
\pgfpathmoveto{\pgfqpoint{0.000000in}{0.000000in}}%
\pgfpathlineto{\pgfqpoint{0.000000in}{0.027778in}}%
\pgfusepath{stroke,fill}%
}%
\begin{pgfscope}%
\pgfsys@transformshift{1.075791in}{2.830646in}%
\pgfsys@useobject{currentmarker}{}%
\end{pgfscope}%
\end{pgfscope}%
\begin{pgfscope}%
\pgfsetbuttcap%
\pgfsetroundjoin%
\definecolor{currentfill}{rgb}{0.000000,0.000000,0.000000}%
\pgfsetfillcolor{currentfill}%
\pgfsetlinewidth{0.602250pt}%
\definecolor{currentstroke}{rgb}{0.000000,0.000000,0.000000}%
\pgfsetstrokecolor{currentstroke}%
\pgfsetdash{}{0pt}%
\pgfsys@defobject{currentmarker}{\pgfqpoint{0.000000in}{0.000000in}}{\pgfqpoint{0.000000in}{0.027778in}}{%
\pgfpathmoveto{\pgfqpoint{0.000000in}{0.000000in}}%
\pgfpathlineto{\pgfqpoint{0.000000in}{0.027778in}}%
\pgfusepath{stroke,fill}%
}%
\begin{pgfscope}%
\pgfsys@transformshift{0.951274in}{2.830646in}%
\pgfsys@useobject{currentmarker}{}%
\end{pgfscope}%
\end{pgfscope}%
\begin{pgfscope}%
\pgfsetbuttcap%
\pgfsetroundjoin%
\definecolor{currentfill}{rgb}{0.000000,0.000000,0.000000}%
\pgfsetfillcolor{currentfill}%
\pgfsetlinewidth{0.602250pt}%
\definecolor{currentstroke}{rgb}{0.000000,0.000000,0.000000}%
\pgfsetstrokecolor{currentstroke}%
\pgfsetdash{}{0pt}%
\pgfsys@defobject{currentmarker}{\pgfqpoint{0.000000in}{0.000000in}}{\pgfqpoint{0.000000in}{0.027778in}}{%
\pgfpathmoveto{\pgfqpoint{0.000000in}{0.000000in}}%
\pgfpathlineto{\pgfqpoint{0.000000in}{0.027778in}}%
\pgfusepath{stroke,fill}%
}%
\begin{pgfscope}%
\pgfsys@transformshift{0.826756in}{2.830646in}%
\pgfsys@useobject{currentmarker}{}%
\end{pgfscope}%
\end{pgfscope}%
\begin{pgfscope}%
\pgfsetbuttcap%
\pgfsetroundjoin%
\definecolor{currentfill}{rgb}{0.000000,0.000000,0.000000}%
\pgfsetfillcolor{currentfill}%
\pgfsetlinewidth{0.602250pt}%
\definecolor{currentstroke}{rgb}{0.000000,0.000000,0.000000}%
\pgfsetstrokecolor{currentstroke}%
\pgfsetdash{}{0pt}%
\pgfsys@defobject{currentmarker}{\pgfqpoint{0.000000in}{0.000000in}}{\pgfqpoint{0.000000in}{0.027778in}}{%
\pgfpathmoveto{\pgfqpoint{0.000000in}{0.000000in}}%
\pgfpathlineto{\pgfqpoint{0.000000in}{0.027778in}}%
\pgfusepath{stroke,fill}%
}%
\begin{pgfscope}%
\pgfsys@transformshift{0.702239in}{2.830646in}%
\pgfsys@useobject{currentmarker}{}%
\end{pgfscope}%
\end{pgfscope}%
\begin{pgfscope}%
\pgfsetbuttcap%
\pgfsetroundjoin%
\definecolor{currentfill}{rgb}{0.000000,0.000000,0.000000}%
\pgfsetfillcolor{currentfill}%
\pgfsetlinewidth{0.602250pt}%
\definecolor{currentstroke}{rgb}{0.000000,0.000000,0.000000}%
\pgfsetstrokecolor{currentstroke}%
\pgfsetdash{}{0pt}%
\pgfsys@defobject{currentmarker}{\pgfqpoint{0.000000in}{0.000000in}}{\pgfqpoint{0.000000in}{0.027778in}}{%
\pgfpathmoveto{\pgfqpoint{0.000000in}{0.000000in}}%
\pgfpathlineto{\pgfqpoint{0.000000in}{0.027778in}}%
\pgfusepath{stroke,fill}%
}%
\begin{pgfscope}%
\pgfsys@transformshift{0.577722in}{2.830646in}%
\pgfsys@useobject{currentmarker}{}%
\end{pgfscope}%
\end{pgfscope}%
\begin{pgfscope}%
\definecolor{textcolor}{rgb}{0.000000,0.000000,0.000000}%
\pgfsetstrokecolor{textcolor}%
\pgfsetfillcolor{textcolor}%
\pgftext[x=3.317012in,y=3.106079in,,base]{\color{textcolor}\rmfamily\fontsize{10.000000}{12.000000}\selectfont Motorposition \(\displaystyle \varphi_{\text{RZP}}\) in Schritten}%
\end{pgfscope}%
\begin{pgfscope}%
\pgfsetrectcap%
\pgfsetmiterjoin%
\pgfsetlinewidth{0.803000pt}%
\definecolor{currentstroke}{rgb}{0.000000,0.000000,0.000000}%
\pgfsetstrokecolor{currentstroke}%
\pgfsetdash{}{0pt}%
\pgfpathmoveto{\pgfqpoint{6.181121in}{2.830646in}}%
\pgfpathlineto{\pgfqpoint{0.452903in}{2.830646in}}%
\pgfusepath{stroke}%
\end{pgfscope}%
\begin{pgfscope}%
\pgfsetbuttcap%
\pgfsetroundjoin%
\definecolor{currentfill}{rgb}{0.000000,0.000000,0.000000}%
\pgfsetfillcolor{currentfill}%
\pgfsetlinewidth{0.803000pt}%
\definecolor{currentstroke}{rgb}{0.000000,0.000000,0.000000}%
\pgfsetstrokecolor{currentstroke}%
\pgfsetdash{}{0pt}%
\pgfsys@defobject{currentmarker}{\pgfqpoint{0.000000in}{-0.048611in}}{\pgfqpoint{0.000000in}{0.000000in}}{%
\pgfpathmoveto{\pgfqpoint{0.000000in}{0.000000in}}%
\pgfpathlineto{\pgfqpoint{0.000000in}{-0.048611in}}%
\pgfusepath{stroke,fill}%
}%
\begin{pgfscope}%
\pgfsys@transformshift{0.494710in}{0.398883in}%
\pgfsys@useobject{currentmarker}{}%
\end{pgfscope}%
\end{pgfscope}%
\begin{pgfscope}%
\definecolor{textcolor}{rgb}{0.000000,0.000000,0.000000}%
\pgfsetstrokecolor{textcolor}%
\pgfsetfillcolor{textcolor}%
\pgftext[x=0.494710in,y=0.301661in,,top]{\color{textcolor}\rmfamily\fontsize{10.000000}{12.000000}\selectfont \(\displaystyle {1160}\)}%
\end{pgfscope}%
\begin{pgfscope}%
\pgfsetbuttcap%
\pgfsetroundjoin%
\definecolor{currentfill}{rgb}{0.000000,0.000000,0.000000}%
\pgfsetfillcolor{currentfill}%
\pgfsetlinewidth{0.803000pt}%
\definecolor{currentstroke}{rgb}{0.000000,0.000000,0.000000}%
\pgfsetstrokecolor{currentstroke}%
\pgfsetdash{}{0pt}%
\pgfsys@defobject{currentmarker}{\pgfqpoint{0.000000in}{-0.048611in}}{\pgfqpoint{0.000000in}{0.000000in}}{%
\pgfpathmoveto{\pgfqpoint{0.000000in}{0.000000in}}%
\pgfpathlineto{\pgfqpoint{0.000000in}{-0.048611in}}%
\pgfusepath{stroke,fill}%
}%
\begin{pgfscope}%
\pgfsys@transformshift{1.324826in}{0.398883in}%
\pgfsys@useobject{currentmarker}{}%
\end{pgfscope}%
\end{pgfscope}%
\begin{pgfscope}%
\definecolor{textcolor}{rgb}{0.000000,0.000000,0.000000}%
\pgfsetstrokecolor{textcolor}%
\pgfsetfillcolor{textcolor}%
\pgftext[x=1.324826in,y=0.301661in,,top]{\color{textcolor}\rmfamily\fontsize{10.000000}{12.000000}\selectfont \(\displaystyle {1170}\)}%
\end{pgfscope}%
\begin{pgfscope}%
\pgfsetbuttcap%
\pgfsetroundjoin%
\definecolor{currentfill}{rgb}{0.000000,0.000000,0.000000}%
\pgfsetfillcolor{currentfill}%
\pgfsetlinewidth{0.803000pt}%
\definecolor{currentstroke}{rgb}{0.000000,0.000000,0.000000}%
\pgfsetstrokecolor{currentstroke}%
\pgfsetdash{}{0pt}%
\pgfsys@defobject{currentmarker}{\pgfqpoint{0.000000in}{-0.048611in}}{\pgfqpoint{0.000000in}{0.000000in}}{%
\pgfpathmoveto{\pgfqpoint{0.000000in}{0.000000in}}%
\pgfpathlineto{\pgfqpoint{0.000000in}{-0.048611in}}%
\pgfusepath{stroke,fill}%
}%
\begin{pgfscope}%
\pgfsys@transformshift{2.154943in}{0.398883in}%
\pgfsys@useobject{currentmarker}{}%
\end{pgfscope}%
\end{pgfscope}%
\begin{pgfscope}%
\definecolor{textcolor}{rgb}{0.000000,0.000000,0.000000}%
\pgfsetstrokecolor{textcolor}%
\pgfsetfillcolor{textcolor}%
\pgftext[x=2.154943in,y=0.301661in,,top]{\color{textcolor}\rmfamily\fontsize{10.000000}{12.000000}\selectfont \(\displaystyle {1180}\)}%
\end{pgfscope}%
\begin{pgfscope}%
\pgfsetbuttcap%
\pgfsetroundjoin%
\definecolor{currentfill}{rgb}{0.000000,0.000000,0.000000}%
\pgfsetfillcolor{currentfill}%
\pgfsetlinewidth{0.803000pt}%
\definecolor{currentstroke}{rgb}{0.000000,0.000000,0.000000}%
\pgfsetstrokecolor{currentstroke}%
\pgfsetdash{}{0pt}%
\pgfsys@defobject{currentmarker}{\pgfqpoint{0.000000in}{-0.048611in}}{\pgfqpoint{0.000000in}{0.000000in}}{%
\pgfpathmoveto{\pgfqpoint{0.000000in}{0.000000in}}%
\pgfpathlineto{\pgfqpoint{0.000000in}{-0.048611in}}%
\pgfusepath{stroke,fill}%
}%
\begin{pgfscope}%
\pgfsys@transformshift{2.985059in}{0.398883in}%
\pgfsys@useobject{currentmarker}{}%
\end{pgfscope}%
\end{pgfscope}%
\begin{pgfscope}%
\definecolor{textcolor}{rgb}{0.000000,0.000000,0.000000}%
\pgfsetstrokecolor{textcolor}%
\pgfsetfillcolor{textcolor}%
\pgftext[x=2.985059in,y=0.301661in,,top]{\color{textcolor}\rmfamily\fontsize{10.000000}{12.000000}\selectfont \(\displaystyle {1190}\)}%
\end{pgfscope}%
\begin{pgfscope}%
\pgfsetbuttcap%
\pgfsetroundjoin%
\definecolor{currentfill}{rgb}{0.000000,0.000000,0.000000}%
\pgfsetfillcolor{currentfill}%
\pgfsetlinewidth{0.803000pt}%
\definecolor{currentstroke}{rgb}{0.000000,0.000000,0.000000}%
\pgfsetstrokecolor{currentstroke}%
\pgfsetdash{}{0pt}%
\pgfsys@defobject{currentmarker}{\pgfqpoint{0.000000in}{-0.048611in}}{\pgfqpoint{0.000000in}{0.000000in}}{%
\pgfpathmoveto{\pgfqpoint{0.000000in}{0.000000in}}%
\pgfpathlineto{\pgfqpoint{0.000000in}{-0.048611in}}%
\pgfusepath{stroke,fill}%
}%
\begin{pgfscope}%
\pgfsys@transformshift{3.815175in}{0.398883in}%
\pgfsys@useobject{currentmarker}{}%
\end{pgfscope}%
\end{pgfscope}%
\begin{pgfscope}%
\definecolor{textcolor}{rgb}{0.000000,0.000000,0.000000}%
\pgfsetstrokecolor{textcolor}%
\pgfsetfillcolor{textcolor}%
\pgftext[x=3.815175in,y=0.301661in,,top]{\color{textcolor}\rmfamily\fontsize{10.000000}{12.000000}\selectfont \(\displaystyle {1200}\)}%
\end{pgfscope}%
\begin{pgfscope}%
\pgfsetbuttcap%
\pgfsetroundjoin%
\definecolor{currentfill}{rgb}{0.000000,0.000000,0.000000}%
\pgfsetfillcolor{currentfill}%
\pgfsetlinewidth{0.803000pt}%
\definecolor{currentstroke}{rgb}{0.000000,0.000000,0.000000}%
\pgfsetstrokecolor{currentstroke}%
\pgfsetdash{}{0pt}%
\pgfsys@defobject{currentmarker}{\pgfqpoint{0.000000in}{-0.048611in}}{\pgfqpoint{0.000000in}{0.000000in}}{%
\pgfpathmoveto{\pgfqpoint{0.000000in}{0.000000in}}%
\pgfpathlineto{\pgfqpoint{0.000000in}{-0.048611in}}%
\pgfusepath{stroke,fill}%
}%
\begin{pgfscope}%
\pgfsys@transformshift{4.645292in}{0.398883in}%
\pgfsys@useobject{currentmarker}{}%
\end{pgfscope}%
\end{pgfscope}%
\begin{pgfscope}%
\definecolor{textcolor}{rgb}{0.000000,0.000000,0.000000}%
\pgfsetstrokecolor{textcolor}%
\pgfsetfillcolor{textcolor}%
\pgftext[x=4.645292in,y=0.301661in,,top]{\color{textcolor}\rmfamily\fontsize{10.000000}{12.000000}\selectfont \(\displaystyle {1210}\)}%
\end{pgfscope}%
\begin{pgfscope}%
\pgfsetbuttcap%
\pgfsetroundjoin%
\definecolor{currentfill}{rgb}{0.000000,0.000000,0.000000}%
\pgfsetfillcolor{currentfill}%
\pgfsetlinewidth{0.803000pt}%
\definecolor{currentstroke}{rgb}{0.000000,0.000000,0.000000}%
\pgfsetstrokecolor{currentstroke}%
\pgfsetdash{}{0pt}%
\pgfsys@defobject{currentmarker}{\pgfqpoint{0.000000in}{-0.048611in}}{\pgfqpoint{0.000000in}{0.000000in}}{%
\pgfpathmoveto{\pgfqpoint{0.000000in}{0.000000in}}%
\pgfpathlineto{\pgfqpoint{0.000000in}{-0.048611in}}%
\pgfusepath{stroke,fill}%
}%
\begin{pgfscope}%
\pgfsys@transformshift{5.475408in}{0.398883in}%
\pgfsys@useobject{currentmarker}{}%
\end{pgfscope}%
\end{pgfscope}%
\begin{pgfscope}%
\definecolor{textcolor}{rgb}{0.000000,0.000000,0.000000}%
\pgfsetstrokecolor{textcolor}%
\pgfsetfillcolor{textcolor}%
\pgftext[x=5.475408in,y=0.301661in,,top]{\color{textcolor}\rmfamily\fontsize{10.000000}{12.000000}\selectfont \(\displaystyle {1220}\)}%
\end{pgfscope}%
\begin{pgfscope}%
\pgfsetbuttcap%
\pgfsetroundjoin%
\definecolor{currentfill}{rgb}{0.000000,0.000000,0.000000}%
\pgfsetfillcolor{currentfill}%
\pgfsetlinewidth{0.602250pt}%
\definecolor{currentstroke}{rgb}{0.000000,0.000000,0.000000}%
\pgfsetstrokecolor{currentstroke}%
\pgfsetdash{}{0pt}%
\pgfsys@defobject{currentmarker}{\pgfqpoint{0.000000in}{-0.027778in}}{\pgfqpoint{0.000000in}{0.000000in}}{%
\pgfpathmoveto{\pgfqpoint{0.000000in}{0.000000in}}%
\pgfpathlineto{\pgfqpoint{0.000000in}{-0.027778in}}%
\pgfusepath{stroke,fill}%
}%
\begin{pgfscope}%
\pgfsys@transformshift{0.577722in}{0.398883in}%
\pgfsys@useobject{currentmarker}{}%
\end{pgfscope}%
\end{pgfscope}%
\begin{pgfscope}%
\pgfsetbuttcap%
\pgfsetroundjoin%
\definecolor{currentfill}{rgb}{0.000000,0.000000,0.000000}%
\pgfsetfillcolor{currentfill}%
\pgfsetlinewidth{0.602250pt}%
\definecolor{currentstroke}{rgb}{0.000000,0.000000,0.000000}%
\pgfsetstrokecolor{currentstroke}%
\pgfsetdash{}{0pt}%
\pgfsys@defobject{currentmarker}{\pgfqpoint{0.000000in}{-0.027778in}}{\pgfqpoint{0.000000in}{0.000000in}}{%
\pgfpathmoveto{\pgfqpoint{0.000000in}{0.000000in}}%
\pgfpathlineto{\pgfqpoint{0.000000in}{-0.027778in}}%
\pgfusepath{stroke,fill}%
}%
\begin{pgfscope}%
\pgfsys@transformshift{0.660733in}{0.398883in}%
\pgfsys@useobject{currentmarker}{}%
\end{pgfscope}%
\end{pgfscope}%
\begin{pgfscope}%
\pgfsetbuttcap%
\pgfsetroundjoin%
\definecolor{currentfill}{rgb}{0.000000,0.000000,0.000000}%
\pgfsetfillcolor{currentfill}%
\pgfsetlinewidth{0.602250pt}%
\definecolor{currentstroke}{rgb}{0.000000,0.000000,0.000000}%
\pgfsetstrokecolor{currentstroke}%
\pgfsetdash{}{0pt}%
\pgfsys@defobject{currentmarker}{\pgfqpoint{0.000000in}{-0.027778in}}{\pgfqpoint{0.000000in}{0.000000in}}{%
\pgfpathmoveto{\pgfqpoint{0.000000in}{0.000000in}}%
\pgfpathlineto{\pgfqpoint{0.000000in}{-0.027778in}}%
\pgfusepath{stroke,fill}%
}%
\begin{pgfscope}%
\pgfsys@transformshift{0.743745in}{0.398883in}%
\pgfsys@useobject{currentmarker}{}%
\end{pgfscope}%
\end{pgfscope}%
\begin{pgfscope}%
\pgfsetbuttcap%
\pgfsetroundjoin%
\definecolor{currentfill}{rgb}{0.000000,0.000000,0.000000}%
\pgfsetfillcolor{currentfill}%
\pgfsetlinewidth{0.602250pt}%
\definecolor{currentstroke}{rgb}{0.000000,0.000000,0.000000}%
\pgfsetstrokecolor{currentstroke}%
\pgfsetdash{}{0pt}%
\pgfsys@defobject{currentmarker}{\pgfqpoint{0.000000in}{-0.027778in}}{\pgfqpoint{0.000000in}{0.000000in}}{%
\pgfpathmoveto{\pgfqpoint{0.000000in}{0.000000in}}%
\pgfpathlineto{\pgfqpoint{0.000000in}{-0.027778in}}%
\pgfusepath{stroke,fill}%
}%
\begin{pgfscope}%
\pgfsys@transformshift{0.826756in}{0.398883in}%
\pgfsys@useobject{currentmarker}{}%
\end{pgfscope}%
\end{pgfscope}%
\begin{pgfscope}%
\pgfsetbuttcap%
\pgfsetroundjoin%
\definecolor{currentfill}{rgb}{0.000000,0.000000,0.000000}%
\pgfsetfillcolor{currentfill}%
\pgfsetlinewidth{0.602250pt}%
\definecolor{currentstroke}{rgb}{0.000000,0.000000,0.000000}%
\pgfsetstrokecolor{currentstroke}%
\pgfsetdash{}{0pt}%
\pgfsys@defobject{currentmarker}{\pgfqpoint{0.000000in}{-0.027778in}}{\pgfqpoint{0.000000in}{0.000000in}}{%
\pgfpathmoveto{\pgfqpoint{0.000000in}{0.000000in}}%
\pgfpathlineto{\pgfqpoint{0.000000in}{-0.027778in}}%
\pgfusepath{stroke,fill}%
}%
\begin{pgfscope}%
\pgfsys@transformshift{0.909768in}{0.398883in}%
\pgfsys@useobject{currentmarker}{}%
\end{pgfscope}%
\end{pgfscope}%
\begin{pgfscope}%
\pgfsetbuttcap%
\pgfsetroundjoin%
\definecolor{currentfill}{rgb}{0.000000,0.000000,0.000000}%
\pgfsetfillcolor{currentfill}%
\pgfsetlinewidth{0.602250pt}%
\definecolor{currentstroke}{rgb}{0.000000,0.000000,0.000000}%
\pgfsetstrokecolor{currentstroke}%
\pgfsetdash{}{0pt}%
\pgfsys@defobject{currentmarker}{\pgfqpoint{0.000000in}{-0.027778in}}{\pgfqpoint{0.000000in}{0.000000in}}{%
\pgfpathmoveto{\pgfqpoint{0.000000in}{0.000000in}}%
\pgfpathlineto{\pgfqpoint{0.000000in}{-0.027778in}}%
\pgfusepath{stroke,fill}%
}%
\begin{pgfscope}%
\pgfsys@transformshift{0.992780in}{0.398883in}%
\pgfsys@useobject{currentmarker}{}%
\end{pgfscope}%
\end{pgfscope}%
\begin{pgfscope}%
\pgfsetbuttcap%
\pgfsetroundjoin%
\definecolor{currentfill}{rgb}{0.000000,0.000000,0.000000}%
\pgfsetfillcolor{currentfill}%
\pgfsetlinewidth{0.602250pt}%
\definecolor{currentstroke}{rgb}{0.000000,0.000000,0.000000}%
\pgfsetstrokecolor{currentstroke}%
\pgfsetdash{}{0pt}%
\pgfsys@defobject{currentmarker}{\pgfqpoint{0.000000in}{-0.027778in}}{\pgfqpoint{0.000000in}{0.000000in}}{%
\pgfpathmoveto{\pgfqpoint{0.000000in}{0.000000in}}%
\pgfpathlineto{\pgfqpoint{0.000000in}{-0.027778in}}%
\pgfusepath{stroke,fill}%
}%
\begin{pgfscope}%
\pgfsys@transformshift{1.075791in}{0.398883in}%
\pgfsys@useobject{currentmarker}{}%
\end{pgfscope}%
\end{pgfscope}%
\begin{pgfscope}%
\pgfsetbuttcap%
\pgfsetroundjoin%
\definecolor{currentfill}{rgb}{0.000000,0.000000,0.000000}%
\pgfsetfillcolor{currentfill}%
\pgfsetlinewidth{0.602250pt}%
\definecolor{currentstroke}{rgb}{0.000000,0.000000,0.000000}%
\pgfsetstrokecolor{currentstroke}%
\pgfsetdash{}{0pt}%
\pgfsys@defobject{currentmarker}{\pgfqpoint{0.000000in}{-0.027778in}}{\pgfqpoint{0.000000in}{0.000000in}}{%
\pgfpathmoveto{\pgfqpoint{0.000000in}{0.000000in}}%
\pgfpathlineto{\pgfqpoint{0.000000in}{-0.027778in}}%
\pgfusepath{stroke,fill}%
}%
\begin{pgfscope}%
\pgfsys@transformshift{1.158803in}{0.398883in}%
\pgfsys@useobject{currentmarker}{}%
\end{pgfscope}%
\end{pgfscope}%
\begin{pgfscope}%
\pgfsetbuttcap%
\pgfsetroundjoin%
\definecolor{currentfill}{rgb}{0.000000,0.000000,0.000000}%
\pgfsetfillcolor{currentfill}%
\pgfsetlinewidth{0.602250pt}%
\definecolor{currentstroke}{rgb}{0.000000,0.000000,0.000000}%
\pgfsetstrokecolor{currentstroke}%
\pgfsetdash{}{0pt}%
\pgfsys@defobject{currentmarker}{\pgfqpoint{0.000000in}{-0.027778in}}{\pgfqpoint{0.000000in}{0.000000in}}{%
\pgfpathmoveto{\pgfqpoint{0.000000in}{0.000000in}}%
\pgfpathlineto{\pgfqpoint{0.000000in}{-0.027778in}}%
\pgfusepath{stroke,fill}%
}%
\begin{pgfscope}%
\pgfsys@transformshift{1.241815in}{0.398883in}%
\pgfsys@useobject{currentmarker}{}%
\end{pgfscope}%
\end{pgfscope}%
\begin{pgfscope}%
\pgfsetbuttcap%
\pgfsetroundjoin%
\definecolor{currentfill}{rgb}{0.000000,0.000000,0.000000}%
\pgfsetfillcolor{currentfill}%
\pgfsetlinewidth{0.602250pt}%
\definecolor{currentstroke}{rgb}{0.000000,0.000000,0.000000}%
\pgfsetstrokecolor{currentstroke}%
\pgfsetdash{}{0pt}%
\pgfsys@defobject{currentmarker}{\pgfqpoint{0.000000in}{-0.027778in}}{\pgfqpoint{0.000000in}{0.000000in}}{%
\pgfpathmoveto{\pgfqpoint{0.000000in}{0.000000in}}%
\pgfpathlineto{\pgfqpoint{0.000000in}{-0.027778in}}%
\pgfusepath{stroke,fill}%
}%
\begin{pgfscope}%
\pgfsys@transformshift{1.407838in}{0.398883in}%
\pgfsys@useobject{currentmarker}{}%
\end{pgfscope}%
\end{pgfscope}%
\begin{pgfscope}%
\pgfsetbuttcap%
\pgfsetroundjoin%
\definecolor{currentfill}{rgb}{0.000000,0.000000,0.000000}%
\pgfsetfillcolor{currentfill}%
\pgfsetlinewidth{0.602250pt}%
\definecolor{currentstroke}{rgb}{0.000000,0.000000,0.000000}%
\pgfsetstrokecolor{currentstroke}%
\pgfsetdash{}{0pt}%
\pgfsys@defobject{currentmarker}{\pgfqpoint{0.000000in}{-0.027778in}}{\pgfqpoint{0.000000in}{0.000000in}}{%
\pgfpathmoveto{\pgfqpoint{0.000000in}{0.000000in}}%
\pgfpathlineto{\pgfqpoint{0.000000in}{-0.027778in}}%
\pgfusepath{stroke,fill}%
}%
\begin{pgfscope}%
\pgfsys@transformshift{1.490850in}{0.398883in}%
\pgfsys@useobject{currentmarker}{}%
\end{pgfscope}%
\end{pgfscope}%
\begin{pgfscope}%
\pgfsetbuttcap%
\pgfsetroundjoin%
\definecolor{currentfill}{rgb}{0.000000,0.000000,0.000000}%
\pgfsetfillcolor{currentfill}%
\pgfsetlinewidth{0.602250pt}%
\definecolor{currentstroke}{rgb}{0.000000,0.000000,0.000000}%
\pgfsetstrokecolor{currentstroke}%
\pgfsetdash{}{0pt}%
\pgfsys@defobject{currentmarker}{\pgfqpoint{0.000000in}{-0.027778in}}{\pgfqpoint{0.000000in}{0.000000in}}{%
\pgfpathmoveto{\pgfqpoint{0.000000in}{0.000000in}}%
\pgfpathlineto{\pgfqpoint{0.000000in}{-0.027778in}}%
\pgfusepath{stroke,fill}%
}%
\begin{pgfscope}%
\pgfsys@transformshift{1.573861in}{0.398883in}%
\pgfsys@useobject{currentmarker}{}%
\end{pgfscope}%
\end{pgfscope}%
\begin{pgfscope}%
\pgfsetbuttcap%
\pgfsetroundjoin%
\definecolor{currentfill}{rgb}{0.000000,0.000000,0.000000}%
\pgfsetfillcolor{currentfill}%
\pgfsetlinewidth{0.602250pt}%
\definecolor{currentstroke}{rgb}{0.000000,0.000000,0.000000}%
\pgfsetstrokecolor{currentstroke}%
\pgfsetdash{}{0pt}%
\pgfsys@defobject{currentmarker}{\pgfqpoint{0.000000in}{-0.027778in}}{\pgfqpoint{0.000000in}{0.000000in}}{%
\pgfpathmoveto{\pgfqpoint{0.000000in}{0.000000in}}%
\pgfpathlineto{\pgfqpoint{0.000000in}{-0.027778in}}%
\pgfusepath{stroke,fill}%
}%
\begin{pgfscope}%
\pgfsys@transformshift{1.656873in}{0.398883in}%
\pgfsys@useobject{currentmarker}{}%
\end{pgfscope}%
\end{pgfscope}%
\begin{pgfscope}%
\pgfsetbuttcap%
\pgfsetroundjoin%
\definecolor{currentfill}{rgb}{0.000000,0.000000,0.000000}%
\pgfsetfillcolor{currentfill}%
\pgfsetlinewidth{0.602250pt}%
\definecolor{currentstroke}{rgb}{0.000000,0.000000,0.000000}%
\pgfsetstrokecolor{currentstroke}%
\pgfsetdash{}{0pt}%
\pgfsys@defobject{currentmarker}{\pgfqpoint{0.000000in}{-0.027778in}}{\pgfqpoint{0.000000in}{0.000000in}}{%
\pgfpathmoveto{\pgfqpoint{0.000000in}{0.000000in}}%
\pgfpathlineto{\pgfqpoint{0.000000in}{-0.027778in}}%
\pgfusepath{stroke,fill}%
}%
\begin{pgfscope}%
\pgfsys@transformshift{1.739884in}{0.398883in}%
\pgfsys@useobject{currentmarker}{}%
\end{pgfscope}%
\end{pgfscope}%
\begin{pgfscope}%
\pgfsetbuttcap%
\pgfsetroundjoin%
\definecolor{currentfill}{rgb}{0.000000,0.000000,0.000000}%
\pgfsetfillcolor{currentfill}%
\pgfsetlinewidth{0.602250pt}%
\definecolor{currentstroke}{rgb}{0.000000,0.000000,0.000000}%
\pgfsetstrokecolor{currentstroke}%
\pgfsetdash{}{0pt}%
\pgfsys@defobject{currentmarker}{\pgfqpoint{0.000000in}{-0.027778in}}{\pgfqpoint{0.000000in}{0.000000in}}{%
\pgfpathmoveto{\pgfqpoint{0.000000in}{0.000000in}}%
\pgfpathlineto{\pgfqpoint{0.000000in}{-0.027778in}}%
\pgfusepath{stroke,fill}%
}%
\begin{pgfscope}%
\pgfsys@transformshift{1.822896in}{0.398883in}%
\pgfsys@useobject{currentmarker}{}%
\end{pgfscope}%
\end{pgfscope}%
\begin{pgfscope}%
\pgfsetbuttcap%
\pgfsetroundjoin%
\definecolor{currentfill}{rgb}{0.000000,0.000000,0.000000}%
\pgfsetfillcolor{currentfill}%
\pgfsetlinewidth{0.602250pt}%
\definecolor{currentstroke}{rgb}{0.000000,0.000000,0.000000}%
\pgfsetstrokecolor{currentstroke}%
\pgfsetdash{}{0pt}%
\pgfsys@defobject{currentmarker}{\pgfqpoint{0.000000in}{-0.027778in}}{\pgfqpoint{0.000000in}{0.000000in}}{%
\pgfpathmoveto{\pgfqpoint{0.000000in}{0.000000in}}%
\pgfpathlineto{\pgfqpoint{0.000000in}{-0.027778in}}%
\pgfusepath{stroke,fill}%
}%
\begin{pgfscope}%
\pgfsys@transformshift{1.905908in}{0.398883in}%
\pgfsys@useobject{currentmarker}{}%
\end{pgfscope}%
\end{pgfscope}%
\begin{pgfscope}%
\pgfsetbuttcap%
\pgfsetroundjoin%
\definecolor{currentfill}{rgb}{0.000000,0.000000,0.000000}%
\pgfsetfillcolor{currentfill}%
\pgfsetlinewidth{0.602250pt}%
\definecolor{currentstroke}{rgb}{0.000000,0.000000,0.000000}%
\pgfsetstrokecolor{currentstroke}%
\pgfsetdash{}{0pt}%
\pgfsys@defobject{currentmarker}{\pgfqpoint{0.000000in}{-0.027778in}}{\pgfqpoint{0.000000in}{0.000000in}}{%
\pgfpathmoveto{\pgfqpoint{0.000000in}{0.000000in}}%
\pgfpathlineto{\pgfqpoint{0.000000in}{-0.027778in}}%
\pgfusepath{stroke,fill}%
}%
\begin{pgfscope}%
\pgfsys@transformshift{1.988919in}{0.398883in}%
\pgfsys@useobject{currentmarker}{}%
\end{pgfscope}%
\end{pgfscope}%
\begin{pgfscope}%
\pgfsetbuttcap%
\pgfsetroundjoin%
\definecolor{currentfill}{rgb}{0.000000,0.000000,0.000000}%
\pgfsetfillcolor{currentfill}%
\pgfsetlinewidth{0.602250pt}%
\definecolor{currentstroke}{rgb}{0.000000,0.000000,0.000000}%
\pgfsetstrokecolor{currentstroke}%
\pgfsetdash{}{0pt}%
\pgfsys@defobject{currentmarker}{\pgfqpoint{0.000000in}{-0.027778in}}{\pgfqpoint{0.000000in}{0.000000in}}{%
\pgfpathmoveto{\pgfqpoint{0.000000in}{0.000000in}}%
\pgfpathlineto{\pgfqpoint{0.000000in}{-0.027778in}}%
\pgfusepath{stroke,fill}%
}%
\begin{pgfscope}%
\pgfsys@transformshift{2.071931in}{0.398883in}%
\pgfsys@useobject{currentmarker}{}%
\end{pgfscope}%
\end{pgfscope}%
\begin{pgfscope}%
\pgfsetbuttcap%
\pgfsetroundjoin%
\definecolor{currentfill}{rgb}{0.000000,0.000000,0.000000}%
\pgfsetfillcolor{currentfill}%
\pgfsetlinewidth{0.602250pt}%
\definecolor{currentstroke}{rgb}{0.000000,0.000000,0.000000}%
\pgfsetstrokecolor{currentstroke}%
\pgfsetdash{}{0pt}%
\pgfsys@defobject{currentmarker}{\pgfqpoint{0.000000in}{-0.027778in}}{\pgfqpoint{0.000000in}{0.000000in}}{%
\pgfpathmoveto{\pgfqpoint{0.000000in}{0.000000in}}%
\pgfpathlineto{\pgfqpoint{0.000000in}{-0.027778in}}%
\pgfusepath{stroke,fill}%
}%
\begin{pgfscope}%
\pgfsys@transformshift{2.237954in}{0.398883in}%
\pgfsys@useobject{currentmarker}{}%
\end{pgfscope}%
\end{pgfscope}%
\begin{pgfscope}%
\pgfsetbuttcap%
\pgfsetroundjoin%
\definecolor{currentfill}{rgb}{0.000000,0.000000,0.000000}%
\pgfsetfillcolor{currentfill}%
\pgfsetlinewidth{0.602250pt}%
\definecolor{currentstroke}{rgb}{0.000000,0.000000,0.000000}%
\pgfsetstrokecolor{currentstroke}%
\pgfsetdash{}{0pt}%
\pgfsys@defobject{currentmarker}{\pgfqpoint{0.000000in}{-0.027778in}}{\pgfqpoint{0.000000in}{0.000000in}}{%
\pgfpathmoveto{\pgfqpoint{0.000000in}{0.000000in}}%
\pgfpathlineto{\pgfqpoint{0.000000in}{-0.027778in}}%
\pgfusepath{stroke,fill}%
}%
\begin{pgfscope}%
\pgfsys@transformshift{2.320966in}{0.398883in}%
\pgfsys@useobject{currentmarker}{}%
\end{pgfscope}%
\end{pgfscope}%
\begin{pgfscope}%
\pgfsetbuttcap%
\pgfsetroundjoin%
\definecolor{currentfill}{rgb}{0.000000,0.000000,0.000000}%
\pgfsetfillcolor{currentfill}%
\pgfsetlinewidth{0.602250pt}%
\definecolor{currentstroke}{rgb}{0.000000,0.000000,0.000000}%
\pgfsetstrokecolor{currentstroke}%
\pgfsetdash{}{0pt}%
\pgfsys@defobject{currentmarker}{\pgfqpoint{0.000000in}{-0.027778in}}{\pgfqpoint{0.000000in}{0.000000in}}{%
\pgfpathmoveto{\pgfqpoint{0.000000in}{0.000000in}}%
\pgfpathlineto{\pgfqpoint{0.000000in}{-0.027778in}}%
\pgfusepath{stroke,fill}%
}%
\begin{pgfscope}%
\pgfsys@transformshift{2.403978in}{0.398883in}%
\pgfsys@useobject{currentmarker}{}%
\end{pgfscope}%
\end{pgfscope}%
\begin{pgfscope}%
\pgfsetbuttcap%
\pgfsetroundjoin%
\definecolor{currentfill}{rgb}{0.000000,0.000000,0.000000}%
\pgfsetfillcolor{currentfill}%
\pgfsetlinewidth{0.602250pt}%
\definecolor{currentstroke}{rgb}{0.000000,0.000000,0.000000}%
\pgfsetstrokecolor{currentstroke}%
\pgfsetdash{}{0pt}%
\pgfsys@defobject{currentmarker}{\pgfqpoint{0.000000in}{-0.027778in}}{\pgfqpoint{0.000000in}{0.000000in}}{%
\pgfpathmoveto{\pgfqpoint{0.000000in}{0.000000in}}%
\pgfpathlineto{\pgfqpoint{0.000000in}{-0.027778in}}%
\pgfusepath{stroke,fill}%
}%
\begin{pgfscope}%
\pgfsys@transformshift{2.486989in}{0.398883in}%
\pgfsys@useobject{currentmarker}{}%
\end{pgfscope}%
\end{pgfscope}%
\begin{pgfscope}%
\pgfsetbuttcap%
\pgfsetroundjoin%
\definecolor{currentfill}{rgb}{0.000000,0.000000,0.000000}%
\pgfsetfillcolor{currentfill}%
\pgfsetlinewidth{0.602250pt}%
\definecolor{currentstroke}{rgb}{0.000000,0.000000,0.000000}%
\pgfsetstrokecolor{currentstroke}%
\pgfsetdash{}{0pt}%
\pgfsys@defobject{currentmarker}{\pgfqpoint{0.000000in}{-0.027778in}}{\pgfqpoint{0.000000in}{0.000000in}}{%
\pgfpathmoveto{\pgfqpoint{0.000000in}{0.000000in}}%
\pgfpathlineto{\pgfqpoint{0.000000in}{-0.027778in}}%
\pgfusepath{stroke,fill}%
}%
\begin{pgfscope}%
\pgfsys@transformshift{2.570001in}{0.398883in}%
\pgfsys@useobject{currentmarker}{}%
\end{pgfscope}%
\end{pgfscope}%
\begin{pgfscope}%
\pgfsetbuttcap%
\pgfsetroundjoin%
\definecolor{currentfill}{rgb}{0.000000,0.000000,0.000000}%
\pgfsetfillcolor{currentfill}%
\pgfsetlinewidth{0.602250pt}%
\definecolor{currentstroke}{rgb}{0.000000,0.000000,0.000000}%
\pgfsetstrokecolor{currentstroke}%
\pgfsetdash{}{0pt}%
\pgfsys@defobject{currentmarker}{\pgfqpoint{0.000000in}{-0.027778in}}{\pgfqpoint{0.000000in}{0.000000in}}{%
\pgfpathmoveto{\pgfqpoint{0.000000in}{0.000000in}}%
\pgfpathlineto{\pgfqpoint{0.000000in}{-0.027778in}}%
\pgfusepath{stroke,fill}%
}%
\begin{pgfscope}%
\pgfsys@transformshift{2.653012in}{0.398883in}%
\pgfsys@useobject{currentmarker}{}%
\end{pgfscope}%
\end{pgfscope}%
\begin{pgfscope}%
\pgfsetbuttcap%
\pgfsetroundjoin%
\definecolor{currentfill}{rgb}{0.000000,0.000000,0.000000}%
\pgfsetfillcolor{currentfill}%
\pgfsetlinewidth{0.602250pt}%
\definecolor{currentstroke}{rgb}{0.000000,0.000000,0.000000}%
\pgfsetstrokecolor{currentstroke}%
\pgfsetdash{}{0pt}%
\pgfsys@defobject{currentmarker}{\pgfqpoint{0.000000in}{-0.027778in}}{\pgfqpoint{0.000000in}{0.000000in}}{%
\pgfpathmoveto{\pgfqpoint{0.000000in}{0.000000in}}%
\pgfpathlineto{\pgfqpoint{0.000000in}{-0.027778in}}%
\pgfusepath{stroke,fill}%
}%
\begin{pgfscope}%
\pgfsys@transformshift{2.736024in}{0.398883in}%
\pgfsys@useobject{currentmarker}{}%
\end{pgfscope}%
\end{pgfscope}%
\begin{pgfscope}%
\pgfsetbuttcap%
\pgfsetroundjoin%
\definecolor{currentfill}{rgb}{0.000000,0.000000,0.000000}%
\pgfsetfillcolor{currentfill}%
\pgfsetlinewidth{0.602250pt}%
\definecolor{currentstroke}{rgb}{0.000000,0.000000,0.000000}%
\pgfsetstrokecolor{currentstroke}%
\pgfsetdash{}{0pt}%
\pgfsys@defobject{currentmarker}{\pgfqpoint{0.000000in}{-0.027778in}}{\pgfqpoint{0.000000in}{0.000000in}}{%
\pgfpathmoveto{\pgfqpoint{0.000000in}{0.000000in}}%
\pgfpathlineto{\pgfqpoint{0.000000in}{-0.027778in}}%
\pgfusepath{stroke,fill}%
}%
\begin{pgfscope}%
\pgfsys@transformshift{2.819036in}{0.398883in}%
\pgfsys@useobject{currentmarker}{}%
\end{pgfscope}%
\end{pgfscope}%
\begin{pgfscope}%
\pgfsetbuttcap%
\pgfsetroundjoin%
\definecolor{currentfill}{rgb}{0.000000,0.000000,0.000000}%
\pgfsetfillcolor{currentfill}%
\pgfsetlinewidth{0.602250pt}%
\definecolor{currentstroke}{rgb}{0.000000,0.000000,0.000000}%
\pgfsetstrokecolor{currentstroke}%
\pgfsetdash{}{0pt}%
\pgfsys@defobject{currentmarker}{\pgfqpoint{0.000000in}{-0.027778in}}{\pgfqpoint{0.000000in}{0.000000in}}{%
\pgfpathmoveto{\pgfqpoint{0.000000in}{0.000000in}}%
\pgfpathlineto{\pgfqpoint{0.000000in}{-0.027778in}}%
\pgfusepath{stroke,fill}%
}%
\begin{pgfscope}%
\pgfsys@transformshift{2.902047in}{0.398883in}%
\pgfsys@useobject{currentmarker}{}%
\end{pgfscope}%
\end{pgfscope}%
\begin{pgfscope}%
\pgfsetbuttcap%
\pgfsetroundjoin%
\definecolor{currentfill}{rgb}{0.000000,0.000000,0.000000}%
\pgfsetfillcolor{currentfill}%
\pgfsetlinewidth{0.602250pt}%
\definecolor{currentstroke}{rgb}{0.000000,0.000000,0.000000}%
\pgfsetstrokecolor{currentstroke}%
\pgfsetdash{}{0pt}%
\pgfsys@defobject{currentmarker}{\pgfqpoint{0.000000in}{-0.027778in}}{\pgfqpoint{0.000000in}{0.000000in}}{%
\pgfpathmoveto{\pgfqpoint{0.000000in}{0.000000in}}%
\pgfpathlineto{\pgfqpoint{0.000000in}{-0.027778in}}%
\pgfusepath{stroke,fill}%
}%
\begin{pgfscope}%
\pgfsys@transformshift{3.068071in}{0.398883in}%
\pgfsys@useobject{currentmarker}{}%
\end{pgfscope}%
\end{pgfscope}%
\begin{pgfscope}%
\pgfsetbuttcap%
\pgfsetroundjoin%
\definecolor{currentfill}{rgb}{0.000000,0.000000,0.000000}%
\pgfsetfillcolor{currentfill}%
\pgfsetlinewidth{0.602250pt}%
\definecolor{currentstroke}{rgb}{0.000000,0.000000,0.000000}%
\pgfsetstrokecolor{currentstroke}%
\pgfsetdash{}{0pt}%
\pgfsys@defobject{currentmarker}{\pgfqpoint{0.000000in}{-0.027778in}}{\pgfqpoint{0.000000in}{0.000000in}}{%
\pgfpathmoveto{\pgfqpoint{0.000000in}{0.000000in}}%
\pgfpathlineto{\pgfqpoint{0.000000in}{-0.027778in}}%
\pgfusepath{stroke,fill}%
}%
\begin{pgfscope}%
\pgfsys@transformshift{3.151082in}{0.398883in}%
\pgfsys@useobject{currentmarker}{}%
\end{pgfscope}%
\end{pgfscope}%
\begin{pgfscope}%
\pgfsetbuttcap%
\pgfsetroundjoin%
\definecolor{currentfill}{rgb}{0.000000,0.000000,0.000000}%
\pgfsetfillcolor{currentfill}%
\pgfsetlinewidth{0.602250pt}%
\definecolor{currentstroke}{rgb}{0.000000,0.000000,0.000000}%
\pgfsetstrokecolor{currentstroke}%
\pgfsetdash{}{0pt}%
\pgfsys@defobject{currentmarker}{\pgfqpoint{0.000000in}{-0.027778in}}{\pgfqpoint{0.000000in}{0.000000in}}{%
\pgfpathmoveto{\pgfqpoint{0.000000in}{0.000000in}}%
\pgfpathlineto{\pgfqpoint{0.000000in}{-0.027778in}}%
\pgfusepath{stroke,fill}%
}%
\begin{pgfscope}%
\pgfsys@transformshift{3.234094in}{0.398883in}%
\pgfsys@useobject{currentmarker}{}%
\end{pgfscope}%
\end{pgfscope}%
\begin{pgfscope}%
\pgfsetbuttcap%
\pgfsetroundjoin%
\definecolor{currentfill}{rgb}{0.000000,0.000000,0.000000}%
\pgfsetfillcolor{currentfill}%
\pgfsetlinewidth{0.602250pt}%
\definecolor{currentstroke}{rgb}{0.000000,0.000000,0.000000}%
\pgfsetstrokecolor{currentstroke}%
\pgfsetdash{}{0pt}%
\pgfsys@defobject{currentmarker}{\pgfqpoint{0.000000in}{-0.027778in}}{\pgfqpoint{0.000000in}{0.000000in}}{%
\pgfpathmoveto{\pgfqpoint{0.000000in}{0.000000in}}%
\pgfpathlineto{\pgfqpoint{0.000000in}{-0.027778in}}%
\pgfusepath{stroke,fill}%
}%
\begin{pgfscope}%
\pgfsys@transformshift{3.317106in}{0.398883in}%
\pgfsys@useobject{currentmarker}{}%
\end{pgfscope}%
\end{pgfscope}%
\begin{pgfscope}%
\pgfsetbuttcap%
\pgfsetroundjoin%
\definecolor{currentfill}{rgb}{0.000000,0.000000,0.000000}%
\pgfsetfillcolor{currentfill}%
\pgfsetlinewidth{0.602250pt}%
\definecolor{currentstroke}{rgb}{0.000000,0.000000,0.000000}%
\pgfsetstrokecolor{currentstroke}%
\pgfsetdash{}{0pt}%
\pgfsys@defobject{currentmarker}{\pgfqpoint{0.000000in}{-0.027778in}}{\pgfqpoint{0.000000in}{0.000000in}}{%
\pgfpathmoveto{\pgfqpoint{0.000000in}{0.000000in}}%
\pgfpathlineto{\pgfqpoint{0.000000in}{-0.027778in}}%
\pgfusepath{stroke,fill}%
}%
\begin{pgfscope}%
\pgfsys@transformshift{3.400117in}{0.398883in}%
\pgfsys@useobject{currentmarker}{}%
\end{pgfscope}%
\end{pgfscope}%
\begin{pgfscope}%
\pgfsetbuttcap%
\pgfsetroundjoin%
\definecolor{currentfill}{rgb}{0.000000,0.000000,0.000000}%
\pgfsetfillcolor{currentfill}%
\pgfsetlinewidth{0.602250pt}%
\definecolor{currentstroke}{rgb}{0.000000,0.000000,0.000000}%
\pgfsetstrokecolor{currentstroke}%
\pgfsetdash{}{0pt}%
\pgfsys@defobject{currentmarker}{\pgfqpoint{0.000000in}{-0.027778in}}{\pgfqpoint{0.000000in}{0.000000in}}{%
\pgfpathmoveto{\pgfqpoint{0.000000in}{0.000000in}}%
\pgfpathlineto{\pgfqpoint{0.000000in}{-0.027778in}}%
\pgfusepath{stroke,fill}%
}%
\begin{pgfscope}%
\pgfsys@transformshift{3.483129in}{0.398883in}%
\pgfsys@useobject{currentmarker}{}%
\end{pgfscope}%
\end{pgfscope}%
\begin{pgfscope}%
\pgfsetbuttcap%
\pgfsetroundjoin%
\definecolor{currentfill}{rgb}{0.000000,0.000000,0.000000}%
\pgfsetfillcolor{currentfill}%
\pgfsetlinewidth{0.602250pt}%
\definecolor{currentstroke}{rgb}{0.000000,0.000000,0.000000}%
\pgfsetstrokecolor{currentstroke}%
\pgfsetdash{}{0pt}%
\pgfsys@defobject{currentmarker}{\pgfqpoint{0.000000in}{-0.027778in}}{\pgfqpoint{0.000000in}{0.000000in}}{%
\pgfpathmoveto{\pgfqpoint{0.000000in}{0.000000in}}%
\pgfpathlineto{\pgfqpoint{0.000000in}{-0.027778in}}%
\pgfusepath{stroke,fill}%
}%
\begin{pgfscope}%
\pgfsys@transformshift{3.566140in}{0.398883in}%
\pgfsys@useobject{currentmarker}{}%
\end{pgfscope}%
\end{pgfscope}%
\begin{pgfscope}%
\pgfsetbuttcap%
\pgfsetroundjoin%
\definecolor{currentfill}{rgb}{0.000000,0.000000,0.000000}%
\pgfsetfillcolor{currentfill}%
\pgfsetlinewidth{0.602250pt}%
\definecolor{currentstroke}{rgb}{0.000000,0.000000,0.000000}%
\pgfsetstrokecolor{currentstroke}%
\pgfsetdash{}{0pt}%
\pgfsys@defobject{currentmarker}{\pgfqpoint{0.000000in}{-0.027778in}}{\pgfqpoint{0.000000in}{0.000000in}}{%
\pgfpathmoveto{\pgfqpoint{0.000000in}{0.000000in}}%
\pgfpathlineto{\pgfqpoint{0.000000in}{-0.027778in}}%
\pgfusepath{stroke,fill}%
}%
\begin{pgfscope}%
\pgfsys@transformshift{3.649152in}{0.398883in}%
\pgfsys@useobject{currentmarker}{}%
\end{pgfscope}%
\end{pgfscope}%
\begin{pgfscope}%
\pgfsetbuttcap%
\pgfsetroundjoin%
\definecolor{currentfill}{rgb}{0.000000,0.000000,0.000000}%
\pgfsetfillcolor{currentfill}%
\pgfsetlinewidth{0.602250pt}%
\definecolor{currentstroke}{rgb}{0.000000,0.000000,0.000000}%
\pgfsetstrokecolor{currentstroke}%
\pgfsetdash{}{0pt}%
\pgfsys@defobject{currentmarker}{\pgfqpoint{0.000000in}{-0.027778in}}{\pgfqpoint{0.000000in}{0.000000in}}{%
\pgfpathmoveto{\pgfqpoint{0.000000in}{0.000000in}}%
\pgfpathlineto{\pgfqpoint{0.000000in}{-0.027778in}}%
\pgfusepath{stroke,fill}%
}%
\begin{pgfscope}%
\pgfsys@transformshift{3.732164in}{0.398883in}%
\pgfsys@useobject{currentmarker}{}%
\end{pgfscope}%
\end{pgfscope}%
\begin{pgfscope}%
\pgfsetbuttcap%
\pgfsetroundjoin%
\definecolor{currentfill}{rgb}{0.000000,0.000000,0.000000}%
\pgfsetfillcolor{currentfill}%
\pgfsetlinewidth{0.602250pt}%
\definecolor{currentstroke}{rgb}{0.000000,0.000000,0.000000}%
\pgfsetstrokecolor{currentstroke}%
\pgfsetdash{}{0pt}%
\pgfsys@defobject{currentmarker}{\pgfqpoint{0.000000in}{-0.027778in}}{\pgfqpoint{0.000000in}{0.000000in}}{%
\pgfpathmoveto{\pgfqpoint{0.000000in}{0.000000in}}%
\pgfpathlineto{\pgfqpoint{0.000000in}{-0.027778in}}%
\pgfusepath{stroke,fill}%
}%
\begin{pgfscope}%
\pgfsys@transformshift{3.898187in}{0.398883in}%
\pgfsys@useobject{currentmarker}{}%
\end{pgfscope}%
\end{pgfscope}%
\begin{pgfscope}%
\pgfsetbuttcap%
\pgfsetroundjoin%
\definecolor{currentfill}{rgb}{0.000000,0.000000,0.000000}%
\pgfsetfillcolor{currentfill}%
\pgfsetlinewidth{0.602250pt}%
\definecolor{currentstroke}{rgb}{0.000000,0.000000,0.000000}%
\pgfsetstrokecolor{currentstroke}%
\pgfsetdash{}{0pt}%
\pgfsys@defobject{currentmarker}{\pgfqpoint{0.000000in}{-0.027778in}}{\pgfqpoint{0.000000in}{0.000000in}}{%
\pgfpathmoveto{\pgfqpoint{0.000000in}{0.000000in}}%
\pgfpathlineto{\pgfqpoint{0.000000in}{-0.027778in}}%
\pgfusepath{stroke,fill}%
}%
\begin{pgfscope}%
\pgfsys@transformshift{3.981199in}{0.398883in}%
\pgfsys@useobject{currentmarker}{}%
\end{pgfscope}%
\end{pgfscope}%
\begin{pgfscope}%
\pgfsetbuttcap%
\pgfsetroundjoin%
\definecolor{currentfill}{rgb}{0.000000,0.000000,0.000000}%
\pgfsetfillcolor{currentfill}%
\pgfsetlinewidth{0.602250pt}%
\definecolor{currentstroke}{rgb}{0.000000,0.000000,0.000000}%
\pgfsetstrokecolor{currentstroke}%
\pgfsetdash{}{0pt}%
\pgfsys@defobject{currentmarker}{\pgfqpoint{0.000000in}{-0.027778in}}{\pgfqpoint{0.000000in}{0.000000in}}{%
\pgfpathmoveto{\pgfqpoint{0.000000in}{0.000000in}}%
\pgfpathlineto{\pgfqpoint{0.000000in}{-0.027778in}}%
\pgfusepath{stroke,fill}%
}%
\begin{pgfscope}%
\pgfsys@transformshift{4.064210in}{0.398883in}%
\pgfsys@useobject{currentmarker}{}%
\end{pgfscope}%
\end{pgfscope}%
\begin{pgfscope}%
\pgfsetbuttcap%
\pgfsetroundjoin%
\definecolor{currentfill}{rgb}{0.000000,0.000000,0.000000}%
\pgfsetfillcolor{currentfill}%
\pgfsetlinewidth{0.602250pt}%
\definecolor{currentstroke}{rgb}{0.000000,0.000000,0.000000}%
\pgfsetstrokecolor{currentstroke}%
\pgfsetdash{}{0pt}%
\pgfsys@defobject{currentmarker}{\pgfqpoint{0.000000in}{-0.027778in}}{\pgfqpoint{0.000000in}{0.000000in}}{%
\pgfpathmoveto{\pgfqpoint{0.000000in}{0.000000in}}%
\pgfpathlineto{\pgfqpoint{0.000000in}{-0.027778in}}%
\pgfusepath{stroke,fill}%
}%
\begin{pgfscope}%
\pgfsys@transformshift{4.147222in}{0.398883in}%
\pgfsys@useobject{currentmarker}{}%
\end{pgfscope}%
\end{pgfscope}%
\begin{pgfscope}%
\pgfsetbuttcap%
\pgfsetroundjoin%
\definecolor{currentfill}{rgb}{0.000000,0.000000,0.000000}%
\pgfsetfillcolor{currentfill}%
\pgfsetlinewidth{0.602250pt}%
\definecolor{currentstroke}{rgb}{0.000000,0.000000,0.000000}%
\pgfsetstrokecolor{currentstroke}%
\pgfsetdash{}{0pt}%
\pgfsys@defobject{currentmarker}{\pgfqpoint{0.000000in}{-0.027778in}}{\pgfqpoint{0.000000in}{0.000000in}}{%
\pgfpathmoveto{\pgfqpoint{0.000000in}{0.000000in}}%
\pgfpathlineto{\pgfqpoint{0.000000in}{-0.027778in}}%
\pgfusepath{stroke,fill}%
}%
\begin{pgfscope}%
\pgfsys@transformshift{4.230234in}{0.398883in}%
\pgfsys@useobject{currentmarker}{}%
\end{pgfscope}%
\end{pgfscope}%
\begin{pgfscope}%
\pgfsetbuttcap%
\pgfsetroundjoin%
\definecolor{currentfill}{rgb}{0.000000,0.000000,0.000000}%
\pgfsetfillcolor{currentfill}%
\pgfsetlinewidth{0.602250pt}%
\definecolor{currentstroke}{rgb}{0.000000,0.000000,0.000000}%
\pgfsetstrokecolor{currentstroke}%
\pgfsetdash{}{0pt}%
\pgfsys@defobject{currentmarker}{\pgfqpoint{0.000000in}{-0.027778in}}{\pgfqpoint{0.000000in}{0.000000in}}{%
\pgfpathmoveto{\pgfqpoint{0.000000in}{0.000000in}}%
\pgfpathlineto{\pgfqpoint{0.000000in}{-0.027778in}}%
\pgfusepath{stroke,fill}%
}%
\begin{pgfscope}%
\pgfsys@transformshift{4.313245in}{0.398883in}%
\pgfsys@useobject{currentmarker}{}%
\end{pgfscope}%
\end{pgfscope}%
\begin{pgfscope}%
\pgfsetbuttcap%
\pgfsetroundjoin%
\definecolor{currentfill}{rgb}{0.000000,0.000000,0.000000}%
\pgfsetfillcolor{currentfill}%
\pgfsetlinewidth{0.602250pt}%
\definecolor{currentstroke}{rgb}{0.000000,0.000000,0.000000}%
\pgfsetstrokecolor{currentstroke}%
\pgfsetdash{}{0pt}%
\pgfsys@defobject{currentmarker}{\pgfqpoint{0.000000in}{-0.027778in}}{\pgfqpoint{0.000000in}{0.000000in}}{%
\pgfpathmoveto{\pgfqpoint{0.000000in}{0.000000in}}%
\pgfpathlineto{\pgfqpoint{0.000000in}{-0.027778in}}%
\pgfusepath{stroke,fill}%
}%
\begin{pgfscope}%
\pgfsys@transformshift{4.396257in}{0.398883in}%
\pgfsys@useobject{currentmarker}{}%
\end{pgfscope}%
\end{pgfscope}%
\begin{pgfscope}%
\pgfsetbuttcap%
\pgfsetroundjoin%
\definecolor{currentfill}{rgb}{0.000000,0.000000,0.000000}%
\pgfsetfillcolor{currentfill}%
\pgfsetlinewidth{0.602250pt}%
\definecolor{currentstroke}{rgb}{0.000000,0.000000,0.000000}%
\pgfsetstrokecolor{currentstroke}%
\pgfsetdash{}{0pt}%
\pgfsys@defobject{currentmarker}{\pgfqpoint{0.000000in}{-0.027778in}}{\pgfqpoint{0.000000in}{0.000000in}}{%
\pgfpathmoveto{\pgfqpoint{0.000000in}{0.000000in}}%
\pgfpathlineto{\pgfqpoint{0.000000in}{-0.027778in}}%
\pgfusepath{stroke,fill}%
}%
\begin{pgfscope}%
\pgfsys@transformshift{4.479268in}{0.398883in}%
\pgfsys@useobject{currentmarker}{}%
\end{pgfscope}%
\end{pgfscope}%
\begin{pgfscope}%
\pgfsetbuttcap%
\pgfsetroundjoin%
\definecolor{currentfill}{rgb}{0.000000,0.000000,0.000000}%
\pgfsetfillcolor{currentfill}%
\pgfsetlinewidth{0.602250pt}%
\definecolor{currentstroke}{rgb}{0.000000,0.000000,0.000000}%
\pgfsetstrokecolor{currentstroke}%
\pgfsetdash{}{0pt}%
\pgfsys@defobject{currentmarker}{\pgfqpoint{0.000000in}{-0.027778in}}{\pgfqpoint{0.000000in}{0.000000in}}{%
\pgfpathmoveto{\pgfqpoint{0.000000in}{0.000000in}}%
\pgfpathlineto{\pgfqpoint{0.000000in}{-0.027778in}}%
\pgfusepath{stroke,fill}%
}%
\begin{pgfscope}%
\pgfsys@transformshift{4.562280in}{0.398883in}%
\pgfsys@useobject{currentmarker}{}%
\end{pgfscope}%
\end{pgfscope}%
\begin{pgfscope}%
\pgfsetbuttcap%
\pgfsetroundjoin%
\definecolor{currentfill}{rgb}{0.000000,0.000000,0.000000}%
\pgfsetfillcolor{currentfill}%
\pgfsetlinewidth{0.602250pt}%
\definecolor{currentstroke}{rgb}{0.000000,0.000000,0.000000}%
\pgfsetstrokecolor{currentstroke}%
\pgfsetdash{}{0pt}%
\pgfsys@defobject{currentmarker}{\pgfqpoint{0.000000in}{-0.027778in}}{\pgfqpoint{0.000000in}{0.000000in}}{%
\pgfpathmoveto{\pgfqpoint{0.000000in}{0.000000in}}%
\pgfpathlineto{\pgfqpoint{0.000000in}{-0.027778in}}%
\pgfusepath{stroke,fill}%
}%
\begin{pgfscope}%
\pgfsys@transformshift{4.728303in}{0.398883in}%
\pgfsys@useobject{currentmarker}{}%
\end{pgfscope}%
\end{pgfscope}%
\begin{pgfscope}%
\pgfsetbuttcap%
\pgfsetroundjoin%
\definecolor{currentfill}{rgb}{0.000000,0.000000,0.000000}%
\pgfsetfillcolor{currentfill}%
\pgfsetlinewidth{0.602250pt}%
\definecolor{currentstroke}{rgb}{0.000000,0.000000,0.000000}%
\pgfsetstrokecolor{currentstroke}%
\pgfsetdash{}{0pt}%
\pgfsys@defobject{currentmarker}{\pgfqpoint{0.000000in}{-0.027778in}}{\pgfqpoint{0.000000in}{0.000000in}}{%
\pgfpathmoveto{\pgfqpoint{0.000000in}{0.000000in}}%
\pgfpathlineto{\pgfqpoint{0.000000in}{-0.027778in}}%
\pgfusepath{stroke,fill}%
}%
\begin{pgfscope}%
\pgfsys@transformshift{4.811315in}{0.398883in}%
\pgfsys@useobject{currentmarker}{}%
\end{pgfscope}%
\end{pgfscope}%
\begin{pgfscope}%
\pgfsetbuttcap%
\pgfsetroundjoin%
\definecolor{currentfill}{rgb}{0.000000,0.000000,0.000000}%
\pgfsetfillcolor{currentfill}%
\pgfsetlinewidth{0.602250pt}%
\definecolor{currentstroke}{rgb}{0.000000,0.000000,0.000000}%
\pgfsetstrokecolor{currentstroke}%
\pgfsetdash{}{0pt}%
\pgfsys@defobject{currentmarker}{\pgfqpoint{0.000000in}{-0.027778in}}{\pgfqpoint{0.000000in}{0.000000in}}{%
\pgfpathmoveto{\pgfqpoint{0.000000in}{0.000000in}}%
\pgfpathlineto{\pgfqpoint{0.000000in}{-0.027778in}}%
\pgfusepath{stroke,fill}%
}%
\begin{pgfscope}%
\pgfsys@transformshift{4.894327in}{0.398883in}%
\pgfsys@useobject{currentmarker}{}%
\end{pgfscope}%
\end{pgfscope}%
\begin{pgfscope}%
\pgfsetbuttcap%
\pgfsetroundjoin%
\definecolor{currentfill}{rgb}{0.000000,0.000000,0.000000}%
\pgfsetfillcolor{currentfill}%
\pgfsetlinewidth{0.602250pt}%
\definecolor{currentstroke}{rgb}{0.000000,0.000000,0.000000}%
\pgfsetstrokecolor{currentstroke}%
\pgfsetdash{}{0pt}%
\pgfsys@defobject{currentmarker}{\pgfqpoint{0.000000in}{-0.027778in}}{\pgfqpoint{0.000000in}{0.000000in}}{%
\pgfpathmoveto{\pgfqpoint{0.000000in}{0.000000in}}%
\pgfpathlineto{\pgfqpoint{0.000000in}{-0.027778in}}%
\pgfusepath{stroke,fill}%
}%
\begin{pgfscope}%
\pgfsys@transformshift{4.977338in}{0.398883in}%
\pgfsys@useobject{currentmarker}{}%
\end{pgfscope}%
\end{pgfscope}%
\begin{pgfscope}%
\pgfsetbuttcap%
\pgfsetroundjoin%
\definecolor{currentfill}{rgb}{0.000000,0.000000,0.000000}%
\pgfsetfillcolor{currentfill}%
\pgfsetlinewidth{0.602250pt}%
\definecolor{currentstroke}{rgb}{0.000000,0.000000,0.000000}%
\pgfsetstrokecolor{currentstroke}%
\pgfsetdash{}{0pt}%
\pgfsys@defobject{currentmarker}{\pgfqpoint{0.000000in}{-0.027778in}}{\pgfqpoint{0.000000in}{0.000000in}}{%
\pgfpathmoveto{\pgfqpoint{0.000000in}{0.000000in}}%
\pgfpathlineto{\pgfqpoint{0.000000in}{-0.027778in}}%
\pgfusepath{stroke,fill}%
}%
\begin{pgfscope}%
\pgfsys@transformshift{5.060350in}{0.398883in}%
\pgfsys@useobject{currentmarker}{}%
\end{pgfscope}%
\end{pgfscope}%
\begin{pgfscope}%
\pgfsetbuttcap%
\pgfsetroundjoin%
\definecolor{currentfill}{rgb}{0.000000,0.000000,0.000000}%
\pgfsetfillcolor{currentfill}%
\pgfsetlinewidth{0.602250pt}%
\definecolor{currentstroke}{rgb}{0.000000,0.000000,0.000000}%
\pgfsetstrokecolor{currentstroke}%
\pgfsetdash{}{0pt}%
\pgfsys@defobject{currentmarker}{\pgfqpoint{0.000000in}{-0.027778in}}{\pgfqpoint{0.000000in}{0.000000in}}{%
\pgfpathmoveto{\pgfqpoint{0.000000in}{0.000000in}}%
\pgfpathlineto{\pgfqpoint{0.000000in}{-0.027778in}}%
\pgfusepath{stroke,fill}%
}%
\begin{pgfscope}%
\pgfsys@transformshift{5.143362in}{0.398883in}%
\pgfsys@useobject{currentmarker}{}%
\end{pgfscope}%
\end{pgfscope}%
\begin{pgfscope}%
\pgfsetbuttcap%
\pgfsetroundjoin%
\definecolor{currentfill}{rgb}{0.000000,0.000000,0.000000}%
\pgfsetfillcolor{currentfill}%
\pgfsetlinewidth{0.602250pt}%
\definecolor{currentstroke}{rgb}{0.000000,0.000000,0.000000}%
\pgfsetstrokecolor{currentstroke}%
\pgfsetdash{}{0pt}%
\pgfsys@defobject{currentmarker}{\pgfqpoint{0.000000in}{-0.027778in}}{\pgfqpoint{0.000000in}{0.000000in}}{%
\pgfpathmoveto{\pgfqpoint{0.000000in}{0.000000in}}%
\pgfpathlineto{\pgfqpoint{0.000000in}{-0.027778in}}%
\pgfusepath{stroke,fill}%
}%
\begin{pgfscope}%
\pgfsys@transformshift{5.226373in}{0.398883in}%
\pgfsys@useobject{currentmarker}{}%
\end{pgfscope}%
\end{pgfscope}%
\begin{pgfscope}%
\pgfsetbuttcap%
\pgfsetroundjoin%
\definecolor{currentfill}{rgb}{0.000000,0.000000,0.000000}%
\pgfsetfillcolor{currentfill}%
\pgfsetlinewidth{0.602250pt}%
\definecolor{currentstroke}{rgb}{0.000000,0.000000,0.000000}%
\pgfsetstrokecolor{currentstroke}%
\pgfsetdash{}{0pt}%
\pgfsys@defobject{currentmarker}{\pgfqpoint{0.000000in}{-0.027778in}}{\pgfqpoint{0.000000in}{0.000000in}}{%
\pgfpathmoveto{\pgfqpoint{0.000000in}{0.000000in}}%
\pgfpathlineto{\pgfqpoint{0.000000in}{-0.027778in}}%
\pgfusepath{stroke,fill}%
}%
\begin{pgfscope}%
\pgfsys@transformshift{5.309385in}{0.398883in}%
\pgfsys@useobject{currentmarker}{}%
\end{pgfscope}%
\end{pgfscope}%
\begin{pgfscope}%
\pgfsetbuttcap%
\pgfsetroundjoin%
\definecolor{currentfill}{rgb}{0.000000,0.000000,0.000000}%
\pgfsetfillcolor{currentfill}%
\pgfsetlinewidth{0.602250pt}%
\definecolor{currentstroke}{rgb}{0.000000,0.000000,0.000000}%
\pgfsetstrokecolor{currentstroke}%
\pgfsetdash{}{0pt}%
\pgfsys@defobject{currentmarker}{\pgfqpoint{0.000000in}{-0.027778in}}{\pgfqpoint{0.000000in}{0.000000in}}{%
\pgfpathmoveto{\pgfqpoint{0.000000in}{0.000000in}}%
\pgfpathlineto{\pgfqpoint{0.000000in}{-0.027778in}}%
\pgfusepath{stroke,fill}%
}%
\begin{pgfscope}%
\pgfsys@transformshift{5.392396in}{0.398883in}%
\pgfsys@useobject{currentmarker}{}%
\end{pgfscope}%
\end{pgfscope}%
\begin{pgfscope}%
\pgfsetbuttcap%
\pgfsetroundjoin%
\definecolor{currentfill}{rgb}{0.000000,0.000000,0.000000}%
\pgfsetfillcolor{currentfill}%
\pgfsetlinewidth{0.602250pt}%
\definecolor{currentstroke}{rgb}{0.000000,0.000000,0.000000}%
\pgfsetstrokecolor{currentstroke}%
\pgfsetdash{}{0pt}%
\pgfsys@defobject{currentmarker}{\pgfqpoint{0.000000in}{-0.027778in}}{\pgfqpoint{0.000000in}{0.000000in}}{%
\pgfpathmoveto{\pgfqpoint{0.000000in}{0.000000in}}%
\pgfpathlineto{\pgfqpoint{0.000000in}{-0.027778in}}%
\pgfusepath{stroke,fill}%
}%
\begin{pgfscope}%
\pgfsys@transformshift{5.558420in}{0.398883in}%
\pgfsys@useobject{currentmarker}{}%
\end{pgfscope}%
\end{pgfscope}%
\begin{pgfscope}%
\pgfsetbuttcap%
\pgfsetroundjoin%
\definecolor{currentfill}{rgb}{0.000000,0.000000,0.000000}%
\pgfsetfillcolor{currentfill}%
\pgfsetlinewidth{0.602250pt}%
\definecolor{currentstroke}{rgb}{0.000000,0.000000,0.000000}%
\pgfsetstrokecolor{currentstroke}%
\pgfsetdash{}{0pt}%
\pgfsys@defobject{currentmarker}{\pgfqpoint{0.000000in}{-0.027778in}}{\pgfqpoint{0.000000in}{0.000000in}}{%
\pgfpathmoveto{\pgfqpoint{0.000000in}{0.000000in}}%
\pgfpathlineto{\pgfqpoint{0.000000in}{-0.027778in}}%
\pgfusepath{stroke,fill}%
}%
\begin{pgfscope}%
\pgfsys@transformshift{5.641431in}{0.398883in}%
\pgfsys@useobject{currentmarker}{}%
\end{pgfscope}%
\end{pgfscope}%
\begin{pgfscope}%
\pgfsetbuttcap%
\pgfsetroundjoin%
\definecolor{currentfill}{rgb}{0.000000,0.000000,0.000000}%
\pgfsetfillcolor{currentfill}%
\pgfsetlinewidth{0.602250pt}%
\definecolor{currentstroke}{rgb}{0.000000,0.000000,0.000000}%
\pgfsetstrokecolor{currentstroke}%
\pgfsetdash{}{0pt}%
\pgfsys@defobject{currentmarker}{\pgfqpoint{0.000000in}{-0.027778in}}{\pgfqpoint{0.000000in}{0.000000in}}{%
\pgfpathmoveto{\pgfqpoint{0.000000in}{0.000000in}}%
\pgfpathlineto{\pgfqpoint{0.000000in}{-0.027778in}}%
\pgfusepath{stroke,fill}%
}%
\begin{pgfscope}%
\pgfsys@transformshift{5.724443in}{0.398883in}%
\pgfsys@useobject{currentmarker}{}%
\end{pgfscope}%
\end{pgfscope}%
\begin{pgfscope}%
\pgfsetbuttcap%
\pgfsetroundjoin%
\definecolor{currentfill}{rgb}{0.000000,0.000000,0.000000}%
\pgfsetfillcolor{currentfill}%
\pgfsetlinewidth{0.602250pt}%
\definecolor{currentstroke}{rgb}{0.000000,0.000000,0.000000}%
\pgfsetstrokecolor{currentstroke}%
\pgfsetdash{}{0pt}%
\pgfsys@defobject{currentmarker}{\pgfqpoint{0.000000in}{-0.027778in}}{\pgfqpoint{0.000000in}{0.000000in}}{%
\pgfpathmoveto{\pgfqpoint{0.000000in}{0.000000in}}%
\pgfpathlineto{\pgfqpoint{0.000000in}{-0.027778in}}%
\pgfusepath{stroke,fill}%
}%
\begin{pgfscope}%
\pgfsys@transformshift{5.807455in}{0.398883in}%
\pgfsys@useobject{currentmarker}{}%
\end{pgfscope}%
\end{pgfscope}%
\begin{pgfscope}%
\pgfsetbuttcap%
\pgfsetroundjoin%
\definecolor{currentfill}{rgb}{0.000000,0.000000,0.000000}%
\pgfsetfillcolor{currentfill}%
\pgfsetlinewidth{0.602250pt}%
\definecolor{currentstroke}{rgb}{0.000000,0.000000,0.000000}%
\pgfsetstrokecolor{currentstroke}%
\pgfsetdash{}{0pt}%
\pgfsys@defobject{currentmarker}{\pgfqpoint{0.000000in}{-0.027778in}}{\pgfqpoint{0.000000in}{0.000000in}}{%
\pgfpathmoveto{\pgfqpoint{0.000000in}{0.000000in}}%
\pgfpathlineto{\pgfqpoint{0.000000in}{-0.027778in}}%
\pgfusepath{stroke,fill}%
}%
\begin{pgfscope}%
\pgfsys@transformshift{5.890466in}{0.398883in}%
\pgfsys@useobject{currentmarker}{}%
\end{pgfscope}%
\end{pgfscope}%
\begin{pgfscope}%
\pgfsetbuttcap%
\pgfsetroundjoin%
\definecolor{currentfill}{rgb}{0.000000,0.000000,0.000000}%
\pgfsetfillcolor{currentfill}%
\pgfsetlinewidth{0.602250pt}%
\definecolor{currentstroke}{rgb}{0.000000,0.000000,0.000000}%
\pgfsetstrokecolor{currentstroke}%
\pgfsetdash{}{0pt}%
\pgfsys@defobject{currentmarker}{\pgfqpoint{0.000000in}{-0.027778in}}{\pgfqpoint{0.000000in}{0.000000in}}{%
\pgfpathmoveto{\pgfqpoint{0.000000in}{0.000000in}}%
\pgfpathlineto{\pgfqpoint{0.000000in}{-0.027778in}}%
\pgfusepath{stroke,fill}%
}%
\begin{pgfscope}%
\pgfsys@transformshift{5.973478in}{0.398883in}%
\pgfsys@useobject{currentmarker}{}%
\end{pgfscope}%
\end{pgfscope}%
\begin{pgfscope}%
\pgfsetbuttcap%
\pgfsetroundjoin%
\definecolor{currentfill}{rgb}{0.000000,0.000000,0.000000}%
\pgfsetfillcolor{currentfill}%
\pgfsetlinewidth{0.602250pt}%
\definecolor{currentstroke}{rgb}{0.000000,0.000000,0.000000}%
\pgfsetstrokecolor{currentstroke}%
\pgfsetdash{}{0pt}%
\pgfsys@defobject{currentmarker}{\pgfqpoint{0.000000in}{-0.027778in}}{\pgfqpoint{0.000000in}{0.000000in}}{%
\pgfpathmoveto{\pgfqpoint{0.000000in}{0.000000in}}%
\pgfpathlineto{\pgfqpoint{0.000000in}{-0.027778in}}%
\pgfusepath{stroke,fill}%
}%
\begin{pgfscope}%
\pgfsys@transformshift{6.056490in}{0.398883in}%
\pgfsys@useobject{currentmarker}{}%
\end{pgfscope}%
\end{pgfscope}%
\begin{pgfscope}%
\pgfsetbuttcap%
\pgfsetroundjoin%
\definecolor{currentfill}{rgb}{0.000000,0.000000,0.000000}%
\pgfsetfillcolor{currentfill}%
\pgfsetlinewidth{0.602250pt}%
\definecolor{currentstroke}{rgb}{0.000000,0.000000,0.000000}%
\pgfsetstrokecolor{currentstroke}%
\pgfsetdash{}{0pt}%
\pgfsys@defobject{currentmarker}{\pgfqpoint{0.000000in}{-0.027778in}}{\pgfqpoint{0.000000in}{0.000000in}}{%
\pgfpathmoveto{\pgfqpoint{0.000000in}{0.000000in}}%
\pgfpathlineto{\pgfqpoint{0.000000in}{-0.027778in}}%
\pgfusepath{stroke,fill}%
}%
\begin{pgfscope}%
\pgfsys@transformshift{6.139501in}{0.398883in}%
\pgfsys@useobject{currentmarker}{}%
\end{pgfscope}%
\end{pgfscope}%
\begin{pgfscope}%
\definecolor{textcolor}{rgb}{0.000000,0.000000,0.000000}%
\pgfsetstrokecolor{textcolor}%
\pgfsetfillcolor{textcolor}%
\pgftext[x=3.317012in,y=0.123450in,,top]{\color{textcolor}\rmfamily\fontsize{10.000000}{12.000000}\selectfont Photonenenergie \(\displaystyle h\nu\) in eV}%
\end{pgfscope}%
\begin{pgfscope}%
\pgfsetbuttcap%
\pgfsetroundjoin%
\definecolor{currentfill}{rgb}{0.000000,0.000000,0.000000}%
\pgfsetfillcolor{currentfill}%
\pgfsetlinewidth{0.803000pt}%
\definecolor{currentstroke}{rgb}{0.000000,0.000000,0.000000}%
\pgfsetstrokecolor{currentstroke}%
\pgfsetdash{}{0pt}%
\pgfsys@defobject{currentmarker}{\pgfqpoint{-0.048611in}{0.000000in}}{\pgfqpoint{-0.000000in}{0.000000in}}{%
\pgfpathmoveto{\pgfqpoint{-0.000000in}{0.000000in}}%
\pgfpathlineto{\pgfqpoint{-0.048611in}{0.000000in}}%
\pgfusepath{stroke,fill}%
}%
\begin{pgfscope}%
\pgfsys@transformshift{0.452903in}{0.500207in}%
\pgfsys@useobject{currentmarker}{}%
\end{pgfscope}%
\end{pgfscope}%
\begin{pgfscope}%
\definecolor{textcolor}{rgb}{0.000000,0.000000,0.000000}%
\pgfsetstrokecolor{textcolor}%
\pgfsetfillcolor{textcolor}%
\pgftext[x=0.178211in, y=0.452382in, left, base]{\color{textcolor}\rmfamily\fontsize{10.000000}{12.000000}\selectfont \num{0.0}}%
\end{pgfscope}%
\begin{pgfscope}%
\pgfsetbuttcap%
\pgfsetroundjoin%
\definecolor{currentfill}{rgb}{0.000000,0.000000,0.000000}%
\pgfsetfillcolor{currentfill}%
\pgfsetlinewidth{0.803000pt}%
\definecolor{currentstroke}{rgb}{0.000000,0.000000,0.000000}%
\pgfsetstrokecolor{currentstroke}%
\pgfsetdash{}{0pt}%
\pgfsys@defobject{currentmarker}{\pgfqpoint{-0.048611in}{0.000000in}}{\pgfqpoint{-0.000000in}{0.000000in}}{%
\pgfpathmoveto{\pgfqpoint{-0.000000in}{0.000000in}}%
\pgfpathlineto{\pgfqpoint{-0.048611in}{0.000000in}}%
\pgfusepath{stroke,fill}%
}%
\begin{pgfscope}%
\pgfsys@transformshift{0.452903in}{0.905500in}%
\pgfsys@useobject{currentmarker}{}%
\end{pgfscope}%
\end{pgfscope}%
\begin{pgfscope}%
\definecolor{textcolor}{rgb}{0.000000,0.000000,0.000000}%
\pgfsetstrokecolor{textcolor}%
\pgfsetfillcolor{textcolor}%
\pgftext[x=0.178211in, y=0.857676in, left, base]{\color{textcolor}\rmfamily\fontsize{10.000000}{12.000000}\selectfont \num{0.2}}%
\end{pgfscope}%
\begin{pgfscope}%
\pgfsetbuttcap%
\pgfsetroundjoin%
\definecolor{currentfill}{rgb}{0.000000,0.000000,0.000000}%
\pgfsetfillcolor{currentfill}%
\pgfsetlinewidth{0.803000pt}%
\definecolor{currentstroke}{rgb}{0.000000,0.000000,0.000000}%
\pgfsetstrokecolor{currentstroke}%
\pgfsetdash{}{0pt}%
\pgfsys@defobject{currentmarker}{\pgfqpoint{-0.048611in}{0.000000in}}{\pgfqpoint{-0.000000in}{0.000000in}}{%
\pgfpathmoveto{\pgfqpoint{-0.000000in}{0.000000in}}%
\pgfpathlineto{\pgfqpoint{-0.048611in}{0.000000in}}%
\pgfusepath{stroke,fill}%
}%
\begin{pgfscope}%
\pgfsys@transformshift{0.452903in}{1.310794in}%
\pgfsys@useobject{currentmarker}{}%
\end{pgfscope}%
\end{pgfscope}%
\begin{pgfscope}%
\definecolor{textcolor}{rgb}{0.000000,0.000000,0.000000}%
\pgfsetstrokecolor{textcolor}%
\pgfsetfillcolor{textcolor}%
\pgftext[x=0.178211in, y=1.262970in, left, base]{\color{textcolor}\rmfamily\fontsize{10.000000}{12.000000}\selectfont \num{0.4}}%
\end{pgfscope}%
\begin{pgfscope}%
\pgfsetbuttcap%
\pgfsetroundjoin%
\definecolor{currentfill}{rgb}{0.000000,0.000000,0.000000}%
\pgfsetfillcolor{currentfill}%
\pgfsetlinewidth{0.803000pt}%
\definecolor{currentstroke}{rgb}{0.000000,0.000000,0.000000}%
\pgfsetstrokecolor{currentstroke}%
\pgfsetdash{}{0pt}%
\pgfsys@defobject{currentmarker}{\pgfqpoint{-0.048611in}{0.000000in}}{\pgfqpoint{-0.000000in}{0.000000in}}{%
\pgfpathmoveto{\pgfqpoint{-0.000000in}{0.000000in}}%
\pgfpathlineto{\pgfqpoint{-0.048611in}{0.000000in}}%
\pgfusepath{stroke,fill}%
}%
\begin{pgfscope}%
\pgfsys@transformshift{0.452903in}{1.716088in}%
\pgfsys@useobject{currentmarker}{}%
\end{pgfscope}%
\end{pgfscope}%
\begin{pgfscope}%
\definecolor{textcolor}{rgb}{0.000000,0.000000,0.000000}%
\pgfsetstrokecolor{textcolor}%
\pgfsetfillcolor{textcolor}%
\pgftext[x=0.178211in, y=1.668264in, left, base]{\color{textcolor}\rmfamily\fontsize{10.000000}{12.000000}\selectfont \num{0.6}}%
\end{pgfscope}%
\begin{pgfscope}%
\pgfsetbuttcap%
\pgfsetroundjoin%
\definecolor{currentfill}{rgb}{0.000000,0.000000,0.000000}%
\pgfsetfillcolor{currentfill}%
\pgfsetlinewidth{0.803000pt}%
\definecolor{currentstroke}{rgb}{0.000000,0.000000,0.000000}%
\pgfsetstrokecolor{currentstroke}%
\pgfsetdash{}{0pt}%
\pgfsys@defobject{currentmarker}{\pgfqpoint{-0.048611in}{0.000000in}}{\pgfqpoint{-0.000000in}{0.000000in}}{%
\pgfpathmoveto{\pgfqpoint{-0.000000in}{0.000000in}}%
\pgfpathlineto{\pgfqpoint{-0.048611in}{0.000000in}}%
\pgfusepath{stroke,fill}%
}%
\begin{pgfscope}%
\pgfsys@transformshift{0.452903in}{2.121382in}%
\pgfsys@useobject{currentmarker}{}%
\end{pgfscope}%
\end{pgfscope}%
\begin{pgfscope}%
\definecolor{textcolor}{rgb}{0.000000,0.000000,0.000000}%
\pgfsetstrokecolor{textcolor}%
\pgfsetfillcolor{textcolor}%
\pgftext[x=0.178211in, y=2.073557in, left, base]{\color{textcolor}\rmfamily\fontsize{10.000000}{12.000000}\selectfont \num{0.8}}%
\end{pgfscope}%
\begin{pgfscope}%
\pgfsetbuttcap%
\pgfsetroundjoin%
\definecolor{currentfill}{rgb}{0.000000,0.000000,0.000000}%
\pgfsetfillcolor{currentfill}%
\pgfsetlinewidth{0.803000pt}%
\definecolor{currentstroke}{rgb}{0.000000,0.000000,0.000000}%
\pgfsetstrokecolor{currentstroke}%
\pgfsetdash{}{0pt}%
\pgfsys@defobject{currentmarker}{\pgfqpoint{-0.048611in}{0.000000in}}{\pgfqpoint{-0.000000in}{0.000000in}}{%
\pgfpathmoveto{\pgfqpoint{-0.000000in}{0.000000in}}%
\pgfpathlineto{\pgfqpoint{-0.048611in}{0.000000in}}%
\pgfusepath{stroke,fill}%
}%
\begin{pgfscope}%
\pgfsys@transformshift{0.452903in}{2.526676in}%
\pgfsys@useobject{currentmarker}{}%
\end{pgfscope}%
\end{pgfscope}%
\begin{pgfscope}%
\definecolor{textcolor}{rgb}{0.000000,0.000000,0.000000}%
\pgfsetstrokecolor{textcolor}%
\pgfsetfillcolor{textcolor}%
\pgftext[x=0.178211in, y=2.478851in, left, base]{\color{textcolor}\rmfamily\fontsize{10.000000}{12.000000}\selectfont \num{1.0}}%
\end{pgfscope}%
\begin{pgfscope}%
\definecolor{textcolor}{rgb}{0.000000,0.000000,0.000000}%
\pgfsetstrokecolor{textcolor}%
\pgfsetfillcolor{textcolor}%
\pgftext[x=0.122655in,y=1.614765in,,bottom,rotate=90.000000]{\color{textcolor}\rmfamily\fontsize{10.000000}{12.000000}\selectfont normierte Absorptionsstärke}%
\end{pgfscope}%
\begin{pgfscope}%
\pgfpathrectangle{\pgfqpoint{0.452903in}{0.398883in}}{\pgfqpoint{5.728219in}{2.431763in}}%
\pgfusepath{clip}%
\pgfsetrectcap%
\pgfsetroundjoin%
\pgfsetlinewidth{1.003750pt}%
\definecolor{currentstroke}{rgb}{0.000000,0.000000,0.000000}%
\pgfsetstrokecolor{currentstroke}%
\pgfsetdash{}{0pt}%
\pgfpathmoveto{\pgfqpoint{2.570001in}{0.398883in}}%
\pgfpathlineto{\pgfqpoint{2.570001in}{2.830646in}}%
\pgfusepath{stroke}%
\end{pgfscope}%
\begin{pgfscope}%
\pgfpathrectangle{\pgfqpoint{0.452903in}{0.398883in}}{\pgfqpoint{5.728219in}{2.431763in}}%
\pgfusepath{clip}%
\pgfsetrectcap%
\pgfsetroundjoin%
\pgfsetlinewidth{1.003750pt}%
\definecolor{currentstroke}{rgb}{0.000000,0.000000,0.000000}%
\pgfsetstrokecolor{currentstroke}%
\pgfsetdash{}{0pt}%
\pgfpathmoveto{\pgfqpoint{5.060350in}{0.398883in}}%
\pgfpathlineto{\pgfqpoint{5.060350in}{2.830646in}}%
\pgfusepath{stroke}%
\end{pgfscope}%
\begin{pgfscope}%
\pgfpathrectangle{\pgfqpoint{0.452903in}{0.398883in}}{\pgfqpoint{5.728219in}{2.431763in}}%
\pgfusepath{clip}%
\pgfsetrectcap%
\pgfsetroundjoin%
\pgfsetlinewidth{1.003750pt}%
\definecolor{currentstroke}{rgb}{0.000000,0.000000,0.000000}%
\pgfsetstrokecolor{currentstroke}%
\pgfsetdash{}{0pt}%
\pgfpathmoveto{\pgfqpoint{0.702239in}{0.398883in}}%
\pgfpathlineto{\pgfqpoint{0.702239in}{2.830646in}}%
\pgfusepath{stroke}%
\end{pgfscope}%
\begin{pgfscope}%
\pgfpathrectangle{\pgfqpoint{0.452903in}{0.398883in}}{\pgfqpoint{5.728219in}{2.431763in}}%
\pgfusepath{clip}%
\pgfsetrectcap%
\pgfsetroundjoin%
\pgfsetlinewidth{1.505625pt}%
\definecolor{currentstroke}{rgb}{0.121569,0.466667,0.705882}%
\pgfsetstrokecolor{currentstroke}%
\pgfsetdash{}{0pt}%
\pgfpathmoveto{\pgfqpoint{0.452903in}{0.697701in}}%
\pgfpathlineto{\pgfqpoint{0.515626in}{0.722901in}}%
\pgfpathlineto{\pgfqpoint{0.577840in}{0.720487in}}%
\pgfpathlineto{\pgfqpoint{0.640063in}{0.689844in}}%
\pgfpathlineto{\pgfqpoint{0.702090in}{0.693774in}}%
\pgfpathlineto{\pgfqpoint{0.764811in}{0.738205in}}%
\pgfpathlineto{\pgfqpoint{0.827166in}{0.736666in}}%
\pgfpathlineto{\pgfqpoint{0.889150in}{0.728678in}}%
\pgfpathlineto{\pgfqpoint{0.951777in}{0.691783in}}%
\pgfpathlineto{\pgfqpoint{1.013310in}{0.702650in}}%
\pgfpathlineto{\pgfqpoint{1.076183in}{0.717677in}}%
\pgfpathlineto{\pgfqpoint{1.138312in}{0.672407in}}%
\pgfpathlineto{\pgfqpoint{1.200690in}{0.716115in}}%
\pgfpathlineto{\pgfqpoint{1.263101in}{0.699041in}}%
\pgfpathlineto{\pgfqpoint{1.324894in}{0.698775in}}%
\pgfpathlineto{\pgfqpoint{1.387278in}{0.703566in}}%
\pgfpathlineto{\pgfqpoint{1.449823in}{0.693107in}}%
\pgfpathlineto{\pgfqpoint{1.511654in}{0.700899in}}%
\pgfpathlineto{\pgfqpoint{1.573984in}{0.686068in}}%
\pgfpathlineto{\pgfqpoint{1.635877in}{0.702173in}}%
\pgfpathlineto{\pgfqpoint{1.698574in}{0.716351in}}%
\pgfpathlineto{\pgfqpoint{1.761110in}{0.700051in}}%
\pgfpathlineto{\pgfqpoint{1.823291in}{0.666814in}}%
\pgfpathlineto{\pgfqpoint{1.885669in}{0.660322in}}%
\pgfpathlineto{\pgfqpoint{1.947311in}{0.601060in}}%
\pgfpathlineto{\pgfqpoint{2.009651in}{0.558177in}}%
\pgfpathlineto{\pgfqpoint{2.071910in}{0.528991in}}%
\pgfpathlineto{\pgfqpoint{2.134263in}{0.500207in}}%
\pgfpathlineto{\pgfqpoint{2.196495in}{0.515686in}}%
\pgfpathlineto{\pgfqpoint{2.258723in}{0.537001in}}%
\pgfpathlineto{\pgfqpoint{2.321226in}{0.657897in}}%
\pgfpathlineto{\pgfqpoint{2.383590in}{0.763386in}}%
\pgfpathlineto{\pgfqpoint{2.445412in}{1.259236in}}%
\pgfpathlineto{\pgfqpoint{2.508271in}{1.705060in}}%
\pgfpathlineto{\pgfqpoint{2.570608in}{2.526676in}}%
\pgfpathlineto{\pgfqpoint{2.632302in}{2.355497in}}%
\pgfpathlineto{\pgfqpoint{2.694793in}{1.621657in}}%
\pgfpathlineto{\pgfqpoint{2.757251in}{1.211595in}}%
\pgfpathlineto{\pgfqpoint{2.819515in}{0.974700in}}%
\pgfpathlineto{\pgfqpoint{2.881704in}{0.769882in}}%
\pgfpathlineto{\pgfqpoint{2.943613in}{0.759700in}}%
\pgfpathlineto{\pgfqpoint{3.006216in}{0.718209in}}%
\pgfpathlineto{\pgfqpoint{3.068451in}{0.761981in}}%
\pgfpathlineto{\pgfqpoint{3.130948in}{0.726313in}}%
\pgfpathlineto{\pgfqpoint{3.192610in}{0.671938in}}%
\pgfpathlineto{\pgfqpoint{3.254895in}{0.690300in}}%
\pgfpathlineto{\pgfqpoint{3.317402in}{0.675669in}}%
\pgfpathlineto{\pgfqpoint{3.379797in}{0.670778in}}%
\pgfpathlineto{\pgfqpoint{3.441716in}{0.669868in}}%
\pgfpathlineto{\pgfqpoint{3.504419in}{0.671934in}}%
\pgfpathlineto{\pgfqpoint{3.566177in}{0.694011in}}%
\pgfpathlineto{\pgfqpoint{3.628696in}{0.683855in}}%
\pgfpathlineto{\pgfqpoint{3.691186in}{0.703141in}}%
\pgfpathlineto{\pgfqpoint{3.753324in}{0.718113in}}%
\pgfpathlineto{\pgfqpoint{3.815675in}{0.728505in}}%
\pgfpathlineto{\pgfqpoint{3.877484in}{0.736027in}}%
\pgfpathlineto{\pgfqpoint{3.939927in}{0.738367in}}%
\pgfpathlineto{\pgfqpoint{4.002356in}{0.752853in}}%
\pgfpathlineto{\pgfqpoint{4.064651in}{0.754939in}}%
\pgfpathlineto{\pgfqpoint{4.126738in}{0.741405in}}%
\pgfpathlineto{\pgfqpoint{4.188702in}{0.726159in}}%
\pgfpathlineto{\pgfqpoint{4.188743in}{0.730865in}}%
\pgfpathlineto{\pgfqpoint{4.250809in}{0.717481in}}%
\pgfpathlineto{\pgfqpoint{4.313545in}{0.687370in}}%
\pgfpathlineto{\pgfqpoint{4.375332in}{0.659527in}}%
\pgfpathlineto{\pgfqpoint{4.437892in}{0.640842in}}%
\pgfpathlineto{\pgfqpoint{4.500376in}{0.640339in}}%
\pgfpathlineto{\pgfqpoint{4.562411in}{0.611392in}}%
\pgfpathlineto{\pgfqpoint{4.624651in}{0.598258in}}%
\pgfpathlineto{\pgfqpoint{4.686916in}{0.619860in}}%
\pgfpathlineto{\pgfqpoint{4.749032in}{0.649658in}}%
\pgfpathlineto{\pgfqpoint{4.811626in}{0.697215in}}%
\pgfpathlineto{\pgfqpoint{4.874009in}{0.831670in}}%
\pgfpathlineto{\pgfqpoint{4.936412in}{0.971893in}}%
\pgfpathlineto{\pgfqpoint{4.997962in}{1.041676in}}%
\pgfpathlineto{\pgfqpoint{5.060488in}{1.149663in}}%
\pgfpathlineto{\pgfqpoint{5.122838in}{1.129922in}}%
\pgfpathlineto{\pgfqpoint{5.185132in}{1.029003in}}%
\pgfpathlineto{\pgfqpoint{5.247497in}{0.988840in}}%
\pgfpathlineto{\pgfqpoint{5.309936in}{0.838958in}}%
\pgfpathlineto{\pgfqpoint{5.371667in}{0.783489in}}%
\pgfpathlineto{\pgfqpoint{5.434242in}{0.735658in}}%
\pgfpathlineto{\pgfqpoint{5.496522in}{0.711451in}}%
\pgfpathlineto{\pgfqpoint{5.559022in}{0.715134in}}%
\pgfpathlineto{\pgfqpoint{5.621247in}{0.687137in}}%
\pgfpathlineto{\pgfqpoint{5.682866in}{0.683786in}}%
\pgfpathlineto{\pgfqpoint{5.745563in}{0.706779in}}%
\pgfpathlineto{\pgfqpoint{5.807796in}{0.708353in}}%
\pgfpathlineto{\pgfqpoint{5.869669in}{0.733422in}}%
\pgfpathlineto{\pgfqpoint{5.932524in}{0.724453in}}%
\pgfpathlineto{\pgfqpoint{5.994290in}{0.729206in}}%
\pgfpathlineto{\pgfqpoint{6.056797in}{0.755616in}}%
\pgfpathlineto{\pgfqpoint{6.119044in}{0.717936in}}%
\pgfpathlineto{\pgfqpoint{6.181121in}{0.746625in}}%
\pgfusepath{stroke}%
\end{pgfscope}%
\begin{pgfscope}%
\pgfpathrectangle{\pgfqpoint{0.452903in}{0.398883in}}{\pgfqpoint{5.728219in}{2.431763in}}%
\pgfusepath{clip}%
\pgfsetrectcap%
\pgfsetroundjoin%
\pgfsetlinewidth{1.505625pt}%
\definecolor{currentstroke}{rgb}{1.000000,0.498039,0.054902}%
\pgfsetstrokecolor{currentstroke}%
\pgfsetdash{}{0pt}%
\pgfpathmoveto{\pgfqpoint{0.548466in}{0.561908in}}%
\pgfpathlineto{\pgfqpoint{1.406858in}{0.561930in}}%
\pgfpathlineto{\pgfqpoint{1.864667in}{0.566024in}}%
\pgfpathlineto{\pgfqpoint{1.921893in}{0.567257in}}%
\pgfpathlineto{\pgfqpoint{1.950506in}{0.567951in}}%
\pgfpathlineto{\pgfqpoint{1.979119in}{0.569051in}}%
\pgfpathlineto{\pgfqpoint{2.064958in}{0.579748in}}%
\pgfpathlineto{\pgfqpoint{2.093571in}{0.584748in}}%
\pgfpathlineto{\pgfqpoint{2.265249in}{0.626358in}}%
\pgfpathlineto{\pgfqpoint{2.293862in}{0.650488in}}%
\pgfpathlineto{\pgfqpoint{2.322476in}{0.703565in}}%
\pgfpathlineto{\pgfqpoint{2.351089in}{0.768299in}}%
\pgfpathlineto{\pgfqpoint{2.379702in}{0.840279in}}%
\pgfpathlineto{\pgfqpoint{2.408315in}{0.932802in}}%
\pgfpathlineto{\pgfqpoint{2.436928in}{1.064553in}}%
\pgfpathlineto{\pgfqpoint{2.465541in}{1.242312in}}%
\pgfpathlineto{\pgfqpoint{2.494154in}{1.507391in}}%
\pgfpathlineto{\pgfqpoint{2.522767in}{1.870948in}}%
\pgfpathlineto{\pgfqpoint{2.551380in}{2.265332in}}%
\pgfpathlineto{\pgfqpoint{2.579993in}{2.526676in}}%
\pgfpathlineto{\pgfqpoint{2.608606in}{2.362752in}}%
\pgfpathlineto{\pgfqpoint{2.637219in}{2.004660in}}%
\pgfpathlineto{\pgfqpoint{2.665832in}{1.642997in}}%
\pgfpathlineto{\pgfqpoint{2.694445in}{1.206237in}}%
\pgfpathlineto{\pgfqpoint{2.723058in}{0.982469in}}%
\pgfpathlineto{\pgfqpoint{2.751671in}{0.880228in}}%
\pgfpathlineto{\pgfqpoint{2.808898in}{0.808554in}}%
\pgfpathlineto{\pgfqpoint{2.837511in}{0.778828in}}%
\pgfpathlineto{\pgfqpoint{2.866124in}{0.774023in}}%
\pgfpathlineto{\pgfqpoint{2.951963in}{0.789708in}}%
\pgfpathlineto{\pgfqpoint{2.980576in}{0.792421in}}%
\pgfpathlineto{\pgfqpoint{3.009189in}{0.755667in}}%
\pgfpathlineto{\pgfqpoint{3.066415in}{0.665701in}}%
\pgfpathlineto{\pgfqpoint{3.095028in}{0.650553in}}%
\pgfpathlineto{\pgfqpoint{3.123641in}{0.639047in}}%
\pgfpathlineto{\pgfqpoint{3.209480in}{0.633578in}}%
\pgfpathlineto{\pgfqpoint{3.238093in}{0.627896in}}%
\pgfpathlineto{\pgfqpoint{3.352546in}{0.596342in}}%
\pgfpathlineto{\pgfqpoint{3.381159in}{0.592581in}}%
\pgfpathlineto{\pgfqpoint{3.409772in}{0.589993in}}%
\pgfpathlineto{\pgfqpoint{3.466998in}{0.586689in}}%
\pgfpathlineto{\pgfqpoint{3.552837in}{0.581961in}}%
\pgfpathlineto{\pgfqpoint{3.581450in}{0.581143in}}%
\pgfpathlineto{\pgfqpoint{3.982033in}{0.578362in}}%
\pgfpathlineto{\pgfqpoint{4.125098in}{0.577428in}}%
\pgfpathlineto{\pgfqpoint{4.153711in}{0.577626in}}%
\pgfpathlineto{\pgfqpoint{4.182324in}{0.578865in}}%
\pgfpathlineto{\pgfqpoint{4.382616in}{0.589549in}}%
\pgfpathlineto{\pgfqpoint{4.468455in}{0.594920in}}%
\pgfpathlineto{\pgfqpoint{4.497068in}{0.596743in}}%
\pgfpathlineto{\pgfqpoint{4.525681in}{0.601266in}}%
\pgfpathlineto{\pgfqpoint{4.582907in}{0.610823in}}%
\pgfpathlineto{\pgfqpoint{4.611520in}{0.616854in}}%
\pgfpathlineto{\pgfqpoint{4.811812in}{0.670036in}}%
\pgfpathlineto{\pgfqpoint{4.840425in}{0.699110in}}%
\pgfpathlineto{\pgfqpoint{4.869038in}{0.756308in}}%
\pgfpathlineto{\pgfqpoint{4.897651in}{0.824537in}}%
\pgfpathlineto{\pgfqpoint{4.926264in}{0.912761in}}%
\pgfpathlineto{\pgfqpoint{4.954877in}{1.037896in}}%
\pgfpathlineto{\pgfqpoint{4.983490in}{1.120069in}}%
\pgfpathlineto{\pgfqpoint{5.012103in}{1.157486in}}%
\pgfpathlineto{\pgfqpoint{5.040716in}{1.113510in}}%
\pgfpathlineto{\pgfqpoint{5.097942in}{1.053281in}}%
\pgfpathlineto{\pgfqpoint{5.126555in}{1.043787in}}%
\pgfpathlineto{\pgfqpoint{5.155168in}{1.071558in}}%
\pgfpathlineto{\pgfqpoint{5.183781in}{1.096941in}}%
\pgfpathlineto{\pgfqpoint{5.212394in}{1.085642in}}%
\pgfpathlineto{\pgfqpoint{5.269621in}{0.967464in}}%
\pgfpathlineto{\pgfqpoint{5.298234in}{0.907830in}}%
\pgfpathlineto{\pgfqpoint{5.326847in}{0.854891in}}%
\pgfpathlineto{\pgfqpoint{5.355460in}{0.802996in}}%
\pgfpathlineto{\pgfqpoint{5.384073in}{0.787638in}}%
\pgfpathlineto{\pgfqpoint{5.498525in}{0.734379in}}%
\pgfpathlineto{\pgfqpoint{5.527138in}{0.723280in}}%
\pgfpathlineto{\pgfqpoint{5.555751in}{0.715524in}}%
\pgfpathlineto{\pgfqpoint{5.727430in}{0.671041in}}%
\pgfpathlineto{\pgfqpoint{5.756043in}{0.667049in}}%
\pgfpathlineto{\pgfqpoint{5.784656in}{0.664732in}}%
\pgfpathlineto{\pgfqpoint{5.984947in}{0.655491in}}%
\pgfpathlineto{\pgfqpoint{6.013560in}{0.654650in}}%
\pgfpathlineto{\pgfqpoint{6.042173in}{0.654041in}}%
\pgfpathlineto{\pgfqpoint{6.183121in}{0.652984in}}%
\pgfpathlineto{\pgfqpoint{6.183121in}{0.652984in}}%
\pgfusepath{stroke}%
\end{pgfscope}%
\begin{pgfscope}%
\pgfsetrectcap%
\pgfsetmiterjoin%
\pgfsetlinewidth{0.803000pt}%
\definecolor{currentstroke}{rgb}{0.000000,0.000000,0.000000}%
\pgfsetstrokecolor{currentstroke}%
\pgfsetdash{}{0pt}%
\pgfpathmoveto{\pgfqpoint{0.452903in}{0.398883in}}%
\pgfpathlineto{\pgfqpoint{0.452903in}{2.830646in}}%
\pgfusepath{stroke}%
\end{pgfscope}%
\begin{pgfscope}%
\pgfsetrectcap%
\pgfsetmiterjoin%
\pgfsetlinewidth{0.803000pt}%
\definecolor{currentstroke}{rgb}{0.000000,0.000000,0.000000}%
\pgfsetstrokecolor{currentstroke}%
\pgfsetdash{}{0pt}%
\pgfpathmoveto{\pgfqpoint{6.181121in}{0.398883in}}%
\pgfpathlineto{\pgfqpoint{6.181121in}{2.830646in}}%
\pgfusepath{stroke}%
\end{pgfscope}%
\begin{pgfscope}%
\pgfsetrectcap%
\pgfsetmiterjoin%
\pgfsetlinewidth{0.803000pt}%
\definecolor{currentstroke}{rgb}{0.000000,0.000000,0.000000}%
\pgfsetstrokecolor{currentstroke}%
\pgfsetdash{}{0pt}%
\pgfpathmoveto{\pgfqpoint{0.452903in}{0.398883in}}%
\pgfpathlineto{\pgfqpoint{6.181121in}{0.398883in}}%
\pgfusepath{stroke}%
\end{pgfscope}%
\begin{pgfscope}%
\pgfsetrectcap%
\pgfsetmiterjoin%
\pgfsetlinewidth{0.803000pt}%
\definecolor{currentstroke}{rgb}{0.000000,0.000000,0.000000}%
\pgfsetstrokecolor{currentstroke}%
\pgfsetdash{}{0pt}%
\pgfpathmoveto{\pgfqpoint{0.452903in}{2.830646in}}%
\pgfpathlineto{\pgfqpoint{6.181121in}{2.830646in}}%
\pgfusepath{stroke}%
\end{pgfscope}%
\begin{pgfscope}%
\definecolor{textcolor}{rgb}{0.000000,0.000000,0.000000}%
\pgfsetstrokecolor{textcolor}%
\pgfsetfillcolor{textcolor}%
\pgftext[x=2.644711in,y=2.627999in,left,base]{\color{textcolor}\rmfamily\fontsize{10.000000}{12.000000}\selectfont Gd M5}%
\end{pgfscope}%
\begin{pgfscope}%
\definecolor{textcolor}{rgb}{0.000000,0.000000,0.000000}%
\pgfsetstrokecolor{textcolor}%
\pgfsetfillcolor{textcolor}%
\pgftext[x=5.135060in,y=2.627999in,left,base]{\color{textcolor}\rmfamily\fontsize{10.000000}{12.000000}\selectfont Gd M4}%
\end{pgfscope}%
\begin{pgfscope}%
\definecolor{textcolor}{rgb}{0.000000,0.000000,0.000000}%
\pgfsetstrokecolor{textcolor}%
\pgfsetfillcolor{textcolor}%
\pgftext[x=0.776949in,y=2.627999in,left,base]{\color{textcolor}\rmfamily\fontsize{10.000000}{12.000000}\selectfont Off-Resonanz}%
\end{pgfscope}%
\begin{pgfscope}%
\pgfsetbuttcap%
\pgfsetmiterjoin%
\definecolor{currentfill}{rgb}{1.000000,1.000000,1.000000}%
\pgfsetfillcolor{currentfill}%
\pgfsetfillopacity{0.800000}%
\pgfsetlinewidth{1.003750pt}%
\definecolor{currentstroke}{rgb}{0.800000,0.800000,0.800000}%
\pgfsetstrokecolor{currentstroke}%
\pgfsetstrokeopacity{0.800000}%
\pgfsetdash{}{0pt}%
\pgfpathmoveto{\pgfqpoint{2.771968in}{1.857941in}}%
\pgfpathlineto{\pgfqpoint{4.849956in}{1.857941in}}%
\pgfpathquadraticcurveto{\pgfqpoint{4.877734in}{1.857941in}}{\pgfqpoint{4.877734in}{1.885719in}}%
\pgfpathlineto{\pgfqpoint{4.877734in}{2.285164in}}%
\pgfpathquadraticcurveto{\pgfqpoint{4.877734in}{2.312941in}}{\pgfqpoint{4.849956in}{2.312941in}}%
\pgfpathlineto{\pgfqpoint{2.771968in}{2.312941in}}%
\pgfpathquadraticcurveto{\pgfqpoint{2.744190in}{2.312941in}}{\pgfqpoint{2.744190in}{2.285164in}}%
\pgfpathlineto{\pgfqpoint{2.744190in}{1.885719in}}%
\pgfpathquadraticcurveto{\pgfqpoint{2.744190in}{1.857941in}}{\pgfqpoint{2.771968in}{1.857941in}}%
\pgfpathlineto{\pgfqpoint{2.771968in}{1.857941in}}%
\pgfpathclose%
\pgfusepath{stroke,fill}%
\end{pgfscope}%
\begin{pgfscope}%
\pgfsetrectcap%
\pgfsetroundjoin%
\pgfsetlinewidth{1.505625pt}%
\definecolor{currentstroke}{rgb}{0.121569,0.466667,0.705882}%
\pgfsetstrokecolor{currentstroke}%
\pgfsetdash{}{0pt}%
\pgfpathmoveto{\pgfqpoint{2.799746in}{2.208775in}}%
\pgfpathlineto{\pgfqpoint{2.938635in}{2.208775in}}%
\pgfpathlineto{\pgfqpoint{3.077523in}{2.208775in}}%
\pgfusepath{stroke}%
\end{pgfscope}%
\begin{pgfscope}%
\definecolor{textcolor}{rgb}{0.000000,0.000000,0.000000}%
\pgfsetstrokecolor{textcolor}%
\pgfsetfillcolor{textcolor}%
\pgftext[x=3.188635in,y=2.160164in,left,base]{\color{textcolor}\rmfamily\fontsize{10.000000}{12.000000}\selectfont Messdaten}%
\end{pgfscope}%
\begin{pgfscope}%
\pgfsetrectcap%
\pgfsetroundjoin%
\pgfsetlinewidth{1.505625pt}%
\definecolor{currentstroke}{rgb}{1.000000,0.498039,0.054902}%
\pgfsetstrokecolor{currentstroke}%
\pgfsetdash{}{0pt}%
\pgfpathmoveto{\pgfqpoint{2.799746in}{1.996821in}}%
\pgfpathlineto{\pgfqpoint{2.938635in}{1.996821in}}%
\pgfpathlineto{\pgfqpoint{3.077523in}{1.996821in}}%
\pgfusepath{stroke}%
\end{pgfscope}%
\begin{pgfscope}%
\definecolor{textcolor}{rgb}{0.000000,0.000000,0.000000}%
\pgfsetstrokecolor{textcolor}%
\pgfsetfillcolor{textcolor}%
\pgftext[x=3.188635in,y=1.948210in,left,base]{\color{textcolor}\rmfamily\fontsize{10.000000}{12.000000}\selectfont Referenzwert \(\displaystyle \bar{\beta}\) \cite[Abb. 2]{prieto-x-ray-2005}}%
\end{pgfscope}%
\end{pgfpicture}%
\makeatother%
\endgroup%

    \caption{Absorptionsspektrum nahe der Resonanzphotonenenergien von Gd (blau) der zu untersuchenden Probe und (orange) Referenzwert für Gd.}
    \label{fig:rzp_phi_ev}
\end{figure}
\noindent
Die Streubilder werden somit an der Motorposition $\varphi_\text{\gls{rzp}} = \num{-67}$ aufgenommen, wo der höchste \gls{xmcd}-Kontrast erwartet wird. An der Motorposition $\varphi_\text{\gls{rzp}} = \num{-52}$ wird eine Kontrollmessung mit der Photonenenergie $h\nu_\text{Gd, Off-Res} \approx \SI{1163}{\eV}$ gemacht, die nachweislich weit von beiden Resonanzenergien $h\nu_\text{Gd, M5}$ und $h\nu_\text{Gd, M4}$ entfernt liegt und demzufolge keine resonante magnetische Streuung beobachtet werden soll. Das erwartete Ein-Photon-Signal an der Photonenenergie $h\nu_\text{Gd, Off-Res}$ 
\begin{equation}
    W_\text{Gd, Off-Res} = \SI{176(1)}{\adu}
\end{equation}
ergibt sich nach Gl. (\ref{eq:adu_to_ev}) und unterscheidet sich lediglich um $\SI{2}{\percent}$ von $W_\text{Gd, M5} = \SI{180(1)}{\adu}$, was im Endeffekt dieselben Parameter des Auswertungsverfahrens für die Photonendetektion zulässt.

\section{Punktspreizfunktion einzelner Photonen-Ereignisse}
\label{text:punktspreizfunktion}
Um die Punktspreizfunktion eines isolierten Photons zu untersuchen, werden einzeln detektierte Photonen in einem Gebiet auf dem Detektor von \qtyproduct{100 x 100}{\px} gesucht, in dem der direkte transmittierte Strahl detektiert wurde. Zunächst werden diejenigen Pixel betrachtet, deren Intensitätswerte im Intervall von \SI{80}{\adu} bis \SI{200}{\adu} liegen. (Zur Erinnerung: $W_{\text{Gd, M5}} = \SI{180}{\adu}$). Um eine Überlagerung der Ladungsverteilung benachbarter Photonen-Ereignisse auszuschließen, werden nur diejenigen Photonen-Ereignisse erfasst, die keine Photonen-Ereignisse im \SI{5}{px}-Umkreis haben. Um die Fehldetektion von Pixeln mit statistisch sehr hohen Rauschwerten möglichst zu verringern, wird die zusätzliche Nebenbedingung auferlegt, dass die Summe über die \qtyproduct{3 x 3}{\px}-Umgebung um ein Photonen-Ereignis im Intervall von $(W_{\text{Gd, M5}}-3\sigma_{3\times 3})$ bis $(W_{\text{Gd, M5}}+3\sigma_{3\times 3})$ liegen muss, wobei $\sigma_{3\times 3}$ die Standardabweichung dieser Summe ist und sich als $\sigma_{3\times 3} = \sqrt{9}\sigma_R = 3 \sigma_R$ berechnet.
% \noindent
% Das Quadrat der Standardabweichung (Varianz) $\sigma_\Sigma^2$ von Summe von $i$ Zufallsvariablen, die voneinander nicht abhängen, ergibt sich als die Summe der Varianzen jeder $i$-ten Zufallsvariable $\sigma_i^2$
% \begin{equation}
%     \sigma_\Sigma^2 = \sum_{i} \sigma_i^2.
%     \label{eq:summ_varianz}
% \end{equation}
% \noindent
% Somit kann die Standardabweichung der Summe über \qtyproduct{3 x 3}{\px}-Umgebung
% \begin{equation}
%     \sigma_{3\times 3}^2 = \sum_{i=1}^9\sigma_R^2 \Rightarrow \sigma_{3\times 3} = 3 \sigma_R
% \end{equation}
% berechnet werden.

\noindent
Zum Schluss werden \qtyproduct{5 x 5}{\px}-Bereiche um diejenigen Pixel, welche die Bedingungen oben erfüllen, ausgeschnitten. Die \qtyproduct{5 x 5}{\px}-Bereiche werden über die Anzahl der Bereiche gemittelt und die Standardabweichung jedes Pixels $\sigma_{S+R}$ im \qtyproduct{5 x 5}{\px}-Bereich wird berechnet. Die Standardabweichung $\sigma_{S+R}$ enthält allerdings die Standardabweichung des Detektorrauschens $\sigma_{R}$ und die Standardabweichung der Verteilung des Ein-Photon-Signals $\sigma_{S}$. Diese sind voneinander unabhängig. Daher kann die Standardabweichung $\sigma_{S}$ wie folgt ermittelt werden:
\begin{equation}
    \sigma_{S+R}^2 = \sigma_{S}^2 + \sigma_{R}^2 \Rightarrow \sigma_{S} = \sqrt{\sigma_{S+R}^2 - \sigma_{R}^2}
    \label{eq:std_entkopplung}
\end{equation}
\begin{figure}[H]
    \centering
    \input{images/auswertung/examples_average_std_5x5_hotspot.pgf}
    \caption{(a) Beispiele der isolierten einzelnen Photonen und wie ihre Gesamtintensität über die benachbarten Pixel verteilt wird. Als Bedingung für die Einzelphotonendetektion wurde der Schwellenwert \SI{90}{\adu} von unten und \SI{180}{\adu} von oben festgelegt. Außerdem wird die Nebenbedingung auferlegt, dass die Intensität über einen \qtyproduct{3 x 3}{\px}-Bereich mehr als \SI{161}{\adu} und weniger als \SI{200}{\adu} beträgt, um lediglich einzelne Photonen zu betrachten. Das über 2944 \qtyproduct{5 x 5}{\px}-Bereiche gemittelte Bild (b) zeigt, dass die Intensität hauptsächlich innerhalb des Kreuzes verteilt wird. Die Gesamtintensität des Zentralpixels und des Kreuzes beträgt \SI{179.6}{\adu}, was gut mit dem Erwartungswert für ein Photon mit der Energie $h\nu_\text{Gd, M5}$ übereinstimmt. In (c) ist die Standardabweichung jedes Pixelwertes $\sigma_{S}$ gezeigt, wobei die Standardabweichung des Detektorrauschens $\sigma_R = \SI{19.94}{\adu}$ nach Gl. (\ref{eq:std_entkopplung}) abgezogen wird.}
    \label{fig:examples_average_std_5x5_hotspot}
\end{figure}
\noindent
Es wurden insgesamt 2944 Photonen-Ereignisse gefunden und gemittelt. Zehn Beispiele davon sind in Abb.~\ref{fig:examples_average_std_5x5_hotspot}a dargestellt. Der resultierende gemittelte \qtyproduct{5 x 5}{\px}-Bereich in Abb.~\ref{fig:examples_average_std_5x5_hotspot}b zeigt, dass der ADU-Wert eines Photons schließlich über das zentrale Pixel und auf die angrenzenden Pixel in vertikaler bzw.\ horizontaler Richtung verteilt wird, wobei der ADU-Wert im zentralen Pixel relativ konstant bleibt und 
\begin{equation}
    W_\text{Gd, M5, reell}  = \SI{116(4)}{\adu} 
\end{equation}
beträgt. Die dargestellten Beispiele der \qtyproduct{5 x 5}{\px}-Bereiche in Abb. \ref{fig:examples_average_std_5x5_hotspot}a und die Standardabweichung der Ladungsverteilung $\sigma_{S}$ in Abb.~\ref{fig:examples_average_std_5x5_hotspot}c lässt ablesen, dass das gesamte Ein-Photon-Signal in einem einzelnen Photonen-Ereignis nicht gleichmäßig über alle benachbarten Pixel verteilt wird, sondern nur in einem davon. 

\noindent
In Anbetracht der Tatsache, dass das hellste Pixel nur \SI{116(4)}{\adu} von dem erwarteten gesamten Ein-Photon-Signal $W_\text{GD, M5} = \SI{180(1)}{\adu}$ enthält, würde die weitere Erhöhung des Schwellenwertes $s_V$ über \SI{116}{\adu} zum signifikanten Verlust der Anzahl detektierter Photonen (und folglich von Signal) führen.

\noindent
Aus diesem Grund wird der Schwellenwert $s_V = \SI{100}{\adu}$ in der Auswertung benutzt. Die Absenkung des Schwellenwertes verringert hingegen die Selektivität des Algorithmus.

\section{Auswertung der Streubilder mit Schwellenwert-Algorithmus}
\label{text:streuung_counting}
Die Messdaten wurden zuerst mit dem Schwellenwert-Algorithmus verarbeitet, dessen Funktionsprinzip in Abschnitt \ref{text:threshold_algorithm} beschrieben wurde. Die einzelnen Anwendungsschritte werden exemplarisch in Abb.~\ref{fig:capture_ped_diff} am Beispiel eines aufgenommenen Streubildes mit dem Schwellenwert $s_V = \SI{100}{\adu}$ demonstriert.
\begin{figure}[H]
    \centering
    \input{images/auswertung/capture_ped_diff.pgf}
    \caption{(a) Eine Einzelaufnahme der gestreuten Photonen, (b) ein gemitteltes Bild über \num{10000} Dunkelbilder.  Die resultierende Differenz (c) der ersten beiden Bilder und (d) der angewandte Schwellenwert \SI{100}{\adu}.}
    \label{fig:capture_ped_diff}
\end{figure}
\noindent
Von dem mit dem \gls{moench03} aufgenommenen Streubild (Abb.~\ref{fig:capture_ped_diff}a) wird der konstante Offset (Abb.~\ref{fig:capture_ped_diff}b), der sich als Mittelung von \num{10000} Dunkelbildern ergibt, subtrahiert. Die Pixelwerte in der resultierenden Differenz (Abb.~\ref{fig:capture_ped_diff}c) sind erheblich geringer als in der ursprünglichen Aufnahme des Streubild, was die Wichtigkeit dieser Korrektur des konstanten Offsets unterstreicht. Diejenigen Pixel, deren Werte den Schwellenwert $s_V$ überschreiten, werden in Abb.~\ref{fig:capture_ped_diff}d dargestellt. Wie man erkennt, werden nur sehr wenige Photonen pro Aufnahme (also pro Röntgenpuls) registriert.

\noindent
Im nächsten Abschnitt wird das resultierende Signal-zu-Rausch-Verhältnis in Bezug auf den Schwellenwert $s_V$ diskutiert.

\subsection{Signal-zu-Rausch-Verhältnis}
\label{text:snr}
Um Signal und Rauschen quantitativ auszuwerten, werden einige Begriffe und Hilfsgrößen definiert und ermittelt. Die Zahl der erfassten Aufnahmen wird mit $N_A$ bezeichnet, die Anzahl der Detektorpixel, die für die Auswertung genutzt werden, mit $N_P$. Weiterhin werden die analoge und die digitale Photonenzahl definiert. Die analoge Photonenzahl in einem Bereich der Aufnahme ergibt sich als Quotient aus der Summe aller Pixelwerte in einem Bereich und des Ein-Photon-Signals (in ADU). Diese wurde folgendermaßen ermittelt: Es werden $N_A = 300$ Streubilder gemittelt, um das Detektorrauschen zu verringern. Die Pixelwerte im Bereich $N_P = \qtyproduct{100 x 100}{\px}$ des transmittierten Strahls werden aufsummiert und durch das Ein-Photon-Signal \SI{180}{\adu} geteilt.
Der \gls{photnenfluss} in einem bestimmten Bereich der Aufnahmen ist dann:
\begin{equation}
    \text{\gls{photnenfluss}} = \frac{\text{analoge Photonenzahl}}{N_A N_P}
\end{equation}

\noindent
Die digitale Photonenzahl in einem Bereich ergibt sich als die Gesamtzahl von Photonen, die mit dem Schwellenwert-Algorithmus mit einem bestimmten Wert $s_V$ in dem Bereich detektiert werden. Diese wurde folgendermaßen ermittelt: Es werden dieselben 300 Streubilder genommen und einzeln mit Schwellenwert-Algorithmus ausgewertet. Als nächstes wird die Zahl der detektierten Photonen in demselben \qtyproduct{100 x 100}{\px}-Bereich gezählt. Dieses Verfahren wird für $s_V$ im Intervall von \SI{50}{\adu} bis \SI{160}{\adu} mit Schrittweite \SI{5}{\adu} wiederholt.

\noindent
Ein Pixel kann fälschlicherweise als von einem Photon getroffen bezeichnet werden, wenn das Detektorrauschen in dem Pixel den entsprechenden Schwellenwert $s_V$ überschreitet. Die Zahl von \gls{fdpa} wird als digitale Photonenzahl in \emph{Dunkelbildern} in Bezug auf $s_V$ ermittelt. Diese wurde folgendermaßen ermittelt: Es werden 5000 Dunkelbilder genommen und einzeln mit Schwellenwert-Algorithmus ausgewertet. Alle detektierten Photonen werden als falsch detektiert betrachtet, da es sich ja um Dunkelbilder handelt. Die gesamte Zahl der fehldetektierten Photonen wird auf die gesamte Pixelzahl \qtyproduct{400 x 400}{\px} sowie die Zahl der Aufnahmen normiert. Dieses Verfahren wird für $s_V$ im Intervall von \SI{50}{\adu} bis \SI{160}{\adu} mit der Schrittweite \SI{5}{\adu} wiederholt.

\noindent
Basierend auf diesen Hilfsgrößen wird ein Maß für die Güte des Schwellenwert-Algorithmus definiert, das die \gls{qe} des Schwellenwert-Algorithmus genannt werden soll:
\begin{equation}
    \text{\gls{qe}}(s_V) = \frac{\text{digitale Photonenzahl}(s_V) - N_A N_P \cdot \text{\gls{fdpa}}(s_V)}{\text{analoge Photonenzahl}}
\end{equation}
Im Prinzip gibt die \gls{qe} das Verhältnis zwischen der Anzahl der Photonen, die mit dem Schwellenwert-Algorithmus gefunden wurden, und der tatsächlichen Zahl an detektierten Photonen aus dem analogen Signal an. Da die Zahl der im Mittel fehldetektierten Photonen bekannt ist, wird die digitalen Photonenzahl noch dahingegen korrigiert. Die \gls{qe} und die \gls{fdpa} sind in Abb.~\ref{fig:qe_fehldetektiert_signal_noise} dargestellt. Die Zahl der fehldetektierten Photonen sinkt mit dem Schwellwert erwartungsgemäß rapide ab. Im interessanten Bereich ab einem Schwellenwert von \SI{100}{\adu}, erreicht man Werte von \num{1e-5} falschen Photonen pro Pixel und Aufnahmen. Dieser Wert erscheint zuerst recht klein, kann aber bei Anzahlen von Pixeln und Aufnahmen von jeweils einigen Zehntausend, über die ein Signal gemittelt wird, schnell zu einem erheblichen Untergrund führen. Mit dem Wissen um die Höhe dieses Untergrunds kann dieser aber leicht korrigiert werden.

\noindent
Auch die Quanteneffizienz nimmt mit größerem Schwellenwert ab, da immer mehr Photonen-Ereignisse mit zu kleinem Signal im zentralen Pixel verloren gehen. Idealerweise sollte die \gls{qe} bei eins liegen. Interessanterweise ist sie aber bei einem Schwellenwert unterhalb von \SI{100}{\adu} größer als eins, was darauf hindeutet, dass viele Photonen mehrfach gezählt wurden, da auch die zum zentralen Pixel benachbarten Pixel den Schwellwert überschreiten. Oberhalb eines Schwellenwertes von \SI{100}{\adu} sinkt die \gls{qe} dann unter eins und erreicht schon für $s_V > \SI{150}{\adu}$ einen Wert von fast null.

\noindent
Bei der Auswertung mit dem Schwellenwert-Algorithmus wird das Schrotrauschen wegen des niedrigen detektierten Photonenflusses als die primäre Rauschquelle betrachtet. Die Erwartungswerte des detektierten Signals $S_{\text{EW}}$ und des dazugehörige Schrotrauschens $N_{\text{EW}}$ lassen sich dann folgendermaßen ausdrücken:
\begin{equation}
        S_{\text{EW}}(s_V, N_A, N_P) = N_{A}N_{P}\left[\text{\gls{photnenfluss}}\cdot\text{\gls{qe}}(s_V)\right]
        \label{eq:signal_ew}
\end{equation}
\begin{equation}
        N_{\text{EW}}(s_V, N_A, N_P) = \sqrt{N_{A}N_{P}\left[\text{\gls{photnenfluss}}\cdot\text{\gls{qe}}(s_V) + \text{\gls{fdpa}}(s_V)\right]}
        \label{eq:noise_ew}
\end{equation}
Die Erwartungswerte hängen u.a.\ vom Photonenfluss ab, der sich je nach Bereich in der Aufnahme unterscheiden kann. So ist das Signal-zu-Rausch-Verhältnis in der Aufnahme ortsabhängig. In Abb.~\ref{fig:qe_fehldetektiert_signal_noise} sind auch die Erwartungswerte des Signals und des Schrotrauschens exemplarisch für den zentralen Bereich des Direktstrahls (siehe auch Abb.~\ref{fig:th_50_100_125_150_170_180_200_220_260}) dargestellt.
Die Erwartungswerte $S_{\text{EW}}$ und $N_{\text{EW}}$ betragen im interessanten Bereich von $s_V = \text{\SIrange{100}{125}{\adu}}$ beide ca.\ \SI{1}{\photon}, wenn das Signal in einem Pixel ($N_P = 1$) über 300 Aufnahmen ($N_A = 300$) gemittelt wird. Damit ist für diesen ausgewählten Fall das \gls{snr} auch ungefähr eins. Das \gls{snr} kann verbessert werden, in dem über eine größere Anzahl von Pixeln oder Aufnahmen gemittelt wird. Später wird diese Methode auch zur Bestimmung des \gls{snr} des Streusignals genutzt.

\begin{figure}[H]
    \centering
    \input{images/auswertung/qe_fehldetektiert_signal_noise.pgf}
    \caption{In Bezug auf den Schwellenwert $s_V$ sind (oben links) Quanteneffizienz und (oben rechts) Zahl der fehldetektierten Photonen aufgetragen. Somit werden Erwartungswerte von Signal (unten links) und Rauschen (unten rechts) über der Summe von \SI{300}{\captures} $S/N_{\text{EW}}(s_V, N_A = 300, N_P = 1)$ aufgetragen. Man kann sehen, dass Signal und Rauschen für den Schwellenwert $s_V$ im Intervall \qtyrange{100}{125}{\adu} vergleichbar sind.}
    \label{fig:qe_fehldetektiert_signal_noise}
\end{figure}


\section{Resonante Röntgenkleinwinkelstreuung}
Für die Messung der resonanten Röntgenkleinwinkelstreuung der Fe/Gd-Probe wurden \num{50000} Aufnahmen mit dem Schwellenwert-Algorithmus einzeln ausgewertet und aufsummiert. In Abb.~\ref{fig:th_50_100_125_150_170_180_200_220_260} sind die Summen der ausgewerteten Aufnahmen auf einer logarithmischen Skala dargestellt. Für die Auswertung wurde der Schwellenwert $s_V$ im Intervall von \SIrange{50}{260}{\adu} variiert, um einen Überblick über die Abhängigkeit der Ergebnisse vom Schwellenwert zu geben. 

\noindent
Der helle rechteckige Bereich im Zentrum ist der Direktstrahl, welcher durch das SiN-Fenster mit der Probe transmittiert wird. Das Fenster schneidet einen sehr breitbandigen Bereich aus dem von der \gls{rzp} fokussierten Strahl heraus. Die Photonenenergie ist weiterhin entlang der vertikalen Richtung dispergiert. Deutlich erkennbar ist die Gd M5 Absorptionslinie im Zentrum des Fensters. Nur Photonen aus diesem Bereich sollten resonant zur XMCD-basierten Streuung beitragen.

\begin{figure}[H]
    \centering
    \input{images/auswertung/th_50_100_125_150_170_180_200_220_260.pgf}
    \caption{Anzahl der detektierten Photonen mithilfe des Schwellenwert-Algorithmus mit dem Schwellenwert $s_V$ (a) \SI{50}{\adu}, (b) \SI{100}{\adu}, (c) \SI{125}{\adu}, (d) \SI{150}{\adu}, (e) \SI{170}{\adu}, (f) \SI{180}{\adu}, (g) \SI{200}{\adu}, (h) \SI{220}{\adu} und (g) \SI{260}{\adu}. Aufsummiert wurden jeweils \num{50000} Aufnahmen. Die dunklere horizontale Linie im Zentrum des direkten Strahls entspricht der Gd M5 Absorptionskante.}
    \label{fig:th_50_100_125_150_170_180_200_220_260}
\end{figure}
\noindent
In Abb.~\ref{fig:th_50_100_125_150_170_180_200_220_260}b, c und d kann eindeutig das ringförmigen Streumuster beobachtet werden, das dem erwarteten Streuring der magnetischen Domänen entspricht. Beim Einsatz höherer Schwellenwerte $s_V = \SI{150}{\adu}$, $\SI{170}{\adu}$ und $\SI{180}{\adu}$ (Abb.~\ref{fig:th_50_100_125_150_170_180_200_220_260}d, e und f) wird zwar weniger Rauschen als Photonen fehldetektiert, aber auch die Streuring sind wesentlich kontrastärmer. Für die weitere Auswertung wird ein Schwellenwert von $s_V = \SI{100}{\adu}$ genutzt.

% \noindent
% Bei der Summe von $N_P = 2\pi\cdot\SI{75}{\pixel} \approx \SI{471}{\pixel}$ ergeben sich die folgenden Erwartungswerte
% \begin{equation}
%     \begin{split}
%         S_\text{EW} &= \SI{926}{\photons}\\
%         N_\text{EW} &= \SI{42}{\photons},
%     \end{split}
% \end{equation}
% und das 
% \begin{equation}
%     \text{\gls{snr}} \approx \num{22.28},
% \end{equation}
% \noindent
% das mit dem Wert $\sqrt{471}$ gut übereinstimmt.
\subsection{Auswertung des Streusignals}
Wichtig zu beachten ist, dass die Summen der Aufnahmen $I_\text{ges}$, die mit dem Schwellenwert-Algo\-rith\-mus ausgewertet sind, die fehldetektierten Photonen enthalten. Daher gilt:
\begin{equation}
    I_\text{ges} = N_A N_P \left[\text{\gls{photnenfluss}}\cdot \text{\gls{qe}} + \text{\gls{fdpa}}\right] = \underbrace{N_A N_P \left[\text{\gls{photnenfluss}}\cdot \text{\gls{qe}}\right]}_{S} + N_P N_A\cdot\text{\gls{fdpa}}
\end{equation}
So muss die \gls{fdpa} (vgl. Gl. (\ref{eq:signal_ew})) von jedem Pixel als Offset abgezogen werden, um das eigentliche Signal $S$ zu erhalten:
\begin{equation}
    S = I_\text{ges} - \underbrace{N_A\cdot\text{\gls{fdpa}}(s_V)}_{\Delta_\text{\gls{fdpa}}}
    \label{eq:signal_int_fpda}
\end{equation}
%
\noindent
Es können die Pixel in den Aufnahmen (Abb.~\ref{fig:th_180_450_600}) gefunden werden, in denen sich die Photonen bei der Erhöhung von $s_V$ bis zu \SI{600}{\adu} detektieren lassen. Solche Pixel werden aussortiert, weil es sich dabei um hochenergetische kosmische Strahlung oder um die von der \gls{pxs} emittierten Elektronen handelt.
\begin{figure}[H]
    \centering
    \input{images/auswertung/th_180_450_600.pgf}
    \caption{Die Summe der \num{50000} ausgewerteten Aufnahmen mit Schwellenwert $s_V$ (a) \SI{180}{\adu}, (b) \SI{450}{\adu} und (c) \SI{600}{\adu}. Diejenigen Photonen, die mit allen Schwellenwerten detektiert wurden, sind grün eingekreist.}
    \label{fig:th_180_450_600}
\end{figure}

\noindent
Die ausgewertete Summe von \SI{50000}{\captures} wird in Polarkoordinaten transformiert. Dafür ist es nötig, den Mittelpunkt der Transformation des Koordinatensystems festzulegen. Die resonante magnetische Streuung ist sehr energieselektiv und findet überwiegend an der Resonanzfrequenz statt, welche sich in den Streubildern als eine schmale, horizontale, dunklere Linie im Direktstrahl identifizieren lässt. Die Absorptionslinie bestimmt also die vertikale Position des Mittelpunktes. Die horizontale Koordinate des Mittelpunktes wird durch die Anpassung eines elliptischen Umrisses ermittelt (Abb.~\ref{fig:th-100-200-maske-radial-transform}a).
\begin{figure}[H]
    \centering
    \input{images/auswertung/th_100_200_masked_radial_transform.pgf}
    \caption{(a) Die Summe von \num{50000} ausgewerteten Aufnahmen, die sowohl mit dem Schwellenwert $s_V = \SI{100}{\adu}$ als auch mit der oberen Grenze \SI{600}{\adu} ausgewertet wurden. Davon wird der Offset $\Delta_\text{\gls{fdpa}} = \SI[per-mode = symbol]{1,7}{\photons\per\pixel}$ abgezogen. Der grüne Punkt bezeichnet den Mittelpunkt des elliptischen Umrisses, der für die Transformation zwischen den Koordinatensystemen benutzt wird. Im Bild (b) ist der direkte Strahl ausmaskiert. Das Pfeilende entspricht dem Winkel $\varphi = \SI{0}{\degree}$, die Pfeilrichtung entspricht der positiven Richtung der Azimutalwinkelkoordinate. In Unterabb. (c) ist die in Polarkoordinaten transformierte Unterabb. (b) zu sehen.}
    \label{fig:th-100-200-maske-radial-transform}
\end{figure}

\noindent
Der direkte Strahl wird vor der Transformation ausmaskiert (Abb.~\ref{fig:th-100-200-maske-radial-transform}b), damit er nicht zur Streuintensität des Streurings im Überlappbereich beiträgt.

\noindent
Der Betrag des Streuvektors $q$ kann aus dem Ringradius $r$ berechnet werden über
\begin{equation}
    q(r) = \frac{4\pi}{\lambda_\text{Gd, M5}}\sin\left(\arctan\frac{r}{l}\right).
    \label{eq:streuvektor_von_radius}
\end{equation}

\noindent
Die Intensität in Polarkoordinaten (Abb.~\ref{fig:th-100-200-maske-radial-transform}c) wird über den Azimutwinkel $\varphi$ aufintegriert. Die radiale Intensität wird zunächst über dem Streuvektorbetrag $q$ aufgetragen. Das resultierende Spektrum der Kleinwinkelstreuung wird mit der Gamma-Funktion
\begin{equation}
    g(q, \beta, \alpha, A) = A\frac{q^{\beta-1}\exp\left[-\frac{q}{\alpha}\right]}{\alpha^\beta\Gamma(\beta)}
\end{equation}
angepasst. Diese Verteilungsfunktion eignet sich gut für die Beschreibung der Streuung von magnetischen Wurmdomänen \cite{bagschik_employing_2016}. Das Maximum der Funktion $g(q, \beta, \alpha, A)$ liegt an der Stelle $q_\text{max} = \alpha\beta$. Das Spektrum der Kleinwinkelstreuung sowie die Fit-Funktion sind zusammen mit dem Betragsquadrat der Fourier-Transformierten des MFM-Scan der Probe (Abb.~\ref{fig:mfm-amplitude-ft}d), der als Referenz dient, in Abb.~\ref{fig:radius_fit} dargestellt.
\begin{figure}[H]
    \centering
    %% Creator: Matplotlib, PGF backend
%%
%% To include the figure in your LaTeX document, write
%%   \input{<filename>.pgf}
%%
%% Make sure the required packages are loaded in your preamble
%%   \usepackage{pgf}
%%
%% Also ensure that all the required font packages are loaded; for instance,
%% the lmodern package is sometimes necessary when using math font.
%%   \usepackage{lmodern}
%%
%% Figures using additional raster images can only be included by \input if
%% they are in the same directory as the main LaTeX file. For loading figures
%% from other directories you can use the `import` package
%%   \usepackage{import}
%%
%% and then include the figures with
%%   \import{<path to file>}{<filename>.pgf}
%%
%% Matplotlib used the following preamble
%%   \usepackage{amsmath} \usepackage[utf8]{inputenc} \usepackage[T1]{fontenc} \usepackage[output-decimal-marker={,},print-unity-mantissa=false]{siunitx} \sisetup{per-mode=fraction, separate-uncertainty = true, locale = DE} \usepackage[acronym, toc, section=section, nonumberlist, nopostdot]{glossaries-extra} \DeclareSIUnit\adu{\text{ADU}} \DeclareSIUnit\px{\text{px}} \DeclareSIUnit\photons{\text{Pho\-to\-nen}} \DeclareSIUnit\photon{\text{Pho\-ton}}
%%
\begingroup%
\makeatletter%
\begin{pgfpicture}%
\pgfpathrectangle{\pgfpointorigin}{\pgfqpoint{6.381121in}{3.407926in}}%
\pgfusepath{use as bounding box, clip}%
\begin{pgfscope}%
\pgfsetbuttcap%
\pgfsetmiterjoin%
\pgfsetlinewidth{0.000000pt}%
\definecolor{currentstroke}{rgb}{1.000000,1.000000,1.000000}%
\pgfsetstrokecolor{currentstroke}%
\pgfsetstrokeopacity{0.000000}%
\pgfsetdash{}{0pt}%
\pgfpathmoveto{\pgfqpoint{0.000000in}{0.000000in}}%
\pgfpathlineto{\pgfqpoint{6.381121in}{0.000000in}}%
\pgfpathlineto{\pgfqpoint{6.381121in}{3.407926in}}%
\pgfpathlineto{\pgfqpoint{0.000000in}{3.407926in}}%
\pgfpathlineto{\pgfqpoint{0.000000in}{0.000000in}}%
\pgfpathclose%
\pgfusepath{}%
\end{pgfscope}%
\begin{pgfscope}%
\pgfsetbuttcap%
\pgfsetmiterjoin%
\definecolor{currentfill}{rgb}{1.000000,1.000000,1.000000}%
\pgfsetfillcolor{currentfill}%
\pgfsetlinewidth{0.000000pt}%
\definecolor{currentstroke}{rgb}{0.000000,0.000000,0.000000}%
\pgfsetstrokecolor{currentstroke}%
\pgfsetstrokeopacity{0.000000}%
\pgfsetdash{}{0pt}%
\pgfpathmoveto{\pgfqpoint{0.552903in}{0.522439in}}%
\pgfpathlineto{\pgfqpoint{6.281121in}{0.522439in}}%
\pgfpathlineto{\pgfqpoint{6.281121in}{3.307926in}}%
\pgfpathlineto{\pgfqpoint{0.552903in}{3.307926in}}%
\pgfpathlineto{\pgfqpoint{0.552903in}{0.522439in}}%
\pgfpathclose%
\pgfusepath{fill}%
\end{pgfscope}%
\begin{pgfscope}%
\pgfsetbuttcap%
\pgfsetroundjoin%
\definecolor{currentfill}{rgb}{0.000000,0.000000,0.000000}%
\pgfsetfillcolor{currentfill}%
\pgfsetlinewidth{0.803000pt}%
\definecolor{currentstroke}{rgb}{0.000000,0.000000,0.000000}%
\pgfsetstrokecolor{currentstroke}%
\pgfsetdash{}{0pt}%
\pgfsys@defobject{currentmarker}{\pgfqpoint{0.000000in}{-0.048611in}}{\pgfqpoint{0.000000in}{0.000000in}}{%
\pgfpathmoveto{\pgfqpoint{0.000000in}{0.000000in}}%
\pgfpathlineto{\pgfqpoint{0.000000in}{-0.048611in}}%
\pgfusepath{stroke,fill}%
}%
\begin{pgfscope}%
\pgfsys@transformshift{0.552903in}{0.522439in}%
\pgfsys@useobject{currentmarker}{}%
\end{pgfscope}%
\end{pgfscope}%
\begin{pgfscope}%
\definecolor{textcolor}{rgb}{0.000000,0.000000,0.000000}%
\pgfsetstrokecolor{textcolor}%
\pgfsetfillcolor{textcolor}%
\pgftext[x=0.552903in,y=0.425217in,,top]{\color{textcolor}\rmfamily\fontsize{10.000000}{12.000000}\selectfont \(\displaystyle {0}\)}%
\end{pgfscope}%
\begin{pgfscope}%
\pgfsetbuttcap%
\pgfsetroundjoin%
\definecolor{currentfill}{rgb}{0.000000,0.000000,0.000000}%
\pgfsetfillcolor{currentfill}%
\pgfsetlinewidth{0.803000pt}%
\definecolor{currentstroke}{rgb}{0.000000,0.000000,0.000000}%
\pgfsetstrokecolor{currentstroke}%
\pgfsetdash{}{0pt}%
\pgfsys@defobject{currentmarker}{\pgfqpoint{0.000000in}{-0.048611in}}{\pgfqpoint{0.000000in}{0.000000in}}{%
\pgfpathmoveto{\pgfqpoint{0.000000in}{0.000000in}}%
\pgfpathlineto{\pgfqpoint{0.000000in}{-0.048611in}}%
\pgfusepath{stroke,fill}%
}%
\begin{pgfscope}%
\pgfsys@transformshift{1.437013in}{0.522439in}%
\pgfsys@useobject{currentmarker}{}%
\end{pgfscope}%
\end{pgfscope}%
\begin{pgfscope}%
\definecolor{textcolor}{rgb}{0.000000,0.000000,0.000000}%
\pgfsetstrokecolor{textcolor}%
\pgfsetfillcolor{textcolor}%
\pgftext[x=1.437013in,y=0.425217in,,top]{\color{textcolor}\rmfamily\fontsize{10.000000}{12.000000}\selectfont \(\displaystyle {10}\)}%
\end{pgfscope}%
\begin{pgfscope}%
\pgfsetbuttcap%
\pgfsetroundjoin%
\definecolor{currentfill}{rgb}{0.000000,0.000000,0.000000}%
\pgfsetfillcolor{currentfill}%
\pgfsetlinewidth{0.803000pt}%
\definecolor{currentstroke}{rgb}{0.000000,0.000000,0.000000}%
\pgfsetstrokecolor{currentstroke}%
\pgfsetdash{}{0pt}%
\pgfsys@defobject{currentmarker}{\pgfqpoint{0.000000in}{-0.048611in}}{\pgfqpoint{0.000000in}{0.000000in}}{%
\pgfpathmoveto{\pgfqpoint{0.000000in}{0.000000in}}%
\pgfpathlineto{\pgfqpoint{0.000000in}{-0.048611in}}%
\pgfusepath{stroke,fill}%
}%
\begin{pgfscope}%
\pgfsys@transformshift{2.321124in}{0.522439in}%
\pgfsys@useobject{currentmarker}{}%
\end{pgfscope}%
\end{pgfscope}%
\begin{pgfscope}%
\definecolor{textcolor}{rgb}{0.000000,0.000000,0.000000}%
\pgfsetstrokecolor{textcolor}%
\pgfsetfillcolor{textcolor}%
\pgftext[x=2.321124in,y=0.425217in,,top]{\color{textcolor}\rmfamily\fontsize{10.000000}{12.000000}\selectfont \(\displaystyle {20}\)}%
\end{pgfscope}%
\begin{pgfscope}%
\pgfsetbuttcap%
\pgfsetroundjoin%
\definecolor{currentfill}{rgb}{0.000000,0.000000,0.000000}%
\pgfsetfillcolor{currentfill}%
\pgfsetlinewidth{0.803000pt}%
\definecolor{currentstroke}{rgb}{0.000000,0.000000,0.000000}%
\pgfsetstrokecolor{currentstroke}%
\pgfsetdash{}{0pt}%
\pgfsys@defobject{currentmarker}{\pgfqpoint{0.000000in}{-0.048611in}}{\pgfqpoint{0.000000in}{0.000000in}}{%
\pgfpathmoveto{\pgfqpoint{0.000000in}{0.000000in}}%
\pgfpathlineto{\pgfqpoint{0.000000in}{-0.048611in}}%
\pgfusepath{stroke,fill}%
}%
\begin{pgfscope}%
\pgfsys@transformshift{3.205234in}{0.522439in}%
\pgfsys@useobject{currentmarker}{}%
\end{pgfscope}%
\end{pgfscope}%
\begin{pgfscope}%
\definecolor{textcolor}{rgb}{0.000000,0.000000,0.000000}%
\pgfsetstrokecolor{textcolor}%
\pgfsetfillcolor{textcolor}%
\pgftext[x=3.205234in,y=0.425217in,,top]{\color{textcolor}\rmfamily\fontsize{10.000000}{12.000000}\selectfont \(\displaystyle {30}\)}%
\end{pgfscope}%
\begin{pgfscope}%
\pgfsetbuttcap%
\pgfsetroundjoin%
\definecolor{currentfill}{rgb}{0.000000,0.000000,0.000000}%
\pgfsetfillcolor{currentfill}%
\pgfsetlinewidth{0.803000pt}%
\definecolor{currentstroke}{rgb}{0.000000,0.000000,0.000000}%
\pgfsetstrokecolor{currentstroke}%
\pgfsetdash{}{0pt}%
\pgfsys@defobject{currentmarker}{\pgfqpoint{0.000000in}{-0.048611in}}{\pgfqpoint{0.000000in}{0.000000in}}{%
\pgfpathmoveto{\pgfqpoint{0.000000in}{0.000000in}}%
\pgfpathlineto{\pgfqpoint{0.000000in}{-0.048611in}}%
\pgfusepath{stroke,fill}%
}%
\begin{pgfscope}%
\pgfsys@transformshift{4.089344in}{0.522439in}%
\pgfsys@useobject{currentmarker}{}%
\end{pgfscope}%
\end{pgfscope}%
\begin{pgfscope}%
\definecolor{textcolor}{rgb}{0.000000,0.000000,0.000000}%
\pgfsetstrokecolor{textcolor}%
\pgfsetfillcolor{textcolor}%
\pgftext[x=4.089344in,y=0.425217in,,top]{\color{textcolor}\rmfamily\fontsize{10.000000}{12.000000}\selectfont \(\displaystyle {40}\)}%
\end{pgfscope}%
\begin{pgfscope}%
\pgfsetbuttcap%
\pgfsetroundjoin%
\definecolor{currentfill}{rgb}{0.000000,0.000000,0.000000}%
\pgfsetfillcolor{currentfill}%
\pgfsetlinewidth{0.803000pt}%
\definecolor{currentstroke}{rgb}{0.000000,0.000000,0.000000}%
\pgfsetstrokecolor{currentstroke}%
\pgfsetdash{}{0pt}%
\pgfsys@defobject{currentmarker}{\pgfqpoint{0.000000in}{-0.048611in}}{\pgfqpoint{0.000000in}{0.000000in}}{%
\pgfpathmoveto{\pgfqpoint{0.000000in}{0.000000in}}%
\pgfpathlineto{\pgfqpoint{0.000000in}{-0.048611in}}%
\pgfusepath{stroke,fill}%
}%
\begin{pgfscope}%
\pgfsys@transformshift{4.973455in}{0.522439in}%
\pgfsys@useobject{currentmarker}{}%
\end{pgfscope}%
\end{pgfscope}%
\begin{pgfscope}%
\definecolor{textcolor}{rgb}{0.000000,0.000000,0.000000}%
\pgfsetstrokecolor{textcolor}%
\pgfsetfillcolor{textcolor}%
\pgftext[x=4.973455in,y=0.425217in,,top]{\color{textcolor}\rmfamily\fontsize{10.000000}{12.000000}\selectfont \(\displaystyle {50}\)}%
\end{pgfscope}%
\begin{pgfscope}%
\pgfsetbuttcap%
\pgfsetroundjoin%
\definecolor{currentfill}{rgb}{0.000000,0.000000,0.000000}%
\pgfsetfillcolor{currentfill}%
\pgfsetlinewidth{0.803000pt}%
\definecolor{currentstroke}{rgb}{0.000000,0.000000,0.000000}%
\pgfsetstrokecolor{currentstroke}%
\pgfsetdash{}{0pt}%
\pgfsys@defobject{currentmarker}{\pgfqpoint{0.000000in}{-0.048611in}}{\pgfqpoint{0.000000in}{0.000000in}}{%
\pgfpathmoveto{\pgfqpoint{0.000000in}{0.000000in}}%
\pgfpathlineto{\pgfqpoint{0.000000in}{-0.048611in}}%
\pgfusepath{stroke,fill}%
}%
\begin{pgfscope}%
\pgfsys@transformshift{5.857565in}{0.522439in}%
\pgfsys@useobject{currentmarker}{}%
\end{pgfscope}%
\end{pgfscope}%
\begin{pgfscope}%
\definecolor{textcolor}{rgb}{0.000000,0.000000,0.000000}%
\pgfsetstrokecolor{textcolor}%
\pgfsetfillcolor{textcolor}%
\pgftext[x=5.857565in,y=0.425217in,,top]{\color{textcolor}\rmfamily\fontsize{10.000000}{12.000000}\selectfont \(\displaystyle {60}\)}%
\end{pgfscope}%
\begin{pgfscope}%
\definecolor{textcolor}{rgb}{0.000000,0.000000,0.000000}%
\pgfsetstrokecolor{textcolor}%
\pgfsetfillcolor{textcolor}%
\pgftext[x=3.417012in,y=0.247007in,,top]{\color{textcolor}\rmfamily\fontsize{10.000000}{12.000000}\selectfont Streuvektorbetrag \(\displaystyle q\) in \(\displaystyle \si{\micro\meter^{-1}}\)}%
\end{pgfscope}%
\begin{pgfscope}%
\pgfsetbuttcap%
\pgfsetroundjoin%
\definecolor{currentfill}{rgb}{0.000000,0.000000,0.000000}%
\pgfsetfillcolor{currentfill}%
\pgfsetlinewidth{0.803000pt}%
\definecolor{currentstroke}{rgb}{0.000000,0.000000,0.000000}%
\pgfsetstrokecolor{currentstroke}%
\pgfsetdash{}{0pt}%
\pgfsys@defobject{currentmarker}{\pgfqpoint{-0.048611in}{0.000000in}}{\pgfqpoint{-0.000000in}{0.000000in}}{%
\pgfpathmoveto{\pgfqpoint{-0.000000in}{0.000000in}}%
\pgfpathlineto{\pgfqpoint{-0.048611in}{0.000000in}}%
\pgfusepath{stroke,fill}%
}%
\begin{pgfscope}%
\pgfsys@transformshift{0.552903in}{0.888104in}%
\pgfsys@useobject{currentmarker}{}%
\end{pgfscope}%
\end{pgfscope}%
\begin{pgfscope}%
\definecolor{textcolor}{rgb}{0.000000,0.000000,0.000000}%
\pgfsetstrokecolor{textcolor}%
\pgfsetfillcolor{textcolor}%
\pgftext[x=0.278211in, y=0.840280in, left, base]{\color{textcolor}\rmfamily\fontsize{10.000000}{12.000000}\selectfont \num{0.0}}%
\end{pgfscope}%
\begin{pgfscope}%
\pgfsetbuttcap%
\pgfsetroundjoin%
\definecolor{currentfill}{rgb}{0.000000,0.000000,0.000000}%
\pgfsetfillcolor{currentfill}%
\pgfsetlinewidth{0.803000pt}%
\definecolor{currentstroke}{rgb}{0.000000,0.000000,0.000000}%
\pgfsetstrokecolor{currentstroke}%
\pgfsetdash{}{0pt}%
\pgfsys@defobject{currentmarker}{\pgfqpoint{-0.048611in}{0.000000in}}{\pgfqpoint{-0.000000in}{0.000000in}}{%
\pgfpathmoveto{\pgfqpoint{-0.000000in}{0.000000in}}%
\pgfpathlineto{\pgfqpoint{-0.048611in}{0.000000in}}%
\pgfusepath{stroke,fill}%
}%
\begin{pgfscope}%
\pgfsys@transformshift{0.552903in}{1.656916in}%
\pgfsys@useobject{currentmarker}{}%
\end{pgfscope}%
\end{pgfscope}%
\begin{pgfscope}%
\definecolor{textcolor}{rgb}{0.000000,0.000000,0.000000}%
\pgfsetstrokecolor{textcolor}%
\pgfsetfillcolor{textcolor}%
\pgftext[x=0.278211in, y=1.609091in, left, base]{\color{textcolor}\rmfamily\fontsize{10.000000}{12.000000}\selectfont \num{1.0}}%
\end{pgfscope}%
\begin{pgfscope}%
\pgfsetbuttcap%
\pgfsetroundjoin%
\definecolor{currentfill}{rgb}{0.000000,0.000000,0.000000}%
\pgfsetfillcolor{currentfill}%
\pgfsetlinewidth{0.803000pt}%
\definecolor{currentstroke}{rgb}{0.000000,0.000000,0.000000}%
\pgfsetstrokecolor{currentstroke}%
\pgfsetdash{}{0pt}%
\pgfsys@defobject{currentmarker}{\pgfqpoint{-0.048611in}{0.000000in}}{\pgfqpoint{-0.000000in}{0.000000in}}{%
\pgfpathmoveto{\pgfqpoint{-0.000000in}{0.000000in}}%
\pgfpathlineto{\pgfqpoint{-0.048611in}{0.000000in}}%
\pgfusepath{stroke,fill}%
}%
\begin{pgfscope}%
\pgfsys@transformshift{0.552903in}{2.425727in}%
\pgfsys@useobject{currentmarker}{}%
\end{pgfscope}%
\end{pgfscope}%
\begin{pgfscope}%
\definecolor{textcolor}{rgb}{0.000000,0.000000,0.000000}%
\pgfsetstrokecolor{textcolor}%
\pgfsetfillcolor{textcolor}%
\pgftext[x=0.278211in, y=2.377902in, left, base]{\color{textcolor}\rmfamily\fontsize{10.000000}{12.000000}\selectfont \num{2.0}}%
\end{pgfscope}%
\begin{pgfscope}%
\pgfsetbuttcap%
\pgfsetroundjoin%
\definecolor{currentfill}{rgb}{0.000000,0.000000,0.000000}%
\pgfsetfillcolor{currentfill}%
\pgfsetlinewidth{0.803000pt}%
\definecolor{currentstroke}{rgb}{0.000000,0.000000,0.000000}%
\pgfsetstrokecolor{currentstroke}%
\pgfsetdash{}{0pt}%
\pgfsys@defobject{currentmarker}{\pgfqpoint{-0.048611in}{0.000000in}}{\pgfqpoint{-0.000000in}{0.000000in}}{%
\pgfpathmoveto{\pgfqpoint{-0.000000in}{0.000000in}}%
\pgfpathlineto{\pgfqpoint{-0.048611in}{0.000000in}}%
\pgfusepath{stroke,fill}%
}%
\begin{pgfscope}%
\pgfsys@transformshift{0.552903in}{3.194538in}%
\pgfsys@useobject{currentmarker}{}%
\end{pgfscope}%
\end{pgfscope}%
\begin{pgfscope}%
\definecolor{textcolor}{rgb}{0.000000,0.000000,0.000000}%
\pgfsetstrokecolor{textcolor}%
\pgfsetfillcolor{textcolor}%
\pgftext[x=0.278211in, y=3.146714in, left, base]{\color{textcolor}\rmfamily\fontsize{10.000000}{12.000000}\selectfont \num{3.0}}%
\end{pgfscope}%
\begin{pgfscope}%
\definecolor{textcolor}{rgb}{0.000000,0.000000,0.000000}%
\pgfsetstrokecolor{textcolor}%
\pgfsetfillcolor{textcolor}%
\pgftext[x=0.222655in,y=1.915183in,,bottom,rotate=90.000000]{\color{textcolor}\rmfamily\fontsize{10.000000}{12.000000}\selectfont radiale Streuintensität in bel. Einheiten}%
\end{pgfscope}%
\begin{pgfscope}%
\pgfpathrectangle{\pgfqpoint{0.552903in}{0.522439in}}{\pgfqpoint{5.728219in}{2.785486in}}%
\pgfusepath{clip}%
\pgfsetrectcap%
\pgfsetroundjoin%
\pgfsetlinewidth{1.505625pt}%
\definecolor{currentstroke}{rgb}{1.000000,0.498039,0.054902}%
\pgfsetstrokecolor{currentstroke}%
\pgfsetdash{}{0pt}%
\pgfpathmoveto{\pgfqpoint{1.276949in}{1.222367in}}%
\pgfpathlineto{\pgfqpoint{1.295280in}{1.238391in}}%
\pgfpathlineto{\pgfqpoint{1.313610in}{1.268595in}}%
\pgfpathlineto{\pgfqpoint{1.331940in}{1.304714in}}%
\pgfpathlineto{\pgfqpoint{1.350271in}{1.334841in}}%
\pgfpathlineto{\pgfqpoint{1.368601in}{1.333933in}}%
\pgfpathlineto{\pgfqpoint{1.386931in}{1.334364in}}%
\pgfpathlineto{\pgfqpoint{1.405262in}{1.303767in}}%
\pgfpathlineto{\pgfqpoint{1.423592in}{1.329195in}}%
\pgfpathlineto{\pgfqpoint{1.441922in}{1.349274in}}%
\pgfpathlineto{\pgfqpoint{1.460252in}{1.359410in}}%
\pgfpathlineto{\pgfqpoint{1.478583in}{1.364356in}}%
\pgfpathlineto{\pgfqpoint{1.496913in}{1.374820in}}%
\pgfpathlineto{\pgfqpoint{1.515243in}{1.423103in}}%
\pgfpathlineto{\pgfqpoint{1.533574in}{1.425599in}}%
\pgfpathlineto{\pgfqpoint{1.551904in}{1.461533in}}%
\pgfpathlineto{\pgfqpoint{1.570234in}{1.506985in}}%
\pgfpathlineto{\pgfqpoint{1.588565in}{1.520465in}}%
\pgfpathlineto{\pgfqpoint{1.606895in}{1.552731in}}%
\pgfpathlineto{\pgfqpoint{1.625225in}{1.562756in}}%
\pgfpathlineto{\pgfqpoint{1.643555in}{1.554645in}}%
\pgfpathlineto{\pgfqpoint{1.661886in}{1.561322in}}%
\pgfpathlineto{\pgfqpoint{1.680216in}{1.555547in}}%
\pgfpathlineto{\pgfqpoint{1.698546in}{1.579216in}}%
\pgfpathlineto{\pgfqpoint{1.716877in}{1.708375in}}%
\pgfpathlineto{\pgfqpoint{1.735207in}{1.773770in}}%
\pgfpathlineto{\pgfqpoint{1.753537in}{1.870226in}}%
\pgfpathlineto{\pgfqpoint{1.771868in}{1.913600in}}%
\pgfpathlineto{\pgfqpoint{1.790198in}{1.992099in}}%
\pgfpathlineto{\pgfqpoint{1.808528in}{2.200571in}}%
\pgfpathlineto{\pgfqpoint{1.826858in}{2.331199in}}%
\pgfpathlineto{\pgfqpoint{1.845189in}{2.354311in}}%
\pgfpathlineto{\pgfqpoint{1.863519in}{2.375043in}}%
\pgfpathlineto{\pgfqpoint{1.881849in}{2.437775in}}%
\pgfpathlineto{\pgfqpoint{1.900180in}{2.544203in}}%
\pgfpathlineto{\pgfqpoint{1.918510in}{2.587929in}}%
\pgfpathlineto{\pgfqpoint{1.936840in}{2.503247in}}%
\pgfpathlineto{\pgfqpoint{1.955171in}{2.559904in}}%
\pgfpathlineto{\pgfqpoint{1.973501in}{2.547973in}}%
\pgfpathlineto{\pgfqpoint{1.991831in}{2.615501in}}%
\pgfpathlineto{\pgfqpoint{2.010161in}{2.751887in}}%
\pgfpathlineto{\pgfqpoint{2.028492in}{2.814614in}}%
\pgfpathlineto{\pgfqpoint{2.046822in}{2.863683in}}%
\pgfpathlineto{\pgfqpoint{2.065152in}{2.952108in}}%
\pgfpathlineto{\pgfqpoint{2.083483in}{3.020847in}}%
\pgfpathlineto{\pgfqpoint{2.101813in}{2.933786in}}%
\pgfpathlineto{\pgfqpoint{2.120143in}{2.832564in}}%
\pgfpathlineto{\pgfqpoint{2.138474in}{2.723731in}}%
\pgfpathlineto{\pgfqpoint{2.156804in}{2.681129in}}%
\pgfpathlineto{\pgfqpoint{2.175134in}{2.616891in}}%
\pgfpathlineto{\pgfqpoint{2.193464in}{2.582685in}}%
\pgfpathlineto{\pgfqpoint{2.211795in}{2.504624in}}%
\pgfpathlineto{\pgfqpoint{2.230125in}{2.427686in}}%
\pgfpathlineto{\pgfqpoint{2.248455in}{2.288815in}}%
\pgfpathlineto{\pgfqpoint{2.266786in}{2.185570in}}%
\pgfpathlineto{\pgfqpoint{2.285116in}{2.040496in}}%
\pgfpathlineto{\pgfqpoint{2.303446in}{1.942840in}}%
\pgfpathlineto{\pgfqpoint{2.321777in}{1.897559in}}%
\pgfpathlineto{\pgfqpoint{2.340107in}{1.850715in}}%
\pgfpathlineto{\pgfqpoint{2.358437in}{1.643127in}}%
\pgfpathlineto{\pgfqpoint{2.376767in}{1.567377in}}%
\pgfpathlineto{\pgfqpoint{2.395098in}{1.508940in}}%
\pgfpathlineto{\pgfqpoint{2.413428in}{1.458675in}}%
\pgfpathlineto{\pgfqpoint{2.431758in}{1.450565in}}%
\pgfpathlineto{\pgfqpoint{2.450089in}{1.346251in}}%
\pgfpathlineto{\pgfqpoint{2.468419in}{1.271344in}}%
\pgfpathlineto{\pgfqpoint{2.486749in}{1.243014in}}%
\pgfpathlineto{\pgfqpoint{2.505080in}{1.203321in}}%
\pgfpathlineto{\pgfqpoint{2.523410in}{1.188704in}}%
\pgfpathlineto{\pgfqpoint{2.541740in}{1.169301in}}%
\pgfpathlineto{\pgfqpoint{2.560070in}{1.128996in}}%
\pgfpathlineto{\pgfqpoint{2.578401in}{1.105615in}}%
\pgfpathlineto{\pgfqpoint{2.596731in}{1.074203in}}%
\pgfpathlineto{\pgfqpoint{2.615061in}{1.053277in}}%
\pgfpathlineto{\pgfqpoint{2.633392in}{1.039171in}}%
\pgfpathlineto{\pgfqpoint{2.651722in}{1.026698in}}%
\pgfpathlineto{\pgfqpoint{2.670052in}{1.009196in}}%
\pgfpathlineto{\pgfqpoint{2.688383in}{1.002542in}}%
\pgfpathlineto{\pgfqpoint{2.706713in}{0.989250in}}%
\pgfpathlineto{\pgfqpoint{2.725043in}{0.982044in}}%
\pgfpathlineto{\pgfqpoint{2.743373in}{0.977785in}}%
\pgfpathlineto{\pgfqpoint{2.780034in}{0.962543in}}%
\pgfpathlineto{\pgfqpoint{2.798364in}{0.954222in}}%
\pgfpathlineto{\pgfqpoint{2.816695in}{0.949674in}}%
\pgfpathlineto{\pgfqpoint{2.835025in}{0.946672in}}%
\pgfpathlineto{\pgfqpoint{2.853355in}{0.941357in}}%
\pgfpathlineto{\pgfqpoint{2.871686in}{0.940230in}}%
\pgfpathlineto{\pgfqpoint{2.890016in}{0.936253in}}%
\pgfpathlineto{\pgfqpoint{2.908346in}{0.931596in}}%
\pgfpathlineto{\pgfqpoint{2.926676in}{0.928109in}}%
\pgfpathlineto{\pgfqpoint{2.945007in}{0.927945in}}%
\pgfpathlineto{\pgfqpoint{2.963337in}{0.925053in}}%
\pgfpathlineto{\pgfqpoint{2.981667in}{0.923391in}}%
\pgfpathlineto{\pgfqpoint{2.999998in}{0.920789in}}%
\pgfpathlineto{\pgfqpoint{3.018328in}{0.920322in}}%
\pgfpathlineto{\pgfqpoint{3.036658in}{0.919079in}}%
\pgfpathlineto{\pgfqpoint{3.073319in}{0.917096in}}%
\pgfpathlineto{\pgfqpoint{3.091649in}{0.914881in}}%
\pgfpathlineto{\pgfqpoint{3.109979in}{0.913589in}}%
\pgfpathlineto{\pgfqpoint{3.128310in}{0.911885in}}%
\pgfpathlineto{\pgfqpoint{3.146640in}{0.910636in}}%
\pgfpathlineto{\pgfqpoint{3.164970in}{0.909076in}}%
\pgfpathlineto{\pgfqpoint{3.183301in}{0.908028in}}%
\pgfpathlineto{\pgfqpoint{3.201631in}{0.906540in}}%
\pgfpathlineto{\pgfqpoint{3.219961in}{0.906422in}}%
\pgfpathlineto{\pgfqpoint{3.256622in}{0.905791in}}%
\pgfpathlineto{\pgfqpoint{3.274952in}{0.905511in}}%
\pgfpathlineto{\pgfqpoint{3.293282in}{0.904871in}}%
\pgfpathlineto{\pgfqpoint{3.329943in}{0.904680in}}%
\pgfpathlineto{\pgfqpoint{3.348273in}{0.905664in}}%
\pgfpathlineto{\pgfqpoint{3.366604in}{0.905608in}}%
\pgfpathlineto{\pgfqpoint{3.384934in}{0.906042in}}%
\pgfpathlineto{\pgfqpoint{3.403264in}{0.905710in}}%
\pgfpathlineto{\pgfqpoint{3.421595in}{0.905022in}}%
\pgfpathlineto{\pgfqpoint{3.439925in}{0.904717in}}%
\pgfpathlineto{\pgfqpoint{3.531576in}{0.901530in}}%
\pgfpathlineto{\pgfqpoint{3.549907in}{0.901184in}}%
\pgfpathlineto{\pgfqpoint{3.586567in}{0.898502in}}%
\pgfpathlineto{\pgfqpoint{3.659888in}{0.897475in}}%
\pgfpathlineto{\pgfqpoint{3.678219in}{0.897718in}}%
\pgfpathlineto{\pgfqpoint{3.733210in}{0.897002in}}%
\pgfpathlineto{\pgfqpoint{3.788200in}{0.896202in}}%
\pgfpathlineto{\pgfqpoint{3.806531in}{0.896047in}}%
\pgfpathlineto{\pgfqpoint{3.824861in}{0.896168in}}%
\pgfpathlineto{\pgfqpoint{3.861522in}{0.895908in}}%
\pgfpathlineto{\pgfqpoint{3.879852in}{0.895980in}}%
\pgfpathlineto{\pgfqpoint{3.916513in}{0.895551in}}%
\pgfpathlineto{\pgfqpoint{3.989834in}{0.894952in}}%
\pgfpathlineto{\pgfqpoint{4.026494in}{0.895266in}}%
\pgfpathlineto{\pgfqpoint{4.044825in}{0.895414in}}%
\pgfpathlineto{\pgfqpoint{4.099816in}{0.894760in}}%
\pgfpathlineto{\pgfqpoint{4.118146in}{0.894675in}}%
\pgfpathlineto{\pgfqpoint{4.173137in}{0.893748in}}%
\pgfpathlineto{\pgfqpoint{4.228128in}{0.894122in}}%
\pgfpathlineto{\pgfqpoint{4.264788in}{0.893551in}}%
\pgfpathlineto{\pgfqpoint{4.301449in}{0.893251in}}%
\pgfpathlineto{\pgfqpoint{4.429761in}{0.892483in}}%
\pgfpathlineto{\pgfqpoint{4.448091in}{0.892252in}}%
\pgfpathlineto{\pgfqpoint{4.484752in}{0.891359in}}%
\pgfpathlineto{\pgfqpoint{4.539743in}{0.891011in}}%
\pgfpathlineto{\pgfqpoint{4.924679in}{0.889670in}}%
\pgfpathlineto{\pgfqpoint{5.034661in}{0.889701in}}%
\pgfpathlineto{\pgfqpoint{5.107982in}{0.889549in}}%
\pgfpathlineto{\pgfqpoint{5.254624in}{0.889629in}}%
\pgfpathlineto{\pgfqpoint{5.291285in}{0.889270in}}%
\pgfpathlineto{\pgfqpoint{5.511249in}{0.889532in}}%
\pgfpathlineto{\pgfqpoint{5.566240in}{0.889633in}}%
\pgfpathlineto{\pgfqpoint{5.639561in}{0.889479in}}%
\pgfpathlineto{\pgfqpoint{5.804533in}{0.889649in}}%
\pgfpathlineto{\pgfqpoint{5.841194in}{0.890017in}}%
\pgfpathlineto{\pgfqpoint{5.896185in}{0.890097in}}%
\pgfpathlineto{\pgfqpoint{5.932846in}{0.890270in}}%
\pgfpathlineto{\pgfqpoint{5.969506in}{0.890302in}}%
\pgfpathlineto{\pgfqpoint{5.987836in}{0.890589in}}%
\pgfpathlineto{\pgfqpoint{6.042827in}{0.890589in}}%
\pgfpathlineto{\pgfqpoint{6.042827in}{0.890589in}}%
\pgfusepath{stroke}%
\end{pgfscope}%
\begin{pgfscope}%
\pgfpathrectangle{\pgfqpoint{0.552903in}{0.522439in}}{\pgfqpoint{5.728219in}{2.785486in}}%
\pgfusepath{clip}%
\pgfsetrectcap%
\pgfsetroundjoin%
\pgfsetlinewidth{1.003750pt}%
\definecolor{currentstroke}{rgb}{0.000000,0.000000,0.000000}%
\pgfsetstrokecolor{currentstroke}%
\pgfsetdash{}{0pt}%
\pgfpathmoveto{\pgfqpoint{0.552903in}{0.888104in}}%
\pgfpathlineto{\pgfqpoint{6.281121in}{0.888104in}}%
\pgfusepath{stroke}%
\end{pgfscope}%
\begin{pgfscope}%
\pgfpathrectangle{\pgfqpoint{0.552903in}{0.522439in}}{\pgfqpoint{5.728219in}{2.785486in}}%
\pgfusepath{clip}%
\pgfsetbuttcap%
\pgfsetroundjoin%
\definecolor{currentfill}{rgb}{0.121569,0.466667,0.705882}%
\pgfsetfillcolor{currentfill}%
\pgfsetfillopacity{0.600000}%
\pgfsetlinewidth{0.000000pt}%
\definecolor{currentstroke}{rgb}{0.121569,0.466667,0.705882}%
\pgfsetstrokecolor{currentstroke}%
\pgfsetstrokeopacity{0.600000}%
\pgfsetdash{}{0pt}%
\pgfsys@defobject{currentmarker}{\pgfqpoint{-0.020833in}{-0.020833in}}{\pgfqpoint{0.020833in}{0.020833in}}{%
\pgfpathmoveto{\pgfqpoint{0.000000in}{-0.020833in}}%
\pgfpathcurveto{\pgfqpoint{0.005525in}{-0.020833in}}{\pgfqpoint{0.010825in}{-0.018638in}}{\pgfqpoint{0.014731in}{-0.014731in}}%
\pgfpathcurveto{\pgfqpoint{0.018638in}{-0.010825in}}{\pgfqpoint{0.020833in}{-0.005525in}}{\pgfqpoint{0.020833in}{0.000000in}}%
\pgfpathcurveto{\pgfqpoint{0.020833in}{0.005525in}}{\pgfqpoint{0.018638in}{0.010825in}}{\pgfqpoint{0.014731in}{0.014731in}}%
\pgfpathcurveto{\pgfqpoint{0.010825in}{0.018638in}}{\pgfqpoint{0.005525in}{0.020833in}}{\pgfqpoint{0.000000in}{0.020833in}}%
\pgfpathcurveto{\pgfqpoint{-0.005525in}{0.020833in}}{\pgfqpoint{-0.010825in}{0.018638in}}{\pgfqpoint{-0.014731in}{0.014731in}}%
\pgfpathcurveto{\pgfqpoint{-0.018638in}{0.010825in}}{\pgfqpoint{-0.020833in}{0.005525in}}{\pgfqpoint{-0.020833in}{0.000000in}}%
\pgfpathcurveto{\pgfqpoint{-0.020833in}{-0.005525in}}{\pgfqpoint{-0.018638in}{-0.010825in}}{\pgfqpoint{-0.014731in}{-0.014731in}}%
\pgfpathcurveto{\pgfqpoint{-0.010825in}{-0.018638in}}{\pgfqpoint{-0.005525in}{-0.020833in}}{\pgfqpoint{0.000000in}{-0.020833in}}%
\pgfpathlineto{\pgfqpoint{0.000000in}{-0.020833in}}%
\pgfpathclose%
\pgfusepath{fill}%
}%
\begin{pgfscope}%
\pgfsys@transformshift{1.529447in}{2.045821in}%
\pgfsys@useobject{currentmarker}{}%
\end{pgfscope}%
\begin{pgfscope}%
\pgfsys@transformshift{1.555174in}{2.252256in}%
\pgfsys@useobject{currentmarker}{}%
\end{pgfscope}%
\begin{pgfscope}%
\pgfsys@transformshift{1.580902in}{2.349282in}%
\pgfsys@useobject{currentmarker}{}%
\end{pgfscope}%
\begin{pgfscope}%
\pgfsys@transformshift{1.606630in}{2.488949in}%
\pgfsys@useobject{currentmarker}{}%
\end{pgfscope}%
\begin{pgfscope}%
\pgfsys@transformshift{1.632357in}{2.499948in}%
\pgfsys@useobject{currentmarker}{}%
\end{pgfscope}%
\begin{pgfscope}%
\pgfsys@transformshift{1.658085in}{2.425526in}%
\pgfsys@useobject{currentmarker}{}%
\end{pgfscope}%
\begin{pgfscope}%
\pgfsys@transformshift{1.683813in}{2.538199in}%
\pgfsys@useobject{currentmarker}{}%
\end{pgfscope}%
\begin{pgfscope}%
\pgfsys@transformshift{1.709540in}{2.912674in}%
\pgfsys@useobject{currentmarker}{}%
\end{pgfscope}%
\begin{pgfscope}%
\pgfsys@transformshift{1.735268in}{2.500038in}%
\pgfsys@useobject{currentmarker}{}%
\end{pgfscope}%
\begin{pgfscope}%
\pgfsys@transformshift{1.760995in}{2.865009in}%
\pgfsys@useobject{currentmarker}{}%
\end{pgfscope}%
\begin{pgfscope}%
\pgfsys@transformshift{1.786723in}{2.572226in}%
\pgfsys@useobject{currentmarker}{}%
\end{pgfscope}%
\begin{pgfscope}%
\pgfsys@transformshift{1.812451in}{2.839372in}%
\pgfsys@useobject{currentmarker}{}%
\end{pgfscope}%
\begin{pgfscope}%
\pgfsys@transformshift{1.838178in}{3.041011in}%
\pgfsys@useobject{currentmarker}{}%
\end{pgfscope}%
\begin{pgfscope}%
\pgfsys@transformshift{1.863906in}{2.905185in}%
\pgfsys@useobject{currentmarker}{}%
\end{pgfscope}%
\begin{pgfscope}%
\pgfsys@transformshift{1.889634in}{2.844486in}%
\pgfsys@useobject{currentmarker}{}%
\end{pgfscope}%
\begin{pgfscope}%
\pgfsys@transformshift{1.915361in}{2.892429in}%
\pgfsys@useobject{currentmarker}{}%
\end{pgfscope}%
\begin{pgfscope}%
\pgfsys@transformshift{1.941089in}{3.059807in}%
\pgfsys@useobject{currentmarker}{}%
\end{pgfscope}%
\begin{pgfscope}%
\pgfsys@transformshift{1.966816in}{3.107434in}%
\pgfsys@useobject{currentmarker}{}%
\end{pgfscope}%
\begin{pgfscope}%
\pgfsys@transformshift{1.992544in}{3.129715in}%
\pgfsys@useobject{currentmarker}{}%
\end{pgfscope}%
\begin{pgfscope}%
\pgfsys@transformshift{2.018272in}{2.919554in}%
\pgfsys@useobject{currentmarker}{}%
\end{pgfscope}%
\begin{pgfscope}%
\pgfsys@transformshift{2.043999in}{2.981874in}%
\pgfsys@useobject{currentmarker}{}%
\end{pgfscope}%
\begin{pgfscope}%
\pgfsys@transformshift{2.069727in}{2.893111in}%
\pgfsys@useobject{currentmarker}{}%
\end{pgfscope}%
\begin{pgfscope}%
\pgfsys@transformshift{2.095454in}{2.754323in}%
\pgfsys@useobject{currentmarker}{}%
\end{pgfscope}%
\begin{pgfscope}%
\pgfsys@transformshift{2.121182in}{2.838068in}%
\pgfsys@useobject{currentmarker}{}%
\end{pgfscope}%
\begin{pgfscope}%
\pgfsys@transformshift{2.146910in}{3.181313in}%
\pgfsys@useobject{currentmarker}{}%
\end{pgfscope}%
\begin{pgfscope}%
\pgfsys@transformshift{2.172637in}{2.911011in}%
\pgfsys@useobject{currentmarker}{}%
\end{pgfscope}%
\begin{pgfscope}%
\pgfsys@transformshift{2.198365in}{2.616940in}%
\pgfsys@useobject{currentmarker}{}%
\end{pgfscope}%
\begin{pgfscope}%
\pgfsys@transformshift{2.224093in}{2.700094in}%
\pgfsys@useobject{currentmarker}{}%
\end{pgfscope}%
\begin{pgfscope}%
\pgfsys@transformshift{2.249820in}{2.656708in}%
\pgfsys@useobject{currentmarker}{}%
\end{pgfscope}%
\begin{pgfscope}%
\pgfsys@transformshift{2.275548in}{2.865694in}%
\pgfsys@useobject{currentmarker}{}%
\end{pgfscope}%
\begin{pgfscope}%
\pgfsys@transformshift{2.301275in}{2.909518in}%
\pgfsys@useobject{currentmarker}{}%
\end{pgfscope}%
\begin{pgfscope}%
\pgfsys@transformshift{2.327003in}{2.700180in}%
\pgfsys@useobject{currentmarker}{}%
\end{pgfscope}%
\begin{pgfscope}%
\pgfsys@transformshift{2.352731in}{2.833419in}%
\pgfsys@useobject{currentmarker}{}%
\end{pgfscope}%
\begin{pgfscope}%
\pgfsys@transformshift{2.378458in}{2.699667in}%
\pgfsys@useobject{currentmarker}{}%
\end{pgfscope}%
\begin{pgfscope}%
\pgfsys@transformshift{2.404186in}{2.487387in}%
\pgfsys@useobject{currentmarker}{}%
\end{pgfscope}%
\begin{pgfscope}%
\pgfsys@transformshift{2.429913in}{2.566527in}%
\pgfsys@useobject{currentmarker}{}%
\end{pgfscope}%
\begin{pgfscope}%
\pgfsys@transformshift{2.455641in}{2.055424in}%
\pgfsys@useobject{currentmarker}{}%
\end{pgfscope}%
\begin{pgfscope}%
\pgfsys@transformshift{2.481369in}{2.150280in}%
\pgfsys@useobject{currentmarker}{}%
\end{pgfscope}%
\begin{pgfscope}%
\pgfsys@transformshift{2.507096in}{2.517765in}%
\pgfsys@useobject{currentmarker}{}%
\end{pgfscope}%
\begin{pgfscope}%
\pgfsys@transformshift{2.532824in}{2.708042in}%
\pgfsys@useobject{currentmarker}{}%
\end{pgfscope}%
\begin{pgfscope}%
\pgfsys@transformshift{2.558552in}{2.410446in}%
\pgfsys@useobject{currentmarker}{}%
\end{pgfscope}%
\begin{pgfscope}%
\pgfsys@transformshift{2.584279in}{2.173230in}%
\pgfsys@useobject{currentmarker}{}%
\end{pgfscope}%
\begin{pgfscope}%
\pgfsys@transformshift{2.610007in}{2.022470in}%
\pgfsys@useobject{currentmarker}{}%
\end{pgfscope}%
\begin{pgfscope}%
\pgfsys@transformshift{2.635734in}{1.741474in}%
\pgfsys@useobject{currentmarker}{}%
\end{pgfscope}%
\begin{pgfscope}%
\pgfsys@transformshift{2.661462in}{1.972332in}%
\pgfsys@useobject{currentmarker}{}%
\end{pgfscope}%
\begin{pgfscope}%
\pgfsys@transformshift{2.687190in}{1.892777in}%
\pgfsys@useobject{currentmarker}{}%
\end{pgfscope}%
\begin{pgfscope}%
\pgfsys@transformshift{2.712917in}{1.686416in}%
\pgfsys@useobject{currentmarker}{}%
\end{pgfscope}%
\begin{pgfscope}%
\pgfsys@transformshift{2.738645in}{1.792405in}%
\pgfsys@useobject{currentmarker}{}%
\end{pgfscope}%
\begin{pgfscope}%
\pgfsys@transformshift{2.764373in}{1.959948in}%
\pgfsys@useobject{currentmarker}{}%
\end{pgfscope}%
\begin{pgfscope}%
\pgfsys@transformshift{2.790100in}{1.827511in}%
\pgfsys@useobject{currentmarker}{}%
\end{pgfscope}%
\begin{pgfscope}%
\pgfsys@transformshift{2.815828in}{1.639653in}%
\pgfsys@useobject{currentmarker}{}%
\end{pgfscope}%
\begin{pgfscope}%
\pgfsys@transformshift{2.841555in}{1.308858in}%
\pgfsys@useobject{currentmarker}{}%
\end{pgfscope}%
\begin{pgfscope}%
\pgfsys@transformshift{2.867283in}{1.667118in}%
\pgfsys@useobject{currentmarker}{}%
\end{pgfscope}%
\begin{pgfscope}%
\pgfsys@transformshift{2.893011in}{1.429541in}%
\pgfsys@useobject{currentmarker}{}%
\end{pgfscope}%
\begin{pgfscope}%
\pgfsys@transformshift{2.918738in}{1.399759in}%
\pgfsys@useobject{currentmarker}{}%
\end{pgfscope}%
\begin{pgfscope}%
\pgfsys@transformshift{2.944466in}{1.126047in}%
\pgfsys@useobject{currentmarker}{}%
\end{pgfscope}%
\begin{pgfscope}%
\pgfsys@transformshift{2.970193in}{1.182455in}%
\pgfsys@useobject{currentmarker}{}%
\end{pgfscope}%
\begin{pgfscope}%
\pgfsys@transformshift{2.995921in}{1.430534in}%
\pgfsys@useobject{currentmarker}{}%
\end{pgfscope}%
\begin{pgfscope}%
\pgfsys@transformshift{3.021649in}{1.305813in}%
\pgfsys@useobject{currentmarker}{}%
\end{pgfscope}%
\begin{pgfscope}%
\pgfsys@transformshift{3.047376in}{1.359893in}%
\pgfsys@useobject{currentmarker}{}%
\end{pgfscope}%
\begin{pgfscope}%
\pgfsys@transformshift{3.073104in}{1.229345in}%
\pgfsys@useobject{currentmarker}{}%
\end{pgfscope}%
\begin{pgfscope}%
\pgfsys@transformshift{3.098832in}{1.117187in}%
\pgfsys@useobject{currentmarker}{}%
\end{pgfscope}%
\begin{pgfscope}%
\pgfsys@transformshift{3.124559in}{1.539769in}%
\pgfsys@useobject{currentmarker}{}%
\end{pgfscope}%
\begin{pgfscope}%
\pgfsys@transformshift{3.150287in}{1.436005in}%
\pgfsys@useobject{currentmarker}{}%
\end{pgfscope}%
\begin{pgfscope}%
\pgfsys@transformshift{3.176014in}{1.191317in}%
\pgfsys@useobject{currentmarker}{}%
\end{pgfscope}%
\begin{pgfscope}%
\pgfsys@transformshift{3.201742in}{1.164530in}%
\pgfsys@useobject{currentmarker}{}%
\end{pgfscope}%
\begin{pgfscope}%
\pgfsys@transformshift{3.227470in}{1.184781in}%
\pgfsys@useobject{currentmarker}{}%
\end{pgfscope}%
\begin{pgfscope}%
\pgfsys@transformshift{3.253197in}{1.351235in}%
\pgfsys@useobject{currentmarker}{}%
\end{pgfscope}%
\begin{pgfscope}%
\pgfsys@transformshift{3.278925in}{1.262235in}%
\pgfsys@useobject{currentmarker}{}%
\end{pgfscope}%
\begin{pgfscope}%
\pgfsys@transformshift{3.304653in}{0.950762in}%
\pgfsys@useobject{currentmarker}{}%
\end{pgfscope}%
\begin{pgfscope}%
\pgfsys@transformshift{3.330380in}{1.204326in}%
\pgfsys@useobject{currentmarker}{}%
\end{pgfscope}%
\begin{pgfscope}%
\pgfsys@transformshift{3.356108in}{1.174978in}%
\pgfsys@useobject{currentmarker}{}%
\end{pgfscope}%
\begin{pgfscope}%
\pgfsys@transformshift{3.381835in}{1.346370in}%
\pgfsys@useobject{currentmarker}{}%
\end{pgfscope}%
\begin{pgfscope}%
\pgfsys@transformshift{3.407563in}{1.136291in}%
\pgfsys@useobject{currentmarker}{}%
\end{pgfscope}%
\begin{pgfscope}%
\pgfsys@transformshift{3.433291in}{0.963504in}%
\pgfsys@useobject{currentmarker}{}%
\end{pgfscope}%
\begin{pgfscope}%
\pgfsys@transformshift{3.459018in}{0.961397in}%
\pgfsys@useobject{currentmarker}{}%
\end{pgfscope}%
\begin{pgfscope}%
\pgfsys@transformshift{3.484746in}{0.970959in}%
\pgfsys@useobject{currentmarker}{}%
\end{pgfscope}%
\begin{pgfscope}%
\pgfsys@transformshift{3.510473in}{1.009575in}%
\pgfsys@useobject{currentmarker}{}%
\end{pgfscope}%
\begin{pgfscope}%
\pgfsys@transformshift{3.536201in}{1.071635in}%
\pgfsys@useobject{currentmarker}{}%
\end{pgfscope}%
\begin{pgfscope}%
\pgfsys@transformshift{3.561929in}{1.039655in}%
\pgfsys@useobject{currentmarker}{}%
\end{pgfscope}%
\begin{pgfscope}%
\pgfsys@transformshift{3.587656in}{0.868036in}%
\pgfsys@useobject{currentmarker}{}%
\end{pgfscope}%
\begin{pgfscope}%
\pgfsys@transformshift{3.613384in}{0.928107in}%
\pgfsys@useobject{currentmarker}{}%
\end{pgfscope}%
\begin{pgfscope}%
\pgfsys@transformshift{3.639112in}{0.769280in}%
\pgfsys@useobject{currentmarker}{}%
\end{pgfscope}%
\begin{pgfscope}%
\pgfsys@transformshift{3.664839in}{0.959837in}%
\pgfsys@useobject{currentmarker}{}%
\end{pgfscope}%
\begin{pgfscope}%
\pgfsys@transformshift{3.690567in}{0.830084in}%
\pgfsys@useobject{currentmarker}{}%
\end{pgfscope}%
\begin{pgfscope}%
\pgfsys@transformshift{3.716294in}{0.890513in}%
\pgfsys@useobject{currentmarker}{}%
\end{pgfscope}%
\begin{pgfscope}%
\pgfsys@transformshift{3.742022in}{0.855083in}%
\pgfsys@useobject{currentmarker}{}%
\end{pgfscope}%
\begin{pgfscope}%
\pgfsys@transformshift{3.767750in}{1.163515in}%
\pgfsys@useobject{currentmarker}{}%
\end{pgfscope}%
\begin{pgfscope}%
\pgfsys@transformshift{3.793477in}{1.145366in}%
\pgfsys@useobject{currentmarker}{}%
\end{pgfscope}%
\begin{pgfscope}%
\pgfsys@transformshift{3.819205in}{1.244720in}%
\pgfsys@useobject{currentmarker}{}%
\end{pgfscope}%
\begin{pgfscope}%
\pgfsys@transformshift{3.844933in}{0.897409in}%
\pgfsys@useobject{currentmarker}{}%
\end{pgfscope}%
\begin{pgfscope}%
\pgfsys@transformshift{3.870660in}{0.976929in}%
\pgfsys@useobject{currentmarker}{}%
\end{pgfscope}%
\begin{pgfscope}%
\pgfsys@transformshift{3.896388in}{0.691537in}%
\pgfsys@useobject{currentmarker}{}%
\end{pgfscope}%
\begin{pgfscope}%
\pgfsys@transformshift{3.922115in}{0.839267in}%
\pgfsys@useobject{currentmarker}{}%
\end{pgfscope}%
\begin{pgfscope}%
\pgfsys@transformshift{3.947843in}{0.923649in}%
\pgfsys@useobject{currentmarker}{}%
\end{pgfscope}%
\begin{pgfscope}%
\pgfsys@transformshift{3.973571in}{0.839738in}%
\pgfsys@useobject{currentmarker}{}%
\end{pgfscope}%
\begin{pgfscope}%
\pgfsys@transformshift{3.999298in}{0.649052in}%
\pgfsys@useobject{currentmarker}{}%
\end{pgfscope}%
\begin{pgfscope}%
\pgfsys@transformshift{4.025026in}{0.821077in}%
\pgfsys@useobject{currentmarker}{}%
\end{pgfscope}%
\begin{pgfscope}%
\pgfsys@transformshift{4.050753in}{0.783748in}%
\pgfsys@useobject{currentmarker}{}%
\end{pgfscope}%
\begin{pgfscope}%
\pgfsys@transformshift{4.076481in}{0.738004in}%
\pgfsys@useobject{currentmarker}{}%
\end{pgfscope}%
\end{pgfscope}%
\begin{pgfscope}%
\pgfpathrectangle{\pgfqpoint{0.552903in}{0.522439in}}{\pgfqpoint{5.728219in}{2.785486in}}%
\pgfusepath{clip}%
\pgfsetrectcap%
\pgfsetroundjoin%
\pgfsetlinewidth{1.505625pt}%
\definecolor{currentstroke}{rgb}{0.121569,0.466667,0.705882}%
\pgfsetstrokecolor{currentstroke}%
\pgfsetdash{}{0pt}%
\pgfpathmoveto{\pgfqpoint{1.529447in}{2.095936in}}%
\pgfpathlineto{\pgfqpoint{1.555174in}{2.182480in}}%
\pgfpathlineto{\pgfqpoint{1.580902in}{2.267505in}}%
\pgfpathlineto{\pgfqpoint{1.606630in}{2.350336in}}%
\pgfpathlineto{\pgfqpoint{1.632357in}{2.430333in}}%
\pgfpathlineto{\pgfqpoint{1.658085in}{2.506899in}}%
\pgfpathlineto{\pgfqpoint{1.683813in}{2.579484in}}%
\pgfpathlineto{\pgfqpoint{1.709540in}{2.647594in}}%
\pgfpathlineto{\pgfqpoint{1.735268in}{2.710791in}}%
\pgfpathlineto{\pgfqpoint{1.760995in}{2.768702in}}%
\pgfpathlineto{\pgfqpoint{1.786723in}{2.821012in}}%
\pgfpathlineto{\pgfqpoint{1.812451in}{2.867475in}}%
\pgfpathlineto{\pgfqpoint{1.838178in}{2.907904in}}%
\pgfpathlineto{\pgfqpoint{1.863906in}{2.942176in}}%
\pgfpathlineto{\pgfqpoint{1.889634in}{2.970226in}}%
\pgfpathlineto{\pgfqpoint{1.915361in}{2.992048in}}%
\pgfpathlineto{\pgfqpoint{1.941089in}{3.007688in}}%
\pgfpathlineto{\pgfqpoint{1.966816in}{3.017240in}}%
\pgfpathlineto{\pgfqpoint{1.992544in}{3.020847in}}%
\pgfpathlineto{\pgfqpoint{2.018272in}{3.018689in}}%
\pgfpathlineto{\pgfqpoint{2.043999in}{3.010983in}}%
\pgfpathlineto{\pgfqpoint{2.069727in}{2.997979in}}%
\pgfpathlineto{\pgfqpoint{2.095454in}{2.979951in}}%
\pgfpathlineto{\pgfqpoint{2.121182in}{2.957197in}}%
\pgfpathlineto{\pgfqpoint{2.146910in}{2.930030in}}%
\pgfpathlineto{\pgfqpoint{2.172637in}{2.898779in}}%
\pgfpathlineto{\pgfqpoint{2.198365in}{2.863781in}}%
\pgfpathlineto{\pgfqpoint{2.224093in}{2.825376in}}%
\pgfpathlineto{\pgfqpoint{2.249820in}{2.783909in}}%
\pgfpathlineto{\pgfqpoint{2.275548in}{2.739721in}}%
\pgfpathlineto{\pgfqpoint{2.301275in}{2.693149in}}%
\pgfpathlineto{\pgfqpoint{2.327003in}{2.644526in}}%
\pgfpathlineto{\pgfqpoint{2.352731in}{2.594170in}}%
\pgfpathlineto{\pgfqpoint{2.378458in}{2.542393in}}%
\pgfpathlineto{\pgfqpoint{2.404186in}{2.489492in}}%
\pgfpathlineto{\pgfqpoint{2.429913in}{2.435748in}}%
\pgfpathlineto{\pgfqpoint{2.455641in}{2.381431in}}%
\pgfpathlineto{\pgfqpoint{2.481369in}{2.326790in}}%
\pgfpathlineto{\pgfqpoint{2.507096in}{2.272061in}}%
\pgfpathlineto{\pgfqpoint{2.532824in}{2.217461in}}%
\pgfpathlineto{\pgfqpoint{2.558552in}{2.163191in}}%
\pgfpathlineto{\pgfqpoint{2.584279in}{2.109433in}}%
\pgfpathlineto{\pgfqpoint{2.610007in}{2.056353in}}%
\pgfpathlineto{\pgfqpoint{2.635734in}{2.004101in}}%
\pgfpathlineto{\pgfqpoint{2.661462in}{1.952809in}}%
\pgfpathlineto{\pgfqpoint{2.687190in}{1.902595in}}%
\pgfpathlineto{\pgfqpoint{2.712917in}{1.853560in}}%
\pgfpathlineto{\pgfqpoint{2.738645in}{1.805792in}}%
\pgfpathlineto{\pgfqpoint{2.764373in}{1.759365in}}%
\pgfpathlineto{\pgfqpoint{2.790100in}{1.714338in}}%
\pgfpathlineto{\pgfqpoint{2.815828in}{1.670762in}}%
\pgfpathlineto{\pgfqpoint{2.841555in}{1.628673in}}%
\pgfpathlineto{\pgfqpoint{2.867283in}{1.588098in}}%
\pgfpathlineto{\pgfqpoint{2.893011in}{1.549053in}}%
\pgfpathlineto{\pgfqpoint{2.918738in}{1.511548in}}%
\pgfpathlineto{\pgfqpoint{2.944466in}{1.475583in}}%
\pgfpathlineto{\pgfqpoint{2.970193in}{1.441150in}}%
\pgfpathlineto{\pgfqpoint{2.995921in}{1.408236in}}%
\pgfpathlineto{\pgfqpoint{3.021649in}{1.376822in}}%
\pgfpathlineto{\pgfqpoint{3.047376in}{1.346882in}}%
\pgfpathlineto{\pgfqpoint{3.073104in}{1.318389in}}%
\pgfpathlineto{\pgfqpoint{3.098832in}{1.291310in}}%
\pgfpathlineto{\pgfqpoint{3.124559in}{1.265608in}}%
\pgfpathlineto{\pgfqpoint{3.150287in}{1.241246in}}%
\pgfpathlineto{\pgfqpoint{3.176014in}{1.218181in}}%
\pgfpathlineto{\pgfqpoint{3.201742in}{1.196371in}}%
\pgfpathlineto{\pgfqpoint{3.227470in}{1.175772in}}%
\pgfpathlineto{\pgfqpoint{3.253197in}{1.156340in}}%
\pgfpathlineto{\pgfqpoint{3.278925in}{1.138028in}}%
\pgfpathlineto{\pgfqpoint{3.304653in}{1.120790in}}%
\pgfpathlineto{\pgfqpoint{3.330380in}{1.104582in}}%
\pgfpathlineto{\pgfqpoint{3.356108in}{1.089356in}}%
\pgfpathlineto{\pgfqpoint{3.381835in}{1.075067in}}%
\pgfpathlineto{\pgfqpoint{3.407563in}{1.061671in}}%
\pgfpathlineto{\pgfqpoint{3.433291in}{1.049125in}}%
\pgfpathlineto{\pgfqpoint{3.459018in}{1.037384in}}%
\pgfpathlineto{\pgfqpoint{3.484746in}{1.026408in}}%
\pgfpathlineto{\pgfqpoint{3.510473in}{1.016155in}}%
\pgfpathlineto{\pgfqpoint{3.536201in}{1.006586in}}%
\pgfpathlineto{\pgfqpoint{3.561929in}{0.997663in}}%
\pgfpathlineto{\pgfqpoint{3.587656in}{0.989350in}}%
\pgfpathlineto{\pgfqpoint{3.613384in}{0.981610in}}%
\pgfpathlineto{\pgfqpoint{3.639112in}{0.974411in}}%
\pgfpathlineto{\pgfqpoint{3.664839in}{0.967719in}}%
\pgfpathlineto{\pgfqpoint{3.690567in}{0.961503in}}%
\pgfpathlineto{\pgfqpoint{3.716294in}{0.955734in}}%
\pgfpathlineto{\pgfqpoint{3.742022in}{0.950384in}}%
\pgfpathlineto{\pgfqpoint{3.767750in}{0.945425in}}%
\pgfpathlineto{\pgfqpoint{3.793477in}{0.940832in}}%
\pgfpathlineto{\pgfqpoint{3.819205in}{0.936582in}}%
\pgfpathlineto{\pgfqpoint{3.844933in}{0.932651in}}%
\pgfpathlineto{\pgfqpoint{3.870660in}{0.929017in}}%
\pgfpathlineto{\pgfqpoint{3.896388in}{0.925661in}}%
\pgfpathlineto{\pgfqpoint{3.922115in}{0.922562in}}%
\pgfpathlineto{\pgfqpoint{3.947843in}{0.919704in}}%
\pgfpathlineto{\pgfqpoint{3.973571in}{0.917069in}}%
\pgfpathlineto{\pgfqpoint{3.999298in}{0.914641in}}%
\pgfpathlineto{\pgfqpoint{4.025026in}{0.912404in}}%
\pgfpathlineto{\pgfqpoint{4.050753in}{0.910346in}}%
\pgfpathlineto{\pgfqpoint{4.076481in}{0.908453in}}%
\pgfusepath{stroke}%
\end{pgfscope}%
\begin{pgfscope}%
\pgfsetrectcap%
\pgfsetmiterjoin%
\pgfsetlinewidth{0.803000pt}%
\definecolor{currentstroke}{rgb}{0.000000,0.000000,0.000000}%
\pgfsetstrokecolor{currentstroke}%
\pgfsetdash{}{0pt}%
\pgfpathmoveto{\pgfqpoint{0.552903in}{0.522439in}}%
\pgfpathlineto{\pgfqpoint{0.552903in}{3.307926in}}%
\pgfusepath{stroke}%
\end{pgfscope}%
\begin{pgfscope}%
\pgfsetrectcap%
\pgfsetmiterjoin%
\pgfsetlinewidth{0.803000pt}%
\definecolor{currentstroke}{rgb}{0.000000,0.000000,0.000000}%
\pgfsetstrokecolor{currentstroke}%
\pgfsetdash{}{0pt}%
\pgfpathmoveto{\pgfqpoint{6.281121in}{0.522439in}}%
\pgfpathlineto{\pgfqpoint{6.281121in}{3.307926in}}%
\pgfusepath{stroke}%
\end{pgfscope}%
\begin{pgfscope}%
\pgfsetrectcap%
\pgfsetmiterjoin%
\pgfsetlinewidth{0.803000pt}%
\definecolor{currentstroke}{rgb}{0.000000,0.000000,0.000000}%
\pgfsetstrokecolor{currentstroke}%
\pgfsetdash{}{0pt}%
\pgfpathmoveto{\pgfqpoint{0.552903in}{0.522439in}}%
\pgfpathlineto{\pgfqpoint{6.281121in}{0.522439in}}%
\pgfusepath{stroke}%
\end{pgfscope}%
\begin{pgfscope}%
\pgfsetrectcap%
\pgfsetmiterjoin%
\pgfsetlinewidth{0.803000pt}%
\definecolor{currentstroke}{rgb}{0.000000,0.000000,0.000000}%
\pgfsetstrokecolor{currentstroke}%
\pgfsetdash{}{0pt}%
\pgfpathmoveto{\pgfqpoint{0.552903in}{3.307926in}}%
\pgfpathlineto{\pgfqpoint{6.281121in}{3.307926in}}%
\pgfusepath{stroke}%
\end{pgfscope}%
\begin{pgfscope}%
\pgfsetbuttcap%
\pgfsetmiterjoin%
\definecolor{currentfill}{rgb}{1.000000,1.000000,1.000000}%
\pgfsetfillcolor{currentfill}%
\pgfsetfillopacity{0.800000}%
\pgfsetlinewidth{1.003750pt}%
\definecolor{currentstroke}{rgb}{0.800000,0.800000,0.800000}%
\pgfsetstrokecolor{currentstroke}%
\pgfsetstrokeopacity{0.800000}%
\pgfsetdash{}{0pt}%
\pgfpathmoveto{\pgfqpoint{3.061868in}{2.077314in}}%
\pgfpathlineto{\pgfqpoint{6.183899in}{2.077314in}}%
\pgfpathquadraticcurveto{\pgfqpoint{6.211677in}{2.077314in}}{\pgfqpoint{6.211677in}{2.105092in}}%
\pgfpathlineto{\pgfqpoint{6.211677in}{3.210704in}}%
\pgfpathquadraticcurveto{\pgfqpoint{6.211677in}{3.238481in}}{\pgfqpoint{6.183899in}{3.238481in}}%
\pgfpathlineto{\pgfqpoint{3.061868in}{3.238481in}}%
\pgfpathquadraticcurveto{\pgfqpoint{3.034090in}{3.238481in}}{\pgfqpoint{3.034090in}{3.210704in}}%
\pgfpathlineto{\pgfqpoint{3.034090in}{2.105092in}}%
\pgfpathquadraticcurveto{\pgfqpoint{3.034090in}{2.077314in}}{\pgfqpoint{3.061868in}{2.077314in}}%
\pgfpathlineto{\pgfqpoint{3.061868in}{2.077314in}}%
\pgfpathclose%
\pgfusepath{stroke,fill}%
\end{pgfscope}%
\begin{pgfscope}%
\pgfsetrectcap%
\pgfsetroundjoin%
\pgfsetlinewidth{1.505625pt}%
\definecolor{currentstroke}{rgb}{1.000000,0.498039,0.054902}%
\pgfsetstrokecolor{currentstroke}%
\pgfsetdash{}{0pt}%
\pgfpathmoveto{\pgfqpoint{3.089646in}{3.042540in}}%
\pgfpathlineto{\pgfqpoint{3.228534in}{3.042540in}}%
\pgfpathlineto{\pgfqpoint{3.367423in}{3.042540in}}%
\pgfusepath{stroke}%
\end{pgfscope}%
\begin{pgfscope}%
\definecolor{textcolor}{rgb}{0.000000,0.000000,0.000000}%
\pgfsetstrokecolor{textcolor}%
\pgfsetfillcolor{textcolor}%
\pgftext[x=3.478534in, y=3.087270in, left, base]{\color{textcolor}\rmfamily\fontsize{10.000000}{12.000000}\selectfont Betragsquadrat der Fourier-Transformierten}%
\end{pgfscope}%
\begin{pgfscope}%
\definecolor{textcolor}{rgb}{0.000000,0.000000,0.000000}%
\pgfsetstrokecolor{textcolor}%
\pgfsetfillcolor{textcolor}%
\pgftext[x=3.478534in, y=2.935301in, left, base]{\color{textcolor}\rmfamily\fontsize{10.000000}{12.000000}\selectfont  vom MFM-Scan der Probe (Abb. \ref{fig:mfm-amplitude-ft}d)}%
\end{pgfscope}%
\begin{pgfscope}%
\pgfsetbuttcap%
\pgfsetroundjoin%
\definecolor{currentfill}{rgb}{0.121569,0.466667,0.705882}%
\pgfsetfillcolor{currentfill}%
\pgfsetfillopacity{0.600000}%
\pgfsetlinewidth{0.000000pt}%
\definecolor{currentstroke}{rgb}{0.121569,0.466667,0.705882}%
\pgfsetstrokecolor{currentstroke}%
\pgfsetstrokeopacity{0.600000}%
\pgfsetdash{}{0pt}%
\pgfsys@defobject{currentmarker}{\pgfqpoint{-0.020833in}{-0.020833in}}{\pgfqpoint{0.020833in}{0.020833in}}{%
\pgfpathmoveto{\pgfqpoint{0.000000in}{-0.020833in}}%
\pgfpathcurveto{\pgfqpoint{0.005525in}{-0.020833in}}{\pgfqpoint{0.010825in}{-0.018638in}}{\pgfqpoint{0.014731in}{-0.014731in}}%
\pgfpathcurveto{\pgfqpoint{0.018638in}{-0.010825in}}{\pgfqpoint{0.020833in}{-0.005525in}}{\pgfqpoint{0.020833in}{0.000000in}}%
\pgfpathcurveto{\pgfqpoint{0.020833in}{0.005525in}}{\pgfqpoint{0.018638in}{0.010825in}}{\pgfqpoint{0.014731in}{0.014731in}}%
\pgfpathcurveto{\pgfqpoint{0.010825in}{0.018638in}}{\pgfqpoint{0.005525in}{0.020833in}}{\pgfqpoint{0.000000in}{0.020833in}}%
\pgfpathcurveto{\pgfqpoint{-0.005525in}{0.020833in}}{\pgfqpoint{-0.010825in}{0.018638in}}{\pgfqpoint{-0.014731in}{0.014731in}}%
\pgfpathcurveto{\pgfqpoint{-0.018638in}{0.010825in}}{\pgfqpoint{-0.020833in}{0.005525in}}{\pgfqpoint{-0.020833in}{0.000000in}}%
\pgfpathcurveto{\pgfqpoint{-0.020833in}{-0.005525in}}{\pgfqpoint{-0.018638in}{-0.010825in}}{\pgfqpoint{-0.014731in}{-0.014731in}}%
\pgfpathcurveto{\pgfqpoint{-0.010825in}{-0.018638in}}{\pgfqpoint{-0.005525in}{-0.020833in}}{\pgfqpoint{0.000000in}{-0.020833in}}%
\pgfpathlineto{\pgfqpoint{0.000000in}{-0.020833in}}%
\pgfpathclose%
\pgfusepath{fill}%
}%
\begin{pgfscope}%
\pgfsys@transformshift{3.228534in}{2.690757in}%
\pgfsys@useobject{currentmarker}{}%
\end{pgfscope}%
\end{pgfscope}%
\begin{pgfscope}%
\definecolor{textcolor}{rgb}{0.000000,0.000000,0.000000}%
\pgfsetstrokecolor{textcolor}%
\pgfsetfillcolor{textcolor}%
\pgftext[x=3.478534in, y=2.735487in, left, base]{\color{textcolor}\rmfamily\fontsize{10.000000}{12.000000}\selectfont ermitteltes Signal}%
\end{pgfscope}%
\begin{pgfscope}%
\definecolor{textcolor}{rgb}{0.000000,0.000000,0.000000}%
\pgfsetstrokecolor{textcolor}%
\pgfsetfillcolor{textcolor}%
\pgftext[x=3.478534in, y=2.583518in, left, base]{\color{textcolor}\rmfamily\fontsize{10.000000}{12.000000}\selectfont  mit maskiertem Direktstrahl (Abb. \ref{fig:th-100-200-maske-radial-transform}b)}%
\end{pgfscope}%
\begin{pgfscope}%
\pgfsetrectcap%
\pgfsetroundjoin%
\pgfsetlinewidth{1.505625pt}%
\definecolor{currentstroke}{rgb}{0.121569,0.466667,0.705882}%
\pgfsetstrokecolor{currentstroke}%
\pgfsetdash{}{0pt}%
\pgfpathmoveto{\pgfqpoint{3.089646in}{2.306898in}}%
\pgfpathlineto{\pgfqpoint{3.367423in}{2.306898in}}%
\pgfusepath{stroke}%
\end{pgfscope}%
\begin{pgfscope}%
\definecolor{textcolor}{rgb}{0.000000,0.000000,0.000000}%
\pgfsetstrokecolor{textcolor}%
\pgfsetfillcolor{textcolor}%
\pgftext[x=3.478534in, y=2.375193in, left, base]{\color{textcolor}\rmfamily\fontsize{10.000000}{12.000000}\selectfont Fit \(\displaystyle g(q)\) vom Signal}%
\end{pgfscope}%
\begin{pgfscope}%
\definecolor{textcolor}{rgb}{0.000000,0.000000,0.000000}%
\pgfsetstrokecolor{textcolor}%
\pgfsetfillcolor{textcolor}%
\pgftext[x=3.478534in, y=2.199664in, left, base]{\color{textcolor}\rmfamily\fontsize{10.000000}{12.000000}\selectfont mit \(\displaystyle q_\text{max}\) = \SI{18(2)}{\per\micro\meter}}%
\end{pgfscope}%
\end{pgfpicture}%
\makeatother%
\endgroup%

    \caption{azimutal (hellblaue Punkte) aufintegrierte Streuintensität und (orange Linie) die Fourier-Transformierte der magnetischen Struktur der Probe (Abb. \ref{fig:mfm-amplitude-ft}d) mit dem Maximum bei $q =\SI{18(1)}{\per\micro\meter}$. Die radiale Streuintensität wird mit (blaue Linie) Fit $g(q)$ mit den Parametern $\beta = \SI{9.5(5)}{}$, $\alpha = \SI{1.93(9)}{}$ und $A = \SI{39.4(8)}{}$ angepasst. Das Maximum des Fits liegt bei $q_\text{max} = \SI{18(2)}{\per\micro\meter}$.}
    \label{fig:radius_fit}
\end{figure}
\noindent
Die MFM-Daten und die Streudaten liegen im $q$-Spektrum sehr nah bei einander. Die gemessene Intensitätsverteilung hat eine größere radiale Ausdehnung des Streurings als die Fourier-Transformierte des MFM-Scan der Probe. Dieses Phänomen lässt sich mit der großen Breite des Strahlprofils erklären. Über die Faltung des Streusignals mit dieser Ausdehnung des Strahls kommt es auch zur Verbreiterung des Spektrums der Kleinwinkelstreuung. Das Maximum der Fourier-transformierten MFM-Daten liegt innerhalb des Fit-Vertrauensbereiches der Röntgenstreuung.

\noindent
In Analogie zu Abschnitt \ref{text:snr} werden die Erwartungswerte $S_{\text{EW}}$ und $N_{\text{EW}}$ je Pixel ($N_P = 1$) für die ausgewerteten Summen von $N_A = \SI{50000}{\captures}$ im Bereich des Streuringes bestimmt.
\begin{equation}
    \begin{split}
        S_\text{EW} &= \SI{1.97}{\photons}\\
        N_\text{EW} &= \SI{1.91}{\photons}\\
    \end{split}
\end{equation}
Damit ist das \gls{snr} je Pixel ungefähr eins und \gls{snr} $\approx \sqrt{N_P}$, wenn über $N_P$ Pixel gemittelt wird. Der mittlere Umfang des Streurings beträgt ca.\ $2\pi\cdot\SI{75}{\px}$. So wird das \gls{snr} mit der azimutalen Integration von eins auf ca.\ 22 für jeden Datenpunkt im Durchschnitt in Abb.~\ref{fig:radius_fit} erhöht.

\subsection{Nicht-resonante Messung}
Es wird eine Kontrollmessung bei einer nicht-resonanten Photonenenergie $h\nu_{\text{Gd, Off-Res}} \approx \SI{1163}{\eV}$ durchgeführt, um zu beweisen, dass ein resonanter magnetischer Effekt der beobachteten Intensitätsverteilung zugrunde liegt.

\noindent
Es werden ebenfalls \SI{50000}{\captures} aufgenommen und mit dem Schwellenwert-Algorithmus ausgewertet, wobei die Schwellenwerte $s_V$ im Intervall von \SIrange{50}{180}{\adu} variiert werden, damit sich die Ergebnisse mit den Ergebnissen an der resonanten Photonenenergie (Abb. \ref{fig:th_50_100_125_150_170_180_200_220_260}) vergleichen lassen.
\begin{figure}[H]
    \centering
    \input{images/auswertung/th_50_100_125_150_170_180_off_resonance.pgf}
    \caption{Anzahl der detektierten Photonen, ausgewertet mithilfe des Schwellenwert-Algorithmus bei den Schwellenwerten $s_V$ (a) \SI{50}{\adu}, (b) \SI{100}{\adu}, (c) \SI{125}{\adu}, (d) \SI{150}{\adu}, (e) \SI{170}{\adu} und (f) \SI{180}{\adu}. Aufsummiert werden jeweils \num{50000} Aufnahmen; aufgenommen bei der Photonenenergie $h\nu_\text{Gd, Off-Res} \approx \SI{1163}{\eV}$.}
    \label{fig:th_50_100_125_150_170_180_off_resonance}
\end{figure}
\noindent
Bis zu einem Schwellenwert von \SI{125}{\adu} wird eine fast homogene Intensitätsverteilung beobachtet, die auf die Probentopographie (Rauigkeit und Kristallinität von Dünnschichten und Substrat) sowie fehldetektierte Photonen zurückzuführen ist. Oberhalb dieses Schwellenwertes wird kaum noch Intensität außerhalb des Direktstrahls detektiert. Ein Streuring ist bei keinem Schwellenwert erkennbar.

\section{Auswertung mit Clustering-Algorithmus}
Die Analyse der Punktspreizfunktion eines isolierten Photons (siehe Abschnitt \ref{text:punktspreizfunktion}) demonstriert, dass bis zu \SI{36}{\percent} des Ein-Photon-Signals außerhalb des zentralen Pixels liegt. In dem Fall kann der Ansatz des Clustering-Algorithmus, der in Abschnitt \ref{text:clustering_algorithm} beschrieben wurde, vorteilhaft sein, um die Gesamtladung eines Photons zurückzugewinnen und dadurch die Sensitivität der Photonenerkennung zu erhöhen.

\noindent
Unter Berücksichtigung der Ergebnisse der durchgeführten Analyse und zwar, dass der gesamte \gls{adu}-Wert eines Photons grundsätzlich innerhalb zweier benachbarter Pixel verteilt wird, wird der Cluster-Kern
\begin{equation}
    \mathbf{K}_2 = \begin{bmatrix}
1 & 1\\
1 & 1
\end{bmatrix}
\end{equation}
benutzt.

\noindent
Die Einzelschritte der Anwendung des Clustering-Algorithmus sind exemplarisch in Abb.~\ref{fig:capture_ped_diff_clustering} am Beispiel desselben Streubildes dargestellt, das zur Demonstration der Einzelschritte des Schwellen\-wert-Algorithmus in Abb.~\ref{fig:capture_ped_diff} benutzt wurde.
\begin{figure}[H]
    \centering
    \input{images/auswertung/capture_ped_diff_clustering.pgf}
    \caption{Die Differenz (c) des aufgenommenen Streubildes und des konstanten Offsets stimmt mit Abb. \ref{fig:capture_ped_diff}c überein. Diese Differenz wird mit dem Cluster-Kern $\mathbf{K}_2$ gefaltet (d). Es werden nur lokale Maxima (f) in \qtyproduct{2 x 2}{\px}-Umgebungen behalten. Zum Schluss (g) wird der Schwellenwert $s_Q = \SI{150}{\adu}$ angewendet.}
    \label{fig:capture_ped_diff_clustering}
\end{figure}
\noindent
Die ersten beiden Schritte, in denen der konstante Offset vom aufgenommenen Streubild abgezogen wird, ist zum Schwellenwert-Algorithmus identisch (Abb.~\ref{fig:capture_ped_diff}a und b) und werden daher in Abb.~\ref{fig:capture_ped_diff_clustering} weggelassen. Die resultierende Differenz (Abb.~\ref{fig:capture_ped_diff_clustering}c) wird mit dem Cluster-Kern  $\mathbf{K}_2$ gefaltet (Abb. \ref{fig:capture_ped_diff_clustering}d). Als Nächstes werden die lokalen Maxima in \qtyproduct{2 x 2}{\px}-Nachbarschaft gesucht und behalten (Abb.~\ref{fig:capture_ped_diff_clustering}f). Zum Schluss werden diejenigen Punkte, die den Wert $s_Q$ überschreiten, als Photonen bezeichnet (Abb.~\ref{fig:capture_ped_diff_clustering}g). Mit dem Einsatz des Clustering-Algorithmus können in demselben Streubild mehr Photonen detektiert werden als mit dem Schwellenwert-Algorithmus (vgl. Abb.~\ref{fig:capture_ped_diff}d).


\noindent
Auf diese Weise werden \SI{50000}{\captures} einzeln ausgewertet und aufsummiert. In Abb.~\ref{fig:cl_2_150_170_180_200_220_250_resonance} sind die Summen der ausgewerteten Aufnahmen in log-Skala dargestellt. Für die Auswertung aller Summen wird ein Schwellenwert $s_Q$ aus dem Intervall von \SIrange{150}{250}{\adu} genommen.
\begin{figure}[H]
    \centering
    \input{images/auswertung/cl_2_150_170_180_200_220_250_resonance.pgf}
    \caption{Anzahl der detektierten Photonen mithilfe des Clustering-Algorithmus mit Cluster-Kern $\mathbf{K}_2$ und den Schwellenwerten $s_Q$ (a) \SI{150}{\adu}, (b) \SI{170}{\adu}, (c) \SI{180}{\adu}, (d) \SI{200}{\adu}, (e) \SI{220}{\adu} und (f) \SI{250}{\adu}. Aufsummiert wurden jeweils \num{50000} Aufnahmen. Aufgenommen an der Gd M5 Resonanzfrequenz.}
    \label{fig:cl_2_150_170_180_200_220_250_resonance}
\end{figure}
\noindent
Man kann in der Auswertung der an der resonanten Photonenenergie aufgenommenen Daten das ringähnliche Muster erkennen. Jedoch ist es nötig, zunächst einen Blick auf die Messdaten an der nicht-resonanten Photonenenergie $h\nu_\text{Gd, Off-Res}$ zu werfen.
\begin{figure}[H]
    \centering
    \input{images/auswertung/cl_150_170_180_200_220_250_off_resonance.pgf}
    \caption{Anzahl der detektierten Photonen mithilfe des Clustering-Algorithmuses mit Cluster-Kern $\mathbf{K}_2$ und dem Schwellenwert $s_Q$ (a) \SI{150}{\adu}, (b) \SI{170}{\adu}, (c) \SI{180}{\adu}, (d) \SI{200}{\adu}, (e) \SI{220}{\adu} und (f) \SI{250}{\adu}. Aufsummiert werden jeweils \num{50000} Aufnahmen. Aufgenommen an der Photonenenergie $h\nu_\text{Gd, Off-Res} \approx \SI{1163}{\eV}$.}
    \label{fig:cl_150_170_180_200_220_250_off_resonance}
\end{figure}
\noindent
Die Anzahl an fehldetektierten Photonen ist so hoch, dass sich die vergleichbaren Photonenzahlen im Bereich des Streurings bei den resonanten und nicht-resonanten Photonenenergien ergeben. Dazu können am wahrscheinlichsten zwei Aspekte beitragen. 

\noindent
Der erste Faktor ist Detektorrauschen mit der Standardabweichung $\sigma_R = \SI{19.94}{\adu}$, die vergleichbar mit dem Ein-Photon-Signal $W_\text{Gd, M5} = \SI{180}{\adu}$ ist. Wird die Summe eines \qtyproduct{2 x 2}{\px}-Clusters erfasst, ist die Standardabweichung dieser Summe $\sigma_{2\times 2} = \sqrt{4}\sigma_R = 2\sigma_R$. 

\noindent
Als Nächstes geht man davon aus, dass das gesamte Ein-Photon-Signal \SI{180}{\adu} eines Photons innerhalb des \qtyproduct{2 x 2}{\px}-Clusters liegt und der Schwellenwert $s_Q$ bis auf \SI{170}{\adu} oder \SI{180}{\adu} erhöht werden kann.

\noindent
Da die Verteilung des Detektorrauschens gut einer Gauß-Verteilung $G(W, \mu, \sigma, A)$ folgt, kann die Zahl der fehldetektierten Photonen in Bezug auf den eingesetzten Schwellenwert $s_{V/Q}$ und die bekannte Standardabweichung des Detektorrauschens $\sigma$ wie folgt ermittelt werden: 
\begin{equation}
    \text{\gls{fdpa}}_{\text{EW, $s_{V/Q}$}}=\int_{s_{V/Q}}^\infty G(W, 0, \sigma, A) \,dW = \frac{A}{2}\left[1 - \erf\left({\frac{s_{V/Q}}{\sqrt{2}\sigma}}\right)\right]
\end{equation}

\noindent
Im Falle eines \qtyproduct{2 x 2}{\px}-Clusters mit der resultierenden Standardabweichung $\sigma_{2\times 2}$ erhält man bei dem Schwellenwert $s_Q = \SI{170}{\adu}$ den folgenden Erwartungswert fehldetektierter Photonen:
\begin{equation}
     \text{\gls{fdpa}}_{\text{EW, \SI{170}{\adu}}} = 1{,}01A\cdot 10^{-5} %\int_{\SI{170}{\adu}}^\infty G(W, 0, \sigma_{2\times 2}, A) \,dW = \frac{A}{2}\left[1 - \erf\left({\frac{\SI{170}{\adu}}{\sqrt{2}\sigma_{2\times 2}}}\right)\right] 
\end{equation}

\noindent
Wird der Wert $s_Q$ bis auf \SI{180}{\adu} erhöht, dann verringert sich der Erwartungswert der fehldetektierten Photonen:
\begin{equation}
    \text{\gls{fdpa}}_{\text{EW, \SI{180}{\adu}}} = 3{,}19A\cdot 10^{-6} %\int_{\SI{180}{\adu}}^\infty G(W, 0, \sigma_{2\times 2}, A) \,dW = \frac{A}{2}\left[1 - \erf\left({\frac{\SI{180}{\adu}}{\sqrt{2}\sigma_{2\times 2}}}\right)\right]
\end{equation}

\noindent
Wird der Schwellenwert-Algorithmus mit dem Wert $s_V = \SI{100}{\adu}$ eingesetzt, ist die Standardabweichung $\sigma_{1\times 1} = \sigma_{R}$ und der Erwartungswert der fehldetektierten Photonen gleich:
\begin{equation}
   \text{\gls{fdpa}}_{\text{EW, \SI{110}{\adu}}} = 2{,}65A\cdot 10^{-7} %\int_{\SI{100}{\adu}}^\infty G(W, 0, \sigma_{1\times 1}, A) \,dW = \frac{A}{2}\left[1 - \erf\left({\frac{\SI{100}{\adu}}{\sqrt{2}\sigma_{1\times 1}}}\right)\right] 
\end{equation}

\noindent
Man sieht, dass die erwartete Zahl der fehldetektierten Photonen beim Ansatz des Clustering-Al\-go\-rith\-mus selbst unter den besten Bedingungen ca.\ 10 mal höher ist als ohne Clustering.

\noindent
Es werden Histogramme über die Pixelwerte und Summen von \qtyproduct{2 x 2}{\px}-Clustern von \num{10000} Dunkel- und Streubildern in Abb.~\ref{fig:no_pr_cl_2_histograms} aufgetragen. In der Regel können mehrere Peaks in einem Histogramm über die Pixelwerte identifiziert werden. So liegt typischerweise der erste Peak bei \SI{0}{\adu} und entspricht dem Detektorrauschen. Weiter sollen die äquidistanten Peaks folgen, die einem, zwei oder mehreren Photonen-Ereignissen entsprechen.
\begin{figure}[H]
    \centering
    %% Creator: Matplotlib, PGF backend
%%
%% To include the figure in your LaTeX document, write
%%   \input{<filename>.pgf}
%%
%% Make sure the required packages are loaded in your preamble
%%   \usepackage{pgf}
%%
%% Also ensure that all the required font packages are loaded; for instance,
%% the lmodern package is sometimes necessary when using math font.
%%   \usepackage{lmodern}
%%
%% Figures using additional raster images can only be included by \input if
%% they are in the same directory as the main LaTeX file. For loading figures
%% from other directories you can use the `import` package
%%   \usepackage{import}
%%
%% and then include the figures with
%%   \import{<path to file>}{<filename>.pgf}
%%
%% Matplotlib used the following preamble
%%   \usepackage{amsmath} \usepackage[utf8]{inputenc} \usepackage[T1]{fontenc} \usepackage[output-decimal-marker={,},print-unity-mantissa=false]{siunitx} \sisetup{per-mode=fraction, separate-uncertainty = true, locale = DE} \usepackage[acronym, toc, section=section, nonumberlist, nopostdot]{glossaries-extra} \DeclareSIUnit\adu{\text{ADU}} \DeclareSIUnit\px{\text{px}} \DeclareSIUnit\photons{\text{Pho\-to\-nen}} \DeclareSIUnit\photon{\text{Pho\-ton}}
%%
\begingroup%
\makeatletter%
\begin{pgfpicture}%
\pgfpathrectangle{\pgfpointorigin}{\pgfqpoint{6.235591in}{4.204223in}}%
\pgfusepath{use as bounding box, clip}%
\begin{pgfscope}%
\pgfsetbuttcap%
\pgfsetmiterjoin%
\pgfsetlinewidth{0.000000pt}%
\definecolor{currentstroke}{rgb}{1.000000,1.000000,1.000000}%
\pgfsetstrokecolor{currentstroke}%
\pgfsetstrokeopacity{0.000000}%
\pgfsetdash{}{0pt}%
\pgfpathmoveto{\pgfqpoint{0.000000in}{0.000000in}}%
\pgfpathlineto{\pgfqpoint{6.235591in}{0.000000in}}%
\pgfpathlineto{\pgfqpoint{6.235591in}{4.204223in}}%
\pgfpathlineto{\pgfqpoint{0.000000in}{4.204223in}}%
\pgfpathlineto{\pgfqpoint{0.000000in}{0.000000in}}%
\pgfpathclose%
\pgfusepath{}%
\end{pgfscope}%
\begin{pgfscope}%
\pgfsetbuttcap%
\pgfsetmiterjoin%
\definecolor{currentfill}{rgb}{1.000000,1.000000,1.000000}%
\pgfsetfillcolor{currentfill}%
\pgfsetlinewidth{0.000000pt}%
\definecolor{currentstroke}{rgb}{0.000000,0.000000,0.000000}%
\pgfsetstrokecolor{currentstroke}%
\pgfsetstrokeopacity{0.000000}%
\pgfsetdash{}{0pt}%
\pgfpathmoveto{\pgfqpoint{0.557402in}{2.573088in}}%
\pgfpathlineto{\pgfqpoint{6.131424in}{2.573088in}}%
\pgfpathlineto{\pgfqpoint{6.131424in}{3.961264in}}%
\pgfpathlineto{\pgfqpoint{0.557402in}{3.961264in}}%
\pgfpathlineto{\pgfqpoint{0.557402in}{2.573088in}}%
\pgfpathclose%
\pgfusepath{fill}%
\end{pgfscope}%
\begin{pgfscope}%
\pgfsetbuttcap%
\pgfsetroundjoin%
\definecolor{currentfill}{rgb}{0.000000,0.000000,0.000000}%
\pgfsetfillcolor{currentfill}%
\pgfsetlinewidth{0.803000pt}%
\definecolor{currentstroke}{rgb}{0.000000,0.000000,0.000000}%
\pgfsetstrokecolor{currentstroke}%
\pgfsetdash{}{0pt}%
\pgfsys@defobject{currentmarker}{\pgfqpoint{0.000000in}{-0.048611in}}{\pgfqpoint{0.000000in}{0.000000in}}{%
\pgfpathmoveto{\pgfqpoint{0.000000in}{0.000000in}}%
\pgfpathlineto{\pgfqpoint{0.000000in}{-0.048611in}}%
\pgfusepath{stroke,fill}%
}%
\begin{pgfscope}%
\pgfsys@transformshift{0.666697in}{2.573088in}%
\pgfsys@useobject{currentmarker}{}%
\end{pgfscope}%
\end{pgfscope}%
\begin{pgfscope}%
\definecolor{textcolor}{rgb}{0.000000,0.000000,0.000000}%
\pgfsetstrokecolor{textcolor}%
\pgfsetfillcolor{textcolor}%
\pgftext[x=0.666697in,y=2.475866in,,top]{\color{textcolor}\rmfamily\fontsize{10.000000}{12.000000}\selectfont \(\displaystyle {\ensuremath{-}200}\)}%
\end{pgfscope}%
\begin{pgfscope}%
\pgfsetbuttcap%
\pgfsetroundjoin%
\definecolor{currentfill}{rgb}{0.000000,0.000000,0.000000}%
\pgfsetfillcolor{currentfill}%
\pgfsetlinewidth{0.803000pt}%
\definecolor{currentstroke}{rgb}{0.000000,0.000000,0.000000}%
\pgfsetstrokecolor{currentstroke}%
\pgfsetdash{}{0pt}%
\pgfsys@defobject{currentmarker}{\pgfqpoint{0.000000in}{-0.048611in}}{\pgfqpoint{0.000000in}{0.000000in}}{%
\pgfpathmoveto{\pgfqpoint{0.000000in}{0.000000in}}%
\pgfpathlineto{\pgfqpoint{0.000000in}{-0.048611in}}%
\pgfusepath{stroke,fill}%
}%
\begin{pgfscope}%
\pgfsys@transformshift{1.759642in}{2.573088in}%
\pgfsys@useobject{currentmarker}{}%
\end{pgfscope}%
\end{pgfscope}%
\begin{pgfscope}%
\definecolor{textcolor}{rgb}{0.000000,0.000000,0.000000}%
\pgfsetstrokecolor{textcolor}%
\pgfsetfillcolor{textcolor}%
\pgftext[x=1.759642in,y=2.475866in,,top]{\color{textcolor}\rmfamily\fontsize{10.000000}{12.000000}\selectfont \(\displaystyle {\ensuremath{-}100}\)}%
\end{pgfscope}%
\begin{pgfscope}%
\pgfsetbuttcap%
\pgfsetroundjoin%
\definecolor{currentfill}{rgb}{0.000000,0.000000,0.000000}%
\pgfsetfillcolor{currentfill}%
\pgfsetlinewidth{0.803000pt}%
\definecolor{currentstroke}{rgb}{0.000000,0.000000,0.000000}%
\pgfsetstrokecolor{currentstroke}%
\pgfsetdash{}{0pt}%
\pgfsys@defobject{currentmarker}{\pgfqpoint{0.000000in}{-0.048611in}}{\pgfqpoint{0.000000in}{0.000000in}}{%
\pgfpathmoveto{\pgfqpoint{0.000000in}{0.000000in}}%
\pgfpathlineto{\pgfqpoint{0.000000in}{-0.048611in}}%
\pgfusepath{stroke,fill}%
}%
\begin{pgfscope}%
\pgfsys@transformshift{2.852588in}{2.573088in}%
\pgfsys@useobject{currentmarker}{}%
\end{pgfscope}%
\end{pgfscope}%
\begin{pgfscope}%
\definecolor{textcolor}{rgb}{0.000000,0.000000,0.000000}%
\pgfsetstrokecolor{textcolor}%
\pgfsetfillcolor{textcolor}%
\pgftext[x=2.852588in,y=2.475866in,,top]{\color{textcolor}\rmfamily\fontsize{10.000000}{12.000000}\selectfont \(\displaystyle {0}\)}%
\end{pgfscope}%
\begin{pgfscope}%
\pgfsetbuttcap%
\pgfsetroundjoin%
\definecolor{currentfill}{rgb}{0.000000,0.000000,0.000000}%
\pgfsetfillcolor{currentfill}%
\pgfsetlinewidth{0.803000pt}%
\definecolor{currentstroke}{rgb}{0.000000,0.000000,0.000000}%
\pgfsetstrokecolor{currentstroke}%
\pgfsetdash{}{0pt}%
\pgfsys@defobject{currentmarker}{\pgfqpoint{0.000000in}{-0.048611in}}{\pgfqpoint{0.000000in}{0.000000in}}{%
\pgfpathmoveto{\pgfqpoint{0.000000in}{0.000000in}}%
\pgfpathlineto{\pgfqpoint{0.000000in}{-0.048611in}}%
\pgfusepath{stroke,fill}%
}%
\begin{pgfscope}%
\pgfsys@transformshift{3.945533in}{2.573088in}%
\pgfsys@useobject{currentmarker}{}%
\end{pgfscope}%
\end{pgfscope}%
\begin{pgfscope}%
\definecolor{textcolor}{rgb}{0.000000,0.000000,0.000000}%
\pgfsetstrokecolor{textcolor}%
\pgfsetfillcolor{textcolor}%
\pgftext[x=3.945533in,y=2.475866in,,top]{\color{textcolor}\rmfamily\fontsize{10.000000}{12.000000}\selectfont \(\displaystyle {100}\)}%
\end{pgfscope}%
\begin{pgfscope}%
\pgfsetbuttcap%
\pgfsetroundjoin%
\definecolor{currentfill}{rgb}{0.000000,0.000000,0.000000}%
\pgfsetfillcolor{currentfill}%
\pgfsetlinewidth{0.803000pt}%
\definecolor{currentstroke}{rgb}{0.000000,0.000000,0.000000}%
\pgfsetstrokecolor{currentstroke}%
\pgfsetdash{}{0pt}%
\pgfsys@defobject{currentmarker}{\pgfqpoint{0.000000in}{-0.048611in}}{\pgfqpoint{0.000000in}{0.000000in}}{%
\pgfpathmoveto{\pgfqpoint{0.000000in}{0.000000in}}%
\pgfpathlineto{\pgfqpoint{0.000000in}{-0.048611in}}%
\pgfusepath{stroke,fill}%
}%
\begin{pgfscope}%
\pgfsys@transformshift{5.038479in}{2.573088in}%
\pgfsys@useobject{currentmarker}{}%
\end{pgfscope}%
\end{pgfscope}%
\begin{pgfscope}%
\definecolor{textcolor}{rgb}{0.000000,0.000000,0.000000}%
\pgfsetstrokecolor{textcolor}%
\pgfsetfillcolor{textcolor}%
\pgftext[x=5.038479in,y=2.475866in,,top]{\color{textcolor}\rmfamily\fontsize{10.000000}{12.000000}\selectfont \(\displaystyle {200}\)}%
\end{pgfscope}%
\begin{pgfscope}%
\pgfsetbuttcap%
\pgfsetroundjoin%
\definecolor{currentfill}{rgb}{0.000000,0.000000,0.000000}%
\pgfsetfillcolor{currentfill}%
\pgfsetlinewidth{0.803000pt}%
\definecolor{currentstroke}{rgb}{0.000000,0.000000,0.000000}%
\pgfsetstrokecolor{currentstroke}%
\pgfsetdash{}{0pt}%
\pgfsys@defobject{currentmarker}{\pgfqpoint{0.000000in}{-0.048611in}}{\pgfqpoint{0.000000in}{0.000000in}}{%
\pgfpathmoveto{\pgfqpoint{0.000000in}{0.000000in}}%
\pgfpathlineto{\pgfqpoint{0.000000in}{-0.048611in}}%
\pgfusepath{stroke,fill}%
}%
\begin{pgfscope}%
\pgfsys@transformshift{6.131424in}{2.573088in}%
\pgfsys@useobject{currentmarker}{}%
\end{pgfscope}%
\end{pgfscope}%
\begin{pgfscope}%
\definecolor{textcolor}{rgb}{0.000000,0.000000,0.000000}%
\pgfsetstrokecolor{textcolor}%
\pgfsetfillcolor{textcolor}%
\pgftext[x=6.131424in,y=2.475866in,,top]{\color{textcolor}\rmfamily\fontsize{10.000000}{12.000000}\selectfont \(\displaystyle {300}\)}%
\end{pgfscope}%
\begin{pgfscope}%
\definecolor{textcolor}{rgb}{0.000000,0.000000,0.000000}%
\pgfsetstrokecolor{textcolor}%
\pgfsetfillcolor{textcolor}%
\pgftext[x=3.344413in,y=2.297655in,,top]{\color{textcolor}\rmfamily\fontsize{10.000000}{12.000000}\selectfont Pixelwert \(\displaystyle W\) in ADU}%
\end{pgfscope}%
\begin{pgfscope}%
\pgfsetbuttcap%
\pgfsetroundjoin%
\definecolor{currentfill}{rgb}{0.000000,0.000000,0.000000}%
\pgfsetfillcolor{currentfill}%
\pgfsetlinewidth{0.803000pt}%
\definecolor{currentstroke}{rgb}{0.000000,0.000000,0.000000}%
\pgfsetstrokecolor{currentstroke}%
\pgfsetdash{}{0pt}%
\pgfsys@defobject{currentmarker}{\pgfqpoint{-0.048611in}{0.000000in}}{\pgfqpoint{-0.000000in}{0.000000in}}{%
\pgfpathmoveto{\pgfqpoint{-0.000000in}{0.000000in}}%
\pgfpathlineto{\pgfqpoint{-0.048611in}{0.000000in}}%
\pgfusepath{stroke,fill}%
}%
\begin{pgfscope}%
\pgfsys@transformshift{0.557402in}{2.636553in}%
\pgfsys@useobject{currentmarker}{}%
\end{pgfscope}%
\end{pgfscope}%
\begin{pgfscope}%
\definecolor{textcolor}{rgb}{0.000000,0.000000,0.000000}%
\pgfsetstrokecolor{textcolor}%
\pgfsetfillcolor{textcolor}%
\pgftext[x=0.282710in, y=2.588728in, left, base]{\color{textcolor}\rmfamily\fontsize{10.000000}{12.000000}\selectfont \num{0.0}}%
\end{pgfscope}%
\begin{pgfscope}%
\pgfsetbuttcap%
\pgfsetroundjoin%
\definecolor{currentfill}{rgb}{0.000000,0.000000,0.000000}%
\pgfsetfillcolor{currentfill}%
\pgfsetlinewidth{0.803000pt}%
\definecolor{currentstroke}{rgb}{0.000000,0.000000,0.000000}%
\pgfsetstrokecolor{currentstroke}%
\pgfsetdash{}{0pt}%
\pgfsys@defobject{currentmarker}{\pgfqpoint{-0.048611in}{0.000000in}}{\pgfqpoint{-0.000000in}{0.000000in}}{%
\pgfpathmoveto{\pgfqpoint{-0.000000in}{0.000000in}}%
\pgfpathlineto{\pgfqpoint{-0.048611in}{0.000000in}}%
\pgfusepath{stroke,fill}%
}%
\begin{pgfscope}%
\pgfsys@transformshift{0.557402in}{3.031530in}%
\pgfsys@useobject{currentmarker}{}%
\end{pgfscope}%
\end{pgfscope}%
\begin{pgfscope}%
\definecolor{textcolor}{rgb}{0.000000,0.000000,0.000000}%
\pgfsetstrokecolor{textcolor}%
\pgfsetfillcolor{textcolor}%
\pgftext[x=0.282710in, y=2.983705in, left, base]{\color{textcolor}\rmfamily\fontsize{10.000000}{12.000000}\selectfont \num{0.1}}%
\end{pgfscope}%
\begin{pgfscope}%
\pgfsetbuttcap%
\pgfsetroundjoin%
\definecolor{currentfill}{rgb}{0.000000,0.000000,0.000000}%
\pgfsetfillcolor{currentfill}%
\pgfsetlinewidth{0.803000pt}%
\definecolor{currentstroke}{rgb}{0.000000,0.000000,0.000000}%
\pgfsetstrokecolor{currentstroke}%
\pgfsetdash{}{0pt}%
\pgfsys@defobject{currentmarker}{\pgfqpoint{-0.048611in}{0.000000in}}{\pgfqpoint{-0.000000in}{0.000000in}}{%
\pgfpathmoveto{\pgfqpoint{-0.000000in}{0.000000in}}%
\pgfpathlineto{\pgfqpoint{-0.048611in}{0.000000in}}%
\pgfusepath{stroke,fill}%
}%
\begin{pgfscope}%
\pgfsys@transformshift{0.557402in}{3.426507in}%
\pgfsys@useobject{currentmarker}{}%
\end{pgfscope}%
\end{pgfscope}%
\begin{pgfscope}%
\definecolor{textcolor}{rgb}{0.000000,0.000000,0.000000}%
\pgfsetstrokecolor{textcolor}%
\pgfsetfillcolor{textcolor}%
\pgftext[x=0.282710in, y=3.378682in, left, base]{\color{textcolor}\rmfamily\fontsize{10.000000}{12.000000}\selectfont \num{0.2}}%
\end{pgfscope}%
\begin{pgfscope}%
\pgfsetbuttcap%
\pgfsetroundjoin%
\definecolor{currentfill}{rgb}{0.000000,0.000000,0.000000}%
\pgfsetfillcolor{currentfill}%
\pgfsetlinewidth{0.803000pt}%
\definecolor{currentstroke}{rgb}{0.000000,0.000000,0.000000}%
\pgfsetstrokecolor{currentstroke}%
\pgfsetdash{}{0pt}%
\pgfsys@defobject{currentmarker}{\pgfqpoint{-0.048611in}{0.000000in}}{\pgfqpoint{-0.000000in}{0.000000in}}{%
\pgfpathmoveto{\pgfqpoint{-0.000000in}{0.000000in}}%
\pgfpathlineto{\pgfqpoint{-0.048611in}{0.000000in}}%
\pgfusepath{stroke,fill}%
}%
\begin{pgfscope}%
\pgfsys@transformshift{0.557402in}{3.821484in}%
\pgfsys@useobject{currentmarker}{}%
\end{pgfscope}%
\end{pgfscope}%
\begin{pgfscope}%
\definecolor{textcolor}{rgb}{0.000000,0.000000,0.000000}%
\pgfsetstrokecolor{textcolor}%
\pgfsetfillcolor{textcolor}%
\pgftext[x=0.282710in, y=3.773659in, left, base]{\color{textcolor}\rmfamily\fontsize{10.000000}{12.000000}\selectfont \num{0.3}}%
\end{pgfscope}%
\begin{pgfscope}%
\definecolor{textcolor}{rgb}{0.000000,0.000000,0.000000}%
\pgfsetstrokecolor{textcolor}%
\pgfsetfillcolor{textcolor}%
\pgftext[x=0.227155in,y=3.267176in,,bottom,rotate=90.000000]{\color{textcolor}\rmfamily\fontsize{10.000000}{12.000000}\selectfont Pixelzahl}%
\end{pgfscope}%
\begin{pgfscope}%
\definecolor{textcolor}{rgb}{0.000000,0.000000,0.000000}%
\pgfsetstrokecolor{textcolor}%
\pgfsetfillcolor{textcolor}%
\pgftext[x=0.557402in,y=4.002931in,left,base]{\color{textcolor}\rmfamily\fontsize{10.000000}{12.000000}\selectfont \(\displaystyle \times{10^{8}}{}\)}%
\end{pgfscope}%
\begin{pgfscope}%
\pgfpathrectangle{\pgfqpoint{0.557402in}{2.573088in}}{\pgfqpoint{5.574022in}{1.388176in}}%
\pgfusepath{clip}%
\pgfsetrectcap%
\pgfsetroundjoin%
\pgfsetlinewidth{1.505625pt}%
\definecolor{currentstroke}{rgb}{0.121569,0.466667,0.705882}%
\pgfsetstrokecolor{currentstroke}%
\pgfsetdash{}{0pt}%
\pgfpathmoveto{\pgfqpoint{0.666697in}{2.636553in}}%
\pgfpathlineto{\pgfqpoint{1.737783in}{2.636761in}}%
\pgfpathlineto{\pgfqpoint{1.836148in}{2.637124in}}%
\pgfpathlineto{\pgfqpoint{1.912655in}{2.637780in}}%
\pgfpathlineto{\pgfqpoint{1.967302in}{2.638651in}}%
\pgfpathlineto{\pgfqpoint{2.021949in}{2.640184in}}%
\pgfpathlineto{\pgfqpoint{2.054738in}{2.641633in}}%
\pgfpathlineto{\pgfqpoint{2.076596in}{2.642820in}}%
\pgfpathlineto{\pgfqpoint{2.098455in}{2.644406in}}%
\pgfpathlineto{\pgfqpoint{2.120314in}{2.646381in}}%
\pgfpathlineto{\pgfqpoint{2.142173in}{2.648836in}}%
\pgfpathlineto{\pgfqpoint{2.164032in}{2.651961in}}%
\pgfpathlineto{\pgfqpoint{2.185891in}{2.655861in}}%
\pgfpathlineto{\pgfqpoint{2.196821in}{2.658123in}}%
\pgfpathlineto{\pgfqpoint{2.218679in}{2.663526in}}%
\pgfpathlineto{\pgfqpoint{2.229609in}{2.666734in}}%
\pgfpathlineto{\pgfqpoint{2.240538in}{2.670343in}}%
\pgfpathlineto{\pgfqpoint{2.251468in}{2.674202in}}%
\pgfpathlineto{\pgfqpoint{2.262397in}{2.678686in}}%
\pgfpathlineto{\pgfqpoint{2.273327in}{2.683543in}}%
\pgfpathlineto{\pgfqpoint{2.284256in}{2.688912in}}%
\pgfpathlineto{\pgfqpoint{2.295186in}{2.694907in}}%
\pgfpathlineto{\pgfqpoint{2.306115in}{2.701428in}}%
\pgfpathlineto{\pgfqpoint{2.317045in}{2.708942in}}%
\pgfpathlineto{\pgfqpoint{2.327974in}{2.716951in}}%
\pgfpathlineto{\pgfqpoint{2.338903in}{2.725771in}}%
\pgfpathlineto{\pgfqpoint{2.349833in}{2.735313in}}%
\pgfpathlineto{\pgfqpoint{2.360762in}{2.745981in}}%
\pgfpathlineto{\pgfqpoint{2.371692in}{2.757583in}}%
\pgfpathlineto{\pgfqpoint{2.382621in}{2.770104in}}%
\pgfpathlineto{\pgfqpoint{2.393551in}{2.783644in}}%
\pgfpathlineto{\pgfqpoint{2.404480in}{2.798529in}}%
\pgfpathlineto{\pgfqpoint{2.415410in}{2.814451in}}%
\pgfpathlineto{\pgfqpoint{2.426339in}{2.831723in}}%
\pgfpathlineto{\pgfqpoint{2.437269in}{2.850091in}}%
\pgfpathlineto{\pgfqpoint{2.448198in}{2.869995in}}%
\pgfpathlineto{\pgfqpoint{2.459127in}{2.891131in}}%
\pgfpathlineto{\pgfqpoint{2.470057in}{2.913760in}}%
\pgfpathlineto{\pgfqpoint{2.480986in}{2.937797in}}%
\pgfpathlineto{\pgfqpoint{2.491916in}{2.963337in}}%
\pgfpathlineto{\pgfqpoint{2.502845in}{2.990020in}}%
\pgfpathlineto{\pgfqpoint{2.513775in}{3.018083in}}%
\pgfpathlineto{\pgfqpoint{2.524704in}{3.047796in}}%
\pgfpathlineto{\pgfqpoint{2.535634in}{3.078541in}}%
\pgfpathlineto{\pgfqpoint{2.557493in}{3.144291in}}%
\pgfpathlineto{\pgfqpoint{2.579351in}{3.213985in}}%
\pgfpathlineto{\pgfqpoint{2.590281in}{3.250467in}}%
\pgfpathlineto{\pgfqpoint{2.612140in}{3.325377in}}%
\pgfpathlineto{\pgfqpoint{2.666787in}{3.516911in}}%
\pgfpathlineto{\pgfqpoint{2.688646in}{3.590523in}}%
\pgfpathlineto{\pgfqpoint{2.699575in}{3.626227in}}%
\pgfpathlineto{\pgfqpoint{2.710505in}{3.660359in}}%
\pgfpathlineto{\pgfqpoint{2.721434in}{3.693245in}}%
\pgfpathlineto{\pgfqpoint{2.732364in}{3.724071in}}%
\pgfpathlineto{\pgfqpoint{2.743293in}{3.752943in}}%
\pgfpathlineto{\pgfqpoint{2.754223in}{3.780184in}}%
\pgfpathlineto{\pgfqpoint{2.765152in}{3.805003in}}%
\pgfpathlineto{\pgfqpoint{2.776082in}{3.826703in}}%
\pgfpathlineto{\pgfqpoint{2.787011in}{3.845467in}}%
\pgfpathlineto{\pgfqpoint{2.797941in}{3.862500in}}%
\pgfpathlineto{\pgfqpoint{2.808870in}{3.876172in}}%
\pgfpathlineto{\pgfqpoint{2.819799in}{3.885971in}}%
\pgfpathlineto{\pgfqpoint{2.830729in}{3.893809in}}%
\pgfpathlineto{\pgfqpoint{2.841658in}{3.896752in}}%
\pgfpathlineto{\pgfqpoint{2.852588in}{3.898165in}}%
\pgfpathlineto{\pgfqpoint{2.863517in}{3.894488in}}%
\pgfpathlineto{\pgfqpoint{2.874447in}{3.888245in}}%
\pgfpathlineto{\pgfqpoint{2.885376in}{3.879277in}}%
\pgfpathlineto{\pgfqpoint{2.896306in}{3.867496in}}%
\pgfpathlineto{\pgfqpoint{2.907235in}{3.851458in}}%
\pgfpathlineto{\pgfqpoint{2.918165in}{3.833139in}}%
\pgfpathlineto{\pgfqpoint{2.929094in}{3.812527in}}%
\pgfpathlineto{\pgfqpoint{2.940023in}{3.788785in}}%
\pgfpathlineto{\pgfqpoint{2.950953in}{3.762767in}}%
\pgfpathlineto{\pgfqpoint{2.961882in}{3.733935in}}%
\pgfpathlineto{\pgfqpoint{2.972812in}{3.703456in}}%
\pgfpathlineto{\pgfqpoint{2.983741in}{3.671545in}}%
\pgfpathlineto{\pgfqpoint{2.994671in}{3.637858in}}%
\pgfpathlineto{\pgfqpoint{3.005600in}{3.602869in}}%
\pgfpathlineto{\pgfqpoint{3.016530in}{3.566777in}}%
\pgfpathlineto{\pgfqpoint{3.027459in}{3.529815in}}%
\pgfpathlineto{\pgfqpoint{3.093036in}{3.300971in}}%
\pgfpathlineto{\pgfqpoint{3.103965in}{3.263490in}}%
\pgfpathlineto{\pgfqpoint{3.125824in}{3.191331in}}%
\pgfpathlineto{\pgfqpoint{3.136754in}{3.156188in}}%
\pgfpathlineto{\pgfqpoint{3.147683in}{3.122836in}}%
\pgfpathlineto{\pgfqpoint{3.158613in}{3.090331in}}%
\pgfpathlineto{\pgfqpoint{3.169542in}{3.059026in}}%
\pgfpathlineto{\pgfqpoint{3.180471in}{3.028846in}}%
\pgfpathlineto{\pgfqpoint{3.191401in}{3.000299in}}%
\pgfpathlineto{\pgfqpoint{3.202330in}{2.973163in}}%
\pgfpathlineto{\pgfqpoint{3.213260in}{2.947175in}}%
\pgfpathlineto{\pgfqpoint{3.224189in}{2.922728in}}%
\pgfpathlineto{\pgfqpoint{3.235119in}{2.899628in}}%
\pgfpathlineto{\pgfqpoint{3.246048in}{2.878026in}}%
\pgfpathlineto{\pgfqpoint{3.256978in}{2.857646in}}%
\pgfpathlineto{\pgfqpoint{3.267907in}{2.838724in}}%
\pgfpathlineto{\pgfqpoint{3.278837in}{2.821136in}}%
\pgfpathlineto{\pgfqpoint{3.289766in}{2.804779in}}%
\pgfpathlineto{\pgfqpoint{3.300695in}{2.789373in}}%
\pgfpathlineto{\pgfqpoint{3.311625in}{2.775306in}}%
\pgfpathlineto{\pgfqpoint{3.322554in}{2.762531in}}%
\pgfpathlineto{\pgfqpoint{3.333484in}{2.750444in}}%
\pgfpathlineto{\pgfqpoint{3.344413in}{2.739490in}}%
\pgfpathlineto{\pgfqpoint{3.355343in}{2.729550in}}%
\pgfpathlineto{\pgfqpoint{3.366272in}{2.720407in}}%
\pgfpathlineto{\pgfqpoint{3.377202in}{2.712190in}}%
\pgfpathlineto{\pgfqpoint{3.388131in}{2.704494in}}%
\pgfpathlineto{\pgfqpoint{3.399061in}{2.697548in}}%
\pgfpathlineto{\pgfqpoint{3.409990in}{2.691402in}}%
\pgfpathlineto{\pgfqpoint{3.420919in}{2.685749in}}%
\pgfpathlineto{\pgfqpoint{3.431849in}{2.680685in}}%
\pgfpathlineto{\pgfqpoint{3.442778in}{2.676035in}}%
\pgfpathlineto{\pgfqpoint{3.453708in}{2.671904in}}%
\pgfpathlineto{\pgfqpoint{3.475567in}{2.664826in}}%
\pgfpathlineto{\pgfqpoint{3.486496in}{2.661884in}}%
\pgfpathlineto{\pgfqpoint{3.508355in}{2.656779in}}%
\pgfpathlineto{\pgfqpoint{3.519285in}{2.654631in}}%
\pgfpathlineto{\pgfqpoint{3.530214in}{2.652726in}}%
\pgfpathlineto{\pgfqpoint{3.552073in}{2.649424in}}%
\pgfpathlineto{\pgfqpoint{3.573932in}{2.646841in}}%
\pgfpathlineto{\pgfqpoint{3.595791in}{2.644773in}}%
\pgfpathlineto{\pgfqpoint{3.617650in}{2.643148in}}%
\pgfpathlineto{\pgfqpoint{3.650438in}{2.641265in}}%
\pgfpathlineto{\pgfqpoint{3.683226in}{2.639959in}}%
\pgfpathlineto{\pgfqpoint{3.716015in}{2.638982in}}%
\pgfpathlineto{\pgfqpoint{3.759733in}{2.638130in}}%
\pgfpathlineto{\pgfqpoint{3.814380in}{2.637453in}}%
\pgfpathlineto{\pgfqpoint{3.890886in}{2.636958in}}%
\pgfpathlineto{\pgfqpoint{4.000181in}{2.636684in}}%
\pgfpathlineto{\pgfqpoint{4.218770in}{2.636564in}}%
\pgfpathlineto{\pgfqpoint{6.133424in}{2.636553in}}%
\pgfpathlineto{\pgfqpoint{6.133424in}{2.636553in}}%
\pgfusepath{stroke}%
\end{pgfscope}%
\begin{pgfscope}%
\pgfpathrectangle{\pgfqpoint{0.557402in}{2.573088in}}{\pgfqpoint{5.574022in}{1.388176in}}%
\pgfusepath{clip}%
\pgfsetrectcap%
\pgfsetroundjoin%
\pgfsetlinewidth{1.505625pt}%
\definecolor{currentstroke}{rgb}{1.000000,0.498039,0.054902}%
\pgfsetstrokecolor{currentstroke}%
\pgfsetdash{}{0pt}%
\pgfpathmoveto{\pgfqpoint{0.666697in}{2.636554in}}%
\pgfpathlineto{\pgfqpoint{1.213170in}{2.636753in}}%
\pgfpathlineto{\pgfqpoint{1.311535in}{2.637245in}}%
\pgfpathlineto{\pgfqpoint{1.377111in}{2.638072in}}%
\pgfpathlineto{\pgfqpoint{1.420829in}{2.639072in}}%
\pgfpathlineto{\pgfqpoint{1.464547in}{2.640643in}}%
\pgfpathlineto{\pgfqpoint{1.497335in}{2.642328in}}%
\pgfpathlineto{\pgfqpoint{1.530124in}{2.644644in}}%
\pgfpathlineto{\pgfqpoint{1.551983in}{2.646574in}}%
\pgfpathlineto{\pgfqpoint{1.573842in}{2.648860in}}%
\pgfpathlineto{\pgfqpoint{1.606630in}{2.653204in}}%
\pgfpathlineto{\pgfqpoint{1.628489in}{2.656770in}}%
\pgfpathlineto{\pgfqpoint{1.639418in}{2.658661in}}%
\pgfpathlineto{\pgfqpoint{1.661277in}{2.663143in}}%
\pgfpathlineto{\pgfqpoint{1.683136in}{2.668179in}}%
\pgfpathlineto{\pgfqpoint{1.704995in}{2.674042in}}%
\pgfpathlineto{\pgfqpoint{1.715924in}{2.677196in}}%
\pgfpathlineto{\pgfqpoint{1.737783in}{2.684215in}}%
\pgfpathlineto{\pgfqpoint{1.759642in}{2.692089in}}%
\pgfpathlineto{\pgfqpoint{1.781501in}{2.700872in}}%
\pgfpathlineto{\pgfqpoint{1.803360in}{2.710593in}}%
\pgfpathlineto{\pgfqpoint{1.825219in}{2.721125in}}%
\pgfpathlineto{\pgfqpoint{1.836148in}{2.726805in}}%
\pgfpathlineto{\pgfqpoint{1.868937in}{2.745236in}}%
\pgfpathlineto{\pgfqpoint{1.879866in}{2.751927in}}%
\pgfpathlineto{\pgfqpoint{1.901725in}{2.765771in}}%
\pgfpathlineto{\pgfqpoint{1.912655in}{2.773338in}}%
\pgfpathlineto{\pgfqpoint{1.923584in}{2.780597in}}%
\pgfpathlineto{\pgfqpoint{1.934514in}{2.788290in}}%
\pgfpathlineto{\pgfqpoint{1.978231in}{2.821208in}}%
\pgfpathlineto{\pgfqpoint{2.011020in}{2.847963in}}%
\pgfpathlineto{\pgfqpoint{2.043808in}{2.875685in}}%
\pgfpathlineto{\pgfqpoint{2.054738in}{2.885567in}}%
\pgfpathlineto{\pgfqpoint{2.076596in}{2.904748in}}%
\pgfpathlineto{\pgfqpoint{2.087526in}{2.914781in}}%
\pgfpathlineto{\pgfqpoint{2.098455in}{2.924486in}}%
\pgfpathlineto{\pgfqpoint{2.109385in}{2.934942in}}%
\pgfpathlineto{\pgfqpoint{2.131244in}{2.955184in}}%
\pgfpathlineto{\pgfqpoint{2.142173in}{2.965445in}}%
\pgfpathlineto{\pgfqpoint{2.153103in}{2.975243in}}%
\pgfpathlineto{\pgfqpoint{2.164032in}{2.985677in}}%
\pgfpathlineto{\pgfqpoint{2.174962in}{2.995697in}}%
\pgfpathlineto{\pgfqpoint{2.185891in}{3.006371in}}%
\pgfpathlineto{\pgfqpoint{2.240538in}{3.057953in}}%
\pgfpathlineto{\pgfqpoint{2.262397in}{3.078162in}}%
\pgfpathlineto{\pgfqpoint{2.273327in}{3.088985in}}%
\pgfpathlineto{\pgfqpoint{2.284256in}{3.099136in}}%
\pgfpathlineto{\pgfqpoint{2.295186in}{3.109745in}}%
\pgfpathlineto{\pgfqpoint{2.306115in}{3.119353in}}%
\pgfpathlineto{\pgfqpoint{2.317045in}{3.129939in}}%
\pgfpathlineto{\pgfqpoint{2.327974in}{3.139864in}}%
\pgfpathlineto{\pgfqpoint{2.338903in}{3.150155in}}%
\pgfpathlineto{\pgfqpoint{2.393551in}{3.199854in}}%
\pgfpathlineto{\pgfqpoint{2.404480in}{3.209973in}}%
\pgfpathlineto{\pgfqpoint{2.459127in}{3.258107in}}%
\pgfpathlineto{\pgfqpoint{2.470057in}{3.267378in}}%
\pgfpathlineto{\pgfqpoint{2.480986in}{3.277060in}}%
\pgfpathlineto{\pgfqpoint{2.491916in}{3.286174in}}%
\pgfpathlineto{\pgfqpoint{2.513775in}{3.303643in}}%
\pgfpathlineto{\pgfqpoint{2.524704in}{3.312786in}}%
\pgfpathlineto{\pgfqpoint{2.535634in}{3.320698in}}%
\pgfpathlineto{\pgfqpoint{2.546563in}{3.329505in}}%
\pgfpathlineto{\pgfqpoint{2.579351in}{3.351935in}}%
\pgfpathlineto{\pgfqpoint{2.601210in}{3.365841in}}%
\pgfpathlineto{\pgfqpoint{2.612140in}{3.372227in}}%
\pgfpathlineto{\pgfqpoint{2.623069in}{3.378133in}}%
\pgfpathlineto{\pgfqpoint{2.633999in}{3.383719in}}%
\pgfpathlineto{\pgfqpoint{2.644928in}{3.388303in}}%
\pgfpathlineto{\pgfqpoint{2.655858in}{3.392639in}}%
\pgfpathlineto{\pgfqpoint{2.677717in}{3.398927in}}%
\pgfpathlineto{\pgfqpoint{2.699575in}{3.403021in}}%
\pgfpathlineto{\pgfqpoint{2.710505in}{3.403951in}}%
\pgfpathlineto{\pgfqpoint{2.721434in}{3.403976in}}%
\pgfpathlineto{\pgfqpoint{2.732364in}{3.403396in}}%
\pgfpathlineto{\pgfqpoint{2.743293in}{3.401783in}}%
\pgfpathlineto{\pgfqpoint{2.754223in}{3.399799in}}%
\pgfpathlineto{\pgfqpoint{2.765152in}{3.396479in}}%
\pgfpathlineto{\pgfqpoint{2.776082in}{3.392340in}}%
\pgfpathlineto{\pgfqpoint{2.787011in}{3.387224in}}%
\pgfpathlineto{\pgfqpoint{2.797941in}{3.381640in}}%
\pgfpathlineto{\pgfqpoint{2.808870in}{3.374634in}}%
\pgfpathlineto{\pgfqpoint{2.819799in}{3.367182in}}%
\pgfpathlineto{\pgfqpoint{2.830729in}{3.358728in}}%
\pgfpathlineto{\pgfqpoint{2.841658in}{3.348528in}}%
\pgfpathlineto{\pgfqpoint{2.852588in}{3.338774in}}%
\pgfpathlineto{\pgfqpoint{2.863517in}{3.327383in}}%
\pgfpathlineto{\pgfqpoint{2.885376in}{3.302031in}}%
\pgfpathlineto{\pgfqpoint{2.896306in}{3.288454in}}%
\pgfpathlineto{\pgfqpoint{2.918165in}{3.259053in}}%
\pgfpathlineto{\pgfqpoint{2.929094in}{3.243563in}}%
\pgfpathlineto{\pgfqpoint{2.950953in}{3.210553in}}%
\pgfpathlineto{\pgfqpoint{2.961882in}{3.193009in}}%
\pgfpathlineto{\pgfqpoint{2.972812in}{3.174868in}}%
\pgfpathlineto{\pgfqpoint{2.983741in}{3.157578in}}%
\pgfpathlineto{\pgfqpoint{2.994671in}{3.139235in}}%
\pgfpathlineto{\pgfqpoint{3.005600in}{3.120456in}}%
\pgfpathlineto{\pgfqpoint{3.016530in}{3.102429in}}%
\pgfpathlineto{\pgfqpoint{3.071177in}{3.009624in}}%
\pgfpathlineto{\pgfqpoint{3.093036in}{2.973786in}}%
\pgfpathlineto{\pgfqpoint{3.114895in}{2.939213in}}%
\pgfpathlineto{\pgfqpoint{3.136754in}{2.906600in}}%
\pgfpathlineto{\pgfqpoint{3.147683in}{2.890727in}}%
\pgfpathlineto{\pgfqpoint{3.158613in}{2.875481in}}%
\pgfpathlineto{\pgfqpoint{3.169542in}{2.860862in}}%
\pgfpathlineto{\pgfqpoint{3.191401in}{2.833083in}}%
\pgfpathlineto{\pgfqpoint{3.202330in}{2.820252in}}%
\pgfpathlineto{\pgfqpoint{3.213260in}{2.807840in}}%
\pgfpathlineto{\pgfqpoint{3.224189in}{2.795884in}}%
\pgfpathlineto{\pgfqpoint{3.235119in}{2.784474in}}%
\pgfpathlineto{\pgfqpoint{3.246048in}{2.773960in}}%
\pgfpathlineto{\pgfqpoint{3.256978in}{2.763969in}}%
\pgfpathlineto{\pgfqpoint{3.267907in}{2.754376in}}%
\pgfpathlineto{\pgfqpoint{3.278837in}{2.745432in}}%
\pgfpathlineto{\pgfqpoint{3.289766in}{2.736950in}}%
\pgfpathlineto{\pgfqpoint{3.300695in}{2.729116in}}%
\pgfpathlineto{\pgfqpoint{3.311625in}{2.721766in}}%
\pgfpathlineto{\pgfqpoint{3.322554in}{2.714765in}}%
\pgfpathlineto{\pgfqpoint{3.333484in}{2.708408in}}%
\pgfpathlineto{\pgfqpoint{3.344413in}{2.702452in}}%
\pgfpathlineto{\pgfqpoint{3.355343in}{2.696873in}}%
\pgfpathlineto{\pgfqpoint{3.366272in}{2.691752in}}%
\pgfpathlineto{\pgfqpoint{3.377202in}{2.687022in}}%
\pgfpathlineto{\pgfqpoint{3.388131in}{2.682635in}}%
\pgfpathlineto{\pgfqpoint{3.399061in}{2.678618in}}%
\pgfpathlineto{\pgfqpoint{3.409990in}{2.674887in}}%
\pgfpathlineto{\pgfqpoint{3.420919in}{2.671478in}}%
\pgfpathlineto{\pgfqpoint{3.442778in}{2.665443in}}%
\pgfpathlineto{\pgfqpoint{3.453708in}{2.662858in}}%
\pgfpathlineto{\pgfqpoint{3.475567in}{2.658359in}}%
\pgfpathlineto{\pgfqpoint{3.486496in}{2.656315in}}%
\pgfpathlineto{\pgfqpoint{3.508355in}{2.652905in}}%
\pgfpathlineto{\pgfqpoint{3.530214in}{2.650001in}}%
\pgfpathlineto{\pgfqpoint{3.563002in}{2.646705in}}%
\pgfpathlineto{\pgfqpoint{3.584861in}{2.645001in}}%
\pgfpathlineto{\pgfqpoint{3.606720in}{2.643529in}}%
\pgfpathlineto{\pgfqpoint{3.650438in}{2.641417in}}%
\pgfpathlineto{\pgfqpoint{3.694156in}{2.640038in}}%
\pgfpathlineto{\pgfqpoint{3.737874in}{2.639083in}}%
\pgfpathlineto{\pgfqpoint{3.792521in}{2.638360in}}%
\pgfpathlineto{\pgfqpoint{3.901816in}{2.637598in}}%
\pgfpathlineto{\pgfqpoint{4.328064in}{2.636929in}}%
\pgfpathlineto{\pgfqpoint{5.399151in}{2.636561in}}%
\pgfpathlineto{\pgfqpoint{6.133424in}{2.636553in}}%
\pgfpathlineto{\pgfqpoint{6.133424in}{2.636553in}}%
\pgfusepath{stroke}%
\end{pgfscope}%
\begin{pgfscope}%
\pgfsetrectcap%
\pgfsetmiterjoin%
\pgfsetlinewidth{0.803000pt}%
\definecolor{currentstroke}{rgb}{0.000000,0.000000,0.000000}%
\pgfsetstrokecolor{currentstroke}%
\pgfsetdash{}{0pt}%
\pgfpathmoveto{\pgfqpoint{0.557402in}{2.573088in}}%
\pgfpathlineto{\pgfqpoint{0.557402in}{3.961264in}}%
\pgfusepath{stroke}%
\end{pgfscope}%
\begin{pgfscope}%
\pgfsetrectcap%
\pgfsetmiterjoin%
\pgfsetlinewidth{0.803000pt}%
\definecolor{currentstroke}{rgb}{0.000000,0.000000,0.000000}%
\pgfsetstrokecolor{currentstroke}%
\pgfsetdash{}{0pt}%
\pgfpathmoveto{\pgfqpoint{6.131424in}{2.573088in}}%
\pgfpathlineto{\pgfqpoint{6.131424in}{3.961264in}}%
\pgfusepath{stroke}%
\end{pgfscope}%
\begin{pgfscope}%
\pgfsetrectcap%
\pgfsetmiterjoin%
\pgfsetlinewidth{0.803000pt}%
\definecolor{currentstroke}{rgb}{0.000000,0.000000,0.000000}%
\pgfsetstrokecolor{currentstroke}%
\pgfsetdash{}{0pt}%
\pgfpathmoveto{\pgfqpoint{0.557402in}{2.573088in}}%
\pgfpathlineto{\pgfqpoint{6.131424in}{2.573088in}}%
\pgfusepath{stroke}%
\end{pgfscope}%
\begin{pgfscope}%
\pgfsetrectcap%
\pgfsetmiterjoin%
\pgfsetlinewidth{0.803000pt}%
\definecolor{currentstroke}{rgb}{0.000000,0.000000,0.000000}%
\pgfsetstrokecolor{currentstroke}%
\pgfsetdash{}{0pt}%
\pgfpathmoveto{\pgfqpoint{0.557402in}{3.961264in}}%
\pgfpathlineto{\pgfqpoint{6.131424in}{3.961264in}}%
\pgfusepath{stroke}%
\end{pgfscope}%
\begin{pgfscope}%
\definecolor{textcolor}{rgb}{0.000000,0.000000,0.000000}%
\pgfsetstrokecolor{textcolor}%
\pgfsetfillcolor{textcolor}%
\pgftext[x=0.000000in,y=4.100082in,left,base]{\color{textcolor}\rmfamily\fontsize{10.000000}{12.000000}\selectfont (a)}%
\end{pgfscope}%
\begin{pgfscope}%
\pgfpathrectangle{\pgfqpoint{0.557402in}{2.573088in}}{\pgfqpoint{5.574022in}{1.388176in}}%
\pgfusepath{clip}%
\pgfsetbuttcap%
\pgfsetmiterjoin%
\pgfsetlinewidth{1.003750pt}%
\definecolor{currentstroke}{rgb}{0.000000,0.000000,0.000000}%
\pgfsetstrokecolor{currentstroke}%
\pgfsetstrokeopacity{0.500000}%
\pgfsetdash{}{0pt}%
\pgfpathmoveto{\pgfqpoint{3.497426in}{2.636187in}}%
\pgfpathlineto{\pgfqpoint{4.699666in}{2.636187in}}%
\pgfpathlineto{\pgfqpoint{4.699666in}{2.644256in}}%
\pgfpathlineto{\pgfqpoint{3.497426in}{2.644256in}}%
\pgfpathlineto{\pgfqpoint{3.497426in}{2.636187in}}%
\pgfpathclose%
\pgfusepath{stroke}%
\end{pgfscope}%
\begin{pgfscope}%
\pgfsetroundcap%
\pgfsetroundjoin%
\pgfsetlinewidth{1.003750pt}%
\definecolor{currentstroke}{rgb}{0.000000,0.000000,0.000000}%
\pgfsetstrokecolor{currentstroke}%
\pgfsetstrokeopacity{0.500000}%
\pgfsetdash{}{0pt}%
\pgfpathmoveto{\pgfqpoint{3.511634in}{3.711392in}}%
\pgfpathquadraticcurveto{\pgfqpoint{3.504530in}{3.177824in}}{\pgfqpoint{3.497426in}{2.644256in}}%
\pgfusepath{stroke}%
\end{pgfscope}%
\begin{pgfscope}%
\pgfsetroundcap%
\pgfsetroundjoin%
\pgfsetlinewidth{1.003750pt}%
\definecolor{currentstroke}{rgb}{0.000000,0.000000,0.000000}%
\pgfsetstrokecolor{currentstroke}%
\pgfsetstrokeopacity{0.500000}%
\pgfsetdash{}{0pt}%
\pgfpathmoveto{\pgfqpoint{5.072360in}{2.920132in}}%
\pgfpathquadraticcurveto{\pgfqpoint{4.886013in}{2.778160in}}{\pgfqpoint{4.699666in}{2.636187in}}%
\pgfusepath{stroke}%
\end{pgfscope}%
\begin{pgfscope}%
\pgfsetbuttcap%
\pgfsetmiterjoin%
\definecolor{currentfill}{rgb}{1.000000,1.000000,1.000000}%
\pgfsetfillcolor{currentfill}%
\pgfsetlinewidth{0.000000pt}%
\definecolor{currentstroke}{rgb}{0.000000,0.000000,0.000000}%
\pgfsetstrokecolor{currentstroke}%
\pgfsetstrokeopacity{0.000000}%
\pgfsetdash{}{0pt}%
\pgfpathmoveto{\pgfqpoint{3.511634in}{2.920132in}}%
\pgfpathlineto{\pgfqpoint{5.072360in}{2.920132in}}%
\pgfpathlineto{\pgfqpoint{5.072360in}{3.711392in}}%
\pgfpathlineto{\pgfqpoint{3.511634in}{3.711392in}}%
\pgfpathlineto{\pgfqpoint{3.511634in}{2.920132in}}%
\pgfpathclose%
\pgfusepath{fill}%
\end{pgfscope}%
\begin{pgfscope}%
\pgfsetbuttcap%
\pgfsetroundjoin%
\definecolor{currentfill}{rgb}{0.000000,0.000000,0.000000}%
\pgfsetfillcolor{currentfill}%
\pgfsetlinewidth{0.803000pt}%
\definecolor{currentstroke}{rgb}{0.000000,0.000000,0.000000}%
\pgfsetstrokecolor{currentstroke}%
\pgfsetdash{}{0pt}%
\pgfsys@defobject{currentmarker}{\pgfqpoint{0.000000in}{0.000000in}}{\pgfqpoint{0.000000in}{0.048611in}}{%
\pgfpathmoveto{\pgfqpoint{0.000000in}{0.000000in}}%
\pgfpathlineto{\pgfqpoint{0.000000in}{0.048611in}}%
\pgfusepath{stroke,fill}%
}%
\begin{pgfscope}%
\pgfsys@transformshift{3.525822in}{3.711392in}%
\pgfsys@useobject{currentmarker}{}%
\end{pgfscope}%
\end{pgfscope}%
\begin{pgfscope}%
\definecolor{textcolor}{rgb}{0.000000,0.000000,0.000000}%
\pgfsetstrokecolor{textcolor}%
\pgfsetfillcolor{textcolor}%
\pgftext[x=3.525822in,y=3.808615in,,bottom]{\color{textcolor}\rmfamily\fontsize{10.000000}{12.000000}\selectfont \(\displaystyle {60}\)}%
\end{pgfscope}%
\begin{pgfscope}%
\pgfsetbuttcap%
\pgfsetroundjoin%
\definecolor{currentfill}{rgb}{0.000000,0.000000,0.000000}%
\pgfsetfillcolor{currentfill}%
\pgfsetlinewidth{0.803000pt}%
\definecolor{currentstroke}{rgb}{0.000000,0.000000,0.000000}%
\pgfsetstrokecolor{currentstroke}%
\pgfsetdash{}{0pt}%
\pgfsys@defobject{currentmarker}{\pgfqpoint{0.000000in}{0.000000in}}{\pgfqpoint{0.000000in}{0.048611in}}{%
\pgfpathmoveto{\pgfqpoint{0.000000in}{0.000000in}}%
\pgfpathlineto{\pgfqpoint{0.000000in}{0.048611in}}%
\pgfusepath{stroke,fill}%
}%
\begin{pgfscope}%
\pgfsys@transformshift{3.951475in}{3.711392in}%
\pgfsys@useobject{currentmarker}{}%
\end{pgfscope}%
\end{pgfscope}%
\begin{pgfscope}%
\definecolor{textcolor}{rgb}{0.000000,0.000000,0.000000}%
\pgfsetstrokecolor{textcolor}%
\pgfsetfillcolor{textcolor}%
\pgftext[x=3.951475in,y=3.808615in,,bottom]{\color{textcolor}\rmfamily\fontsize{10.000000}{12.000000}\selectfont \(\displaystyle {90}\)}%
\end{pgfscope}%
\begin{pgfscope}%
\pgfsetbuttcap%
\pgfsetroundjoin%
\definecolor{currentfill}{rgb}{0.000000,0.000000,0.000000}%
\pgfsetfillcolor{currentfill}%
\pgfsetlinewidth{0.803000pt}%
\definecolor{currentstroke}{rgb}{0.000000,0.000000,0.000000}%
\pgfsetstrokecolor{currentstroke}%
\pgfsetdash{}{0pt}%
\pgfsys@defobject{currentmarker}{\pgfqpoint{0.000000in}{0.000000in}}{\pgfqpoint{0.000000in}{0.048611in}}{%
\pgfpathmoveto{\pgfqpoint{0.000000in}{0.000000in}}%
\pgfpathlineto{\pgfqpoint{0.000000in}{0.048611in}}%
\pgfusepath{stroke,fill}%
}%
\begin{pgfscope}%
\pgfsys@transformshift{4.377128in}{3.711392in}%
\pgfsys@useobject{currentmarker}{}%
\end{pgfscope}%
\end{pgfscope}%
\begin{pgfscope}%
\definecolor{textcolor}{rgb}{0.000000,0.000000,0.000000}%
\pgfsetstrokecolor{textcolor}%
\pgfsetfillcolor{textcolor}%
\pgftext[x=4.377128in,y=3.808615in,,bottom]{\color{textcolor}\rmfamily\fontsize{10.000000}{12.000000}\selectfont \(\displaystyle {120}\)}%
\end{pgfscope}%
\begin{pgfscope}%
\pgfsetbuttcap%
\pgfsetroundjoin%
\definecolor{currentfill}{rgb}{0.000000,0.000000,0.000000}%
\pgfsetfillcolor{currentfill}%
\pgfsetlinewidth{0.803000pt}%
\definecolor{currentstroke}{rgb}{0.000000,0.000000,0.000000}%
\pgfsetstrokecolor{currentstroke}%
\pgfsetdash{}{0pt}%
\pgfsys@defobject{currentmarker}{\pgfqpoint{0.000000in}{0.000000in}}{\pgfqpoint{0.000000in}{0.048611in}}{%
\pgfpathmoveto{\pgfqpoint{0.000000in}{0.000000in}}%
\pgfpathlineto{\pgfqpoint{0.000000in}{0.048611in}}%
\pgfusepath{stroke,fill}%
}%
\begin{pgfscope}%
\pgfsys@transformshift{4.802780in}{3.711392in}%
\pgfsys@useobject{currentmarker}{}%
\end{pgfscope}%
\end{pgfscope}%
\begin{pgfscope}%
\definecolor{textcolor}{rgb}{0.000000,0.000000,0.000000}%
\pgfsetstrokecolor{textcolor}%
\pgfsetfillcolor{textcolor}%
\pgftext[x=4.802780in,y=3.808615in,,bottom]{\color{textcolor}\rmfamily\fontsize{10.000000}{12.000000}\selectfont \(\displaystyle {150}\)}%
\end{pgfscope}%
\begin{pgfscope}%
\pgfsetbuttcap%
\pgfsetroundjoin%
\definecolor{currentfill}{rgb}{0.000000,0.000000,0.000000}%
\pgfsetfillcolor{currentfill}%
\pgfsetlinewidth{0.602250pt}%
\definecolor{currentstroke}{rgb}{0.000000,0.000000,0.000000}%
\pgfsetstrokecolor{currentstroke}%
\pgfsetdash{}{0pt}%
\pgfsys@defobject{currentmarker}{\pgfqpoint{0.000000in}{0.000000in}}{\pgfqpoint{0.000000in}{0.027778in}}{%
\pgfpathmoveto{\pgfqpoint{0.000000in}{0.000000in}}%
\pgfpathlineto{\pgfqpoint{0.000000in}{0.027778in}}%
\pgfusepath{stroke,fill}%
}%
\begin{pgfscope}%
\pgfsys@transformshift{3.632236in}{3.711392in}%
\pgfsys@useobject{currentmarker}{}%
\end{pgfscope}%
\end{pgfscope}%
\begin{pgfscope}%
\pgfsetbuttcap%
\pgfsetroundjoin%
\definecolor{currentfill}{rgb}{0.000000,0.000000,0.000000}%
\pgfsetfillcolor{currentfill}%
\pgfsetlinewidth{0.602250pt}%
\definecolor{currentstroke}{rgb}{0.000000,0.000000,0.000000}%
\pgfsetstrokecolor{currentstroke}%
\pgfsetdash{}{0pt}%
\pgfsys@defobject{currentmarker}{\pgfqpoint{0.000000in}{0.000000in}}{\pgfqpoint{0.000000in}{0.027778in}}{%
\pgfpathmoveto{\pgfqpoint{0.000000in}{0.000000in}}%
\pgfpathlineto{\pgfqpoint{0.000000in}{0.027778in}}%
\pgfusepath{stroke,fill}%
}%
\begin{pgfscope}%
\pgfsys@transformshift{3.738649in}{3.711392in}%
\pgfsys@useobject{currentmarker}{}%
\end{pgfscope}%
\end{pgfscope}%
\begin{pgfscope}%
\pgfsetbuttcap%
\pgfsetroundjoin%
\definecolor{currentfill}{rgb}{0.000000,0.000000,0.000000}%
\pgfsetfillcolor{currentfill}%
\pgfsetlinewidth{0.602250pt}%
\definecolor{currentstroke}{rgb}{0.000000,0.000000,0.000000}%
\pgfsetstrokecolor{currentstroke}%
\pgfsetdash{}{0pt}%
\pgfsys@defobject{currentmarker}{\pgfqpoint{0.000000in}{0.000000in}}{\pgfqpoint{0.000000in}{0.027778in}}{%
\pgfpathmoveto{\pgfqpoint{0.000000in}{0.000000in}}%
\pgfpathlineto{\pgfqpoint{0.000000in}{0.027778in}}%
\pgfusepath{stroke,fill}%
}%
\begin{pgfscope}%
\pgfsys@transformshift{3.845062in}{3.711392in}%
\pgfsys@useobject{currentmarker}{}%
\end{pgfscope}%
\end{pgfscope}%
\begin{pgfscope}%
\pgfsetbuttcap%
\pgfsetroundjoin%
\definecolor{currentfill}{rgb}{0.000000,0.000000,0.000000}%
\pgfsetfillcolor{currentfill}%
\pgfsetlinewidth{0.602250pt}%
\definecolor{currentstroke}{rgb}{0.000000,0.000000,0.000000}%
\pgfsetstrokecolor{currentstroke}%
\pgfsetdash{}{0pt}%
\pgfsys@defobject{currentmarker}{\pgfqpoint{0.000000in}{0.000000in}}{\pgfqpoint{0.000000in}{0.027778in}}{%
\pgfpathmoveto{\pgfqpoint{0.000000in}{0.000000in}}%
\pgfpathlineto{\pgfqpoint{0.000000in}{0.027778in}}%
\pgfusepath{stroke,fill}%
}%
\begin{pgfscope}%
\pgfsys@transformshift{4.057888in}{3.711392in}%
\pgfsys@useobject{currentmarker}{}%
\end{pgfscope}%
\end{pgfscope}%
\begin{pgfscope}%
\pgfsetbuttcap%
\pgfsetroundjoin%
\definecolor{currentfill}{rgb}{0.000000,0.000000,0.000000}%
\pgfsetfillcolor{currentfill}%
\pgfsetlinewidth{0.602250pt}%
\definecolor{currentstroke}{rgb}{0.000000,0.000000,0.000000}%
\pgfsetstrokecolor{currentstroke}%
\pgfsetdash{}{0pt}%
\pgfsys@defobject{currentmarker}{\pgfqpoint{0.000000in}{0.000000in}}{\pgfqpoint{0.000000in}{0.027778in}}{%
\pgfpathmoveto{\pgfqpoint{0.000000in}{0.000000in}}%
\pgfpathlineto{\pgfqpoint{0.000000in}{0.027778in}}%
\pgfusepath{stroke,fill}%
}%
\begin{pgfscope}%
\pgfsys@transformshift{4.164301in}{3.711392in}%
\pgfsys@useobject{currentmarker}{}%
\end{pgfscope}%
\end{pgfscope}%
\begin{pgfscope}%
\pgfsetbuttcap%
\pgfsetroundjoin%
\definecolor{currentfill}{rgb}{0.000000,0.000000,0.000000}%
\pgfsetfillcolor{currentfill}%
\pgfsetlinewidth{0.602250pt}%
\definecolor{currentstroke}{rgb}{0.000000,0.000000,0.000000}%
\pgfsetstrokecolor{currentstroke}%
\pgfsetdash{}{0pt}%
\pgfsys@defobject{currentmarker}{\pgfqpoint{0.000000in}{0.000000in}}{\pgfqpoint{0.000000in}{0.027778in}}{%
\pgfpathmoveto{\pgfqpoint{0.000000in}{0.000000in}}%
\pgfpathlineto{\pgfqpoint{0.000000in}{0.027778in}}%
\pgfusepath{stroke,fill}%
}%
\begin{pgfscope}%
\pgfsys@transformshift{4.270714in}{3.711392in}%
\pgfsys@useobject{currentmarker}{}%
\end{pgfscope}%
\end{pgfscope}%
\begin{pgfscope}%
\pgfsetbuttcap%
\pgfsetroundjoin%
\definecolor{currentfill}{rgb}{0.000000,0.000000,0.000000}%
\pgfsetfillcolor{currentfill}%
\pgfsetlinewidth{0.602250pt}%
\definecolor{currentstroke}{rgb}{0.000000,0.000000,0.000000}%
\pgfsetstrokecolor{currentstroke}%
\pgfsetdash{}{0pt}%
\pgfsys@defobject{currentmarker}{\pgfqpoint{0.000000in}{0.000000in}}{\pgfqpoint{0.000000in}{0.027778in}}{%
\pgfpathmoveto{\pgfqpoint{0.000000in}{0.000000in}}%
\pgfpathlineto{\pgfqpoint{0.000000in}{0.027778in}}%
\pgfusepath{stroke,fill}%
}%
\begin{pgfscope}%
\pgfsys@transformshift{4.483541in}{3.711392in}%
\pgfsys@useobject{currentmarker}{}%
\end{pgfscope}%
\end{pgfscope}%
\begin{pgfscope}%
\pgfsetbuttcap%
\pgfsetroundjoin%
\definecolor{currentfill}{rgb}{0.000000,0.000000,0.000000}%
\pgfsetfillcolor{currentfill}%
\pgfsetlinewidth{0.602250pt}%
\definecolor{currentstroke}{rgb}{0.000000,0.000000,0.000000}%
\pgfsetstrokecolor{currentstroke}%
\pgfsetdash{}{0pt}%
\pgfsys@defobject{currentmarker}{\pgfqpoint{0.000000in}{0.000000in}}{\pgfqpoint{0.000000in}{0.027778in}}{%
\pgfpathmoveto{\pgfqpoint{0.000000in}{0.000000in}}%
\pgfpathlineto{\pgfqpoint{0.000000in}{0.027778in}}%
\pgfusepath{stroke,fill}%
}%
\begin{pgfscope}%
\pgfsys@transformshift{4.589954in}{3.711392in}%
\pgfsys@useobject{currentmarker}{}%
\end{pgfscope}%
\end{pgfscope}%
\begin{pgfscope}%
\pgfsetbuttcap%
\pgfsetroundjoin%
\definecolor{currentfill}{rgb}{0.000000,0.000000,0.000000}%
\pgfsetfillcolor{currentfill}%
\pgfsetlinewidth{0.602250pt}%
\definecolor{currentstroke}{rgb}{0.000000,0.000000,0.000000}%
\pgfsetstrokecolor{currentstroke}%
\pgfsetdash{}{0pt}%
\pgfsys@defobject{currentmarker}{\pgfqpoint{0.000000in}{0.000000in}}{\pgfqpoint{0.000000in}{0.027778in}}{%
\pgfpathmoveto{\pgfqpoint{0.000000in}{0.000000in}}%
\pgfpathlineto{\pgfqpoint{0.000000in}{0.027778in}}%
\pgfusepath{stroke,fill}%
}%
\begin{pgfscope}%
\pgfsys@transformshift{4.696367in}{3.711392in}%
\pgfsys@useobject{currentmarker}{}%
\end{pgfscope}%
\end{pgfscope}%
\begin{pgfscope}%
\pgfsetbuttcap%
\pgfsetroundjoin%
\definecolor{currentfill}{rgb}{0.000000,0.000000,0.000000}%
\pgfsetfillcolor{currentfill}%
\pgfsetlinewidth{0.602250pt}%
\definecolor{currentstroke}{rgb}{0.000000,0.000000,0.000000}%
\pgfsetstrokecolor{currentstroke}%
\pgfsetdash{}{0pt}%
\pgfsys@defobject{currentmarker}{\pgfqpoint{0.000000in}{0.000000in}}{\pgfqpoint{0.000000in}{0.027778in}}{%
\pgfpathmoveto{\pgfqpoint{0.000000in}{0.000000in}}%
\pgfpathlineto{\pgfqpoint{0.000000in}{0.027778in}}%
\pgfusepath{stroke,fill}%
}%
\begin{pgfscope}%
\pgfsys@transformshift{4.909193in}{3.711392in}%
\pgfsys@useobject{currentmarker}{}%
\end{pgfscope}%
\end{pgfscope}%
\begin{pgfscope}%
\pgfsetbuttcap%
\pgfsetroundjoin%
\definecolor{currentfill}{rgb}{0.000000,0.000000,0.000000}%
\pgfsetfillcolor{currentfill}%
\pgfsetlinewidth{0.602250pt}%
\definecolor{currentstroke}{rgb}{0.000000,0.000000,0.000000}%
\pgfsetstrokecolor{currentstroke}%
\pgfsetdash{}{0pt}%
\pgfsys@defobject{currentmarker}{\pgfqpoint{0.000000in}{0.000000in}}{\pgfqpoint{0.000000in}{0.027778in}}{%
\pgfpathmoveto{\pgfqpoint{0.000000in}{0.000000in}}%
\pgfpathlineto{\pgfqpoint{0.000000in}{0.027778in}}%
\pgfusepath{stroke,fill}%
}%
\begin{pgfscope}%
\pgfsys@transformshift{5.015607in}{3.711392in}%
\pgfsys@useobject{currentmarker}{}%
\end{pgfscope}%
\end{pgfscope}%
\begin{pgfscope}%
\pgfsetbuttcap%
\pgfsetroundjoin%
\definecolor{currentfill}{rgb}{0.000000,0.000000,0.000000}%
\pgfsetfillcolor{currentfill}%
\pgfsetlinewidth{0.803000pt}%
\definecolor{currentstroke}{rgb}{0.000000,0.000000,0.000000}%
\pgfsetstrokecolor{currentstroke}%
\pgfsetdash{}{0pt}%
\pgfsys@defobject{currentmarker}{\pgfqpoint{0.000000in}{0.000000in}}{\pgfqpoint{0.048611in}{0.000000in}}{%
\pgfpathmoveto{\pgfqpoint{0.000000in}{0.000000in}}%
\pgfpathlineto{\pgfqpoint{0.048611in}{0.000000in}}%
\pgfusepath{stroke,fill}%
}%
\begin{pgfscope}%
\pgfsys@transformshift{5.072360in}{2.955982in}%
\pgfsys@useobject{currentmarker}{}%
\end{pgfscope}%
\end{pgfscope}%
\begin{pgfscope}%
\definecolor{textcolor}{rgb}{0.000000,0.000000,0.000000}%
\pgfsetstrokecolor{textcolor}%
\pgfsetfillcolor{textcolor}%
\pgftext[x=5.169582in, y=2.908154in, left, base]{\color{textcolor}\rmfamily\fontsize{10.000000}{12.000000}\selectfont \(\displaystyle {0}\)}%
\end{pgfscope}%
\begin{pgfscope}%
\pgfsetbuttcap%
\pgfsetroundjoin%
\definecolor{currentfill}{rgb}{0.000000,0.000000,0.000000}%
\pgfsetfillcolor{currentfill}%
\pgfsetlinewidth{0.803000pt}%
\definecolor{currentstroke}{rgb}{0.000000,0.000000,0.000000}%
\pgfsetstrokecolor{currentstroke}%
\pgfsetdash{}{0pt}%
\pgfsys@defobject{currentmarker}{\pgfqpoint{0.000000in}{0.000000in}}{\pgfqpoint{0.048611in}{0.000000in}}{%
\pgfpathmoveto{\pgfqpoint{0.000000in}{0.000000in}}%
\pgfpathlineto{\pgfqpoint{0.048611in}{0.000000in}}%
\pgfusepath{stroke,fill}%
}%
\begin{pgfscope}%
\pgfsys@transformshift{5.072360in}{3.343286in}%
\pgfsys@useobject{currentmarker}{}%
\end{pgfscope}%
\end{pgfscope}%
\begin{pgfscope}%
\definecolor{textcolor}{rgb}{0.000000,0.000000,0.000000}%
\pgfsetstrokecolor{textcolor}%
\pgfsetfillcolor{textcolor}%
\pgftext[x=5.169582in, y=3.295458in, left, base]{\color{textcolor}\rmfamily\fontsize{10.000000}{12.000000}\selectfont \(\displaystyle {100000}\)}%
\end{pgfscope}%
\begin{pgfscope}%
\pgfsetbuttcap%
\pgfsetroundjoin%
\definecolor{currentfill}{rgb}{0.000000,0.000000,0.000000}%
\pgfsetfillcolor{currentfill}%
\pgfsetlinewidth{0.602250pt}%
\definecolor{currentstroke}{rgb}{0.000000,0.000000,0.000000}%
\pgfsetstrokecolor{currentstroke}%
\pgfsetdash{}{0pt}%
\pgfsys@defobject{currentmarker}{\pgfqpoint{0.000000in}{0.000000in}}{\pgfqpoint{0.027778in}{0.000000in}}{%
\pgfpathmoveto{\pgfqpoint{0.000000in}{0.000000in}}%
\pgfpathlineto{\pgfqpoint{0.027778in}{0.000000in}}%
\pgfusepath{stroke,fill}%
}%
\begin{pgfscope}%
\pgfsys@transformshift{5.072360in}{3.033443in}%
\pgfsys@useobject{currentmarker}{}%
\end{pgfscope}%
\end{pgfscope}%
\begin{pgfscope}%
\pgfsetbuttcap%
\pgfsetroundjoin%
\definecolor{currentfill}{rgb}{0.000000,0.000000,0.000000}%
\pgfsetfillcolor{currentfill}%
\pgfsetlinewidth{0.602250pt}%
\definecolor{currentstroke}{rgb}{0.000000,0.000000,0.000000}%
\pgfsetstrokecolor{currentstroke}%
\pgfsetdash{}{0pt}%
\pgfsys@defobject{currentmarker}{\pgfqpoint{0.000000in}{0.000000in}}{\pgfqpoint{0.027778in}{0.000000in}}{%
\pgfpathmoveto{\pgfqpoint{0.000000in}{0.000000in}}%
\pgfpathlineto{\pgfqpoint{0.027778in}{0.000000in}}%
\pgfusepath{stroke,fill}%
}%
\begin{pgfscope}%
\pgfsys@transformshift{5.072360in}{3.110904in}%
\pgfsys@useobject{currentmarker}{}%
\end{pgfscope}%
\end{pgfscope}%
\begin{pgfscope}%
\pgfsetbuttcap%
\pgfsetroundjoin%
\definecolor{currentfill}{rgb}{0.000000,0.000000,0.000000}%
\pgfsetfillcolor{currentfill}%
\pgfsetlinewidth{0.602250pt}%
\definecolor{currentstroke}{rgb}{0.000000,0.000000,0.000000}%
\pgfsetstrokecolor{currentstroke}%
\pgfsetdash{}{0pt}%
\pgfsys@defobject{currentmarker}{\pgfqpoint{0.000000in}{0.000000in}}{\pgfqpoint{0.027778in}{0.000000in}}{%
\pgfpathmoveto{\pgfqpoint{0.000000in}{0.000000in}}%
\pgfpathlineto{\pgfqpoint{0.027778in}{0.000000in}}%
\pgfusepath{stroke,fill}%
}%
\begin{pgfscope}%
\pgfsys@transformshift{5.072360in}{3.188365in}%
\pgfsys@useobject{currentmarker}{}%
\end{pgfscope}%
\end{pgfscope}%
\begin{pgfscope}%
\pgfsetbuttcap%
\pgfsetroundjoin%
\definecolor{currentfill}{rgb}{0.000000,0.000000,0.000000}%
\pgfsetfillcolor{currentfill}%
\pgfsetlinewidth{0.602250pt}%
\definecolor{currentstroke}{rgb}{0.000000,0.000000,0.000000}%
\pgfsetstrokecolor{currentstroke}%
\pgfsetdash{}{0pt}%
\pgfsys@defobject{currentmarker}{\pgfqpoint{0.000000in}{0.000000in}}{\pgfqpoint{0.027778in}{0.000000in}}{%
\pgfpathmoveto{\pgfqpoint{0.000000in}{0.000000in}}%
\pgfpathlineto{\pgfqpoint{0.027778in}{0.000000in}}%
\pgfusepath{stroke,fill}%
}%
\begin{pgfscope}%
\pgfsys@transformshift{5.072360in}{3.265825in}%
\pgfsys@useobject{currentmarker}{}%
\end{pgfscope}%
\end{pgfscope}%
\begin{pgfscope}%
\pgfsetbuttcap%
\pgfsetroundjoin%
\definecolor{currentfill}{rgb}{0.000000,0.000000,0.000000}%
\pgfsetfillcolor{currentfill}%
\pgfsetlinewidth{0.602250pt}%
\definecolor{currentstroke}{rgb}{0.000000,0.000000,0.000000}%
\pgfsetstrokecolor{currentstroke}%
\pgfsetdash{}{0pt}%
\pgfsys@defobject{currentmarker}{\pgfqpoint{0.000000in}{0.000000in}}{\pgfqpoint{0.027778in}{0.000000in}}{%
\pgfpathmoveto{\pgfqpoint{0.000000in}{0.000000in}}%
\pgfpathlineto{\pgfqpoint{0.027778in}{0.000000in}}%
\pgfusepath{stroke,fill}%
}%
\begin{pgfscope}%
\pgfsys@transformshift{5.072360in}{3.420747in}%
\pgfsys@useobject{currentmarker}{}%
\end{pgfscope}%
\end{pgfscope}%
\begin{pgfscope}%
\pgfsetbuttcap%
\pgfsetroundjoin%
\definecolor{currentfill}{rgb}{0.000000,0.000000,0.000000}%
\pgfsetfillcolor{currentfill}%
\pgfsetlinewidth{0.602250pt}%
\definecolor{currentstroke}{rgb}{0.000000,0.000000,0.000000}%
\pgfsetstrokecolor{currentstroke}%
\pgfsetdash{}{0pt}%
\pgfsys@defobject{currentmarker}{\pgfqpoint{0.000000in}{0.000000in}}{\pgfqpoint{0.027778in}{0.000000in}}{%
\pgfpathmoveto{\pgfqpoint{0.000000in}{0.000000in}}%
\pgfpathlineto{\pgfqpoint{0.027778in}{0.000000in}}%
\pgfusepath{stroke,fill}%
}%
\begin{pgfscope}%
\pgfsys@transformshift{5.072360in}{3.498207in}%
\pgfsys@useobject{currentmarker}{}%
\end{pgfscope}%
\end{pgfscope}%
\begin{pgfscope}%
\pgfsetbuttcap%
\pgfsetroundjoin%
\definecolor{currentfill}{rgb}{0.000000,0.000000,0.000000}%
\pgfsetfillcolor{currentfill}%
\pgfsetlinewidth{0.602250pt}%
\definecolor{currentstroke}{rgb}{0.000000,0.000000,0.000000}%
\pgfsetstrokecolor{currentstroke}%
\pgfsetdash{}{0pt}%
\pgfsys@defobject{currentmarker}{\pgfqpoint{0.000000in}{0.000000in}}{\pgfqpoint{0.027778in}{0.000000in}}{%
\pgfpathmoveto{\pgfqpoint{0.000000in}{0.000000in}}%
\pgfpathlineto{\pgfqpoint{0.027778in}{0.000000in}}%
\pgfusepath{stroke,fill}%
}%
\begin{pgfscope}%
\pgfsys@transformshift{5.072360in}{3.575668in}%
\pgfsys@useobject{currentmarker}{}%
\end{pgfscope}%
\end{pgfscope}%
\begin{pgfscope}%
\pgfsetbuttcap%
\pgfsetroundjoin%
\definecolor{currentfill}{rgb}{0.000000,0.000000,0.000000}%
\pgfsetfillcolor{currentfill}%
\pgfsetlinewidth{0.602250pt}%
\definecolor{currentstroke}{rgb}{0.000000,0.000000,0.000000}%
\pgfsetstrokecolor{currentstroke}%
\pgfsetdash{}{0pt}%
\pgfsys@defobject{currentmarker}{\pgfqpoint{0.000000in}{0.000000in}}{\pgfqpoint{0.027778in}{0.000000in}}{%
\pgfpathmoveto{\pgfqpoint{0.000000in}{0.000000in}}%
\pgfpathlineto{\pgfqpoint{0.027778in}{0.000000in}}%
\pgfusepath{stroke,fill}%
}%
\begin{pgfscope}%
\pgfsys@transformshift{5.072360in}{3.653129in}%
\pgfsys@useobject{currentmarker}{}%
\end{pgfscope}%
\end{pgfscope}%
\begin{pgfscope}%
\pgfpathrectangle{\pgfqpoint{3.511634in}{2.920132in}}{\pgfqpoint{1.560726in}{0.791260in}}%
\pgfusepath{clip}%
\pgfsetrectcap%
\pgfsetroundjoin%
\pgfsetlinewidth{1.505625pt}%
\definecolor{currentstroke}{rgb}{0.121569,0.466667,0.705882}%
\pgfsetstrokecolor{currentstroke}%
\pgfsetdash{}{0pt}%
\pgfpathmoveto{\pgfqpoint{3.653518in}{3.675426in}}%
\pgfpathlineto{\pgfqpoint{3.667707in}{3.602714in}}%
\pgfpathlineto{\pgfqpoint{3.681895in}{3.535183in}}%
\pgfpathlineto{\pgfqpoint{3.696083in}{3.471379in}}%
\pgfpathlineto{\pgfqpoint{3.710272in}{3.418113in}}%
\pgfpathlineto{\pgfqpoint{3.724460in}{3.369673in}}%
\pgfpathlineto{\pgfqpoint{3.738649in}{3.328588in}}%
\pgfpathlineto{\pgfqpoint{3.752837in}{3.289985in}}%
\pgfpathlineto{\pgfqpoint{3.767026in}{3.254102in}}%
\pgfpathlineto{\pgfqpoint{3.781214in}{3.221436in}}%
\pgfpathlineto{\pgfqpoint{3.795402in}{3.194217in}}%
\pgfpathlineto{\pgfqpoint{3.809591in}{3.169615in}}%
\pgfpathlineto{\pgfqpoint{3.823779in}{3.147248in}}%
\pgfpathlineto{\pgfqpoint{3.837968in}{3.125830in}}%
\pgfpathlineto{\pgfqpoint{3.852156in}{3.110671in}}%
\pgfpathlineto{\pgfqpoint{3.866344in}{3.093979in}}%
\pgfpathlineto{\pgfqpoint{3.880533in}{3.078781in}}%
\pgfpathlineto{\pgfqpoint{3.894721in}{3.066155in}}%
\pgfpathlineto{\pgfqpoint{3.908910in}{3.055353in}}%
\pgfpathlineto{\pgfqpoint{3.923098in}{3.044284in}}%
\pgfpathlineto{\pgfqpoint{3.937287in}{3.034888in}}%
\pgfpathlineto{\pgfqpoint{3.951475in}{3.027184in}}%
\pgfpathlineto{\pgfqpoint{3.965663in}{3.019097in}}%
\pgfpathlineto{\pgfqpoint{3.979852in}{3.012928in}}%
\pgfpathlineto{\pgfqpoint{3.994040in}{3.005801in}}%
\pgfpathlineto{\pgfqpoint{4.008229in}{3.001130in}}%
\pgfpathlineto{\pgfqpoint{4.022417in}{2.995723in}}%
\pgfpathlineto{\pgfqpoint{4.036606in}{2.991014in}}%
\pgfpathlineto{\pgfqpoint{4.050794in}{2.988365in}}%
\pgfpathlineto{\pgfqpoint{4.064982in}{2.984116in}}%
\pgfpathlineto{\pgfqpoint{4.079171in}{2.981219in}}%
\pgfpathlineto{\pgfqpoint{4.093359in}{2.977931in}}%
\pgfpathlineto{\pgfqpoint{4.107548in}{2.976134in}}%
\pgfpathlineto{\pgfqpoint{4.121736in}{2.973353in}}%
\pgfpathlineto{\pgfqpoint{4.135924in}{2.971935in}}%
\pgfpathlineto{\pgfqpoint{4.150113in}{2.970138in}}%
\pgfpathlineto{\pgfqpoint{4.164301in}{2.968825in}}%
\pgfpathlineto{\pgfqpoint{4.178490in}{2.967226in}}%
\pgfpathlineto{\pgfqpoint{4.192678in}{2.966195in}}%
\pgfpathlineto{\pgfqpoint{4.206867in}{2.964778in}}%
\pgfpathlineto{\pgfqpoint{4.221055in}{2.963856in}}%
\pgfpathlineto{\pgfqpoint{4.235243in}{2.963198in}}%
\pgfpathlineto{\pgfqpoint{4.249432in}{2.962036in}}%
\pgfpathlineto{\pgfqpoint{4.263620in}{2.961424in}}%
\pgfpathlineto{\pgfqpoint{4.277809in}{2.960692in}}%
\pgfpathlineto{\pgfqpoint{4.291997in}{2.960370in}}%
\pgfpathlineto{\pgfqpoint{4.306186in}{2.960014in}}%
\pgfpathlineto{\pgfqpoint{4.320374in}{2.959425in}}%
\pgfpathlineto{\pgfqpoint{4.334562in}{2.959030in}}%
\pgfpathlineto{\pgfqpoint{4.348751in}{2.958848in}}%
\pgfpathlineto{\pgfqpoint{4.362939in}{2.958395in}}%
\pgfpathlineto{\pgfqpoint{4.377128in}{2.958023in}}%
\pgfpathlineto{\pgfqpoint{4.391316in}{2.957884in}}%
\pgfpathlineto{\pgfqpoint{4.405504in}{2.957756in}}%
\pgfpathlineto{\pgfqpoint{4.419693in}{2.957640in}}%
\pgfpathlineto{\pgfqpoint{4.433881in}{2.957419in}}%
\pgfpathlineto{\pgfqpoint{4.448070in}{2.957067in}}%
\pgfpathlineto{\pgfqpoint{4.462258in}{2.957105in}}%
\pgfpathlineto{\pgfqpoint{4.476447in}{2.957152in}}%
\pgfpathlineto{\pgfqpoint{4.490635in}{2.956900in}}%
\pgfpathlineto{\pgfqpoint{4.504823in}{2.956846in}}%
\pgfpathlineto{\pgfqpoint{4.519012in}{2.956683in}}%
\pgfpathlineto{\pgfqpoint{4.533200in}{2.956776in}}%
\pgfpathlineto{\pgfqpoint{4.547389in}{2.956714in}}%
\pgfpathlineto{\pgfqpoint{4.561577in}{2.956679in}}%
\pgfpathlineto{\pgfqpoint{4.575765in}{2.956544in}}%
\pgfpathlineto{\pgfqpoint{4.589954in}{2.956509in}}%
\pgfpathlineto{\pgfqpoint{4.604142in}{2.956513in}}%
\pgfpathlineto{\pgfqpoint{4.618331in}{2.956459in}}%
\pgfpathlineto{\pgfqpoint{4.632519in}{2.956366in}}%
\pgfpathlineto{\pgfqpoint{4.646708in}{2.956397in}}%
\pgfpathlineto{\pgfqpoint{4.660896in}{2.956377in}}%
\pgfpathlineto{\pgfqpoint{4.675084in}{2.956362in}}%
\pgfpathlineto{\pgfqpoint{4.689273in}{2.956381in}}%
\pgfpathlineto{\pgfqpoint{4.703461in}{2.956362in}}%
\pgfpathlineto{\pgfqpoint{4.717650in}{2.956288in}}%
\pgfpathlineto{\pgfqpoint{4.731838in}{2.956234in}}%
\pgfpathlineto{\pgfqpoint{4.746027in}{2.956199in}}%
\pgfpathlineto{\pgfqpoint{4.760215in}{2.956300in}}%
\pgfpathlineto{\pgfqpoint{4.774403in}{2.956195in}}%
\pgfpathlineto{\pgfqpoint{4.788592in}{2.956215in}}%
\pgfpathlineto{\pgfqpoint{4.802780in}{2.956219in}}%
\pgfpathlineto{\pgfqpoint{4.816969in}{2.956191in}}%
\pgfpathlineto{\pgfqpoint{4.831157in}{2.956207in}}%
\pgfpathlineto{\pgfqpoint{4.845345in}{2.956195in}}%
\pgfpathlineto{\pgfqpoint{4.859534in}{2.956176in}}%
\pgfpathlineto{\pgfqpoint{4.873722in}{2.956168in}}%
\pgfpathlineto{\pgfqpoint{4.887911in}{2.956211in}}%
\pgfpathlineto{\pgfqpoint{4.902099in}{2.956160in}}%
\pgfpathlineto{\pgfqpoint{4.916288in}{2.956164in}}%
\pgfpathlineto{\pgfqpoint{4.930476in}{2.956199in}}%
\pgfpathlineto{\pgfqpoint{4.944664in}{2.956129in}}%
\pgfpathlineto{\pgfqpoint{4.958853in}{2.956172in}}%
\pgfpathlineto{\pgfqpoint{4.973041in}{2.956122in}}%
\pgfpathlineto{\pgfqpoint{4.987230in}{2.956114in}}%
\pgfpathlineto{\pgfqpoint{5.001418in}{2.956110in}}%
\pgfpathlineto{\pgfqpoint{5.015607in}{2.956106in}}%
\pgfpathlineto{\pgfqpoint{5.029795in}{2.956184in}}%
\pgfpathlineto{\pgfqpoint{5.043983in}{2.956141in}}%
\pgfpathlineto{\pgfqpoint{5.058172in}{2.956098in}}%
\pgfusepath{stroke}%
\end{pgfscope}%
\begin{pgfscope}%
\pgfpathrectangle{\pgfqpoint{3.511634in}{2.920132in}}{\pgfqpoint{1.560726in}{0.791260in}}%
\pgfusepath{clip}%
\pgfsetrectcap%
\pgfsetroundjoin%
\pgfsetlinewidth{1.505625pt}%
\definecolor{currentstroke}{rgb}{1.000000,0.498039,0.054902}%
\pgfsetstrokecolor{currentstroke}%
\pgfsetdash{}{0pt}%
\pgfpathmoveto{\pgfqpoint{3.653518in}{3.640096in}}%
\pgfpathlineto{\pgfqpoint{3.667707in}{3.583724in}}%
\pgfpathlineto{\pgfqpoint{3.681895in}{3.525501in}}%
\pgfpathlineto{\pgfqpoint{3.696083in}{3.479233in}}%
\pgfpathlineto{\pgfqpoint{3.710272in}{3.432982in}}%
\pgfpathlineto{\pgfqpoint{3.724460in}{3.395196in}}%
\pgfpathlineto{\pgfqpoint{3.738649in}{3.360598in}}%
\pgfpathlineto{\pgfqpoint{3.752837in}{3.324664in}}%
\pgfpathlineto{\pgfqpoint{3.767026in}{3.297758in}}%
\pgfpathlineto{\pgfqpoint{3.781214in}{3.271096in}}%
\pgfpathlineto{\pgfqpoint{3.795402in}{3.248796in}}%
\pgfpathlineto{\pgfqpoint{3.809591in}{3.225003in}}%
\pgfpathlineto{\pgfqpoint{3.823779in}{3.204097in}}%
\pgfpathlineto{\pgfqpoint{3.837968in}{3.187017in}}%
\pgfpathlineto{\pgfqpoint{3.852156in}{3.171033in}}%
\pgfpathlineto{\pgfqpoint{3.866344in}{3.157361in}}%
\pgfpathlineto{\pgfqpoint{3.880533in}{3.143809in}}%
\pgfpathlineto{\pgfqpoint{3.894721in}{3.133166in}}%
\pgfpathlineto{\pgfqpoint{3.908910in}{3.123863in}}%
\pgfpathlineto{\pgfqpoint{3.923098in}{3.110222in}}%
\pgfpathlineto{\pgfqpoint{3.937287in}{3.102193in}}%
\pgfpathlineto{\pgfqpoint{3.951475in}{3.093587in}}%
\pgfpathlineto{\pgfqpoint{3.965663in}{3.086872in}}%
\pgfpathlineto{\pgfqpoint{3.979852in}{3.080717in}}%
\pgfpathlineto{\pgfqpoint{3.994040in}{3.072890in}}%
\pgfpathlineto{\pgfqpoint{4.008229in}{3.066941in}}%
\pgfpathlineto{\pgfqpoint{4.022417in}{3.062832in}}%
\pgfpathlineto{\pgfqpoint{4.036606in}{3.058494in}}%
\pgfpathlineto{\pgfqpoint{4.050794in}{3.056271in}}%
\pgfpathlineto{\pgfqpoint{4.064982in}{3.051073in}}%
\pgfpathlineto{\pgfqpoint{4.079171in}{3.046309in}}%
\pgfpathlineto{\pgfqpoint{4.093359in}{3.043122in}}%
\pgfpathlineto{\pgfqpoint{4.107548in}{3.039609in}}%
\pgfpathlineto{\pgfqpoint{4.121736in}{3.038745in}}%
\pgfpathlineto{\pgfqpoint{4.135924in}{3.034934in}}%
\pgfpathlineto{\pgfqpoint{4.150113in}{3.032537in}}%
\pgfpathlineto{\pgfqpoint{4.164301in}{3.031173in}}%
\pgfpathlineto{\pgfqpoint{4.178490in}{3.028354in}}%
\pgfpathlineto{\pgfqpoint{4.192678in}{3.027308in}}%
\pgfpathlineto{\pgfqpoint{4.206867in}{3.025527in}}%
\pgfpathlineto{\pgfqpoint{4.221055in}{3.023997in}}%
\pgfpathlineto{\pgfqpoint{4.235243in}{3.021584in}}%
\pgfpathlineto{\pgfqpoint{4.249432in}{3.020186in}}%
\pgfpathlineto{\pgfqpoint{4.263620in}{3.018509in}}%
\pgfpathlineto{\pgfqpoint{4.277809in}{3.017413in}}%
\pgfpathlineto{\pgfqpoint{4.291997in}{3.016398in}}%
\pgfpathlineto{\pgfqpoint{4.306186in}{3.014082in}}%
\pgfpathlineto{\pgfqpoint{4.320374in}{3.013811in}}%
\pgfpathlineto{\pgfqpoint{4.334562in}{3.012126in}}%
\pgfpathlineto{\pgfqpoint{4.348751in}{3.011049in}}%
\pgfpathlineto{\pgfqpoint{4.362939in}{3.009728in}}%
\pgfpathlineto{\pgfqpoint{4.377128in}{3.009794in}}%
\pgfpathlineto{\pgfqpoint{4.391316in}{3.007315in}}%
\pgfpathlineto{\pgfqpoint{4.405504in}{3.006804in}}%
\pgfpathlineto{\pgfqpoint{4.419693in}{3.005356in}}%
\pgfpathlineto{\pgfqpoint{4.433881in}{3.004465in}}%
\pgfpathlineto{\pgfqpoint{4.448070in}{3.003396in}}%
\pgfpathlineto{\pgfqpoint{4.462258in}{3.002033in}}%
\pgfpathlineto{\pgfqpoint{4.476447in}{3.000940in}}%
\pgfpathlineto{\pgfqpoint{4.490635in}{3.000495in}}%
\pgfpathlineto{\pgfqpoint{4.504823in}{2.999399in}}%
\pgfpathlineto{\pgfqpoint{4.519012in}{2.997850in}}%
\pgfpathlineto{\pgfqpoint{4.533200in}{2.996556in}}%
\pgfpathlineto{\pgfqpoint{4.547389in}{2.995967in}}%
\pgfpathlineto{\pgfqpoint{4.561577in}{2.995615in}}%
\pgfpathlineto{\pgfqpoint{4.575765in}{2.993233in}}%
\pgfpathlineto{\pgfqpoint{4.589954in}{2.992935in}}%
\pgfpathlineto{\pgfqpoint{4.604142in}{2.992350in}}%
\pgfpathlineto{\pgfqpoint{4.618331in}{2.990696in}}%
\pgfpathlineto{\pgfqpoint{4.632519in}{2.989937in}}%
\pgfpathlineto{\pgfqpoint{4.646708in}{2.989713in}}%
\pgfpathlineto{\pgfqpoint{4.660896in}{2.988175in}}%
\pgfpathlineto{\pgfqpoint{4.675084in}{2.986572in}}%
\pgfpathlineto{\pgfqpoint{4.689273in}{2.985615in}}%
\pgfpathlineto{\pgfqpoint{4.703461in}{2.984999in}}%
\pgfpathlineto{\pgfqpoint{4.717650in}{2.983717in}}%
\pgfpathlineto{\pgfqpoint{4.731838in}{2.982648in}}%
\pgfpathlineto{\pgfqpoint{4.746027in}{2.981924in}}%
\pgfpathlineto{\pgfqpoint{4.760215in}{2.981362in}}%
\pgfpathlineto{\pgfqpoint{4.774403in}{2.980464in}}%
\pgfpathlineto{\pgfqpoint{4.788592in}{2.979507in}}%
\pgfpathlineto{\pgfqpoint{4.802780in}{2.978976in}}%
\pgfpathlineto{\pgfqpoint{4.816969in}{2.978020in}}%
\pgfpathlineto{\pgfqpoint{4.831157in}{2.976734in}}%
\pgfpathlineto{\pgfqpoint{4.845345in}{2.975835in}}%
\pgfpathlineto{\pgfqpoint{4.859534in}{2.975622in}}%
\pgfpathlineto{\pgfqpoint{4.873722in}{2.974623in}}%
\pgfpathlineto{\pgfqpoint{4.887911in}{2.973806in}}%
\pgfpathlineto{\pgfqpoint{4.902099in}{2.972896in}}%
\pgfpathlineto{\pgfqpoint{4.916288in}{2.971800in}}%
\pgfpathlineto{\pgfqpoint{4.930476in}{2.971219in}}%
\pgfpathlineto{\pgfqpoint{4.944664in}{2.970518in}}%
\pgfpathlineto{\pgfqpoint{4.958853in}{2.970130in}}%
\pgfpathlineto{\pgfqpoint{4.973041in}{2.969224in}}%
\pgfpathlineto{\pgfqpoint{4.987230in}{2.968845in}}%
\pgfpathlineto{\pgfqpoint{5.001418in}{2.968159in}}%
\pgfpathlineto{\pgfqpoint{5.015607in}{2.967338in}}%
\pgfpathlineto{\pgfqpoint{5.029795in}{2.967009in}}%
\pgfpathlineto{\pgfqpoint{5.043983in}{2.966273in}}%
\pgfpathlineto{\pgfqpoint{5.058172in}{2.965529in}}%
\pgfusepath{stroke}%
\end{pgfscope}%
\begin{pgfscope}%
\pgfpathrectangle{\pgfqpoint{3.511634in}{2.920132in}}{\pgfqpoint{1.560726in}{0.791260in}}%
\pgfusepath{clip}%
\pgfsetrectcap%
\pgfsetroundjoin%
\pgfsetlinewidth{1.003750pt}%
\definecolor{currentstroke}{rgb}{0.000000,0.000000,0.000000}%
\pgfsetstrokecolor{currentstroke}%
\pgfsetdash{}{0pt}%
\pgfpathmoveto{\pgfqpoint{3.696083in}{2.920132in}}%
\pgfpathlineto{\pgfqpoint{3.696083in}{3.711392in}}%
\pgfusepath{stroke}%
\end{pgfscope}%
\begin{pgfscope}%
\pgfsetrectcap%
\pgfsetmiterjoin%
\pgfsetlinewidth{0.803000pt}%
\definecolor{currentstroke}{rgb}{0.000000,0.000000,0.000000}%
\pgfsetstrokecolor{currentstroke}%
\pgfsetdash{}{0pt}%
\pgfpathmoveto{\pgfqpoint{3.511634in}{2.920132in}}%
\pgfpathlineto{\pgfqpoint{3.511634in}{3.711392in}}%
\pgfusepath{stroke}%
\end{pgfscope}%
\begin{pgfscope}%
\pgfsetrectcap%
\pgfsetmiterjoin%
\pgfsetlinewidth{0.803000pt}%
\definecolor{currentstroke}{rgb}{0.000000,0.000000,0.000000}%
\pgfsetstrokecolor{currentstroke}%
\pgfsetdash{}{0pt}%
\pgfpathmoveto{\pgfqpoint{5.072360in}{2.920132in}}%
\pgfpathlineto{\pgfqpoint{5.072360in}{3.711392in}}%
\pgfusepath{stroke}%
\end{pgfscope}%
\begin{pgfscope}%
\pgfsetrectcap%
\pgfsetmiterjoin%
\pgfsetlinewidth{0.803000pt}%
\definecolor{currentstroke}{rgb}{0.000000,0.000000,0.000000}%
\pgfsetstrokecolor{currentstroke}%
\pgfsetdash{}{0pt}%
\pgfpathmoveto{\pgfqpoint{3.511634in}{2.920132in}}%
\pgfpathlineto{\pgfqpoint{5.072360in}{2.920132in}}%
\pgfusepath{stroke}%
\end{pgfscope}%
\begin{pgfscope}%
\pgfsetrectcap%
\pgfsetmiterjoin%
\pgfsetlinewidth{0.803000pt}%
\definecolor{currentstroke}{rgb}{0.000000,0.000000,0.000000}%
\pgfsetstrokecolor{currentstroke}%
\pgfsetdash{}{0pt}%
\pgfpathmoveto{\pgfqpoint{3.511634in}{3.711392in}}%
\pgfpathlineto{\pgfqpoint{5.072360in}{3.711392in}}%
\pgfusepath{stroke}%
\end{pgfscope}%
\begin{pgfscope}%
\pgfsetbuttcap%
\pgfsetmiterjoin%
\definecolor{currentfill}{rgb}{1.000000,1.000000,1.000000}%
\pgfsetfillcolor{currentfill}%
\pgfsetfillopacity{0.800000}%
\pgfsetlinewidth{1.003750pt}%
\definecolor{currentstroke}{rgb}{0.800000,0.800000,0.800000}%
\pgfsetstrokecolor{currentstroke}%
\pgfsetstrokeopacity{0.800000}%
\pgfsetdash{}{0pt}%
\pgfpathmoveto{\pgfqpoint{0.654624in}{3.462821in}}%
\pgfpathlineto{\pgfqpoint{1.876657in}{3.462821in}}%
\pgfpathquadraticcurveto{\pgfqpoint{1.904435in}{3.462821in}}{\pgfqpoint{1.904435in}{3.490598in}}%
\pgfpathlineto{\pgfqpoint{1.904435in}{3.864042in}}%
\pgfpathquadraticcurveto{\pgfqpoint{1.904435in}{3.891820in}}{\pgfqpoint{1.876657in}{3.891820in}}%
\pgfpathlineto{\pgfqpoint{0.654624in}{3.891820in}}%
\pgfpathquadraticcurveto{\pgfqpoint{0.626847in}{3.891820in}}{\pgfqpoint{0.626847in}{3.864042in}}%
\pgfpathlineto{\pgfqpoint{0.626847in}{3.490598in}}%
\pgfpathquadraticcurveto{\pgfqpoint{0.626847in}{3.462821in}}{\pgfqpoint{0.654624in}{3.462821in}}%
\pgfpathlineto{\pgfqpoint{0.654624in}{3.462821in}}%
\pgfpathclose%
\pgfusepath{stroke,fill}%
\end{pgfscope}%
\begin{pgfscope}%
\pgfsetrectcap%
\pgfsetroundjoin%
\pgfsetlinewidth{1.505625pt}%
\definecolor{currentstroke}{rgb}{0.121569,0.466667,0.705882}%
\pgfsetstrokecolor{currentstroke}%
\pgfsetdash{}{0pt}%
\pgfpathmoveto{\pgfqpoint{0.682402in}{3.787653in}}%
\pgfpathlineto{\pgfqpoint{0.821291in}{3.787653in}}%
\pgfpathlineto{\pgfqpoint{0.960180in}{3.787653in}}%
\pgfusepath{stroke}%
\end{pgfscope}%
\begin{pgfscope}%
\definecolor{textcolor}{rgb}{0.000000,0.000000,0.000000}%
\pgfsetstrokecolor{textcolor}%
\pgfsetfillcolor{textcolor}%
\pgftext[x=1.071291in,y=3.739042in,left,base]{\color{textcolor}\rmfamily\fontsize{10.000000}{12.000000}\selectfont Dunkelbilder}%
\end{pgfscope}%
\begin{pgfscope}%
\pgfsetrectcap%
\pgfsetroundjoin%
\pgfsetlinewidth{1.505625pt}%
\definecolor{currentstroke}{rgb}{1.000000,0.498039,0.054902}%
\pgfsetstrokecolor{currentstroke}%
\pgfsetdash{}{0pt}%
\pgfpathmoveto{\pgfqpoint{0.682402in}{3.593987in}}%
\pgfpathlineto{\pgfqpoint{0.821291in}{3.593987in}}%
\pgfpathlineto{\pgfqpoint{0.960180in}{3.593987in}}%
\pgfusepath{stroke}%
\end{pgfscope}%
\begin{pgfscope}%
\definecolor{textcolor}{rgb}{0.000000,0.000000,0.000000}%
\pgfsetstrokecolor{textcolor}%
\pgfsetfillcolor{textcolor}%
\pgftext[x=1.071291in,y=3.545376in,left,base]{\color{textcolor}\rmfamily\fontsize{10.000000}{12.000000}\selectfont Streubilder}%
\end{pgfscope}%
\begin{pgfscope}%
\pgfsetbuttcap%
\pgfsetmiterjoin%
\definecolor{currentfill}{rgb}{1.000000,1.000000,1.000000}%
\pgfsetfillcolor{currentfill}%
\pgfsetlinewidth{0.000000pt}%
\definecolor{currentstroke}{rgb}{0.000000,0.000000,0.000000}%
\pgfsetstrokecolor{currentstroke}%
\pgfsetstrokeopacity{0.000000}%
\pgfsetdash{}{0pt}%
\pgfpathmoveto{\pgfqpoint{0.557402in}{0.398088in}}%
\pgfpathlineto{\pgfqpoint{6.131424in}{0.398088in}}%
\pgfpathlineto{\pgfqpoint{6.131424in}{1.786264in}}%
\pgfpathlineto{\pgfqpoint{0.557402in}{1.786264in}}%
\pgfpathlineto{\pgfqpoint{0.557402in}{0.398088in}}%
\pgfpathclose%
\pgfusepath{fill}%
\end{pgfscope}%
\begin{pgfscope}%
\pgfsetbuttcap%
\pgfsetroundjoin%
\definecolor{currentfill}{rgb}{0.000000,0.000000,0.000000}%
\pgfsetfillcolor{currentfill}%
\pgfsetlinewidth{0.803000pt}%
\definecolor{currentstroke}{rgb}{0.000000,0.000000,0.000000}%
\pgfsetstrokecolor{currentstroke}%
\pgfsetdash{}{0pt}%
\pgfsys@defobject{currentmarker}{\pgfqpoint{0.000000in}{-0.048611in}}{\pgfqpoint{0.000000in}{0.000000in}}{%
\pgfpathmoveto{\pgfqpoint{0.000000in}{0.000000in}}%
\pgfpathlineto{\pgfqpoint{0.000000in}{-0.048611in}}%
\pgfusepath{stroke,fill}%
}%
\begin{pgfscope}%
\pgfsys@transformshift{0.900419in}{0.398088in}%
\pgfsys@useobject{currentmarker}{}%
\end{pgfscope}%
\end{pgfscope}%
\begin{pgfscope}%
\definecolor{textcolor}{rgb}{0.000000,0.000000,0.000000}%
\pgfsetstrokecolor{textcolor}%
\pgfsetfillcolor{textcolor}%
\pgftext[x=0.900419in,y=0.300866in,,top]{\color{textcolor}\rmfamily\fontsize{10.000000}{12.000000}\selectfont \(\displaystyle {0}\)}%
\end{pgfscope}%
\begin{pgfscope}%
\pgfsetbuttcap%
\pgfsetroundjoin%
\definecolor{currentfill}{rgb}{0.000000,0.000000,0.000000}%
\pgfsetfillcolor{currentfill}%
\pgfsetlinewidth{0.803000pt}%
\definecolor{currentstroke}{rgb}{0.000000,0.000000,0.000000}%
\pgfsetstrokecolor{currentstroke}%
\pgfsetdash{}{0pt}%
\pgfsys@defobject{currentmarker}{\pgfqpoint{0.000000in}{-0.048611in}}{\pgfqpoint{0.000000in}{0.000000in}}{%
\pgfpathmoveto{\pgfqpoint{0.000000in}{0.000000in}}%
\pgfpathlineto{\pgfqpoint{0.000000in}{-0.048611in}}%
\pgfusepath{stroke,fill}%
}%
\begin{pgfscope}%
\pgfsys@transformshift{1.757961in}{0.398088in}%
\pgfsys@useobject{currentmarker}{}%
\end{pgfscope}%
\end{pgfscope}%
\begin{pgfscope}%
\definecolor{textcolor}{rgb}{0.000000,0.000000,0.000000}%
\pgfsetstrokecolor{textcolor}%
\pgfsetfillcolor{textcolor}%
\pgftext[x=1.757961in,y=0.300866in,,top]{\color{textcolor}\rmfamily\fontsize{10.000000}{12.000000}\selectfont \(\displaystyle {50}\)}%
\end{pgfscope}%
\begin{pgfscope}%
\pgfsetbuttcap%
\pgfsetroundjoin%
\definecolor{currentfill}{rgb}{0.000000,0.000000,0.000000}%
\pgfsetfillcolor{currentfill}%
\pgfsetlinewidth{0.803000pt}%
\definecolor{currentstroke}{rgb}{0.000000,0.000000,0.000000}%
\pgfsetstrokecolor{currentstroke}%
\pgfsetdash{}{0pt}%
\pgfsys@defobject{currentmarker}{\pgfqpoint{0.000000in}{-0.048611in}}{\pgfqpoint{0.000000in}{0.000000in}}{%
\pgfpathmoveto{\pgfqpoint{0.000000in}{0.000000in}}%
\pgfpathlineto{\pgfqpoint{0.000000in}{-0.048611in}}%
\pgfusepath{stroke,fill}%
}%
\begin{pgfscope}%
\pgfsys@transformshift{2.615503in}{0.398088in}%
\pgfsys@useobject{currentmarker}{}%
\end{pgfscope}%
\end{pgfscope}%
\begin{pgfscope}%
\definecolor{textcolor}{rgb}{0.000000,0.000000,0.000000}%
\pgfsetstrokecolor{textcolor}%
\pgfsetfillcolor{textcolor}%
\pgftext[x=2.615503in,y=0.300866in,,top]{\color{textcolor}\rmfamily\fontsize{10.000000}{12.000000}\selectfont \(\displaystyle {100}\)}%
\end{pgfscope}%
\begin{pgfscope}%
\pgfsetbuttcap%
\pgfsetroundjoin%
\definecolor{currentfill}{rgb}{0.000000,0.000000,0.000000}%
\pgfsetfillcolor{currentfill}%
\pgfsetlinewidth{0.803000pt}%
\definecolor{currentstroke}{rgb}{0.000000,0.000000,0.000000}%
\pgfsetstrokecolor{currentstroke}%
\pgfsetdash{}{0pt}%
\pgfsys@defobject{currentmarker}{\pgfqpoint{0.000000in}{-0.048611in}}{\pgfqpoint{0.000000in}{0.000000in}}{%
\pgfpathmoveto{\pgfqpoint{0.000000in}{0.000000in}}%
\pgfpathlineto{\pgfqpoint{0.000000in}{-0.048611in}}%
\pgfusepath{stroke,fill}%
}%
\begin{pgfscope}%
\pgfsys@transformshift{3.473045in}{0.398088in}%
\pgfsys@useobject{currentmarker}{}%
\end{pgfscope}%
\end{pgfscope}%
\begin{pgfscope}%
\definecolor{textcolor}{rgb}{0.000000,0.000000,0.000000}%
\pgfsetstrokecolor{textcolor}%
\pgfsetfillcolor{textcolor}%
\pgftext[x=3.473045in,y=0.300866in,,top]{\color{textcolor}\rmfamily\fontsize{10.000000}{12.000000}\selectfont \(\displaystyle {150}\)}%
\end{pgfscope}%
\begin{pgfscope}%
\pgfsetbuttcap%
\pgfsetroundjoin%
\definecolor{currentfill}{rgb}{0.000000,0.000000,0.000000}%
\pgfsetfillcolor{currentfill}%
\pgfsetlinewidth{0.803000pt}%
\definecolor{currentstroke}{rgb}{0.000000,0.000000,0.000000}%
\pgfsetstrokecolor{currentstroke}%
\pgfsetdash{}{0pt}%
\pgfsys@defobject{currentmarker}{\pgfqpoint{0.000000in}{-0.048611in}}{\pgfqpoint{0.000000in}{0.000000in}}{%
\pgfpathmoveto{\pgfqpoint{0.000000in}{0.000000in}}%
\pgfpathlineto{\pgfqpoint{0.000000in}{-0.048611in}}%
\pgfusepath{stroke,fill}%
}%
\begin{pgfscope}%
\pgfsys@transformshift{4.330586in}{0.398088in}%
\pgfsys@useobject{currentmarker}{}%
\end{pgfscope}%
\end{pgfscope}%
\begin{pgfscope}%
\definecolor{textcolor}{rgb}{0.000000,0.000000,0.000000}%
\pgfsetstrokecolor{textcolor}%
\pgfsetfillcolor{textcolor}%
\pgftext[x=4.330586in,y=0.300866in,,top]{\color{textcolor}\rmfamily\fontsize{10.000000}{12.000000}\selectfont \(\displaystyle {200}\)}%
\end{pgfscope}%
\begin{pgfscope}%
\pgfsetbuttcap%
\pgfsetroundjoin%
\definecolor{currentfill}{rgb}{0.000000,0.000000,0.000000}%
\pgfsetfillcolor{currentfill}%
\pgfsetlinewidth{0.803000pt}%
\definecolor{currentstroke}{rgb}{0.000000,0.000000,0.000000}%
\pgfsetstrokecolor{currentstroke}%
\pgfsetdash{}{0pt}%
\pgfsys@defobject{currentmarker}{\pgfqpoint{0.000000in}{-0.048611in}}{\pgfqpoint{0.000000in}{0.000000in}}{%
\pgfpathmoveto{\pgfqpoint{0.000000in}{0.000000in}}%
\pgfpathlineto{\pgfqpoint{0.000000in}{-0.048611in}}%
\pgfusepath{stroke,fill}%
}%
\begin{pgfscope}%
\pgfsys@transformshift{5.188128in}{0.398088in}%
\pgfsys@useobject{currentmarker}{}%
\end{pgfscope}%
\end{pgfscope}%
\begin{pgfscope}%
\definecolor{textcolor}{rgb}{0.000000,0.000000,0.000000}%
\pgfsetstrokecolor{textcolor}%
\pgfsetfillcolor{textcolor}%
\pgftext[x=5.188128in,y=0.300866in,,top]{\color{textcolor}\rmfamily\fontsize{10.000000}{12.000000}\selectfont \(\displaystyle {250}\)}%
\end{pgfscope}%
\begin{pgfscope}%
\pgfsetbuttcap%
\pgfsetroundjoin%
\definecolor{currentfill}{rgb}{0.000000,0.000000,0.000000}%
\pgfsetfillcolor{currentfill}%
\pgfsetlinewidth{0.803000pt}%
\definecolor{currentstroke}{rgb}{0.000000,0.000000,0.000000}%
\pgfsetstrokecolor{currentstroke}%
\pgfsetdash{}{0pt}%
\pgfsys@defobject{currentmarker}{\pgfqpoint{0.000000in}{-0.048611in}}{\pgfqpoint{0.000000in}{0.000000in}}{%
\pgfpathmoveto{\pgfqpoint{0.000000in}{0.000000in}}%
\pgfpathlineto{\pgfqpoint{0.000000in}{-0.048611in}}%
\pgfusepath{stroke,fill}%
}%
\begin{pgfscope}%
\pgfsys@transformshift{6.045670in}{0.398088in}%
\pgfsys@useobject{currentmarker}{}%
\end{pgfscope}%
\end{pgfscope}%
\begin{pgfscope}%
\definecolor{textcolor}{rgb}{0.000000,0.000000,0.000000}%
\pgfsetstrokecolor{textcolor}%
\pgfsetfillcolor{textcolor}%
\pgftext[x=6.045670in,y=0.300866in,,top]{\color{textcolor}\rmfamily\fontsize{10.000000}{12.000000}\selectfont \(\displaystyle {300}\)}%
\end{pgfscope}%
\begin{pgfscope}%
\definecolor{textcolor}{rgb}{0.000000,0.000000,0.000000}%
\pgfsetstrokecolor{textcolor}%
\pgfsetfillcolor{textcolor}%
\pgftext[x=3.344413in,y=0.122655in,,top]{\color{textcolor}\rmfamily\fontsize{10.000000}{12.000000}\selectfont Clusterwert \(\displaystyle W\) in ADU}%
\end{pgfscope}%
\begin{pgfscope}%
\pgfsetbuttcap%
\pgfsetroundjoin%
\definecolor{currentfill}{rgb}{0.000000,0.000000,0.000000}%
\pgfsetfillcolor{currentfill}%
\pgfsetlinewidth{0.803000pt}%
\definecolor{currentstroke}{rgb}{0.000000,0.000000,0.000000}%
\pgfsetstrokecolor{currentstroke}%
\pgfsetdash{}{0pt}%
\pgfsys@defobject{currentmarker}{\pgfqpoint{-0.048611in}{0.000000in}}{\pgfqpoint{-0.000000in}{0.000000in}}{%
\pgfpathmoveto{\pgfqpoint{-0.000000in}{0.000000in}}%
\pgfpathlineto{\pgfqpoint{-0.048611in}{0.000000in}}%
\pgfusepath{stroke,fill}%
}%
\begin{pgfscope}%
\pgfsys@transformshift{0.557402in}{0.461188in}%
\pgfsys@useobject{currentmarker}{}%
\end{pgfscope}%
\end{pgfscope}%
\begin{pgfscope}%
\definecolor{textcolor}{rgb}{0.000000,0.000000,0.000000}%
\pgfsetstrokecolor{textcolor}%
\pgfsetfillcolor{textcolor}%
\pgftext[x=0.282710in, y=0.413363in, left, base]{\color{textcolor}\rmfamily\fontsize{10.000000}{12.000000}\selectfont \num{0.0}}%
\end{pgfscope}%
\begin{pgfscope}%
\pgfsetbuttcap%
\pgfsetroundjoin%
\definecolor{currentfill}{rgb}{0.000000,0.000000,0.000000}%
\pgfsetfillcolor{currentfill}%
\pgfsetlinewidth{0.803000pt}%
\definecolor{currentstroke}{rgb}{0.000000,0.000000,0.000000}%
\pgfsetstrokecolor{currentstroke}%
\pgfsetdash{}{0pt}%
\pgfsys@defobject{currentmarker}{\pgfqpoint{-0.048611in}{0.000000in}}{\pgfqpoint{-0.000000in}{0.000000in}}{%
\pgfpathmoveto{\pgfqpoint{-0.000000in}{0.000000in}}%
\pgfpathlineto{\pgfqpoint{-0.048611in}{0.000000in}}%
\pgfusepath{stroke,fill}%
}%
\begin{pgfscope}%
\pgfsys@transformshift{0.557402in}{0.882809in}%
\pgfsys@useobject{currentmarker}{}%
\end{pgfscope}%
\end{pgfscope}%
\begin{pgfscope}%
\definecolor{textcolor}{rgb}{0.000000,0.000000,0.000000}%
\pgfsetstrokecolor{textcolor}%
\pgfsetfillcolor{textcolor}%
\pgftext[x=0.282710in, y=0.834984in, left, base]{\color{textcolor}\rmfamily\fontsize{10.000000}{12.000000}\selectfont \num{0.5}}%
\end{pgfscope}%
\begin{pgfscope}%
\pgfsetbuttcap%
\pgfsetroundjoin%
\definecolor{currentfill}{rgb}{0.000000,0.000000,0.000000}%
\pgfsetfillcolor{currentfill}%
\pgfsetlinewidth{0.803000pt}%
\definecolor{currentstroke}{rgb}{0.000000,0.000000,0.000000}%
\pgfsetstrokecolor{currentstroke}%
\pgfsetdash{}{0pt}%
\pgfsys@defobject{currentmarker}{\pgfqpoint{-0.048611in}{0.000000in}}{\pgfqpoint{-0.000000in}{0.000000in}}{%
\pgfpathmoveto{\pgfqpoint{-0.000000in}{0.000000in}}%
\pgfpathlineto{\pgfqpoint{-0.048611in}{0.000000in}}%
\pgfusepath{stroke,fill}%
}%
\begin{pgfscope}%
\pgfsys@transformshift{0.557402in}{1.304429in}%
\pgfsys@useobject{currentmarker}{}%
\end{pgfscope}%
\end{pgfscope}%
\begin{pgfscope}%
\definecolor{textcolor}{rgb}{0.000000,0.000000,0.000000}%
\pgfsetstrokecolor{textcolor}%
\pgfsetfillcolor{textcolor}%
\pgftext[x=0.282710in, y=1.256605in, left, base]{\color{textcolor}\rmfamily\fontsize{10.000000}{12.000000}\selectfont \num{1.0}}%
\end{pgfscope}%
\begin{pgfscope}%
\pgfsetbuttcap%
\pgfsetroundjoin%
\definecolor{currentfill}{rgb}{0.000000,0.000000,0.000000}%
\pgfsetfillcolor{currentfill}%
\pgfsetlinewidth{0.803000pt}%
\definecolor{currentstroke}{rgb}{0.000000,0.000000,0.000000}%
\pgfsetstrokecolor{currentstroke}%
\pgfsetdash{}{0pt}%
\pgfsys@defobject{currentmarker}{\pgfqpoint{-0.048611in}{0.000000in}}{\pgfqpoint{-0.000000in}{0.000000in}}{%
\pgfpathmoveto{\pgfqpoint{-0.000000in}{0.000000in}}%
\pgfpathlineto{\pgfqpoint{-0.048611in}{0.000000in}}%
\pgfusepath{stroke,fill}%
}%
\begin{pgfscope}%
\pgfsys@transformshift{0.557402in}{1.726050in}%
\pgfsys@useobject{currentmarker}{}%
\end{pgfscope}%
\end{pgfscope}%
\begin{pgfscope}%
\definecolor{textcolor}{rgb}{0.000000,0.000000,0.000000}%
\pgfsetstrokecolor{textcolor}%
\pgfsetfillcolor{textcolor}%
\pgftext[x=0.282710in, y=1.678226in, left, base]{\color{textcolor}\rmfamily\fontsize{10.000000}{12.000000}\selectfont \num{1.5}}%
\end{pgfscope}%
\begin{pgfscope}%
\definecolor{textcolor}{rgb}{0.000000,0.000000,0.000000}%
\pgfsetstrokecolor{textcolor}%
\pgfsetfillcolor{textcolor}%
\pgftext[x=0.227155in,y=1.092176in,,bottom,rotate=90.000000]{\color{textcolor}\rmfamily\fontsize{10.000000}{12.000000}\selectfont Clusterzahl}%
\end{pgfscope}%
\begin{pgfscope}%
\definecolor{textcolor}{rgb}{0.000000,0.000000,0.000000}%
\pgfsetstrokecolor{textcolor}%
\pgfsetfillcolor{textcolor}%
\pgftext[x=0.557402in,y=1.827931in,left,base]{\color{textcolor}\rmfamily\fontsize{10.000000}{12.000000}\selectfont \(\displaystyle \times{10^{9}}{}\)}%
\end{pgfscope}%
\begin{pgfscope}%
\pgfpathrectangle{\pgfqpoint{0.557402in}{0.398088in}}{\pgfqpoint{5.574022in}{1.388176in}}%
\pgfusepath{clip}%
\pgfsetrectcap%
\pgfsetroundjoin%
\pgfsetlinewidth{1.505625pt}%
\definecolor{currentstroke}{rgb}{0.121569,0.466667,0.705882}%
\pgfsetstrokecolor{currentstroke}%
\pgfsetdash{}{0pt}%
\pgfpathmoveto{\pgfqpoint{0.555402in}{0.461219in}}%
\pgfpathlineto{\pgfqpoint{0.883268in}{0.461329in}}%
\pgfpathlineto{\pgfqpoint{0.900419in}{1.723165in}}%
\pgfpathlineto{\pgfqpoint{0.917570in}{0.461360in}}%
\pgfpathlineto{\pgfqpoint{2.203883in}{0.462118in}}%
\pgfpathlineto{\pgfqpoint{3.833212in}{0.461202in}}%
\pgfpathlineto{\pgfqpoint{6.133424in}{0.461188in}}%
\pgfpathlineto{\pgfqpoint{6.133424in}{0.461188in}}%
\pgfusepath{stroke}%
\end{pgfscope}%
\begin{pgfscope}%
\pgfpathrectangle{\pgfqpoint{0.557402in}{0.398088in}}{\pgfqpoint{5.574022in}{1.388176in}}%
\pgfusepath{clip}%
\pgfsetrectcap%
\pgfsetroundjoin%
\pgfsetlinewidth{1.505625pt}%
\definecolor{currentstroke}{rgb}{1.000000,0.498039,0.054902}%
\pgfsetstrokecolor{currentstroke}%
\pgfsetdash{}{0pt}%
\pgfpathmoveto{\pgfqpoint{0.555402in}{0.461579in}}%
\pgfpathlineto{\pgfqpoint{0.883268in}{0.461631in}}%
\pgfpathlineto{\pgfqpoint{0.900419in}{1.721983in}}%
\pgfpathlineto{\pgfqpoint{0.917570in}{0.461638in}}%
\pgfpathlineto{\pgfqpoint{2.221033in}{0.461521in}}%
\pgfpathlineto{\pgfqpoint{3.850363in}{0.461201in}}%
\pgfpathlineto{\pgfqpoint{6.133424in}{0.461189in}}%
\pgfpathlineto{\pgfqpoint{6.133424in}{0.461189in}}%
\pgfusepath{stroke}%
\end{pgfscope}%
\begin{pgfscope}%
\pgfsetrectcap%
\pgfsetmiterjoin%
\pgfsetlinewidth{0.803000pt}%
\definecolor{currentstroke}{rgb}{0.000000,0.000000,0.000000}%
\pgfsetstrokecolor{currentstroke}%
\pgfsetdash{}{0pt}%
\pgfpathmoveto{\pgfqpoint{0.557402in}{0.398088in}}%
\pgfpathlineto{\pgfqpoint{0.557402in}{1.786264in}}%
\pgfusepath{stroke}%
\end{pgfscope}%
\begin{pgfscope}%
\pgfsetrectcap%
\pgfsetmiterjoin%
\pgfsetlinewidth{0.803000pt}%
\definecolor{currentstroke}{rgb}{0.000000,0.000000,0.000000}%
\pgfsetstrokecolor{currentstroke}%
\pgfsetdash{}{0pt}%
\pgfpathmoveto{\pgfqpoint{6.131424in}{0.398088in}}%
\pgfpathlineto{\pgfqpoint{6.131424in}{1.786264in}}%
\pgfusepath{stroke}%
\end{pgfscope}%
\begin{pgfscope}%
\pgfsetrectcap%
\pgfsetmiterjoin%
\pgfsetlinewidth{0.803000pt}%
\definecolor{currentstroke}{rgb}{0.000000,0.000000,0.000000}%
\pgfsetstrokecolor{currentstroke}%
\pgfsetdash{}{0pt}%
\pgfpathmoveto{\pgfqpoint{0.557402in}{0.398088in}}%
\pgfpathlineto{\pgfqpoint{6.131424in}{0.398088in}}%
\pgfusepath{stroke}%
\end{pgfscope}%
\begin{pgfscope}%
\pgfsetrectcap%
\pgfsetmiterjoin%
\pgfsetlinewidth{0.803000pt}%
\definecolor{currentstroke}{rgb}{0.000000,0.000000,0.000000}%
\pgfsetstrokecolor{currentstroke}%
\pgfsetdash{}{0pt}%
\pgfpathmoveto{\pgfqpoint{0.557402in}{1.786264in}}%
\pgfpathlineto{\pgfqpoint{6.131424in}{1.786264in}}%
\pgfusepath{stroke}%
\end{pgfscope}%
\begin{pgfscope}%
\definecolor{textcolor}{rgb}{0.000000,0.000000,0.000000}%
\pgfsetstrokecolor{textcolor}%
\pgfsetfillcolor{textcolor}%
\pgftext[x=0.000000in,y=1.925082in,left,base]{\color{textcolor}\rmfamily\fontsize{10.000000}{12.000000}\selectfont (b)}%
\end{pgfscope}%
\begin{pgfscope}%
\pgfpathrectangle{\pgfqpoint{0.557402in}{0.398088in}}{\pgfqpoint{5.574022in}{1.388176in}}%
\pgfusepath{clip}%
\pgfsetbuttcap%
\pgfsetmiterjoin%
\pgfsetlinewidth{1.003750pt}%
\definecolor{currentstroke}{rgb}{0.000000,0.000000,0.000000}%
\pgfsetstrokecolor{currentstroke}%
\pgfsetstrokeopacity{0.500000}%
\pgfsetdash{}{0pt}%
\pgfpathmoveto{\pgfqpoint{3.610251in}{0.461187in}}%
\pgfpathlineto{\pgfqpoint{6.131424in}{0.461187in}}%
\pgfpathlineto{\pgfqpoint{6.131424in}{0.461205in}}%
\pgfpathlineto{\pgfqpoint{3.610251in}{0.461205in}}%
\pgfpathlineto{\pgfqpoint{3.610251in}{0.461187in}}%
\pgfpathclose%
\pgfusepath{stroke}%
\end{pgfscope}%
\begin{pgfscope}%
\pgfsetroundcap%
\pgfsetroundjoin%
\pgfsetlinewidth{1.003750pt}%
\definecolor{currentstroke}{rgb}{0.000000,0.000000,0.000000}%
\pgfsetstrokecolor{currentstroke}%
\pgfsetstrokeopacity{0.500000}%
\pgfsetdash{}{0pt}%
\pgfpathmoveto{\pgfqpoint{1.839427in}{0.745132in}}%
\pgfpathquadraticcurveto{\pgfqpoint{2.724839in}{0.603160in}}{\pgfqpoint{3.610251in}{0.461187in}}%
\pgfusepath{stroke}%
\end{pgfscope}%
\begin{pgfscope}%
\pgfsetroundcap%
\pgfsetroundjoin%
\pgfsetlinewidth{1.003750pt}%
\definecolor{currentstroke}{rgb}{0.000000,0.000000,0.000000}%
\pgfsetstrokecolor{currentstroke}%
\pgfsetstrokeopacity{0.500000}%
\pgfsetdash{}{0pt}%
\pgfpathmoveto{\pgfqpoint{3.790335in}{1.536392in}}%
\pgfpathquadraticcurveto{\pgfqpoint{4.960880in}{0.998799in}}{\pgfqpoint{6.131424in}{0.461205in}}%
\pgfusepath{stroke}%
\end{pgfscope}%
\begin{pgfscope}%
\pgfsetbuttcap%
\pgfsetmiterjoin%
\definecolor{currentfill}{rgb}{1.000000,1.000000,1.000000}%
\pgfsetfillcolor{currentfill}%
\pgfsetlinewidth{0.000000pt}%
\definecolor{currentstroke}{rgb}{0.000000,0.000000,0.000000}%
\pgfsetstrokecolor{currentstroke}%
\pgfsetstrokeopacity{0.000000}%
\pgfsetdash{}{0pt}%
\pgfpathmoveto{\pgfqpoint{1.839427in}{0.745132in}}%
\pgfpathlineto{\pgfqpoint{3.790335in}{0.745132in}}%
\pgfpathlineto{\pgfqpoint{3.790335in}{1.536392in}}%
\pgfpathlineto{\pgfqpoint{1.839427in}{1.536392in}}%
\pgfpathlineto{\pgfqpoint{1.839427in}{0.745132in}}%
\pgfpathclose%
\pgfusepath{fill}%
\end{pgfscope}%
\begin{pgfscope}%
\pgfsetbuttcap%
\pgfsetroundjoin%
\definecolor{currentfill}{rgb}{0.000000,0.000000,0.000000}%
\pgfsetfillcolor{currentfill}%
\pgfsetlinewidth{0.803000pt}%
\definecolor{currentstroke}{rgb}{0.000000,0.000000,0.000000}%
\pgfsetstrokecolor{currentstroke}%
\pgfsetdash{}{0pt}%
\pgfsys@defobject{currentmarker}{\pgfqpoint{0.000000in}{0.000000in}}{\pgfqpoint{0.000000in}{0.048611in}}{%
\pgfpathmoveto{\pgfqpoint{0.000000in}{0.000000in}}%
\pgfpathlineto{\pgfqpoint{0.000000in}{0.048611in}}%
\pgfusepath{stroke,fill}%
}%
\begin{pgfscope}%
\pgfsys@transformshift{2.131400in}{1.536392in}%
\pgfsys@useobject{currentmarker}{}%
\end{pgfscope}%
\end{pgfscope}%
\begin{pgfscope}%
\definecolor{textcolor}{rgb}{0.000000,0.000000,0.000000}%
\pgfsetstrokecolor{textcolor}%
\pgfsetfillcolor{textcolor}%
\pgftext[x=2.131400in,y=1.633615in,,bottom]{\color{textcolor}\rmfamily\fontsize{10.000000}{12.000000}\selectfont \(\displaystyle {180}\)}%
\end{pgfscope}%
\begin{pgfscope}%
\pgfsetbuttcap%
\pgfsetroundjoin%
\definecolor{currentfill}{rgb}{0.000000,0.000000,0.000000}%
\pgfsetfillcolor{currentfill}%
\pgfsetlinewidth{0.803000pt}%
\definecolor{currentstroke}{rgb}{0.000000,0.000000,0.000000}%
\pgfsetstrokecolor{currentstroke}%
\pgfsetdash{}{0pt}%
\pgfsys@defobject{currentmarker}{\pgfqpoint{0.000000in}{0.000000in}}{\pgfqpoint{0.000000in}{0.048611in}}{%
\pgfpathmoveto{\pgfqpoint{0.000000in}{0.000000in}}%
\pgfpathlineto{\pgfqpoint{0.000000in}{0.048611in}}%
\pgfusepath{stroke,fill}%
}%
\begin{pgfscope}%
\pgfsys@transformshift{2.529544in}{1.536392in}%
\pgfsys@useobject{currentmarker}{}%
\end{pgfscope}%
\end{pgfscope}%
\begin{pgfscope}%
\definecolor{textcolor}{rgb}{0.000000,0.000000,0.000000}%
\pgfsetstrokecolor{textcolor}%
\pgfsetfillcolor{textcolor}%
\pgftext[x=2.529544in,y=1.633615in,,bottom]{\color{textcolor}\rmfamily\fontsize{10.000000}{12.000000}\selectfont \(\displaystyle {210}\)}%
\end{pgfscope}%
\begin{pgfscope}%
\pgfsetbuttcap%
\pgfsetroundjoin%
\definecolor{currentfill}{rgb}{0.000000,0.000000,0.000000}%
\pgfsetfillcolor{currentfill}%
\pgfsetlinewidth{0.803000pt}%
\definecolor{currentstroke}{rgb}{0.000000,0.000000,0.000000}%
\pgfsetstrokecolor{currentstroke}%
\pgfsetdash{}{0pt}%
\pgfsys@defobject{currentmarker}{\pgfqpoint{0.000000in}{0.000000in}}{\pgfqpoint{0.000000in}{0.048611in}}{%
\pgfpathmoveto{\pgfqpoint{0.000000in}{0.000000in}}%
\pgfpathlineto{\pgfqpoint{0.000000in}{0.048611in}}%
\pgfusepath{stroke,fill}%
}%
\begin{pgfscope}%
\pgfsys@transformshift{2.927689in}{1.536392in}%
\pgfsys@useobject{currentmarker}{}%
\end{pgfscope}%
\end{pgfscope}%
\begin{pgfscope}%
\definecolor{textcolor}{rgb}{0.000000,0.000000,0.000000}%
\pgfsetstrokecolor{textcolor}%
\pgfsetfillcolor{textcolor}%
\pgftext[x=2.927689in,y=1.633615in,,bottom]{\color{textcolor}\rmfamily\fontsize{10.000000}{12.000000}\selectfont \(\displaystyle {240}\)}%
\end{pgfscope}%
\begin{pgfscope}%
\pgfsetbuttcap%
\pgfsetroundjoin%
\definecolor{currentfill}{rgb}{0.000000,0.000000,0.000000}%
\pgfsetfillcolor{currentfill}%
\pgfsetlinewidth{0.803000pt}%
\definecolor{currentstroke}{rgb}{0.000000,0.000000,0.000000}%
\pgfsetstrokecolor{currentstroke}%
\pgfsetdash{}{0pt}%
\pgfsys@defobject{currentmarker}{\pgfqpoint{0.000000in}{0.000000in}}{\pgfqpoint{0.000000in}{0.048611in}}{%
\pgfpathmoveto{\pgfqpoint{0.000000in}{0.000000in}}%
\pgfpathlineto{\pgfqpoint{0.000000in}{0.048611in}}%
\pgfusepath{stroke,fill}%
}%
\begin{pgfscope}%
\pgfsys@transformshift{3.325833in}{1.536392in}%
\pgfsys@useobject{currentmarker}{}%
\end{pgfscope}%
\end{pgfscope}%
\begin{pgfscope}%
\definecolor{textcolor}{rgb}{0.000000,0.000000,0.000000}%
\pgfsetstrokecolor{textcolor}%
\pgfsetfillcolor{textcolor}%
\pgftext[x=3.325833in,y=1.633615in,,bottom]{\color{textcolor}\rmfamily\fontsize{10.000000}{12.000000}\selectfont \(\displaystyle {270}\)}%
\end{pgfscope}%
\begin{pgfscope}%
\pgfsetbuttcap%
\pgfsetroundjoin%
\definecolor{currentfill}{rgb}{0.000000,0.000000,0.000000}%
\pgfsetfillcolor{currentfill}%
\pgfsetlinewidth{0.803000pt}%
\definecolor{currentstroke}{rgb}{0.000000,0.000000,0.000000}%
\pgfsetstrokecolor{currentstroke}%
\pgfsetdash{}{0pt}%
\pgfsys@defobject{currentmarker}{\pgfqpoint{0.000000in}{0.000000in}}{\pgfqpoint{0.000000in}{0.048611in}}{%
\pgfpathmoveto{\pgfqpoint{0.000000in}{0.000000in}}%
\pgfpathlineto{\pgfqpoint{0.000000in}{0.048611in}}%
\pgfusepath{stroke,fill}%
}%
\begin{pgfscope}%
\pgfsys@transformshift{3.723978in}{1.536392in}%
\pgfsys@useobject{currentmarker}{}%
\end{pgfscope}%
\end{pgfscope}%
\begin{pgfscope}%
\definecolor{textcolor}{rgb}{0.000000,0.000000,0.000000}%
\pgfsetstrokecolor{textcolor}%
\pgfsetfillcolor{textcolor}%
\pgftext[x=3.723978in,y=1.633615in,,bottom]{\color{textcolor}\rmfamily\fontsize{10.000000}{12.000000}\selectfont \(\displaystyle {300}\)}%
\end{pgfscope}%
\begin{pgfscope}%
\pgfsetbuttcap%
\pgfsetroundjoin%
\definecolor{currentfill}{rgb}{0.000000,0.000000,0.000000}%
\pgfsetfillcolor{currentfill}%
\pgfsetlinewidth{0.602250pt}%
\definecolor{currentstroke}{rgb}{0.000000,0.000000,0.000000}%
\pgfsetstrokecolor{currentstroke}%
\pgfsetdash{}{0pt}%
\pgfsys@defobject{currentmarker}{\pgfqpoint{0.000000in}{0.000000in}}{\pgfqpoint{0.000000in}{0.027778in}}{%
\pgfpathmoveto{\pgfqpoint{0.000000in}{0.000000in}}%
\pgfpathlineto{\pgfqpoint{0.000000in}{0.027778in}}%
\pgfusepath{stroke,fill}%
}%
\begin{pgfscope}%
\pgfsys@transformshift{1.932328in}{1.536392in}%
\pgfsys@useobject{currentmarker}{}%
\end{pgfscope}%
\end{pgfscope}%
\begin{pgfscope}%
\pgfsetbuttcap%
\pgfsetroundjoin%
\definecolor{currentfill}{rgb}{0.000000,0.000000,0.000000}%
\pgfsetfillcolor{currentfill}%
\pgfsetlinewidth{0.602250pt}%
\definecolor{currentstroke}{rgb}{0.000000,0.000000,0.000000}%
\pgfsetstrokecolor{currentstroke}%
\pgfsetdash{}{0pt}%
\pgfsys@defobject{currentmarker}{\pgfqpoint{0.000000in}{0.000000in}}{\pgfqpoint{0.000000in}{0.027778in}}{%
\pgfpathmoveto{\pgfqpoint{0.000000in}{0.000000in}}%
\pgfpathlineto{\pgfqpoint{0.000000in}{0.027778in}}%
\pgfusepath{stroke,fill}%
}%
\begin{pgfscope}%
\pgfsys@transformshift{2.031864in}{1.536392in}%
\pgfsys@useobject{currentmarker}{}%
\end{pgfscope}%
\end{pgfscope}%
\begin{pgfscope}%
\pgfsetbuttcap%
\pgfsetroundjoin%
\definecolor{currentfill}{rgb}{0.000000,0.000000,0.000000}%
\pgfsetfillcolor{currentfill}%
\pgfsetlinewidth{0.602250pt}%
\definecolor{currentstroke}{rgb}{0.000000,0.000000,0.000000}%
\pgfsetstrokecolor{currentstroke}%
\pgfsetdash{}{0pt}%
\pgfsys@defobject{currentmarker}{\pgfqpoint{0.000000in}{0.000000in}}{\pgfqpoint{0.000000in}{0.027778in}}{%
\pgfpathmoveto{\pgfqpoint{0.000000in}{0.000000in}}%
\pgfpathlineto{\pgfqpoint{0.000000in}{0.027778in}}%
\pgfusepath{stroke,fill}%
}%
\begin{pgfscope}%
\pgfsys@transformshift{2.230936in}{1.536392in}%
\pgfsys@useobject{currentmarker}{}%
\end{pgfscope}%
\end{pgfscope}%
\begin{pgfscope}%
\pgfsetbuttcap%
\pgfsetroundjoin%
\definecolor{currentfill}{rgb}{0.000000,0.000000,0.000000}%
\pgfsetfillcolor{currentfill}%
\pgfsetlinewidth{0.602250pt}%
\definecolor{currentstroke}{rgb}{0.000000,0.000000,0.000000}%
\pgfsetstrokecolor{currentstroke}%
\pgfsetdash{}{0pt}%
\pgfsys@defobject{currentmarker}{\pgfqpoint{0.000000in}{0.000000in}}{\pgfqpoint{0.000000in}{0.027778in}}{%
\pgfpathmoveto{\pgfqpoint{0.000000in}{0.000000in}}%
\pgfpathlineto{\pgfqpoint{0.000000in}{0.027778in}}%
\pgfusepath{stroke,fill}%
}%
\begin{pgfscope}%
\pgfsys@transformshift{2.330472in}{1.536392in}%
\pgfsys@useobject{currentmarker}{}%
\end{pgfscope}%
\end{pgfscope}%
\begin{pgfscope}%
\pgfsetbuttcap%
\pgfsetroundjoin%
\definecolor{currentfill}{rgb}{0.000000,0.000000,0.000000}%
\pgfsetfillcolor{currentfill}%
\pgfsetlinewidth{0.602250pt}%
\definecolor{currentstroke}{rgb}{0.000000,0.000000,0.000000}%
\pgfsetstrokecolor{currentstroke}%
\pgfsetdash{}{0pt}%
\pgfsys@defobject{currentmarker}{\pgfqpoint{0.000000in}{0.000000in}}{\pgfqpoint{0.000000in}{0.027778in}}{%
\pgfpathmoveto{\pgfqpoint{0.000000in}{0.000000in}}%
\pgfpathlineto{\pgfqpoint{0.000000in}{0.027778in}}%
\pgfusepath{stroke,fill}%
}%
\begin{pgfscope}%
\pgfsys@transformshift{2.430008in}{1.536392in}%
\pgfsys@useobject{currentmarker}{}%
\end{pgfscope}%
\end{pgfscope}%
\begin{pgfscope}%
\pgfsetbuttcap%
\pgfsetroundjoin%
\definecolor{currentfill}{rgb}{0.000000,0.000000,0.000000}%
\pgfsetfillcolor{currentfill}%
\pgfsetlinewidth{0.602250pt}%
\definecolor{currentstroke}{rgb}{0.000000,0.000000,0.000000}%
\pgfsetstrokecolor{currentstroke}%
\pgfsetdash{}{0pt}%
\pgfsys@defobject{currentmarker}{\pgfqpoint{0.000000in}{0.000000in}}{\pgfqpoint{0.000000in}{0.027778in}}{%
\pgfpathmoveto{\pgfqpoint{0.000000in}{0.000000in}}%
\pgfpathlineto{\pgfqpoint{0.000000in}{0.027778in}}%
\pgfusepath{stroke,fill}%
}%
\begin{pgfscope}%
\pgfsys@transformshift{2.629080in}{1.536392in}%
\pgfsys@useobject{currentmarker}{}%
\end{pgfscope}%
\end{pgfscope}%
\begin{pgfscope}%
\pgfsetbuttcap%
\pgfsetroundjoin%
\definecolor{currentfill}{rgb}{0.000000,0.000000,0.000000}%
\pgfsetfillcolor{currentfill}%
\pgfsetlinewidth{0.602250pt}%
\definecolor{currentstroke}{rgb}{0.000000,0.000000,0.000000}%
\pgfsetstrokecolor{currentstroke}%
\pgfsetdash{}{0pt}%
\pgfsys@defobject{currentmarker}{\pgfqpoint{0.000000in}{0.000000in}}{\pgfqpoint{0.000000in}{0.027778in}}{%
\pgfpathmoveto{\pgfqpoint{0.000000in}{0.000000in}}%
\pgfpathlineto{\pgfqpoint{0.000000in}{0.027778in}}%
\pgfusepath{stroke,fill}%
}%
\begin{pgfscope}%
\pgfsys@transformshift{2.728617in}{1.536392in}%
\pgfsys@useobject{currentmarker}{}%
\end{pgfscope}%
\end{pgfscope}%
\begin{pgfscope}%
\pgfsetbuttcap%
\pgfsetroundjoin%
\definecolor{currentfill}{rgb}{0.000000,0.000000,0.000000}%
\pgfsetfillcolor{currentfill}%
\pgfsetlinewidth{0.602250pt}%
\definecolor{currentstroke}{rgb}{0.000000,0.000000,0.000000}%
\pgfsetstrokecolor{currentstroke}%
\pgfsetdash{}{0pt}%
\pgfsys@defobject{currentmarker}{\pgfqpoint{0.000000in}{0.000000in}}{\pgfqpoint{0.000000in}{0.027778in}}{%
\pgfpathmoveto{\pgfqpoint{0.000000in}{0.000000in}}%
\pgfpathlineto{\pgfqpoint{0.000000in}{0.027778in}}%
\pgfusepath{stroke,fill}%
}%
\begin{pgfscope}%
\pgfsys@transformshift{2.828153in}{1.536392in}%
\pgfsys@useobject{currentmarker}{}%
\end{pgfscope}%
\end{pgfscope}%
\begin{pgfscope}%
\pgfsetbuttcap%
\pgfsetroundjoin%
\definecolor{currentfill}{rgb}{0.000000,0.000000,0.000000}%
\pgfsetfillcolor{currentfill}%
\pgfsetlinewidth{0.602250pt}%
\definecolor{currentstroke}{rgb}{0.000000,0.000000,0.000000}%
\pgfsetstrokecolor{currentstroke}%
\pgfsetdash{}{0pt}%
\pgfsys@defobject{currentmarker}{\pgfqpoint{0.000000in}{0.000000in}}{\pgfqpoint{0.000000in}{0.027778in}}{%
\pgfpathmoveto{\pgfqpoint{0.000000in}{0.000000in}}%
\pgfpathlineto{\pgfqpoint{0.000000in}{0.027778in}}%
\pgfusepath{stroke,fill}%
}%
\begin{pgfscope}%
\pgfsys@transformshift{3.027225in}{1.536392in}%
\pgfsys@useobject{currentmarker}{}%
\end{pgfscope}%
\end{pgfscope}%
\begin{pgfscope}%
\pgfsetbuttcap%
\pgfsetroundjoin%
\definecolor{currentfill}{rgb}{0.000000,0.000000,0.000000}%
\pgfsetfillcolor{currentfill}%
\pgfsetlinewidth{0.602250pt}%
\definecolor{currentstroke}{rgb}{0.000000,0.000000,0.000000}%
\pgfsetstrokecolor{currentstroke}%
\pgfsetdash{}{0pt}%
\pgfsys@defobject{currentmarker}{\pgfqpoint{0.000000in}{0.000000in}}{\pgfqpoint{0.000000in}{0.027778in}}{%
\pgfpathmoveto{\pgfqpoint{0.000000in}{0.000000in}}%
\pgfpathlineto{\pgfqpoint{0.000000in}{0.027778in}}%
\pgfusepath{stroke,fill}%
}%
\begin{pgfscope}%
\pgfsys@transformshift{3.126761in}{1.536392in}%
\pgfsys@useobject{currentmarker}{}%
\end{pgfscope}%
\end{pgfscope}%
\begin{pgfscope}%
\pgfsetbuttcap%
\pgfsetroundjoin%
\definecolor{currentfill}{rgb}{0.000000,0.000000,0.000000}%
\pgfsetfillcolor{currentfill}%
\pgfsetlinewidth{0.602250pt}%
\definecolor{currentstroke}{rgb}{0.000000,0.000000,0.000000}%
\pgfsetstrokecolor{currentstroke}%
\pgfsetdash{}{0pt}%
\pgfsys@defobject{currentmarker}{\pgfqpoint{0.000000in}{0.000000in}}{\pgfqpoint{0.000000in}{0.027778in}}{%
\pgfpathmoveto{\pgfqpoint{0.000000in}{0.000000in}}%
\pgfpathlineto{\pgfqpoint{0.000000in}{0.027778in}}%
\pgfusepath{stroke,fill}%
}%
\begin{pgfscope}%
\pgfsys@transformshift{3.226297in}{1.536392in}%
\pgfsys@useobject{currentmarker}{}%
\end{pgfscope}%
\end{pgfscope}%
\begin{pgfscope}%
\pgfsetbuttcap%
\pgfsetroundjoin%
\definecolor{currentfill}{rgb}{0.000000,0.000000,0.000000}%
\pgfsetfillcolor{currentfill}%
\pgfsetlinewidth{0.602250pt}%
\definecolor{currentstroke}{rgb}{0.000000,0.000000,0.000000}%
\pgfsetstrokecolor{currentstroke}%
\pgfsetdash{}{0pt}%
\pgfsys@defobject{currentmarker}{\pgfqpoint{0.000000in}{0.000000in}}{\pgfqpoint{0.000000in}{0.027778in}}{%
\pgfpathmoveto{\pgfqpoint{0.000000in}{0.000000in}}%
\pgfpathlineto{\pgfqpoint{0.000000in}{0.027778in}}%
\pgfusepath{stroke,fill}%
}%
\begin{pgfscope}%
\pgfsys@transformshift{3.425369in}{1.536392in}%
\pgfsys@useobject{currentmarker}{}%
\end{pgfscope}%
\end{pgfscope}%
\begin{pgfscope}%
\pgfsetbuttcap%
\pgfsetroundjoin%
\definecolor{currentfill}{rgb}{0.000000,0.000000,0.000000}%
\pgfsetfillcolor{currentfill}%
\pgfsetlinewidth{0.602250pt}%
\definecolor{currentstroke}{rgb}{0.000000,0.000000,0.000000}%
\pgfsetstrokecolor{currentstroke}%
\pgfsetdash{}{0pt}%
\pgfsys@defobject{currentmarker}{\pgfqpoint{0.000000in}{0.000000in}}{\pgfqpoint{0.000000in}{0.027778in}}{%
\pgfpathmoveto{\pgfqpoint{0.000000in}{0.000000in}}%
\pgfpathlineto{\pgfqpoint{0.000000in}{0.027778in}}%
\pgfusepath{stroke,fill}%
}%
\begin{pgfscope}%
\pgfsys@transformshift{3.524905in}{1.536392in}%
\pgfsys@useobject{currentmarker}{}%
\end{pgfscope}%
\end{pgfscope}%
\begin{pgfscope}%
\pgfsetbuttcap%
\pgfsetroundjoin%
\definecolor{currentfill}{rgb}{0.000000,0.000000,0.000000}%
\pgfsetfillcolor{currentfill}%
\pgfsetlinewidth{0.602250pt}%
\definecolor{currentstroke}{rgb}{0.000000,0.000000,0.000000}%
\pgfsetstrokecolor{currentstroke}%
\pgfsetdash{}{0pt}%
\pgfsys@defobject{currentmarker}{\pgfqpoint{0.000000in}{0.000000in}}{\pgfqpoint{0.000000in}{0.027778in}}{%
\pgfpathmoveto{\pgfqpoint{0.000000in}{0.000000in}}%
\pgfpathlineto{\pgfqpoint{0.000000in}{0.027778in}}%
\pgfusepath{stroke,fill}%
}%
\begin{pgfscope}%
\pgfsys@transformshift{3.624442in}{1.536392in}%
\pgfsys@useobject{currentmarker}{}%
\end{pgfscope}%
\end{pgfscope}%
\begin{pgfscope}%
\pgfsetbuttcap%
\pgfsetroundjoin%
\definecolor{currentfill}{rgb}{0.000000,0.000000,0.000000}%
\pgfsetfillcolor{currentfill}%
\pgfsetlinewidth{0.803000pt}%
\definecolor{currentstroke}{rgb}{0.000000,0.000000,0.000000}%
\pgfsetstrokecolor{currentstroke}%
\pgfsetdash{}{0pt}%
\pgfsys@defobject{currentmarker}{\pgfqpoint{-0.048611in}{0.000000in}}{\pgfqpoint{-0.000000in}{0.000000in}}{%
\pgfpathmoveto{\pgfqpoint{-0.000000in}{0.000000in}}%
\pgfpathlineto{\pgfqpoint{-0.048611in}{0.000000in}}%
\pgfusepath{stroke,fill}%
}%
\begin{pgfscope}%
\pgfsys@transformshift{1.839427in}{0.781098in}%
\pgfsys@useobject{currentmarker}{}%
\end{pgfscope}%
\end{pgfscope}%
\begin{pgfscope}%
\definecolor{textcolor}{rgb}{0.000000,0.000000,0.000000}%
\pgfsetstrokecolor{textcolor}%
\pgfsetfillcolor{textcolor}%
\pgftext[x=1.672760in, y=0.733271in, left, base]{\color{textcolor}\rmfamily\fontsize{10.000000}{12.000000}\selectfont \(\displaystyle {0}\)}%
\end{pgfscope}%
\begin{pgfscope}%
\pgfsetbuttcap%
\pgfsetroundjoin%
\definecolor{currentfill}{rgb}{0.000000,0.000000,0.000000}%
\pgfsetfillcolor{currentfill}%
\pgfsetlinewidth{0.803000pt}%
\definecolor{currentstroke}{rgb}{0.000000,0.000000,0.000000}%
\pgfsetstrokecolor{currentstroke}%
\pgfsetdash{}{0pt}%
\pgfsys@defobject{currentmarker}{\pgfqpoint{-0.048611in}{0.000000in}}{\pgfqpoint{-0.000000in}{0.000000in}}{%
\pgfpathmoveto{\pgfqpoint{-0.000000in}{0.000000in}}%
\pgfpathlineto{\pgfqpoint{-0.048611in}{0.000000in}}%
\pgfusepath{stroke,fill}%
}%
\begin{pgfscope}%
\pgfsys@transformshift{1.839427in}{1.503460in}%
\pgfsys@useobject{currentmarker}{}%
\end{pgfscope}%
\end{pgfscope}%
\begin{pgfscope}%
\definecolor{textcolor}{rgb}{0.000000,0.000000,0.000000}%
\pgfsetstrokecolor{textcolor}%
\pgfsetfillcolor{textcolor}%
\pgftext[x=1.394982in, y=1.455632in, left, base]{\color{textcolor}\rmfamily\fontsize{10.000000}{12.000000}\selectfont \(\displaystyle {20000}\)}%
\end{pgfscope}%
\begin{pgfscope}%
\pgfsetbuttcap%
\pgfsetroundjoin%
\definecolor{currentfill}{rgb}{0.000000,0.000000,0.000000}%
\pgfsetfillcolor{currentfill}%
\pgfsetlinewidth{0.602250pt}%
\definecolor{currentstroke}{rgb}{0.000000,0.000000,0.000000}%
\pgfsetstrokecolor{currentstroke}%
\pgfsetdash{}{0pt}%
\pgfsys@defobject{currentmarker}{\pgfqpoint{-0.027778in}{0.000000in}}{\pgfqpoint{-0.000000in}{0.000000in}}{%
\pgfpathmoveto{\pgfqpoint{-0.000000in}{0.000000in}}%
\pgfpathlineto{\pgfqpoint{-0.027778in}{0.000000in}}%
\pgfusepath{stroke,fill}%
}%
\begin{pgfscope}%
\pgfsys@transformshift{1.839427in}{0.961689in}%
\pgfsys@useobject{currentmarker}{}%
\end{pgfscope}%
\end{pgfscope}%
\begin{pgfscope}%
\pgfsetbuttcap%
\pgfsetroundjoin%
\definecolor{currentfill}{rgb}{0.000000,0.000000,0.000000}%
\pgfsetfillcolor{currentfill}%
\pgfsetlinewidth{0.602250pt}%
\definecolor{currentstroke}{rgb}{0.000000,0.000000,0.000000}%
\pgfsetstrokecolor{currentstroke}%
\pgfsetdash{}{0pt}%
\pgfsys@defobject{currentmarker}{\pgfqpoint{-0.027778in}{0.000000in}}{\pgfqpoint{-0.000000in}{0.000000in}}{%
\pgfpathmoveto{\pgfqpoint{-0.000000in}{0.000000in}}%
\pgfpathlineto{\pgfqpoint{-0.027778in}{0.000000in}}%
\pgfusepath{stroke,fill}%
}%
\begin{pgfscope}%
\pgfsys@transformshift{1.839427in}{1.142279in}%
\pgfsys@useobject{currentmarker}{}%
\end{pgfscope}%
\end{pgfscope}%
\begin{pgfscope}%
\pgfsetbuttcap%
\pgfsetroundjoin%
\definecolor{currentfill}{rgb}{0.000000,0.000000,0.000000}%
\pgfsetfillcolor{currentfill}%
\pgfsetlinewidth{0.602250pt}%
\definecolor{currentstroke}{rgb}{0.000000,0.000000,0.000000}%
\pgfsetstrokecolor{currentstroke}%
\pgfsetdash{}{0pt}%
\pgfsys@defobject{currentmarker}{\pgfqpoint{-0.027778in}{0.000000in}}{\pgfqpoint{-0.000000in}{0.000000in}}{%
\pgfpathmoveto{\pgfqpoint{-0.000000in}{0.000000in}}%
\pgfpathlineto{\pgfqpoint{-0.027778in}{0.000000in}}%
\pgfusepath{stroke,fill}%
}%
\begin{pgfscope}%
\pgfsys@transformshift{1.839427in}{1.322870in}%
\pgfsys@useobject{currentmarker}{}%
\end{pgfscope}%
\end{pgfscope}%
\begin{pgfscope}%
\pgfpathrectangle{\pgfqpoint{1.839427in}{0.745132in}}{\pgfqpoint{1.950908in}{0.791260in}}%
\pgfusepath{clip}%
\pgfsetrectcap%
\pgfsetroundjoin%
\pgfsetlinewidth{1.505625pt}%
\definecolor{currentstroke}{rgb}{0.121569,0.466667,0.705882}%
\pgfsetstrokecolor{currentstroke}%
\pgfsetdash{}{0pt}%
\pgfpathmoveto{\pgfqpoint{1.972142in}{1.500426in}}%
\pgfpathlineto{\pgfqpoint{1.985414in}{1.454014in}}%
\pgfpathlineto{\pgfqpoint{1.998685in}{1.415801in}}%
\pgfpathlineto{\pgfqpoint{2.011957in}{1.368812in}}%
\pgfpathlineto{\pgfqpoint{2.025228in}{1.348224in}}%
\pgfpathlineto{\pgfqpoint{2.038500in}{1.306002in}}%
\pgfpathlineto{\pgfqpoint{2.051771in}{1.267356in}}%
\pgfpathlineto{\pgfqpoint{2.065043in}{1.231202in}}%
\pgfpathlineto{\pgfqpoint{2.078314in}{1.212168in}}%
\pgfpathlineto{\pgfqpoint{2.091585in}{1.179950in}}%
\pgfpathlineto{\pgfqpoint{2.104857in}{1.148744in}}%
\pgfpathlineto{\pgfqpoint{2.118128in}{1.132672in}}%
\pgfpathlineto{\pgfqpoint{2.131400in}{1.108689in}}%
\pgfpathlineto{\pgfqpoint{2.144671in}{1.080517in}}%
\pgfpathlineto{\pgfqpoint{2.157943in}{1.065889in}}%
\pgfpathlineto{\pgfqpoint{2.171214in}{1.041112in}}%
\pgfpathlineto{\pgfqpoint{2.184486in}{1.025148in}}%
\pgfpathlineto{\pgfqpoint{2.197757in}{1.008462in}}%
\pgfpathlineto{\pgfqpoint{2.211029in}{0.996254in}}%
\pgfpathlineto{\pgfqpoint{2.224300in}{0.979314in}}%
\pgfpathlineto{\pgfqpoint{2.237572in}{0.969743in}}%
\pgfpathlineto{\pgfqpoint{2.250843in}{0.959124in}}%
\pgfpathlineto{\pgfqpoint{2.264115in}{0.939729in}}%
\pgfpathlineto{\pgfqpoint{2.277386in}{0.931747in}}%
\pgfpathlineto{\pgfqpoint{2.290658in}{0.918744in}}%
\pgfpathlineto{\pgfqpoint{2.303929in}{0.912604in}}%
\pgfpathlineto{\pgfqpoint{2.317201in}{0.902202in}}%
\pgfpathlineto{\pgfqpoint{2.330472in}{0.893895in}}%
\pgfpathlineto{\pgfqpoint{2.343744in}{0.888477in}}%
\pgfpathlineto{\pgfqpoint{2.357015in}{0.877317in}}%
\pgfpathlineto{\pgfqpoint{2.370287in}{0.873019in}}%
\pgfpathlineto{\pgfqpoint{2.383558in}{0.869190in}}%
\pgfpathlineto{\pgfqpoint{2.396830in}{0.860992in}}%
\pgfpathlineto{\pgfqpoint{2.410101in}{0.853985in}}%
\pgfpathlineto{\pgfqpoint{2.423373in}{0.847375in}}%
\pgfpathlineto{\pgfqpoint{2.436644in}{0.844739in}}%
\pgfpathlineto{\pgfqpoint{2.449915in}{0.839899in}}%
\pgfpathlineto{\pgfqpoint{2.463187in}{0.836323in}}%
\pgfpathlineto{\pgfqpoint{2.489730in}{0.827835in}}%
\pgfpathlineto{\pgfqpoint{2.503001in}{0.823357in}}%
\pgfpathlineto{\pgfqpoint{2.516273in}{0.821298in}}%
\pgfpathlineto{\pgfqpoint{2.529544in}{0.819817in}}%
\pgfpathlineto{\pgfqpoint{2.542816in}{0.815627in}}%
\pgfpathlineto{\pgfqpoint{2.556087in}{0.812377in}}%
\pgfpathlineto{\pgfqpoint{2.582630in}{0.808765in}}%
\pgfpathlineto{\pgfqpoint{2.595902in}{0.805695in}}%
\pgfpathlineto{\pgfqpoint{2.609173in}{0.805550in}}%
\pgfpathlineto{\pgfqpoint{2.622445in}{0.803853in}}%
\pgfpathlineto{\pgfqpoint{2.635716in}{0.801614in}}%
\pgfpathlineto{\pgfqpoint{2.648988in}{0.801686in}}%
\pgfpathlineto{\pgfqpoint{2.662259in}{0.798796in}}%
\pgfpathlineto{\pgfqpoint{2.675531in}{0.796701in}}%
\pgfpathlineto{\pgfqpoint{2.688802in}{0.796701in}}%
\pgfpathlineto{\pgfqpoint{2.702074in}{0.796954in}}%
\pgfpathlineto{\pgfqpoint{2.715345in}{0.794896in}}%
\pgfpathlineto{\pgfqpoint{2.755160in}{0.790886in}}%
\pgfpathlineto{\pgfqpoint{2.781702in}{0.790056in}}%
\pgfpathlineto{\pgfqpoint{2.821517in}{0.787889in}}%
\pgfpathlineto{\pgfqpoint{2.848060in}{0.786913in}}%
\pgfpathlineto{\pgfqpoint{2.861331in}{0.787275in}}%
\pgfpathlineto{\pgfqpoint{2.874603in}{0.785830in}}%
\pgfpathlineto{\pgfqpoint{2.887874in}{0.785180in}}%
\pgfpathlineto{\pgfqpoint{2.914417in}{0.784602in}}%
\pgfpathlineto{\pgfqpoint{2.927689in}{0.784999in}}%
\pgfpathlineto{\pgfqpoint{2.954232in}{0.783952in}}%
\pgfpathlineto{\pgfqpoint{2.967503in}{0.785035in}}%
\pgfpathlineto{\pgfqpoint{2.980775in}{0.783771in}}%
\pgfpathlineto{\pgfqpoint{2.994046in}{0.783771in}}%
\pgfpathlineto{\pgfqpoint{3.007318in}{0.783085in}}%
\pgfpathlineto{\pgfqpoint{3.020589in}{0.782760in}}%
\pgfpathlineto{\pgfqpoint{3.033861in}{0.783013in}}%
\pgfpathlineto{\pgfqpoint{3.047132in}{0.782868in}}%
\pgfpathlineto{\pgfqpoint{3.060404in}{0.783518in}}%
\pgfpathlineto{\pgfqpoint{3.073675in}{0.782904in}}%
\pgfpathlineto{\pgfqpoint{3.086947in}{0.782579in}}%
\pgfpathlineto{\pgfqpoint{3.100218in}{0.782688in}}%
\pgfpathlineto{\pgfqpoint{3.113490in}{0.782074in}}%
\pgfpathlineto{\pgfqpoint{3.126761in}{0.782363in}}%
\pgfpathlineto{\pgfqpoint{3.140032in}{0.781965in}}%
\pgfpathlineto{\pgfqpoint{3.166575in}{0.782001in}}%
\pgfpathlineto{\pgfqpoint{3.179847in}{0.781821in}}%
\pgfpathlineto{\pgfqpoint{3.193118in}{0.782146in}}%
\pgfpathlineto{\pgfqpoint{3.232933in}{0.781676in}}%
\pgfpathlineto{\pgfqpoint{3.246204in}{0.782001in}}%
\pgfpathlineto{\pgfqpoint{3.272747in}{0.781640in}}%
\pgfpathlineto{\pgfqpoint{3.312562in}{0.781424in}}%
\pgfpathlineto{\pgfqpoint{3.339105in}{0.781460in}}%
\pgfpathlineto{\pgfqpoint{3.352376in}{0.781568in}}%
\pgfpathlineto{\pgfqpoint{3.378919in}{0.781424in}}%
\pgfpathlineto{\pgfqpoint{3.485091in}{0.781207in}}%
\pgfpathlineto{\pgfqpoint{3.511634in}{0.781315in}}%
\pgfpathlineto{\pgfqpoint{3.551448in}{0.781135in}}%
\pgfpathlineto{\pgfqpoint{3.591263in}{0.781135in}}%
\pgfpathlineto{\pgfqpoint{3.657620in}{0.781135in}}%
\pgfpathlineto{\pgfqpoint{3.684163in}{0.781207in}}%
\pgfpathlineto{\pgfqpoint{3.723978in}{0.781171in}}%
\pgfpathlineto{\pgfqpoint{3.750521in}{0.781171in}}%
\pgfpathlineto{\pgfqpoint{3.792335in}{0.781135in}}%
\pgfpathlineto{\pgfqpoint{3.792335in}{0.781135in}}%
\pgfusepath{stroke}%
\end{pgfscope}%
\begin{pgfscope}%
\pgfpathrectangle{\pgfqpoint{1.839427in}{0.745132in}}{\pgfqpoint{1.950908in}{0.791260in}}%
\pgfusepath{clip}%
\pgfsetrectcap%
\pgfsetroundjoin%
\pgfsetlinewidth{1.505625pt}%
\definecolor{currentstroke}{rgb}{1.000000,0.498039,0.054902}%
\pgfsetstrokecolor{currentstroke}%
\pgfsetdash{}{0pt}%
\pgfpathmoveto{\pgfqpoint{1.972142in}{1.446646in}}%
\pgfpathlineto{\pgfqpoint{1.985414in}{1.431729in}}%
\pgfpathlineto{\pgfqpoint{1.998685in}{1.411576in}}%
\pgfpathlineto{\pgfqpoint{2.011957in}{1.388677in}}%
\pgfpathlineto{\pgfqpoint{2.025228in}{1.360432in}}%
\pgfpathlineto{\pgfqpoint{2.038500in}{1.347610in}}%
\pgfpathlineto{\pgfqpoint{2.051771in}{1.328901in}}%
\pgfpathlineto{\pgfqpoint{2.065043in}{1.311962in}}%
\pgfpathlineto{\pgfqpoint{2.078314in}{1.298598in}}%
\pgfpathlineto{\pgfqpoint{2.091585in}{1.285704in}}%
\pgfpathlineto{\pgfqpoint{2.104857in}{1.262950in}}%
\pgfpathlineto{\pgfqpoint{2.118128in}{1.252945in}}%
\pgfpathlineto{\pgfqpoint{2.131400in}{1.240665in}}%
\pgfpathlineto{\pgfqpoint{2.144671in}{1.229035in}}%
\pgfpathlineto{\pgfqpoint{2.157943in}{1.218019in}}%
\pgfpathlineto{\pgfqpoint{2.171214in}{1.207834in}}%
\pgfpathlineto{\pgfqpoint{2.184486in}{1.189341in}}%
\pgfpathlineto{\pgfqpoint{2.197757in}{1.185079in}}%
\pgfpathlineto{\pgfqpoint{2.211029in}{1.171860in}}%
\pgfpathlineto{\pgfqpoint{2.224300in}{1.163192in}}%
\pgfpathlineto{\pgfqpoint{2.237572in}{1.163372in}}%
\pgfpathlineto{\pgfqpoint{2.250843in}{1.155101in}}%
\pgfpathlineto{\pgfqpoint{2.264115in}{1.144013in}}%
\pgfpathlineto{\pgfqpoint{2.277386in}{1.137873in}}%
\pgfpathlineto{\pgfqpoint{2.290658in}{1.128663in}}%
\pgfpathlineto{\pgfqpoint{2.303929in}{1.127326in}}%
\pgfpathlineto{\pgfqpoint{2.317201in}{1.118514in}}%
\pgfpathlineto{\pgfqpoint{2.330472in}{1.110893in}}%
\pgfpathlineto{\pgfqpoint{2.343744in}{1.110387in}}%
\pgfpathlineto{\pgfqpoint{2.357015in}{1.097132in}}%
\pgfpathlineto{\pgfqpoint{2.370287in}{1.100924in}}%
\pgfpathlineto{\pgfqpoint{2.383558in}{1.093159in}}%
\pgfpathlineto{\pgfqpoint{2.396830in}{1.088644in}}%
\pgfpathlineto{\pgfqpoint{2.410101in}{1.086332in}}%
\pgfpathlineto{\pgfqpoint{2.423373in}{1.077014in}}%
\pgfpathlineto{\pgfqpoint{2.436644in}{1.069285in}}%
\pgfpathlineto{\pgfqpoint{2.449915in}{1.067623in}}%
\pgfpathlineto{\pgfqpoint{2.463187in}{1.062639in}}%
\pgfpathlineto{\pgfqpoint{2.476458in}{1.064373in}}%
\pgfpathlineto{\pgfqpoint{2.489730in}{1.053465in}}%
\pgfpathlineto{\pgfqpoint{2.503001in}{1.053934in}}%
\pgfpathlineto{\pgfqpoint{2.516273in}{1.053357in}}%
\pgfpathlineto{\pgfqpoint{2.529544in}{1.048192in}}%
\pgfpathlineto{\pgfqpoint{2.542816in}{1.047108in}}%
\pgfpathlineto{\pgfqpoint{2.569359in}{1.028074in}}%
\pgfpathlineto{\pgfqpoint{2.595902in}{1.029446in}}%
\pgfpathlineto{\pgfqpoint{2.609173in}{1.021789in}}%
\pgfpathlineto{\pgfqpoint{2.622445in}{1.022078in}}%
\pgfpathlineto{\pgfqpoint{2.635716in}{1.021031in}}%
\pgfpathlineto{\pgfqpoint{2.648988in}{1.016444in}}%
\pgfpathlineto{\pgfqpoint{2.662259in}{1.009220in}}%
\pgfpathlineto{\pgfqpoint{2.675531in}{1.010484in}}%
\pgfpathlineto{\pgfqpoint{2.688802in}{1.008137in}}%
\pgfpathlineto{\pgfqpoint{2.702074in}{1.003261in}}%
\pgfpathlineto{\pgfqpoint{2.715345in}{0.996615in}}%
\pgfpathlineto{\pgfqpoint{2.728617in}{0.991956in}}%
\pgfpathlineto{\pgfqpoint{2.741888in}{0.987874in}}%
\pgfpathlineto{\pgfqpoint{2.755160in}{0.993545in}}%
\pgfpathlineto{\pgfqpoint{2.768431in}{0.989716in}}%
\pgfpathlineto{\pgfqpoint{2.781702in}{0.983938in}}%
\pgfpathlineto{\pgfqpoint{2.794974in}{0.981120in}}%
\pgfpathlineto{\pgfqpoint{2.808245in}{0.979278in}}%
\pgfpathlineto{\pgfqpoint{2.821517in}{0.976642in}}%
\pgfpathlineto{\pgfqpoint{2.834788in}{0.969490in}}%
\pgfpathlineto{\pgfqpoint{2.848060in}{0.965951in}}%
\pgfpathlineto{\pgfqpoint{2.861331in}{0.964398in}}%
\pgfpathlineto{\pgfqpoint{2.874603in}{0.962375in}}%
\pgfpathlineto{\pgfqpoint{2.887874in}{0.960750in}}%
\pgfpathlineto{\pgfqpoint{2.901146in}{0.958799in}}%
\pgfpathlineto{\pgfqpoint{2.914417in}{0.956343in}}%
\pgfpathlineto{\pgfqpoint{2.927689in}{0.955404in}}%
\pgfpathlineto{\pgfqpoint{2.940960in}{0.951251in}}%
\pgfpathlineto{\pgfqpoint{2.954232in}{0.946230in}}%
\pgfpathlineto{\pgfqpoint{2.967503in}{0.941860in}}%
\pgfpathlineto{\pgfqpoint{2.980775in}{0.942871in}}%
\pgfpathlineto{\pgfqpoint{2.994046in}{0.936876in}}%
\pgfpathlineto{\pgfqpoint{3.007318in}{0.935684in}}%
\pgfpathlineto{\pgfqpoint{3.020589in}{0.934022in}}%
\pgfpathlineto{\pgfqpoint{3.033861in}{0.930808in}}%
\pgfpathlineto{\pgfqpoint{3.047132in}{0.925824in}}%
\pgfpathlineto{\pgfqpoint{3.060404in}{0.924776in}}%
\pgfpathlineto{\pgfqpoint{3.073675in}{0.924596in}}%
\pgfpathlineto{\pgfqpoint{3.086947in}{0.917155in}}%
\pgfpathlineto{\pgfqpoint{3.100218in}{0.916975in}}%
\pgfpathlineto{\pgfqpoint{3.126761in}{0.907620in}}%
\pgfpathlineto{\pgfqpoint{3.140032in}{0.908306in}}%
\pgfpathlineto{\pgfqpoint{3.153304in}{0.905742in}}%
\pgfpathlineto{\pgfqpoint{3.166575in}{0.901769in}}%
\pgfpathlineto{\pgfqpoint{3.179847in}{0.900649in}}%
\pgfpathlineto{\pgfqpoint{3.193118in}{0.900541in}}%
\pgfpathlineto{\pgfqpoint{3.206390in}{0.898952in}}%
\pgfpathlineto{\pgfqpoint{3.219661in}{0.896207in}}%
\pgfpathlineto{\pgfqpoint{3.232933in}{0.894581in}}%
\pgfpathlineto{\pgfqpoint{3.246204in}{0.886888in}}%
\pgfpathlineto{\pgfqpoint{3.259476in}{0.889633in}}%
\pgfpathlineto{\pgfqpoint{3.272747in}{0.880929in}}%
\pgfpathlineto{\pgfqpoint{3.286019in}{0.884071in}}%
\pgfpathlineto{\pgfqpoint{3.299290in}{0.880387in}}%
\pgfpathlineto{\pgfqpoint{3.312562in}{0.878762in}}%
\pgfpathlineto{\pgfqpoint{3.325833in}{0.875764in}}%
\pgfpathlineto{\pgfqpoint{3.339105in}{0.879520in}}%
\pgfpathlineto{\pgfqpoint{3.352376in}{0.872297in}}%
\pgfpathlineto{\pgfqpoint{3.365648in}{0.871574in}}%
\pgfpathlineto{\pgfqpoint{3.378919in}{0.869190in}}%
\pgfpathlineto{\pgfqpoint{3.392191in}{0.869010in}}%
\pgfpathlineto{\pgfqpoint{3.418734in}{0.863592in}}%
\pgfpathlineto{\pgfqpoint{3.432005in}{0.861642in}}%
\pgfpathlineto{\pgfqpoint{3.445277in}{0.863267in}}%
\pgfpathlineto{\pgfqpoint{3.458548in}{0.862292in}}%
\pgfpathlineto{\pgfqpoint{3.471820in}{0.855429in}}%
\pgfpathlineto{\pgfqpoint{3.485091in}{0.858102in}}%
\pgfpathlineto{\pgfqpoint{3.498362in}{0.856188in}}%
\pgfpathlineto{\pgfqpoint{3.511634in}{0.855032in}}%
\pgfpathlineto{\pgfqpoint{3.538177in}{0.853154in}}%
\pgfpathlineto{\pgfqpoint{3.551448in}{0.850409in}}%
\pgfpathlineto{\pgfqpoint{3.564720in}{0.847231in}}%
\pgfpathlineto{\pgfqpoint{3.577991in}{0.850156in}}%
\pgfpathlineto{\pgfqpoint{3.591263in}{0.847375in}}%
\pgfpathlineto{\pgfqpoint{3.604534in}{0.845750in}}%
\pgfpathlineto{\pgfqpoint{3.617806in}{0.848025in}}%
\pgfpathlineto{\pgfqpoint{3.631077in}{0.843763in}}%
\pgfpathlineto{\pgfqpoint{3.644349in}{0.844377in}}%
\pgfpathlineto{\pgfqpoint{3.657620in}{0.843691in}}%
\pgfpathlineto{\pgfqpoint{3.670892in}{0.840404in}}%
\pgfpathlineto{\pgfqpoint{3.684163in}{0.837768in}}%
\pgfpathlineto{\pgfqpoint{3.697435in}{0.838815in}}%
\pgfpathlineto{\pgfqpoint{3.710706in}{0.835131in}}%
\pgfpathlineto{\pgfqpoint{3.723978in}{0.834156in}}%
\pgfpathlineto{\pgfqpoint{3.737249in}{0.834914in}}%
\pgfpathlineto{\pgfqpoint{3.750521in}{0.834734in}}%
\pgfpathlineto{\pgfqpoint{3.763792in}{0.833614in}}%
\pgfpathlineto{\pgfqpoint{3.777064in}{0.833000in}}%
\pgfpathlineto{\pgfqpoint{3.792335in}{0.831287in}}%
\pgfpathlineto{\pgfqpoint{3.792335in}{0.831287in}}%
\pgfusepath{stroke}%
\end{pgfscope}%
\begin{pgfscope}%
\pgfpathrectangle{\pgfqpoint{1.839427in}{0.745132in}}{\pgfqpoint{1.950908in}{0.791260in}}%
\pgfusepath{clip}%
\pgfsetrectcap%
\pgfsetroundjoin%
\pgfsetlinewidth{1.003750pt}%
\definecolor{currentstroke}{rgb}{0.000000,0.000000,0.000000}%
\pgfsetstrokecolor{currentstroke}%
\pgfsetdash{}{0pt}%
\pgfpathmoveto{\pgfqpoint{2.011957in}{0.745132in}}%
\pgfpathlineto{\pgfqpoint{2.011957in}{1.536392in}}%
\pgfusepath{stroke}%
\end{pgfscope}%
\begin{pgfscope}%
\pgfsetrectcap%
\pgfsetmiterjoin%
\pgfsetlinewidth{0.803000pt}%
\definecolor{currentstroke}{rgb}{0.000000,0.000000,0.000000}%
\pgfsetstrokecolor{currentstroke}%
\pgfsetdash{}{0pt}%
\pgfpathmoveto{\pgfqpoint{1.839427in}{0.745132in}}%
\pgfpathlineto{\pgfqpoint{1.839427in}{1.536392in}}%
\pgfusepath{stroke}%
\end{pgfscope}%
\begin{pgfscope}%
\pgfsetrectcap%
\pgfsetmiterjoin%
\pgfsetlinewidth{0.803000pt}%
\definecolor{currentstroke}{rgb}{0.000000,0.000000,0.000000}%
\pgfsetstrokecolor{currentstroke}%
\pgfsetdash{}{0pt}%
\pgfpathmoveto{\pgfqpoint{3.790335in}{0.745132in}}%
\pgfpathlineto{\pgfqpoint{3.790335in}{1.536392in}}%
\pgfusepath{stroke}%
\end{pgfscope}%
\begin{pgfscope}%
\pgfsetrectcap%
\pgfsetmiterjoin%
\pgfsetlinewidth{0.803000pt}%
\definecolor{currentstroke}{rgb}{0.000000,0.000000,0.000000}%
\pgfsetstrokecolor{currentstroke}%
\pgfsetdash{}{0pt}%
\pgfpathmoveto{\pgfqpoint{1.839427in}{0.745132in}}%
\pgfpathlineto{\pgfqpoint{3.790335in}{0.745132in}}%
\pgfusepath{stroke}%
\end{pgfscope}%
\begin{pgfscope}%
\pgfsetrectcap%
\pgfsetmiterjoin%
\pgfsetlinewidth{0.803000pt}%
\definecolor{currentstroke}{rgb}{0.000000,0.000000,0.000000}%
\pgfsetstrokecolor{currentstroke}%
\pgfsetdash{}{0pt}%
\pgfpathmoveto{\pgfqpoint{1.839427in}{1.536392in}}%
\pgfpathlineto{\pgfqpoint{3.790335in}{1.536392in}}%
\pgfusepath{stroke}%
\end{pgfscope}%
\end{pgfpicture}%
\makeatother%
\endgroup%

    %\input{images/auswertung/no_pr_cl_2_cl_3_histograms.pgf}  und (c) $\mathbf{K}_3$ 
    \caption{Histogramme der (a) direkten Pixelwerte und der mit (b) $\mathbf{K}_2$-Clustering-Kern gefalteten Pixelwerte, wobei nur die lokalen Maxima im \qtyproduct{2 x 2}{\px}-Umgebungen behalten werden. In den eingefügten Graphen sind die Bereiche detailliert dargestellt, in denen sich die Schnittpunkte (vertikale Linien) der Dunkelbilder- und Streubilder-Histogramme befinden. Für jedes Histogramm wurden \num{10000} Dunkelbilder sowie Streubilder verwendet.}
    \label{fig:no_pr_cl_2_histograms}
\end{figure}
\noindent
In beiden Histogrammen können keine erkennbaren Peaks beobachtet werden, die den Photonenen-Ereignissen entsprechen würden. Allerdings existieren Intervalle in den beiden Histogrammen, wo die Streubilder mehr Pixel bzw. \qtyproduct{2 x 2}{\px}-Cluster als Dunkelbilder enthalten. Im Histogramm von den nicht-geclusterten Pixeln (Abb.~\ref{fig:no_pr_cl_2_histograms}a) beginnt das Intervall mit dem Wert \SI{72}{\adu}, obwohl das hellste Pixel ca.\ \SI{116}{\adu} pro Photon enthält. Die Asymmetrie des ersten Peaks im Histogramm der Streubilder konnte bisher nicht erklärt werden.  Im Histogramm von den geclusterten Pixeln (Abb.~\ref{fig:no_pr_cl_2_histograms}b) beginnt das Intervall mit dem Wert \SI{172}{\adu}, was mit dem erwarteten Ein-Photon-Signal \SI{180}{\adu} gut korreliert. 

\noindent
Der zweite Faktor ist der konstante Offset, der sich aus der Mittelung der Dunkelbilder ergibt und von jedem Streubild subtrahiert wird. Es scheint so zu sein, dass der statische Hintergrund sich im Laufe der Zeit verändert. Darüber hinaus hängt die Veränderung vom Photonenfluss ab. So wird ein zusätzlicher Offset $W_\Delta$ zu jedem Pixel addiert.

\noindent
Dieser Offset taucht wohl bei der Anwendung des Schwellenwert-Algorithmus auf, scheint aber vernachlässigbar klein gegenüber dem einzelnen Pixelwert zu sein. Bei der Anwendung des Clustering-Algorithmus mit dem \qtyproduct{2 x 2}{\px} vervierfacht sich der Offset $W_\Delta$ und erhöht die Zahl der fehldetektierten Photonen.

\noindent
Der Cluster-Kern kann in Bezug auf die vorliegende Gesamtladungsverteilung, die grundlegend zwischen zwei benachbarten Pixeln stattfindet, angepasst werden. Zum Beispiel können jeweils die Cluster-Kerne
\begin{equation}
    \mathbf{K}_{2\times1} = \begin{bmatrix}
1\\
1
\end{bmatrix}
\text{ bzw. }
    \mathbf{K}_{1\times2} = \begin{bmatrix}
1 & 1
\end{bmatrix}
\end{equation}
benutzt werden. So wird die Standardabweichung eines Clusters von $2\sigma_R$ auf $\sqrt{2}\sigma_R$ verringert. Ebenso wird der gesamte Offset nur $2W_\Delta$ statt $4W_\Delta$ sein. Der Offset $W_\Delta$ kann durch die häufigere Aufnahme von Dunkelbildern gesenkt werden.


% 12 Photonenereignisse zu Photonen-Ereignisse gemacht, weil es ab 6.3 immer so geschrieben wird
% 20 Ich würde es lieber "Dunkelbild-Analyse" schreiben, aber ich weiß nicht, ob ich irgendein \label zerschieße, wenn ich dir in ne Überschrift reinpfusche
% 45 unten "in willkürlichen Einheiten" - können wir es als "Schritte" bezeichnen? Nur das "beliebige Einheiten" über Abb. 16 ist noch schlimmer ;)
% 48 Motorpositionsveränderung... du hast ein Hang zu langen Worten, aber ich zerlege das total gern :)
% 62 XMCD im PDF über zwei Zeilen :/ bekommen wir den Satz umgestellt, damit das nicht mehr passiert?
% 68 Kombination aus Kommentaren zu den Zeilen 12 und 20: Ich würds gern entsprechend Zeile 12 ändern, aber im Sinne von Zeile 20 trau ich mich nicht
% 70 "Um die Überlagerung der Ladungsverteilung der benachbarten Photonen-Ereignisse auszuschließen, werden nur diejenigen Photonen-Ereignisse erfasst, die keine Photonen-Ereignisse im \SI{5}{px}-Umkreis haben." Ganz schön viele Photonen-Ereignisse in einem Satz. Ist aber vllt auch egal.
% 85 oben "welche die Bedingungen im oberen Paragraph erfüllen" Paragraph = Absatz?
% 85 bin mit dem Satz immer noch überfordert unten "enthält allerdings die Standardabweichung des Detektorrauschens $\sigma_{R}$ und die Standardabweichung der ADU-Wert Verteilung eines Photons" Ab ADU-Wert Verteilung eines Photons kapier ich den Satz nicht mehr
% 101 gleichmäßig? unten "dass der gesamte ADU-Wert eines Photons in einem einzelnen Photonen-Ereignis nicht simultan über die benachbarten Pixel verteilt wird, sondern nur in einem davon. " Ist "simultan" (gleichzeitig) hier wirklich das richtige Wort?
% 124 Signal-zu-Rausch
% 141 "Die Zahl zeigt das Verhältnis zwischen der Zahl der Photonen, die an dem Detektor detektiert wurden, und die mit dem Schwellenwert-Algorithmus als Photon gekennzeichnet wird." Das Wort "Zahl" an zweiter Stelle würde ich gern gegen die Abkürzung QE austauschen. Muss ich da nur \gls{qe} reinkopieren? Und anhand des Satzbaus kann ich die darüber stehende Gleichung nicht nachvollziehen
% 180 "Für ein erfasstes Pixel in der Summe" - Im Durchschnitt?
% 197 "Der hellste rechförmige Fleck im Zentrum ist der Direktstrahl" rechförmig = recht unförmig, rechteckig, ...?
% 263 oben "Diese Verteilungsfunktionen" wieso Plural?
% 263 unten "zusammen mit der Referenz" evtl noch mal den Querverweis zur Quelle anbringen (?)
% 266 Legende zur orangen Linie innerhalb der Graphik: Betragsquadrat der Fourier-TransformierteN voM MFM-Scan
% 271 "Die gemessene Intensitätsvertilung hat eine größere radiale Verbreitung des Streuringes" Verbreitung = Ausbreitung, Verbreiterung?
% 288 "(Rauligkeit und Kristalline von Dünnschichten und Substrat)" Rauhigkeit und Kristallinität?
% 291 unten "dadurch die Sensitivität der Photonenerkennung zu erhöhen." Was tut LaTeX den Photonen da betreffs Worttrennung an x.x
% 304 "Schwellen\-wert-Algorithmus in Abb. \ref{fig:capture_ped_diff} benutzt wurde." Dann kann man den Zeilenumbruch wohl doch an gewünschter Stelle erzwingen
% 308 "Die Unterabbildungen (a), (b) und (c) stimmen mit Abb. \ref{fig:capture_ped_diff} überein." Darfst du Unterabbildungen ansprechen, die du gar nicht abgebildet hast?
% 316 "Für die Auswertung jeder Summen" aller Summen?
% 371-372 Zum Aufhübschen von Graphiken haben wir vermutlich keine Zeit mehr ^^ Abb. 28 gefällt mir nicht. Die horizontale Skalierung der beiden Untergraphiken stehen ohne Zusammenhang, dennoch wird in (b) die Ordinate unnötig weit nach rechts ausgedehnt. Der Peak links in b ist praktisch nicht zu sehen. Also die ganze Anordnung in der b-Subgraphik gefällt mir nicht
% 377 Ich glaub, ich habs kapiert - bitte gegenprüfen :) erste Satz macht mir schwammig im Kopp
% 380 "Der zweitwichtigste Faktor" wann hab ich die Anmoderation des allerwichtigsten Faktors verpasst?

% 150 das "kommt" gefällt mir nicht
% 170 wieso eigentlich oben-unten-links-rechts und nicht (a)-(b)-(c)-(d)?
% 267 "Das Maximum vom Fit" Also genau genommen müsste hier wieder der Genitiv ran: "Das Maximum des Fits", aber das klingt mir dann doch etwas zu gestelzt
% 271 "Fourier-Transformierte vom MFM-Scan" genau wie bei 267: Genitiv wäre nötig, ist aber hässlich. Ich hatte den Genitiv vorher schon ein paar Mal ignoriert... ich vertraue einfach meinem Bauchgefühl, an welchen Stellen man den Genitiv zum Dativ machen kann
% 274 habe ich den Satz richtig interpretiert?
% 341 auf welches Wort bezieht sich das Detektorrauschen? Ich glaube, das ist noch im falschen Fall gefangen.
% 365 ist die Zahl der fehldetektierten Photonen beim Clustering zehn mal HÖHER? Es war immerhin vom best-case-Szenario die Rede?? Das klingt wie "Wenns richtig gut läuft, ist das Schlimme nur Faktor 10x so schlimm, wie das Gute gut ist."