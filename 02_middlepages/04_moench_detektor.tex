\chapter{Röntgen-Detektor}
\label{text:detector_theorie}
\newacronym[user1=\emph{engl. analog-digital untis}]{adu}{ADU}{Analog-Digital-Einheiten}
\newacronym[user1=\emph{engl. analog-to-digital converter}]{adc}{ADC}{Analog-Digital-Umsetzer}
\newacronym[user1=\emph{engl. com\-ple\-men\-tary metal-oxide-semi\-con\-duc\-tor}]{cmos}{CMOS}{CMOS-Sensor}
\newacronym[user1=\emph{engl. charge-coupled device}]{ccd}{CCD}{CCD-Sensor}
\newacronym[user1=\emph{engl. active pixel sensor}]{aps}{APS}{aktiver Pixelsensor}
\newacronym{ptb}{PTB}{Physikalisch-Technischen Bundesanstalt}
\newacronym{psi}{PSI}{Paul Scherrer Institut}
\newacronym{sls}{SLS}{Swiss Light Source}
\newacronym{moench03}{MÖNCH03}{dritte Generation des MÖNCH-Detektors}
\newacronym[user1=\emph{engl. signal to noise ratio}]{snr}{SNR}{Signal-zu-Rausch-Verhältnis}
\newacronym[user1=\emph{engl. point spread function}]{psf}{PSF}{Punktspreizfunktion}
\newacronym{hzb}{HZB}{Helmholtz-Zentrum Berlin}
Um das Kleinwinkelstreusignal möglichst effizient aufzunehmen, werden heutzutage fast ausschließlich Flächendetektoren verwendet. Ein photonenempfindlicher Sensor ist das wichtigste Bauteil aller Detektoren. Die einfallenden Photonen erzeugen Ladung durch den inneren Fotoeffekt in den Fotodioden der Sensorfläche. Für den weiteren Ladungstransport und die Datenverarbeitung gibt es grundsätzlich zwei Primärtechnologien, wonach sich alle Sensoren aufteilen lassen.

\section{CCD-Sensor}
\label{text:ccd_theorie}
Die erste Sensorfamilie heißt \gls{ccd}. Bei diesem Sensortyp wird zuerst die von Fotodioden erzeugte Ladung an einen Leiter übertragen. Danach werden die Ladungen im Leiter durch die koordinierte zeitliche Änderung des lokalen elektrischen Potentials der Reihe nach verschoben. Typischerweise erfolgt die Verschiebung vertikal (entlang der Spalten) bis zu einem einzigen horizontalen Leiter. Dieser wird auch Schieberegister genannt und ermöglicht den weiteren Ladungstransport zum Verstärker. Dieses Verfahren ist genauerer in Abb. \ref{fig:ccd_scheme} dargestellt.
\begin{figure}[H]
    % \centering
    % \subfloat[]{\includegraphics[width=0.24\textwidth]{images/ccd_scheme/ccd_frame_1_amp.pdf} \label{fig:ccd_scheme_frame_1}}
    % \subfloat[]{\includegraphics[width=0.24\textwidth]{images/ccd_scheme/ccd_frame_2_amp.pdf} \label{fig:ccd_scheme_frame_2}}
    % \subfloat[]{\includegraphics[width=0.24\textwidth]{images/ccd_scheme/ccd_frame_3_amp.pdf} \label{fig:ccd_scheme_frame_3}}
    % \subfloat[]{\includegraphics[width=0.24\textwidth]{images/ccd_scheme/ccd_frame_4_amp.pdf} \label{fig:ccd_scheme_frame_4}}
    \input{ccd_auslese_all.pdf_tex}
    \caption{Die prinzipiellen Schritte des Ausleseprozesses am Beispiel eines \qtyproduct{2 x 2}{\px} \gls{ccd}-Sensors. Die hellblauen Bereiche sind als einzelne Fotodioden zu verstehen, die gelben und braunen Tafeln entsprechen den Leiterbereichen, in denen der Ladungstransport vertikal bzw. horizontal erfolgt. Der horizontale Leiter wird auch Schieberegister genannt. Das grüne Dreieck stellt einen Verstärker dar. Die Ladung wird (a) zuerst durch den inneren Fotoeffekt an den Fotodioden erzeugt und (b) dann an den vertikalen Leiter übertragen. Die Ladungen von Fotodioden der untersten Zeile (c) werden auf dem Schieberegister zuerst ankommen, während von oben nachgerückt wird. Im Laufe der Zeit (d) werden die Ladungen im Schieberegister weiter nach rechts zum Verstärker verschoben, wo die Ladung in eine Spannung umgewandelt wird. Während die Ladungen der untersten Zeile verarbeitet werden, wiederholen sich die Schritte (c) und (d) bis zum Ende der vertikalen Sensorauflösung.}
    \label{fig:ccd_scheme}
\end{figure}
\noindent
Die Ladungen werden seriell mit einem einzigen Verstärker zuerst in eine Spannung und dann mit einem \gls{adc} in das digitale Signal umgewandelt. So wird das Gesamtbild auf dem Sensor der Reihe nach ausgelesen.

\noindent
Die Dauer dieses Verfahrens hängt von der Sensorauflösung (Pixelanzahl) ab und kann relativ lang sein. Als Beispiel eines im Röntgenbereich aktiv benutzten \gls{ccd}-Detektors kann ein Princeton Instruments PI-MTE: 2048B$^{\text{®}}$ dienen. Im Falle der höchsten Ausleserate von \SI{1}{\mega\hertz} dauert das Auslesen des gesamten Sensors mit \qtyproduct{2048 x 2048}{\px} \SI{4,2}{\second} \cite[s. 81, Readout-Charakterstik]{mte-manual}. Die Auslesezeit kann  durch die Auslese eines kleineren Sensorbereichs verkürzt werden und beträgt im Falle eines \qtyproduct{400 x 400}{\px} Sensorunterbereichs \SI{160}{\milli\second}. Hauptvorteil und Grund für große Verbreitung von \gls{ccd}-Detektoren im wissenschaftlichen Kontext sind das sehr geringe Hintergrund- und Ausleserauschen, die im Falle von PI-MTE: 2048B$^{\text{®}}$ \SI[per-mode=symbol]{0.12}{\electron\per\pixel\per\second}. Für eine Belichtungszeit von \SI{3}{\second} würde es das Rauschen ca.\ \SI{15}{\electron} (RMS) betragen.

\section{CMOS-Sensor}
\label{text:cmos_theorie}
Die zweite große Sensorfamilie heißt \gls{cmos}. Sie wird auch häufig als \gls{aps} bezeichnet. Bei diesem Sensortyp besitzt jede Fotodiode einen eigenen Verstärker und die Ladung wird unmittelbar an der Fotodiode ohne weiteren Ladungstransport in die Spannung umgewandelt. Die Digitalisierung der Spannung einer bestimmten Fotodiode erfolgt durch das Schließen der entsprechenden Schalter für die Zeile und Spalte des Pixels, wobei alle anderen Schalter offen bleiben. Dieses Verfahren ist in Abb. \ref{fig:cmos_scheme} schematisch dargestellt.
\begin{figure}[H]
    % \centering
    % \subfloat[]{\includegraphics[width=0.30\textwidth]{images/cmos_scheme/cmos_auslese_frame_1_fat.pdf} \label{fig:cmos_scheme_frame_1}}
    % \hfill
    % \subfloat[]{\includegraphics[width=0.30\textwidth]{images/cmos_scheme/cmos_auslese_frame_2_fat.pdf} \label{fig:cmos_scheme_frame_2}}
    % \hfill
    % \subfloat[]{\includegraphics[width=0.30\textwidth]{images/cmos_scheme/cmos_auslese_frame_3_fat.pdf} \label{fig:cmos_scheme_frame_3}}
    \input{cmos_auslese_all.pdf_tex}
    % \hfill
    % \subfloat[]{\includegraphics[width=0.24\textwidth]{images/cmos_scheme/cmos_auslese_frame_4_fat.pdf} \label{fig:cmos_scheme_frame_4}}
    \caption{Die prinzipiellen Schritte vom Ausleseprozess am Beispiel eines \qtyproduct{2 x 2}{\px} CMOS-Sensors. Die hellblauen Bereiche sind als einzelne Fotodioden zu verstehen. Jedes grüne Dreieck ist ein Verstärker. Die Ladung wird (a) zuerst durch den inneren Fotoeffekt an den Fotodioden erzeugt und (b) dann durch den Verstärker an jedem Pixel in eine Spannung umgewandelt. Die Schalter entsprechender Zeilen und Spalten (c) werden geschlossen. Durch ihre Permutationen werden die Spannungen jedes Pixels übertragen. Im Gegensatz zu einem CCD-Sensor kann das Auslesen von verschiedenen Spalten simultan laufen.}
    \label{fig:cmos_scheme}
\end{figure}
\noindent
Dadurch können die Spannungen parallel in Spalten mit mehreren \gls{adc}s parallel in allen Spalten digitalisiert und Auslesezeiten somt wesentlich verkürzt werden. Die typische Auslesezeit einer Zeile liegt im \si{\micro\second}-Bereich, was im Falle eines \qtyproduct{2048 x 2048}{\px} Bildes einige \si{\milli\second} für den ganzen Sensor ergibt.

\noindent
Diese Sensorfamilie ist allerdings nicht frei von Nachteilen. Der entscheidende Faktor, warum \gls{cmos}-Sensoren die \gls{ccd}-Sensoren nicht in allen Bereichen verdrängt haben, ist der wesentlich höhere Rauschpegel. Die Wahl des geeigneten Detektors hängt von der verfügbaren Zeit, gewünschten Auflösung und Art der Messung ab.

\section{MÖNCH-Detektor}
\label{text:moench_theorie}
Der MÖNCH-Detektor ist ein Röntgendetektorprototyp, der am \gls{psi} entwickelt wurde. Dieser stellt einen Hybrid-Detektor dar, dessen Sensor seinem Aufbau nach hauptsächlich einem \gls{cmos}-Sensor ähnelt. Die technischen Nachbesserungen (Abb. \ref{fig:moench_pixel}) ermöglichen, den Verstärkungsgrad zu variieren und, durch den Einbau eines zusätzlichen Kondensatorspeichers, die Auslesezeit zu beschleunigen.
\begin{figure}[H]
    \centering
    \includegraphics[width=0.6\textwidth]{images/moench/paper_crop.png}
    \caption{schematischer Aufbau eines Pixels des MÖNCH-Sensors. Quelle: \cite{dinapoli_monch_2014}}
    \label{fig:moench_pixel}
\end{figure}
\noindent
Für die Messungen in dieser Arbeit stand die \gls{moench03} zur Verfügung, die eine Auflösung von \qtyproduct{400 x 400}{\px} und eine verbesserte Rauschcharakteristik im Vergleich zu vorherigen Generationen hat. Die Pixelgröße beträgt \SI{25}{\micro\meter}. Der gesamte Sensor ist \qtyproduct{10 x 10}{\milli\meter} groß. Um eine unerwünschte Belichtung der Pixel mit sichtbarem Licht während der Messung zu verhindern, hat die Sensorfläche eine ca.\ \SI{500}{\nano\meter} dicke Al-Beschichtung.

\noindent
Die Belichtungszeit $\tau$ des \gls{moench03} kann minimal bis auf \SI{100}{\nano\second} gesetzt werden. Die Totzeit des Detektors nach einer Aufnahme beträgt \SI{250}{\micro\second}. Der Aufnahmevorgang des Detektors kann mit einem externen Triggersignal gestartet werden, was die zeitliche Synchronisation mit der Röntgenquelle ermöglicht. Die gemessenen Intensitätswerte werden vom Detektor in spezifischen Einheiten zur Verfügung gestellt und werden als \gls{adu} bezeichnet.

\subsection{Rauschcharakteristik}
Das Detektorsignal des Hintergrundrauschen folgt gut einer Gauß-Verteilung. Der Mittelwert $\bar{x}$ und die Standardabweichung $\sigma$ vom Hintergrundrauschen hängen direkt von der Belichtungszeit $\tau$ ab. Der Wert jedes Pixels hat einen für eine bestimmte Belichtungszeit $\tau$ konstanten Offset, der durch den Abzug der gemittelten Dunkelbilder bei der Datennachbearbeitung eliminiert werden kann. Aus diesem Grund ist die Standardabweichung $\sigma$ die entscheidende Größe, die Auskunft über das \gls{snr} des Detektors gibt.

\noindent
Da der Rauschpegel des Detektors von den einzelnen Bauelementen abhängt und sich innerhalb einer Baureihe unterscheiden kann, wurde für diese Arbeit die  Rauschcharakteristik des eingesetzten Detektormoduls bestimmt.
\begin{figure}[H]
    \centering
    \input{images/moench/noise_referenz_two_plots.pgf}
    \caption{Die Rauschcharakteristik des \gls{moench03} in Bezug auf die Belichtungszeit $\tau$ ist in (a) in \gls{adu} und logarithmisch bzw. (b) in den erzeugten Elementarladungen e$^-$ linear aufgetragen. Die entsprechenden Messungen (blau) wurden am 2. März 2022 unmittelbar mit dem \gls{moench03} durchgeführt. Die Koeffizienten für den Referenz-Fit (b) $\sigma(\tau)$ stammen aus \cite{ramilli-measurements-2017} und lauten $a=\num{0,2}$ e${^-}^2\si{\per\micro\second}$ und $b=\num{34.3}$ e${^-}$. Im Vergleich dazu weist die Messreihe einen vergleichbaren Rauschpegel-Offset im Bereich der minimal erreichbaren Belichtungszeiten auf; für größere $\tau$ zeigt sich ein proportional stärkeres Rauschen beim verwendeten Detektor.}
    \label{fig:noise_moench}
\end{figure}
\noindent
Die Ergebnisse einer Untersuchung des Detektorrauschens in Abhängigkeit von der Belichtungszeit $\tau$ sind in Abb. \ref{fig:noise_moench} dargestellt. Deutlich ist die Abhängigkeit des Rauschpegels von der Belichtungszeit zu erkennen, insbesondere für Belichtungszeiten oberhalb von \SI{0.1}{\milli\second}. Generell sollte die Belichtugnszeit so kurz wie möglich gewählt werden.

\noindent
In Abb. \ref{fig:noise_moench}b wird die Standardabweichung des Rauschens vom verfügbaren \gls{moench03}, die von \si{\adu} in die äquivalente Zahl der erzeugten Elektronen umgerechnet wurde, mit der Rauschcharakteristik vom Referenzblatt des \gls{psi} verglichen \cite{ramilli-measurements-2017}. Dafür wird das gemessene Detektorsignal anhand der bekannten Responsecharakteristik von \si{\adu} in \si{\eV} (siehe Umrechnung nach Gl. (\ref{eq:adu_to_ev}) später in diesem Kapitel) und dann in der äquivalenten Zahl der in Si erzeugten Elektronen-Loch-Paare ($\SI{3,66}{\eV}$ pro Paar) \cite{scholze_mean_1998} dargestellt.

\noindent
Obwohl die Rauschcharakteristik des verfügbaren \gls{moench03} zeigt eine deutlich höhere Standardabweichung bei den größeren Belichtungzeiten $\tau$, nähert die Rauschcharakteristik bei der kürzesten Belichtungszeit von \SI{100}{\nano\second} dem Referenzwert. Bei der hier verwendeten Belichtungszeit von \SI{1}{\micro\second} ergibt sich ein Rauschen von \SI[per-mode=symbol]{36}{\electron\per\pixel}.

\subsection{Responsecharakteristik}
Das Detektorsignal eines Pixels in \gls{adu} hängt unmittelbar von der Energie $h\nu$ des einfallenden Photons ab. Der Zusammenhang dieser Größen nennt sich Responsecharakteristik. Diese hängt auch von der Einstellung des Verstärkungsgrades am Detektor ab. Bei der Aufnahme der Responsecharakteristik und im Laufe der Messungen zu dieser Arbeit wurde der höchste Verstärkungsgrad (G4\_HIGHGAIN) benutzt, um ein möglichst hohes Einzelphotonsignal zu erhalten, das sich weit vom Rauschen absetzt. Die Elektronik des Detektors wurde mithilfe eines äußeren Wasserdurchlaufkühlers auf eine Temperatur von \SI{17}{\celsius} gekühlt. 
% \begin{figure}[H]
%     \centering
%     \input{images/moench/response_700_1400ev.pgf}
%     \caption{Die Sensorresponse und die berechnete Quantumeffizienz des MÖNCH-Detektors in Bezug auf die Photonenenergie $h\nu$. Die entsprechenden Messungen an der kalibrierten Röntgenquelle von \gls{ptb} wurden am 8. Oktober 2020 durchgeführt.}
%     \label{fig:response_moench}
% \end{figure}

\noindent
Die Responsecharakteristik des verfügbaren \gls{moench03} wurde bei der \gls{ptb} an den kalibrierten Synchrotronstrahlungsquellen MLS und BESSY II vermessen. Die Messung wurde im Vorfeld dieser Arbeit durchgeführt. Die Photonenenergie wurde jeweils im Bereich von \SI{30}{\eV} bis \SI{70}{\eV} und von \SI{400}{\eV} bis \SI{1900}{\eV} frei durchgestimmt. Der Energiebereich zwischen \SI{70}{\eV} und \SI{400}{\eV} oberhalb der L-Kanten von Al und Si wurde ausgelassen, weil die Quanteneffizienz des Detektors in diesem Bereich extrem gering ist. An beiden Synchrotronstrahlungsquellen waren der Strom am Beschleuniger in \si{\ampere} und die Leistung in \si[per-mode = symbol]{\photons\per\ampere\per\second} bekannt und ließen somit den Photonenfluss in \si[per-mode = symbol]{\photons\per\second} ermitteln. Die Belichtungszeit wurde dem Photonenfluss so angepasst, dass die Überbelichtung der Pixel auszuschließen und dadurch die Abhängigkeit des Detektorsignals von der Photonenenergie möglichst linear genähert werden konnte. Das Detektorsignal in \si{\adu} wurde über den detektierten Strahlquerschnitt aufsummiert und durch die Photonenzahl, die sich als Produkt von Belichtungszeit und Photonenfluss ergibt, geteilt. Dieser Photonenfluss wurde über der Photonenenergie aufgetragen.

\noindent
Wie bereits erwähnt wurde, hat der Sensor eine \SI{500}{\nano\meter} Al-Beschichtung. Während diese die Al-Beschichtung niederenergetische Strahlung (z.B. sichtbares Licht) vollständig blockt, müssen für weiche Röntgentstrahlung Verluste gemäß den Absoprtions-Koeffizienten in Kauf genommen werden.

\noindent
Die Messpunkte werden mit der Funktion
\begin{equation}
    R(h\nu, r, \{d_m\}) = h\nu\cdot r\cdot\underbrace{\exp\left[-\sum_{m}\mu_m(h\nu)d_m\right]}_{\text{Transmissionsfaktor}}
\label{eq:response_durchschnitt}
\end{equation}
angepasst, wobei $r$ - konstanter Responsekoeffizient in \si{\adu\per\eV} pro Photon, $\mu_m(h\nu)$ - materialspezifischer energieabhängiger, Absorptionskoeffizient für $m \in \{\text{Al}, \text{Si}, \text{Al$_2$O$_3$}\}$,  $d_m$ - entsprechende Materialdicke. Die Absorptionskoeffizienten $\mu_m(h\nu)$ stammen aus \cite{xray-coeffs}. Die gesuchten Koeffizienten sind $r$ und $\{d_m\}$.

\noindent
Die Messpunkte sowie die Fit-Funktion sind in Abb. \ref{fig:response_moench} dargestellt.
\begin{figure}[H]
    \centering
    \input{images/moench/response_200_1900ev.pgf}
    \caption{Die Sensorresponse (orange Linie) und die berechnete Quanteneffizienz (rote Linie) des MÖNCH-Detektors in Bezug auf die Photonenenergie $h\nu$. Die entsprechenden Messungen an der kalibrierten Röntgenquelle der \gls{ptb} wurden am 8. Oktober 2020 durchgeführt. Der starke Knick von Sensorresponse und Quanteneffizienz bei der $h\nu \approx \SI{1559}{\eV}$ entspricht der K-Absorptionskante von Al.}
    \label{fig:response_moench}
\end{figure}
\noindent
Die Anpassung ergab die Werte
$r = \SI[per-mode = symbol]{1.51e-1}{\adu\per\eV\per\photon}$, $d_\text{Si} = \SI{26.2}{\nano\meter}$, $d_\text{Al} = \SI{454}{\nano\meter}$,
$d_\text{Al$_2$O$_3$} = \SI{4.74}{\nano\meter}$, die mit der gegebenen Dicke der Al-Beschichtung plausibel scheinen.

\noindent
Die in Abb. \ref{fig:response_moench} dargestellte Responsecharakteristik bezeichnet die durchschnittliche Sensorresponse. Das heißt, dass das Responsesignal gleichmäßig auf die gesamte Photonenzahl geteilt wird, ungeachtet dessen, ob das Photon tatsächlich detektiert wird oder in der Al-Beschichtung absorbiert wird und den Sensor nicht erreicht.

\noindent
Wenn es um das Responsesignal eines einzelnen, tatsächlich detektierten Photons geht, muss der Transmissionsfaktor in Gl. (\ref{eq:response_durchschnitt}) der Fit-Funktion $R$ weggelassen werden. So ergibt sich die Proportionalität
\begin{equation}
    R_\text{detek}(h\nu) = h\nu \cdot r
    \label{eq:adu_to_ev}
\end{equation}
zwischen dem detektierten Signal eines tatsächlich detektierten Photons und seiner Photonenenergie $h\nu$, wobei derselbe Koeffizient $r = \SI[per-mode = symbol]{1.51e-1}{\adu\per\eV\per\photon}$ eingesetzt werden kann. Somit lassen sich die erwarteten \gls{adu}-Werte $W$ für tatsächlich detektierte Photonen mit den Enenergien $h\nu_{\text{Fe, L3}} = \SI{706.97}{\eV}$ und $h\nu_{\text{Gd, M5}} = \SI{1184.79}{\eV}$ ermitteln. Diese ergeben sich zu:
\begin{equation}
\begin{split}
     W_{\text{Fe, L3}} &=  \SI{107(1)}{\adu}\\
     W_{\text{Gd, M5}} &=  \SI{180(1)}{\adu}
\end{split}
\label{eq:auselesewerte_fe_gd}
\end{equation}


\section{Detektion einzelner Photonen}
\label{text:single_photon_theorie}
Der \gls{moench03} stellt einen Detektor dar, der sich aufgrund seines Auslesezeit-Rausch-Ver\-hält\-nis\-ses insbesondere für die Detektion von einzelnen Photonen gut geeignet ist \cite{bergamaschi_monch_2018}. Das Triggerinterface des \gls{moench03} in Kombination mit der Totzeit von ca.\ \SI{250}{\micro\second}, die deutlich kürzer als die Pulsperiode T$_\text{PXS} = \SI{10}{\milli\second}$ ist, ermöglicht, die von der \gls{pxs} ausgelösten Pulse einzeln aufzunehmen. Die Standardabweichung des Detektorrauschens $\sigma = \SI{19}{\adu}$ bei der kürzesten Belichtungszeit $\tau = \SI{100}{\nano\second}$, das grundlegend Ausleserauschen ist, ist signifkant kleiner als die erwarteten $W_{\text{Fe, L3}}$ und $W_{\text{Gd, M5}}$.

\noindent
Der Photonenfluss der \gls{pxs} liegt bei ca.\ \SI{617}{\photons} pro Puls (s. Kap. \ref{text:quelle_roentgen}) und wird in Hinblick auf die niedrige erwartete Transmissionsrate der Probe (s. Abb. \ref{fig:proben_vergleich_centered}) stark abgeschwächt. So ist es entscheidend, die einzelnen Photonen in einer Aufnahme zu erkennen und sie, als das gewünschte Signal, von ungewünschtem Hintergrundrauschen zu trennen. Für die bessere Effizienz der Nachbearbeitung soll die Differenz zwischen den \gls{adu}-Werten von Rauschen und Ein-Photon-Signal maximiert werden. Daher ist die Photonenenergie $h\nu_\text{Gd, M5}$ mit dem Wert $W_{\text{Gd, M5}} = \SI{180}{\adu}$ gegenüber der Photonenenergie $h\nu_\text{Fe, L3}$ mit $W_{\text{Fe, L3}} = \SI{107}{\adu}$ bevorzugt.

\noindent
Die Erkennung der einzelnen Photonen (in der Literatur oft als \emph{engl. single-photon counting} bezeichnet) ist eine besondere nummerische Herausforderung, weil ein Kompromiss zwischen Sensitivität und Spezifität gefunden werden muss. Es existieren zahlreiche Algorithmen für diese Aufgabe und es wird weiter nach besseren Lösungen gesucht. Die Effizienz der Algorithmen hängt von der Detektorauflösung, dem Photonenfluss, dem \gls{snr} und weiteren Parametern der Messdaten ab. In folgenden Unterabschnitten werden die Grundlagen der einzusetzenden Algorithmen dargestellt.

\subsection{Schwellenwert-Algorithmus}
\label{text:threshold_algorithm}
Eine der einfachsten Lösungen ist die Anwendung des Schwellenwert-Filters für die Pixelwerte. So werden die Pixel, deren \gls{adu}-Werte größer als der Schwellenwert $s_V$ sind, als Pixel mit einem Photon bezeichnet. Andernfalls werden die Pixel als Rauschen identifiziert und auf null gesetzt.

\noindent
Der Algorithmus hat trotz seiner einfachen Anwendung oft eine gute Selektivität. Es ergibt sich aber ein entscheidender Nachteil: Bei realen Sensoren tritt häufig der Effekt auf, dass die Ladung (als Wolke) über einander benachbarten Pixeln verteilt wird. So wird auch der \gls{adu}-Wert dementsprechend verteilt. Das Ein-Photon-Signal kann so stark über die Pixel-Nachbarschaft verteilt werden, dass keines der Pixel den Schwellenwert $s_V$ überschreitet und das eingefallene Photon nicht als solches identifiziert wird. Die Charakteristik, die die Stärke der Ladungsverteilung beschreibt, ist spezifisch für Detektor und Photonenenergie und heißt \gls{psf}. Ein Algorithmus, bei dem dieser Mangel behoben sein soll, wird im nächsten Unterabschnitt beschrieben.

\subsection{Clustering-Algorithmus}
\label{text:clustering_algorithm}
Der \emph{Clustering}-Algorithmus wurde entwickelt, um die mehrere Pixel vreteilte Ladung zusammenzufassen. Die Funktionsfähigkeit dieses Algorithmus wurde mit einem \gls{moench03} in \cite{cartier_micron_2014} bewiesen. Abweichend wurde dabei eine Photonenenergie von \SI{8}{\kilo\eV} genutzt. Photonen dieser Energie erzeugen ein deutlich stärkeres Signal als Photonen an den Absorptionskanten, die für \gls{xmcd}-basierte Messungen interessant sind.

\noindent
Die Grundidee dieses Algorithmus ist die Abbildung jedes $i,j$-ten Pixels in seiner Summe mit den benachbarten Pixeln. Als Nachbarbereich wird typischerweise ein Rechteck mit den Seiten $M$ und $L$ gewählt. Die Form und die Größe des Nachbarbereichs kann dem Detektor entsprechend angepasst werden, um die Genauigkeit des Algorithmus zu erhöhen.

\noindent
In anderen Worten wird zur Gesamtbildmatrix $\mathbf{V}$ eine Clustermatrix $\mathbf{Q}$ zugeordnet. Dieses Verfahren lässt sich formal wie folgt aufschreiben:
\begin{equation}
     \mathbf{V} * \mathbf{K}_{M\times L} =  \mathbf{Q} \Leftrightarrow Q_{i, j} = \sum_{l=i-\left[\frac{M}{2}\right]}^{i+\left[\frac{M}{2}\right]-1} \sum_{m=j-\left[\frac{L}{2}\right]}^{j+\left[\frac{L}{2}\right]-1} V_{l,m},
\end{equation}
wobei $\mathbf{V}$ - Auslesewerte des Detektors (Gesamtbild), dargestellt als Matrix, $\mathbf{K}_{M \times L}$ - Cluster-Kern, mit $*$ wird die Faltungs-Operation gekennzeichnet. Der Cluster-Kern gleicht im Allgemeinen
\begin{equation}
    \mathbf{K}_{M \times L}  = \begin{bmatrix}
1 & \cdots & 1\\
\vdots & \ddots & \vdots\\
1 & \cdots & 1
\end{bmatrix}
\in \mathbb{Z}^{M \times L} \text{ mit } \mathbf{K}_N = \mathbf{K}_{N \times N}
\label{eq:cluster-kern}
\end{equation}

\noindent
Im nächsten Schritt werden die Einträge der Clustermatrix $Q_{i,j}$ mit dem Schwellenwert 
\begin{equation}
    s_Q(M, L, \sigma, c) = c\sigma\sqrt{ML}
\end{equation}
verglichen, wobei $\sigma$ - Standardabweichung des Hintergrundrauschens und $c$ - freier Parameter. Die Einträge, die größer als der Schwellenwert $s_Q$ sind, werden behalten und als Cluster mit einem Photon bezeichnet. Diejenigen Einträge, die der Ungleichung nicht genügen, werden als Hintergrundrauschen identifiziert und auf null gesetzt.

\noindent
Anschließend wird nach den lokalen Maxima in der Matrix $\mathbf{Q}$ gesucht und diese behalten. Es ist wichtig zu beachten, dass der \gls{adu}-Wert jedes Pixels in $M\cdot L$ Clustern auftaucht. Das kann dazu führen, dass ein detektiertes Photon mehr als einmal gezählt wird. Um diese Fehlinterpretation zu vermeiden, muss eine Nebenbedingung gestellt werden, dass der Mindestabstand zwischen den lokalen Maxima größer als $\max\{M,L\}$ sein muss.

\noindent
Für den \gls{moench03} mit der Auflösung \qtyproduct{400 x 400}{\px} ist $\mathbf{V} \in \mathbb{Z}^{400\times 400}$. Die sinnvollen Cluster-Kerne für den Detektor sind rechteckig $M=L=N$ (vgl. Gl. (\ref{eq:cluster-kern})) mit $N = 2$ oder 3 \cite[Abschnitt 4]{cartier_micron_2014}. So wurden die Cluster-Kerne
\begin{equation}
    \mathbf{K}_{2} = \begin{bmatrix}
1 & 1\\
1 & 1
\end{bmatrix}
\text{ bzw. }
    \mathbf{K}_{3} = \begin{bmatrix}
1 & 1 & 1\\
1 & 1 & 1\\
1 & 1 & 1
\end{bmatrix}
\end{equation}
bei der Datenauswertung benutzt.

\noindent
In der Referenzarbeit wurde der Parameter $c$ von 3 bis 5 variiert. Der Schwellenwert $s_Q = cN\sigma$ mit $c=5$ und $N=2$ gleicht dem Wert $\SI{200}{\adu}$, was bereits höher als der erwartete \gls{adu}-Wert eines Photons mit Photonenenergie $h\nu_\text{Gd, M5}$ ist. Daher muss der Schwellenwert $s_Q$ im Falle der niedrigeren Photonenenergien genauer angepasst werden.